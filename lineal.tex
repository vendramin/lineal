\documentclass[11pt,a4paper,oneside,spanish]{amsbook}

\usepackage[T1]{fontenc}
\usepackage[latin9]{inputenc}
\usepackage{amsthm}
\usepackage{amssymb}
\usepackage{amstext}
\usepackage{mathtools}
\usepackage{stmaryrd}
\usepackage{lmodern}
\usepackage{xcolor} 
\usepackage[left=1.45in, right=1in, top=01in, bottom=1in, includefoot, textwidth = 5.77in, textheight=9.4in]{geometry}

%%% fonts
%\usepackage[osf,noBBpl]{mathpazo}
%\usepackage{eulervm}
\usepackage{mathrsfs}

\usepackage[notref,notcite]{showkeys}

\usepackage{babel}
\addto\shorthandsspanish{\spanishdeactivate{~<>}}

\usepackage[all]{xy}

\makeatletter
\swapnumbers

\numberwithin{section}{chapter}
\numberwithin{equation}{section}
\numberwithin{figure}{section}

\theoremstyle{plain}
\newtheorem{thm}{\protect\theoremname}[section]

%%% Enumearate
\es@enumerate{1)}{a)}{i)}{a$¿$)}

\makeatletter
\let\c@paragraph\c@thm
\let\c@equation\c@thm
\let\c@subsubsection\c@paragraph
\makeatother

\theoremstyle{definition}
\newtheorem{defn}[thm]{\protect\definitionname}
\newtheorem*{defn*}{\protect\definitionname}
\theoremstyle{definition}
\newtheorem{example}[thm]{\protect\examplename}
\newtheorem*{example*}{\protect\examplename}
\newtheorem{examples}[thm]{\protect\examplesname}
\theoremstyle{definition}
\newtheorem{xca}[thm]{\protect\exercisename}
\newtheorem*{xca*}{\protect\exercisename}
\theoremstyle{remark}
\newtheorem*{solution}{\protect\solutionname}
\theoremstyle{remark}
\newtheorem{remark}[thm]{\protect\remarkname}
\newtheorem*{remark*}{\protect\remarkname}
\newtheorem*{claim*}{\protect\claimname}
\newtheorem{remarks}[thm]{\protect\remarksname}
\newtheorem*{remarks*}{\protect\remarksname}
\theoremstyle{plain}
\newtheorem{prop}[thm]{\protect\propositionname}
\newtheorem*{prop*}{\protect\propositionname}
\newtheorem*{thm*}{\protect\theoremname}
\theoremstyle{plain}
\newtheorem{lem}[thm]{\protect\lemmaname}
\newtheorem*{lem*}{\protect\lemmaname}
\newtheorem{problem}[thm]{\protect\problemname}
\theoremstyle{plain}
\newtheorem{cor}[thm]{\protect\corollaryname}
\newtheorem*{cor*}{\protect\corollaryname}
\newtheorem{corollaries}[thm]{\protect\corollariesname}
\theoremstyle{remark}
\newtheorem{block}[thm]{}

\makeatother

\providecommand{\corollaryname}{Corolario}
\providecommand{\corollariesname}{Corolarios}
\providecommand{\definitionname}{Definición}
\providecommand{\examplename}{Ejemplo}
\providecommand{\exercisename}{Ejercicio}
\providecommand{\lemmaname}{Lema}
\providecommand{\propositionname}{Proposición}
\providecommand{\remarkname}{Observación}
\providecommand{\remarksname}{Observaciones}
\providecommand{\solutionname}{Solución}
\providecommand{\theoremname}{Teorema}
\providecommand{\examplesname}{Ejemplos}
\providecommand{\claimname}{Afirmación}
\providecommand{\problemname}{Problema}

\newcommand{\N}{\mathbb{N}}
\newcommand{\Q}{\mathbb{Q}}
\newcommand{\Z}{\mathbb{Z}}
\newcommand{\R}{\mathbb{R}}
\newcommand{\C}{\mathbb{C}}
\newcommand{\K}{\mathbb{K}}
\newcommand{\T}{\mathbb{T}}
\newcommand{\D}{\mathbb{D}}

\newcommand{\cB}{\mathscr{B}}
\newcommand{\cE}{\mathscr{E}}

%%% Definitions
\newcommand{\Aff}{\mathrm{Aff}}
\newcommand{\Rad}{\mathrm{rad}}
\newcommand{\Inn}{\mathrm{Inn}}
\newcommand{\Ext}{\mathrm{Ext}}
\newcommand{\Img}{\mathrm{Img}}
\newcommand{\Syl}{\mathrm{Syl}}
\newcommand{\id}{\operatorname{id}}
\newcommand{\Aut}{\operatorname{Aut}}
\newcommand{\End}{\operatorname{End}}
\newcommand{\Irr}{\operatorname{Irr}}
\newcommand{\GL}{\mathbf{GL}}
\newcommand{\SL}{\mathbf{SL}}
\newcommand{\Alt}{\mathbb{A}}
\newcommand{\Sym}{\mathbb{S}}
\newcommand{\lcm}{\mathrm{mcm}}

\newcommand{\sgn}{\operatorname{sgn}}
\newcommand{\tr}{\operatorname{tr}}
\newcommand{\rg}{\operatorname{rg}}
\newcommand{\im}{\operatorname{im}}
\newcommand{\inv}{\operatorname{I}}
\newcommand{\adj}{\operatorname{adj}}
\newcommand{\spec}{\operatorname{spec}}

\newcommand{\codim}{\operatorname{codim}}
\newcommand{\diag}{\operatorname{diag}}
\newcommand{\ann}{\operatorname{Ann}}
\newcommand{\dist}{\operatorname{dist}}
\newcommand{\re}{\operatorname{re}}

% column vector
\newcount\colveccount
\newcommand*\colvec[1]{
\global\colveccount#1
\begin{pmatrix}
	\colvecnext
	}
	\def\colvecnext#1{
	#1
	\global\advance\colveccount-1
	\ifnum\colveccount>0
	\\
	\expandafter\colvecnext
	\else
\end{pmatrix}
\fi
}

%\title{Notas de álgebra lineal}
%\author{Leandro Vendramin}
%\address{Departamento de Matemática
%Facultad de Ciencias Exactas y Naturales
%Universidad de Buenos Aires
%Ciudad Universitaria, Pabellón I
%Buenos Aires, Argentina}
%\email{lvendramin@dm.uba.ar}
%%\url{http://mate.dm.uba.ar/~lvendram/}

\makeindex

\begin{document}

%\maketitle


%\setlength{\oddsidemargin}{0mm} % Adjust margins to center the colored title box
%\setlength{\evensidemargin}{0mm} % Margins on even pages - only necessary if adding more content to this template

\newcommand{\HRule}[1]{\hfill \rule{0.2\linewidth}{#1}} 
\definecolor{grey}{rgb}{0.9,0.9,0.9} 

\thispagestyle{empty} 

\colorbox{grey}{
\parbox[t]{1.0\linewidth}{
\centering \fontsize{50pt}{80pt}\selectfont 
\vspace*{0.7cm} 

%\hfill 
\'Algebra lineal\par

\vspace*{0.7cm} 
}
}

\vfill 

{\centering \large 
\hfill Leandro Vendramin \\
\hfill Departamento de Matemática \\
\hfill Facultad de Ciencias Exactas y Naturales \\
\hfill Universidad de Buenos Aires \\
\hfill \verb+http://mate.dm.uba.ar/~lvendram/+ \\
\hfill \texttt{lvendramin@dm.uba.ar} \\

\HRule{1pt}} 

\clearpage 



\frontmatter
\tableofcontents

\mainmatter

%\chapter*{Agradecimientos}

A Santiago Laplagne.

\chapter{Sistemas lineales y matrices}

Dos clases. Temas: Repaso de sistemas lineales y matrices. 

\section{Cuerpos}

\begin{block}
    Un \textbf{cuerpo} $\K$ es un conjunto con dos operaciones binarias en $\K$, 
    %$\K\times\K\to\K$, 
    una llamada \textbf{suma} y denotada por $(x,y)\mapsto
    x+y$, y otra llamada \textbf{producto} y denotada por $(x,y)\mapsto xy$,
    que satisfacen las siguientes propiedades:
	\begin{enumerate}
		\item $(x+y)+z=x+(y+z)$ para todo $x,y,z\in\K$,
		\item existe un único $0\in\K$ tal que $x+0=0+x=x$ para todo $x\in\K$,
		\item para cada $x\in\K$ existe un único $-x\in\K$ tal que 
            \[
                x+(-x)=(-x)+x=0.
                \]
		\item $x+y=y+x$ para todo $x,y\in\K$,
		\item $(xy)z=x(yz)$ para todo $x,y,z\in\K$,
        \item existe un único $1\in\K\setminus\{0\}$ tal que $x1=1x=x$ para todo $x\in\K$,
        \item para cada $x\in\K\setminus\{0\}$ existe un único $x^{-1}\in\K$
            tal que \[
            xx^{-1}=x^{-1}x=1,
            \]
		\item $xy=yx$ para todo $x,y\in\K$,
		\item $x(y+z)=xy+xz$ para todo $x,y,z\in\K$.
	\end{enumerate}
\end{block}

\begin{examples}\
	\begin{enumerate}
		\item $\Q$, $\R$ y $\C$ son cuerpos.
		\item $\Z_p$ es cuerpo si y sólo si $p$ es un número primo.
        \item $\Q[\sqrt{2}]=\{a+b\sqrt{2}\mid a,b\in\Q\}$ es un cuerpo.
            Determine el inverso de un elemento de la forma $a+b\sqrt{2}$,
            donde $a,b\in\Q$.
	\end{enumerate}
\end{examples}

\section{Sistemas lineales y matrices}

\begin{block}
	Estudiaremos \textbf{sistemas de ecuaciones lineales} con $m$
	ecuaciones y $n$ incógnitas. Dados $a_{ij}$, donde $i\in\{1,\dots,m\}$ y
	$j\in\{1,\dots,n\}$, el problema consiste en encontrar $n$
	elementos de $\K$, digamos $x_1,\dots,x_n$, que satisfagan las condiciones
	\begin{equation}
		\label{eq:sistema}
		\begin{cases}
			\begin{aligned}
				a_{11}x_1+a_{12}x_2+\cdots+a_{1n}x_n&=b_1,\\
				a_{21}x_1+a_{22}x_2+\cdots+a_{2n}x_n&=b_2,\\
				\vdots\\
				a_{m1}x_1+a_{m2}x_2+\cdots+a_{mn}x_n&=b_m,
			\end{aligned}
		\end{cases}
	\end{equation}
	Toda tupla
	$(x_1,\dots,x_n)$ de elementos de $\K$ que satisface~\eqref{eq:sistema} se
	denomina \textbf{solución} del sistema. Si $b_1=\cdots=b_m=0$ el
	sistema~\eqref{eq:sistema} se denomina sistema \textbf{homogéneo}.
\end{block}

\begin{block}
	Un sistema lineal es \textbf{compatible determinado} cuando admite una
	única solución, \textbf{compatible indeterminado} cuando admite más de una
	solución e \textbf{incompatible} si no tiene soluciones.  Observemos que un
	sistema homogéneo siempre tiene solución: basta tomar la \textbf{solución
	trivial} $x_1=x_2=\cdots=x_n=0$.
\end{block}

\begin{example}
    Muchos problemas no lineales pueden resolverse mediante un sistema lineal.
    Como ejemplo demostraremos que existe una única terna $(a,b,c)$ que
    satisface las ecuaciones 
    \begin{equation}
        \label{eq:sistema_nolineal}
        \begin{aligned}
            abc=1, && a^2b^2c^3&=8, && a(bc)^{-1}=16, 
        \end{aligned}
    \end{equation}
    En efecto, si aplicamos $\log_2$, las
    ecuaciones~\eqref{eq:sistema_nolineal}, se transforman en el sistema lineal 
    \[
        \begin{cases}
            \begin{aligned}
                x+y+z &= 0,\\
                2x+2y+3z &= 3,\\
                x-y-z &= 4,
            \end{aligned}
        \end{cases}
    \]
    que tiene a $(x,y,z)=(2,-5,3)$ como única solución. Luego
    la única solución del
    sistema~\eqref{eq:sistema_nolineal} es $(a,b,c)=(4,2^{-5},8)$.
\end{example}

\begin{block}
	En ocasiones resulta conveniente escribir al sistema~\eqref{eq:sistema} de
	la siguiente forma:
	\begin{equation}
		\begin{cases}	
			F_1(x_1,\dots,x_n)=b_1,\\
			\quad\vdots\\
			F_m(x_1,\dots,x_n)=b_m,
		\end{cases}
	\end{equation}
	donde $F_1(x_1,\dots,x_n),\dots,F_m(x_1,\dots,x_n)$ son las funciones 
	definidas por 
	\[
	F_i(x_1,\dots,x_n)=a_{i1}x_1+a_{i2}x_2+\cdots+a_{in}x_n
	\]
	para $i\in\{1,\dots,m\}$. Observemos que las $F_i$ satisfacen 
	\begin{align*}
		& F_i(x_1+y_1,\dots,x_n+y_n)=F_i(x_1,\dots,x_n)+F_i(y_1,\dots,y_n),\\
		& F_i(\lambda x_1,\dots,\lambda x_n)=\lambda F_i(x_1,\dots,x_n),
	\end{align*}
	para todo $\lambda\in\K$.
\end{block}

\begin{prop}
    Supongamos que $\K$ tiene infinitos elementos y que el sistema
    lineal~\eqref{eq:sistema} tiene más de una solución.
    Entonces~\eqref{eq:sistema} tiene infinitas soluciones.

	\begin{proof}
        Sean $c$ y $d$ dos soluciones y $\lambda\in\K$. Demostremos que $r=\lambda
        c+(1-\lambda)d$ es solución del sistema~\eqref{eq:sistema}. En efecto, 
		\[
			F_i(r)=\lambda F_i(c)+(1-\lambda)F_i(d)=\lambda b_i+(1-\lambda)b_i=b_i
		\]
		para todo $i\in\{1,\dots,m\}$. Para ver que hay infinitas soluciones
		basta observar que si $\lambda,\mu\in\K$ con $\lambda\ne\mu$ entonces
		$\lambda c+(1-\lambda)d\ne \mu c+(1-\mu)d$.
	\end{proof}
\end{prop}

\begin{example}
	En $\R^3$ el sistema
	\begin{equation}
		\label{eq:sistema:2x3}
		\begin{cases}
			\begin{aligned}
				x+y+z &= 0,\\
				y+z &= 0,
			\end{aligned}
		\end{cases}
	\end{equation}
	tiene infinitas soluciones. De hecho, el conjunto de soluciones es la recta 
	\[
		\{\lambda(0,1,-1): \lambda\in\R\}.
	\]
	En cambio, el conjunto de soluciones en $\Z_3^3$ del
	sistema~\eqref{eq:sistema:2x3} es el conjunto finito
	$\{(0,0,0),(0,1,-1),(0,-1,1)\}$. 
\end{example}

\begin{block}
	Diremos que dos sistemas lineales son \textbf{equivalentes} si tienen
	exactamente el mismo conjunto de soluciones.	
\end{block}

\begin{block}
	Los sistemas lineales 
	\begin{align*}
		\begin{cases}
			\begin{aligned}
				x+y+z&=0,\\
				x+y&=0,
			\end{aligned}
		\end{cases}
		&&
		\begin{cases}
			\begin{aligned}
				x+y+z&=0,\\
				z&=0,
			\end{aligned}
		\end{cases}
	\end{align*}
	son equivalentes. En efecto, el conjunto de soluciones es
	$\{(\lambda,-\lambda,0): \lambda\in\K\}$.
\end{block}

\begin{block}
    Dado un sistema lineal como en~\eqref{eq:sistema} vamos a definir dos
    \textbf{operaciones elementales} que no alteran el conjunto de soluciones
    del sistema:
	\begin{enumerate}
		\item Intercambiar el lugar de dos ecuaciones.
		\item Reemplazar una ecuación, digamos $F_i(x_1,\dots,x_n)=b_i$, por la
			ecuación $(F_i+\lambda F_j)(x_1,\dots,x_n)=b_i+\lambda b_j$, donde
			$\lambda\in\K$. 
	\end{enumerate}

	\begin{thm*}
		Al aplicar estas operaciones elementales se obtiene un sistema
		equivalente al original.

		\begin{proof}
			El conjunto de soluciones de un sistema lineal es la intersección
			del conjunto de soluciones de cada ecuación lineal
			\[
                F_i(x_1,\dots,x_n)=b_i.
            \]
            Como la intersección de conjuntos es conmutativa, intercambiar dos
            ecuaciones no altera el conjunto de soluciones del sistema lineal.

			Para demostrar que al aplicar la segunda operación elemental el conjunto 
			de soluciones del sistema lineal no se altera, 
			observemos que 
            \[
            \begin{cases}
                F_i(x_1,\dots,x_n)=b_i,\\
                F_j(x_1,\dots,x_n)=b_j,
            \end{cases}
            \]
            si y sólo si  
            \[
            \begin{cases}
                (F_i+\lambda F_j)(x_1,\dots,x_n)=b_i+\lambda b_j,\\
                F_j(x_1,\dots,x_n)=b_j,
            \end{cases}
            \]
            para todo $\lambda\in\K$.
		\end{proof}
	\end{thm*}
\end{block}

\begin{block}[Método de Gauss]
	\label{block:sistemas:gauss}
    Supongamos que se tiene un sistema lineal como en~\eqref{eq:sistema} tal
    que $a_{11}\ne0$. Para cada $i\in\{2,\dots,m\}$ podemos utilizar la segunda
    operación elemental y reemplazar la $i$-ésima ecuación
    $F_i(x_1,\dots,x_n)=b_i$ por 
	\[
		F_i(x_1,\dots,x_n)-\frac{a_{i1}}{a_{11}}F_1(x_1,\dots,x_n)=b_i-\frac{a_{i1}}{a_{11}}b_1.
	\]
	De esta forma obtenemos un sistema lineal 
	\begin{equation*}
		\begin{cases}
			\begin{aligned}
				a_{11}x_1+a_{12}x_2+\cdots+a_{1n}x_n&=b_1,\\
				a_{22}'x_2+\cdots+a'_{2n}x_{n}&=b'_2,\\
				\vdots\\
				a_{m2}'x_2+\cdots+a'_{mn}x_{n}&=b'_m,
			\end{aligned}
		\end{cases}
	\end{equation*}
	equivalente a~\eqref{eq:sistema}. 
\end{block}

\begin{thm}
    \label{thm:sistemas:n>m}
    Sean $n,m\in\N$ tales que $n>m$. Un sistema de $m$ ecuaciones con $n$
    incógnitas es incompatible o compatible indeterminado.

    \begin{proof}
        Procederemos por inducción en $m$. 
        
		Supongamos primero que $m=1$. En este caso tenemos una ecuación tipo
		$\sum_{i=1}^n a_ix_i=b$. Si todos los $a_i$ son cero, el sistema es
		incompatible si $b\ne0$ y compatible indeterminado si $b=0$.  En
		cambio, si existe algún $a_i\ne0$, se tienen infinitas soluciones: las
		incógnitas $x_j$ con $j\ne i$ pueden tomar cualquier valor en $\K$ y
		$x_i=\frac1{a_i}\sum_{j\ne i}a_jx_j$.

		Supongamos ahora que $m>1$. Si todos los $a_{ij}$ son cero el resultado
		es trivialmente válido.  Si suponemos que existe al menos un
		$a_{ij}\ne0$, después de utilizar la operación elemental de intercambiar filas
		y (si fuera necesario) de cambiar el nombre de las variables, podemos
		suponer que $a_{11}\ne0$. El procedimiento visto
		en~\ref{block:sistemas:gauss} nos da un sistema de la forma
        \begin{equation*}
            \begin{cases}
                \begin{aligned}
                    a_{11}x_1+a_{12}x_2+\cdots+a_{1n}x_n&=b_1,\\
                    a_{22}'x_2+\cdots+a'_{2n}x_{n}&=b'_2,\\
                    \vdots\\
                    a_{m2}'x_2+\cdots+a'_{mn}x_{n}&=b'_m,
                \end{aligned}
            \end{cases}
        \end{equation*}
        que es equivalente al sistema original. Como 
        $n-1>m-1$, la hipótesis inductiva aplicada al sistema 
        \begin{equation*}
            \begin{cases}
                \begin{aligned}
                    a_{22}'x_2+\cdots+a'_{2n}x_{n}&=b'_2,\\
                    \vdots\\
                    a_{m2}'x_2+\cdots+a'_{mn}x_{n}&=b'_m,
                \end{aligned}
            \end{cases}
        \end{equation*}
		de $m-1$ ecuaciones y $n-1$ incógnitas nos dice que ese sistema es
		compatible indeterminado o incompatible.  Esto implica que el sistema
		original es compatible indeterminado o incompatible.
    \end{proof}
\end{thm}

\begin{cor}
	\label{cor:homogeneo}
	En un sistema homogéneo que tiene más incógnitas que ecuaciones existe al
	menos una solución no trivial.

	\begin{proof}
		Es consecuencia inmediata del teorema~\ref{thm:sistemas:n>m} ya que los
		sistemas homogéneos son siempre compatibles.
	\end{proof}
\end{cor}

\begin{block}
    Sean $\K$ un cuerpo y $n,m\in\N$. Una \textbf{matriz} $A=(a_{ij})$ sobre
    $\K$ de tamaño $m\times n$ es un arreglo de elementos de $\K$ de la forma
	\[
		A=\begin{pmatrix}
			a_{11} & a_{12} & \cdots & a_{1n}\\
			a_{21} & a_{22} & \cdots & a_{2n}\\
			\vdots & \vdots & \ddots & \vdots\\
			a_{m1} & a_{m2} & \cdots & a_{mn}
		\end{pmatrix}.
	\]
	Los $a_{ij}$ son los \textbf{elementos} de la matriz $A$.  El conjunto de
	matrices de tamaño $m\times n$ con elementos en $\K$ se denotará por
	$\K^{m\times n}$.
\end{block}

\begin{block}[Suma de matrices]
    Sean las matrices $A=(a_{ij})\in\K^{m\times n}$ y $B=(b_{ij})\in\K^{m\times
    n}$. Se define la \textbf{suma} de $A$ y $B$ como la matriz de $\K^{m\times
    n}$ dada por 
	\[
		(A+B)_{ij}=a_{ij}+b_{ij}
	\]
    para todo $i,j$.  Queda como ejercicio demostrar que el neutro para la suma
    es la matriz $0\in\K^{m\times n}$ dada por $0_{ij}=0$ para todo $i,j$, que
    el inverso es $(-A)_{ij}=-a_{ij}$ para todo $i,j$, y que la suma de
    matrices es asociativa y conmutativa. 
\end{block}

\begin{block}[Producto por un escalar]
    Sean $A=(a_{ij})\in\K^{m\times n}$ y $\lambda\in\K$. Se define $\lambda A$ como la
	matriz de $m\times n$ cuyos elementos son 
	\[
		(\lambda A)_{ij}=\lambda a_{ij}
	\]
    para todo $i,j$. Queda como ejercicio demostrar que las siguientes
    propiedades: $(\lambda\mu)A=\lambda(\mu A)$, $(\lambda+\mu)A=\lambda A+\mu
    A$ y $\lambda(A+B)=\lambda A+\lambda B$ para todo $\lambda,\mu\in\K$ y
    $A,B\in\K^{m\times n}$.
\end{block}

\begin{block}[Producto de matrices]
    Sean las matrices $A=(a_{ij})\in\K^{m\times n}$ y $B=(b_{ij})\in\K^{n\times
    p}$. Se define el \textbf{producto} $AB$ como la matriz de $m\times p$
    cuyos elementos son
	\[
		(AB)_{ij}=\sum_{k=1}^n a_{ik}b_{kj}
	\]
	para todo $i,j$.
\end{block}

\begin{example}
	El producto de matrices en general no es conmutativo: 
	\begin{align*}
		A=\begin{pmatrix}
			0 & 1\\
			0 & 0
		\end{pmatrix},
		&&
		B=\begin{pmatrix}
			1 & 0\\
			0 & 0
		\end{pmatrix},
		&&
		AB=\begin{pmatrix}
			0 & 0\\
			0 & 0
		\end{pmatrix},
		&&
		BA=\begin{pmatrix}
			0 & 1\\
			0 & 0
		\end{pmatrix}.
	\end{align*}
\end{example}

\begin{example}
    Sea $A=(A_{ij})\in\K^{m\times n}$.  Para $i\in\{1,\dots,n\}$ consideremos
    el vector columna $e_j\in\K^{n\times1}$ dado por
	\[
		(e_i)_j=\delta_{ij}=
		\begin{cases} 
			1 & \text{si $i=j$},\\ 
			0 & \text{si $i\ne j$}.
		\end{cases}
	\]
	Entonces $Ae_i\in\K^{m\times 1}$ y $Ae_i$ es la $i$-ésima columna de la
	matriz $A$.
\end{example}

\begin{block}
	Se define la \textbf{matriz identidad} $I$ de tamaño $n$ como la matriz
	cuadrada de $n\times n$ dada por $I_{ij}=\delta_{ij}$ para todo $i,j$. 
	Si $A\in\K^{m\times n}$, se tienen las siguientes dos propiedades:
	\begin{enumerate}
		\item Si $I$ es la identidad de $m\times m$ entonces $IA=A$.
		\item Si $I$ es la identidad de $n\times n$ entonces $AI=A$.
	\end{enumerate}
\end{block}

\begin{thm}
	El producto de matrices es asociativo. 
	\begin{proof}
		Supongamos que $A\in\K^{m\times n}$, $B\in\K^{n\times p}$ y
		$C\in\K^{p\times q}$.  Como el producto en $\K$ es asociativo,
		\begin{align*}
			\left(A(BC)\right)_{ij}
			&=\sum_{k=1}^n A_{ik}(BC)_{kj}
			=\sum_{k=1}^n A_{ik}\left(\sum_{l=1}^p B_{kl}C_{lj}\right)
			=\sum_{k=1}^n\sum_{l=1}^p A_{ik}B_{kl}C_{lj}\\
			&=\sum_{l=1}^p\left(\sum_{k=1}^n A_{ik}B_{kl}\right)C_{lj}
			=\sum_{l=1}^p (AB)_{il}C_{lj}
			=\left((AB)C\right)_{ij},
		\end{align*}
		tal como se quería demostrar.
	\end{proof}
\end{thm}

\begin{block}
	Observemos que el producto de matrices nos permite escribir el sistema
	lineal~\eqref{eq:sistema} como la ecuación matricial $Ax=b$, donde
	\begin{align*}
		A=\begin{pmatrix}
			a_{11} & a_{12} & \cdots & a_{1n}\\
			a_{21} & a_{22} & \cdots & a_{2n}\\
			\vdots & \vdots & \ddots & \vdots\\
			a_{m1} & a_{m2} & \cdots & a_{mn}
		\end{pmatrix},\quad
		x=\begin{pmatrix}
			x_1\\
			\vdots\\
			x_n\\
		\end{pmatrix},\quad
		b=\begin{pmatrix}
			b_1\\
			\vdots\\
			b_m\\
		\end{pmatrix}.
	\end{align*}
    La matriz $A$ se denomina \textbf{matriz del sistema}. La \textbf{matriz
    ampliada} es la matriz de tamaño $m\times(n+1)$ dada por
	\[
		(A|b)=
        \left(\begin{array}{cccc|c}
			a_{11} & a_{12} & \cdots & a_{1n} & b_1\\
			a_{21} & a_{22} & \cdots & a_{2n} & b_2\\
			\vdots & \vdots & \ddots & \vdots & \vdots\\
			a_{m1} & a_{m2} & \cdots & a_{mn} & b_m
		\end{array}\right).
	\]
\end{block}

\begin{xca}
	\label{xca:Ax=b_y_Ax=0}
	Sea $p$ una solución del sistema $Ax=b$ y $S$ el conjunto de soluciones del
	sistema homogéneo $As=0$. Entonces toda solucion de $Ax=b$ es de la forma
	$s+p$ para algún $s\in S$.
\end{xca}

\begin{block}
    \label{block:traspuesta}
	Si $A\in\K^{m\times n}$ se define la \textbf{traspuesta} de $A$ como la
	matriz $A^T\in\K^{n\times m}$ dada por
	\[
		(A^T)_{ij}=A_{ji}
	\]
    para todo $i,j$. De la definición es evidente que
    $(A^T)^T=A$. Vale además que $(AB)^T=B^TA^T$ pues si $A\in\K^{m\times n}$ y
    $B\in\K^{n\times p}$ entonces
	\begin{align*}
		(B^TA^T)_{ij}=\sum_{k=1}^n (B^T)_{ik}(A^T)_{kj}=\sum_{k=1}^n B_{ki}A_{jk}=\sum_{k=1}^n A_{jk}B_{ki}=(AB)_{ji}.
	\end{align*}
\end{block}

\begin{xca}
	\label{xca:I^T=I}
	Demuestre que si $I$ es la matriz identidad de $n\times n$ entonces
	$I^T=I$.
\end{xca}

\begin{block}
	Si $A\in\K^{n\times n}$ se define la \textbf{traza} de $A$ como
	\[
        \tr(A)=A_{11}+A_{22}+\cdots+A_{nn}.
    \]
	Valen las siguientes propiedades:
	\begin{enumerate}
		\item $\tr(A+B)=\tr(A)+\tr(B)$ para todo $A,B\in\K^{n\times n}$.
		\item $\tr(\lambda A)=\lambda\tr(A)$ para todo $A\in\K^{n\times n}$ y $\lambda\in\K$.
		\item $\tr(AB)=\tr(BA)$ para todo $A,B\in\K^{n\times n}$.
	\end{enumerate}
	Demostremos la tercera afirmación: 
	\begin{align*}
		\tr(AB)
		&=\sum_{i=1}^n (AB)_{ii}
		=\sum_{i=1}^n \sum_{j=1}^n A_{ij}B_{ji}\\
		&=\sum_{j=1}^n\sum_{i=1}^n B_{ji}A_{ij}
		=\sum_{j=1}^n (BA)_{jj}
		=\tr(BA).
	\end{align*}
\end{block}

\begin{prop}
	\label{pro:Av=Bv=>A=B}
	Sean $A$ y $B$ dos matrices de $n\times n$ tales que $Av=Bv$ para todo
	$v\in\K^{n\times1}$.  Entonces $A=B$.
	
	\begin{proof}
		Basta tomar $v=e_i$ para todo $i\in\{1,\dots,n\}$, donde los $e_i$
		están dados por $(e_i)_j=\delta_{ij}$.
	\end{proof}
\end{prop}

\section{Matrices inversibles}

\begin{block}
	Una matriz cuadrada $A\in\K^{n\times n}$ es \textbf{inversible} si existe
	$B\in\K^{n\times n}$ tal que $AB=BA=I$. El conjunto de matrices inversibles
	de $n\times n$ con elementos en $\K$ se denota por
	\[
		\GL(n,\K)=\{A\in\K^{n\times n}\mid A\text{ es inversible}\}.
	\]
\end{block}

\begin{remarks}\
	\label{rem:inversa}
	\begin{enumerate}
		\item La identidad $I$ es inversible pues $II=I$.
		\item Si $AB=I$ y $CA=I$ entonces $B=C$ pues
			\[
			B=IB=(CA)B=C(AB)=CI=C.
			\]
		\item La inversa de una matriz, si existe, es única.
		\item Si $A$ es inversible entonces $A^{-1}$ es inversible y
			$(A^{-1})^{-1}=A$.
		\item Si $A$ y $B$ son matrices inversibles entonces $AB$ es inversible y
			$(AB)^{-1}=B^{-1}A^{-1}$. 
	\end{enumerate}
\end{remarks}

\begin{remark}
	Un \textbf{grupo} es un conjunto $G$ junto con una operación binaria
	$G\times G\to G$, que denotamos por $(x,y)\mapsto xy$, tal que:
	\begin{enumerate}
		\item $x(yz)=(xy)z$ para todo $x,y,z\in G$.
		\item Existe $e\in G$ tal que $xe=ex=x$ para todo $x\in G$.
		\item Para cada $x\in G$ existe $x^{-1}\in G$ tal que $xx^{-1}=x^{-1}x=e$.
	\end{enumerate}

	Lo hecho en~\ref{rem:inversa} nos permite demostrar que $\GL(n,\K)$ es un
	\textbf{grupo}. Este grupo se denomina \textbf{grupo lineal general} de
	grado $n$ sobre $\K$. En general $\GL(n,\K)$ es no conmutativo. 
\end{remark}

\begin{xca}
	Encuentre dos matrices inversibles $A,B\in\R^{2\times2}$ tales que $AB\ne
	BA$.
\end{xca}

\begin{example}
	Vamos a calcular la inversa de la matriz $A=\begin{pmatrix}2 & 7\\1 &
		4\end{pmatrix}$. Sea $B=(b_{ij})$ tal que $AB=I$. Las ecuaciones
		correspondientes a $AB=I$ son
	\[
			\begin{aligned}
			2b_{11}+7b_{21}&=1, 
			&& 
			b_{11}+4b_{21}&=0, 
			&& 
			2b_{12}+7b_{22}&=0,
			&& 
			b_{12}+4b_{22}&=1.
		\end{aligned}
	\]
	Observemos que para resolver este sistema de $4\times 4$ alcanza con
	aplicar el método de Gauss en la matriz ampliada 
	\[
	\left(
	\begin{array}{cc|cc}
		2 & 7 & 1 & 0\\
		1 & 4 & 0 & 1
	\end{array}
	\right)
	\]

	El método de Gauss nos da un matriz 
	$B=\begin{pmatrix}
		4 & -7
		\\-1& 2
	\end{pmatrix}$
	que cumple $AB=I$. Esta matriz $B$ 
	también cumple que $BA=I$. ¿Por qué?
\end{example}

\begin{xca}
	Calcule la inversa de la matriz 
	$\begin{pmatrix}
		1 & 0 & 1\\
		1 & 1 & 1\\
		1 & 1 & 0
	\end{pmatrix}$.
\end{xca}

\begin{block}
    Si $A$ es inversible entonces $A^T$ es inversible y
    $(A^T)^{-1}=(A^{-1})^T$. En efecto, si $A$ es inversible entonces
    $AA^{-1}=A^{-1}A=I$. Entonces, al aplicar la traspuesta y
    utilizar las propiedades de la traza vistas en~\ref{block:traspuesta},
    \[
    (A^{-1})^TA^T=I=A^T(A^{-1})^T, 
    \]
    es decir: $A^T$ es inversible y $(A^T)^{-1}=(A^{-1})^T$.
\end{block}

\section{Una aplicación a la criptografía}

\begin{block}%[Una aplicación a la criptografía]
    Vamos a utilizar matrices inversibles para codificar mensajes.  Supongamos
    que queremos codificar el mensaje:
    \begin{center}
        \textsf{la pelota no se mancha}
    \end{center}
    Si a cada letra del alfabeto le asignamos un número tal como vemos en la tabla 
    \begin{align*}
        \textsf{a} && \textsf{b} && \textsf{c} && \textsf{d} && \textsf{e} && \cdots && \textsf{z}\\
        1 && 2 && 3 && 4 && 5 && \cdots && 26
    \end{align*}
    y al espacio le asignamos el número 27, el mensaje a encriptar se
    transforma en la siguiente tira de números:
    \begin{align}
        \label{eq:lapelotanosemancha}
        12 \; 1 \; 27 \; 16 \; 5 \; 12 \; 15 \; 20 \; 1 \; 27 \; 14 \; 15 \; 27 \; 19 \; 5 \; 27 \; 13 \; 1 \; 14 \; 3 \; 8 \; 1
    \end{align}
    Identificaremos a nuestro mensaje con la tira~\eqref{eq:lapelotanosemancha}.
    Para encriptar el mensaje utilizaremos una matriz inversible $A$ de
    $\R^{2\times2}$ y es por eso que necesitamos escribir
    a~\eqref{eq:lapelotanosemancha} como una matriz en $\R^{2\times 11}$:
    \[
    B=\left(\begin{array}{ccccccccccc}
        12 & 27 &  5 & 15 &  1 & 14 & 27 &  5 & 13 & 14 & 8\\
         1 & 16 & 12 & 20 & 27 & 15 & 19 & 27 &  1 &  3 & 1
        \end{array}
        \right).
    \]
    Nuestro método de encriptación consiste en multiplicar a izquierda a la
    matriz $B$ por $A$. Si, por ejemplo, elegimos la matriz inversible 
    \[
    A=\begin{pmatrix}
        -4 & 3\\
        3 & -2
    \end{pmatrix},
    \]
    el mensaje encriptado será entonces
    \[
        AB=\left(\begin{array}{ccccccccccc}
        -45 & -60 & 16 & 0 & 77 & -11 & -51 & 61 & -49 & -47 & -29\\
         34 & 49 & -9 & 5 & -51 & 12 & 43 & -39 & 37 & 36 & 22
        \end{array}
        \right).
    \]
    Observemos que al multiplicar al mensaje encriptado por $A^{-1}$ a
    izquierda se obtiene el mensaje original.
%    \begin{xca*}
%        Encuentre el mensaje encriptado con la matriz en la matriz 
%        \[
%        \begin{pmatrix}
%               -11 & -48 &  22 & 17 & -48 & -65 & -20 &  34\\
%                12 &  41 & -13 & -9 &  41 &  51 &  20 & -21
%        \end{pmatrix}.
%        \]
%    \end{xca*}
\end{block}

%%% TODO
%%% * algún sistemita con números complejos
%%% * toda matriz es producto de elementales
%%% * en la sección XXX veremos otros resultados como aplicación


\chapter{Espacios vectoriales}

Tres clases. Temas: Espacios vectoriales y subespacios. Suma e intersección.
Sistemas de generadores e independencia lineal. Bases y dimensión. Rango.
Coordenadas y cambio de base. Aplicaciones. 

\section{Espacios vectoriales}

\begin{block}
	Un \textbf{espacio vectorial} sobre un cuerpo $\K$ es un conjunto no vacío
	$V$ con dos operaciones, una suma $V\times V\to V$ denotada por
	$(x,y)\mapsto x+y$, y un producto por escalares $\K\times V\to V$, denotado
	por $(\lambda,x)\mapsto \lambda x$, tal que
    \begin{enumerate}
        \item $(x+y)+z=x+(y+z)$ para todo $x,y,z\in V$.        
        \item Existe un único $0_V\in V$ tal que $x+0_V=0_V+x=x$ para todo $x\in V$.
        \item Para cada $x\in V$ existe un único $(-x)\in V$ tal que \[x+(-x)=(-x)+x=0_V.\]
        \item $x+y=y+x$ para todo $x,y\in V$.
        \item $1x=x$ para todo $x\in V$.
        \item $(\lambda\mu)x=\lambda(\mu x)$ para todo $\lambda,\mu\in\K$ y $x\in V$.
        \item $(\lambda+\mu)x=\lambda x+\mu x$ para todo $\lambda,\mu\in\K$ y $x\in V$.
        \item $\lambda(x+y)=\lambda x+\lambda y$ para todo $\lambda\in\K$ y $x,y\in V$.
    \end{enumerate}
	Los elementos de $V$ se denominan \textbf{vectores}. Los elementos de $\K$
	se denominan \textbf{escalares}. El elemento $0_V\in V$ se denomina el
	\textbf{origen} de $V$.  Cuando no haya peligro de confusión, denotaremos
	por $0$ al cero del cuerpo $\K$ y también al origen de $V$.
\end{block}

\begin{block}
	Demostremos algunas propiedades básicas:
	\begin{enumerate}
		\item $0v=0_V$ para todo $v\in V$. En efecto, $0v=(0+0)v=0v+0v$ y luego, al sumar en
			ambos miembros el elemento $-(0v)$, obtenemos $0v=0_V$.
		\item $(-1)v=-v$ para todo $v\in V$. En efecto, como $0v=0_V$, entonces
			\[
			0_V=0v=(1+(-1))v=1v+(-1)v=v+(-1)v
			\]
			y luego $-v=(-1)v$, por la unicidad
			del inverso aditivo.
		\item $\lambda0_V=0_V$ para todo $\lambda\in\K$ pues
			$\lambda0_V=\lambda (0_V+0_V)=\lambda0_V=\lambda0_V$ y luego
			$\lambda0_V=0_V$. 
	\end{enumerate}
\end{block}

\begin{xca}
    Sea $I$ un conjunto no vacío y sea $\{V_i:i\in I\}$ una colección de
    espacios vectoriales sobre $\K$. Se define el \textbf{producto directo} de
    los $V_i$ como
    \[
        \prod_{i\in I}V_i=\left\{f\colon I\to\bigcup_{i\in I}V_i:f(i)\in V_i\text{ para todo $i\in I$}\right\}.
    \]
    Demuestre que con las operaciones 
    \begin{align*}
    &(f+g)(i)=f(i)+g(i),\\
    &(\lambda f)(i)=\lambda f(i),
    \end{align*}
    donde la suma es la suma del espacio vectorial $V_i$ y
    el producto por escalares es el producto por escalares de $V_i$, es un
    espacio vectorial sobre $\K$.  Un poco de notación: si $I=\{1,\dots,n\}$ es un
    conjunto finito escribiremos 
    $\prod_{i\in I}V_i=V_1\times\cdots\times V_n$. 
    Si todos los $V_i$ son iguales a un espacio vectorial $V$ entonces
    $\prod_{i\in I}V_i=V^{I}$. 
    Por último, si $I=\{1,\dots,n\}$ y todos los
    $V_i$ son iguales a $V$ entonces $\prod_{i\in I}V_i=V^n$. 
\end{xca}


\begin{examples}\
	\begin{enumerate}
        \item Un cuerpo $\K$ con las operaciones usuales es un espacio
            vectorial sobre $\K$ donde vectores y escalares coinciden.
        \item El conjunto $\K^n$ de $n$-tuplas de elementos de $\K$ con las
            operaciones
            \begin{align*}
                &(x_1,\dots,x_n)+(y_1,\dots,y_n)=(x_1+y_1,\dots,x_n+y_n),\\
                &\lambda(x_1,\dots,x_n)=(\lambda x_1,\dots,\lambda x_n),
            \end{align*}
            es un espacio vectorial sobre $\K$.
        \item El conjunto $\K^{m\times n}$ de matrices de $m\times n$ con las operaciones usuales
            es un espacio vectorial sobre $\K$.
        \item El conjunto $\K[X]$ de polinomios con coeficientes en $\K$ es un
            espacio vectorial sobre $\K$.
		\item $V=C(\R)=\{f\colon\R\to\R:f\text{ es continua}\}$
        \item $V=C^\infty(\R)=\{f\colon\R\to\R:f\text{ es de clase $C^\infty$}\}$.
        \item Las sucesiones en $\K$ con las operaciones
            \begin{align*}
                &(x_1,x_2,\dots)+(y_1,y_2,\dots)=(x_1+y_1,x_2+y_2,\dots),\\
                &\lambda (x_1,x_2,\dots)=(\lambda x_1,\lambda x_2,\dots),
            \end{align*}
            forman un espacio vectorial sobre $\K$. Este espacio vectorial será
            denotado por $\K^{\infty}$.
	\end{enumerate}
\end{examples}

\begin{xca}
	Si $V$ un espacio vectorial sobre $\K$ y $p\in V$. Pruebe que las operaciones
	$v+_pw=v+w-p$ y $\lambda\cdot_p v=p+\lambda (x-p)$ definen una estructura
	de espacio vectorial sobre $V$ cuyo origen es $p$.
\end{xca}

\begin{xca}
	Sea $V$ un espacio vectorial sobre $\R$. Pruebe que las operaciones
    \begin{align*}
      &(x,y)+(x',y')=(x+x',y+y'),
      &&(a+bi)(x,y)=(ax-by,ay+bx),
    \end{align*}
    definen sobre $V\times V$ una estructura de espacio vectorial sobre $\C$.
    Este espacio vectorial se denomina la \textbf{complexificación} de $V$ y se
    denota por $V_{\C}$.
\end{xca}

\section{Subespacios, suma e intersección}

\begin{block}
    Sea $V$ un espacio vectorial sobre $\K$.  Un subconjunto no vacío
    $S\subseteq V$ es un \textbf{subespacio} de $V$ si $0\in S$ y si $x,y\in S$
    y $\lambda\in\K$ entonces $x+\lambda y\in S$. Observemos que si $S$ es un
    subespacio de $V$ entonces $S$, con las operaciones de $V$ restringidas a
    $S$, es un espacio vectorial.
\end{block}

\begin{xca}
    \label{xca:subespacio}
	Sea $V$ un espacio vectorial sobre $\K$ y sea $S\subseteq V$ un subconjunto
	no vacío. Pruebe que $S$ es subespacio si y sólo si $v+\lambda w\in S$ para
	todo $v,w\in S$ y $\lambda\in\K$.
\end{xca}

\begin{xca}
    Sea $I$ un conjunto no vacío y $\{V_i:i\in I\}$ una colección de espacios
    vectoriales sobre $\K$. Demuestre que el conjunto
    \[
        \coprod_{i\in I}V_i=\left\{f\in\prod_{i\in I}V_i:f(i)=0\text{ salvo para finitos $i\in I$}\right\}
    \]
    es un subespacio de $\prod_{i\in I}V_i$. Este subespacio se conoce como el
    \textbf{coproducto directo} de los $V_i$. Notación: si todos los $V_i$ son
    iguales a un espacio vectorial $V$ escribimos $\coprod_{i\in
    I}V_i=V^{(I)}$. 
\end{xca}

\begin{examples}\
    \begin{enumerate}
        \item Si $V$ es un espacio vectorial entonces $\{0\}$ y $V$ son
            subespacios de $V$.
        \item Si $A=(a_{ij})\in\K^{m\times n}$ entonces
            \[
                \left\{(x_1,\dots,x_n)\in\K^n: \sum_{j=1}^n a_{ij}x_j=0\text{ para todo $i\in\{1,\dots,m\}$}\right\}
            \]
            es un subespacio de $\K^n$.
        \item Para cada $n\in\N$, el conjunto 
            \[
            \K_n[X]=\{f\in\K[X]: \deg f\leq n\text{ o }f=0\}
            \]
            es un subespacio de $\K[X]$.
        \item El conjunto $C^{\infty}(\R)$ es un subespacio de $C(\R)$.
		\item El conjunto $\{f\in C^\infty(\R):f''+f=0\}$ es un subespacio de $C^{\infty}(\R)$.
		\item El conjunto de sucesiones en $\K$ con un número finito de elementos no nulos
			es un subespacio de $\K^{\infty}$.
    \end{enumerate}
\end{examples}

\begin{xca}
    Si $n\in\N$ entonces $\{f\in\R[X]:\deg f\geq n\text{ o }f=0\}$ no es un
    subespacio de $\R[X]$. ¿Por qué?
\end{xca}

\begin{xca} 
    Sea $I$ un conjunto de índices y sea $(S_i)_{i\in I}$ una colección
    arbitraria de subespacios de un espacio vectorial $V$. Demuestre que 
    $\cap_{i\in I}S_i$ es un subespacio de $V$. 
\end{xca}

\begin{block}
    Si $S$ y $T$ son dos subespacios de un espacio vectorial $V$ entonces
    $S\cap T$ es un subespacio de $V$.  Observemos que $S\cap T$ es el mayor
    subespacio de $V$ contenido en $S$ y en $T$. 
\end{block}

\begin{xca}
    Sea $I$ un conjunto no vacío y sea $\{W_i:i\in I\}$ una colección de
    subespacios de un espacio vectorial $V$.  Demuestre que 
    \[
    \sum_{i\in I}W_i=\left\{v\in V:v=\sum_j w_j\text{ (suma finita), donde $w_j\in W_j$ para todo $j$}\right\}
    \]
    es un subespacio de $V$ y que es el menor subespacio de $V$ que contiene a $\cup_{i\in I}W_i$. 
\end{xca}

\begin{block}
	Sean $V$ un espacio vectorial y $S$ y $T$ dos subespacios de $V$.  Entonces 
    \[
        S+T=\{s+t\in V: s\in S,\;t\in T\}
    \]
	es un subespacio de $V$.  Demostremos que $S+T$ es el menor subespacio de
	$V$ que contiene a $S\cup T$. En efecto, si $W$ es un subespacio de $V$ que
	contiene a $S\cup T$ entonces, como $S\subseteq S\cup T\subseteq W$ y
	$T\subseteq S\cup T\subseteq W$, se tiene que $S+T\subseteq W$. 
\end{block}

\begin{remark}
    La unión de subespacios no siempre es subespacio. Por ejemplo: si $V=\R^2$,
    $S=\{(x,y):x=0\}$ y $T=\{(x,y):y=0\}$ entonces $(1,1)\not\in S\cup T$.
\end{remark}

\begin{example}
	\label{exa:R2x2}
	Consideremos el subespacio $U$ de $\R^{2\times 2}$ dado por 
	\[
	U=\left\{\begin{pmatrix}
		a_{11} & a_{12}\\
		0 & a_{22}
	\end{pmatrix}
	:a_{11},a_{12},a_{22}\in\K\right\}
	= \left\{\begin{pmatrix}
			\star & \star\\
			0 & \star
		\end{pmatrix}
		\right\},
	\]
	y los siguientes subespacios
	\begin{align*}
		U' = \left\{\begin{pmatrix}
			0 & \star\\
			0 & 0
		\end{pmatrix}
		\right\},
		&&
		L = \left\{\begin{pmatrix}
			\star & 0\\
			\star & \star
		\end{pmatrix}
		\right\},
		&&
		L' = \left\{\begin{pmatrix}
			0 & 0\\
			\star & 0
		\end{pmatrix}
		\right\},
		&&
		H = \left\{\begin{pmatrix}
			\star & 0\\
			0 & \star
		\end{pmatrix}
		\right\},
	\end{align*}
	Entonces, por ejemplo, se demuestra fácilmente que:
	\begin{enumerate}
		\item $V=U+L$.
		\item $V=U+L'$.
	\end{enumerate}
\end{example}

\begin{block}
    Sean $V$ un espacio vectorial y $S$ y $T$ dos subespacios. Se dice que $V$
    es \textbf{suma directa} de $S$ y $T$ si $V=S+T$ y $S\cap T=\{0\}$.  Si $V$
    es suma directa de $S$ y $T$ la notación es: $V=S\oplus T$. 
\end{block}

\begin{example}
    En el ejemplo~\ref{exa:R2x2} vimos que $V=U+L$. Es fácil ver que $V$ no es
    suma directa de $U$ y $L$ pues $U\cap L\ne\{0\}$. En cambio, $V=U\oplus L'$
    pues $V=U+L'$ y además $U\cap L'=\{0\}$. 
\end{example}

\begin{example}
    Sea $V$ el espacio vectorial de las funciones $f\colon\R\to\R$.
    Consideremos los subespacios $V_P=\{f\in V:f(x)=f(-x)\text{ para todo
    $x$}\}$ y $V_I=\{f\in V: f(x)=-f(-x)\text{ para todo $x$}\}$. Como toda $f$
    puede escribirse como $f=f_P+f_I$, donde
	\[
	f_P(x)=\frac{f(x)+f(-x)}{2}\in V_P,\quad
	f_I(x)=\frac{f(x)-f(-x)}{2}\in V_I,
	\]
    entonces $V=V_P+V_I$. Más aún, si $f\in V_P\cap V_I$ entonces $f(x)=0$ para
    todo $x$. Luego $V_P+F_I=\{0\}$ y entonces $V=V_P\oplus V_I$.
\end{example}

\begin{prop}
    Son equivalentes:
    \begin{enumerate}
        \item $V=S\oplus T$.
		\item $V=S+T$ y si $s+t=0$ con $s\in S$ y $t\in T$ entonces $s=0$ y
			$t=0$.
		\item Para cada $v\in V$ existen únicos $s\in S$ y $t\in T$ tales que
			$v=s+t$.
    \end{enumerate}

    \begin{proof}
		Demostremos que $(1)$ implica $(2)$. Como $V=S\oplus T$, entonces, por definición, 
	    $V=S+T$. Si $s+t=0$ con $s\in S$ y $t\in T$ entonces
		$s=-t\in S\cap T=\{0\}$ y por lo tanto $s=0$ y $t=0$. Demostremos ahora
		que $(2)$ implica $(3)$. Como la existencia es trivial, demostraremos
		la unicidad: si $s+t=s'+t'$ con $s,s'\in S$ y $t,t'\in T$ entonces
		$(s-s')+(t-t')=0$ con $s-s'\in S$ y $t-t'\in T$. Luego $s-s'=0$ y
		$t-t'=0$ como se quería demostrar. Por último, demostremos
		$(3)\Rightarrow(1)$. Por hipótesis, $V=S+T$. Si $v\in S\cap T$ entonces
		$v\in S$.  Además $-v\in S\cap T$ y entonces $-v\in T$. Como $0=v+(-v)$
		con $v\in S$ y $-v\in T$, entonces $v=0$.
    \end{proof}
\end{prop}

\begin{block}
    Si $V$ es un espacio vectorial y $S_1,\dots,S_n$ son subespacios de $V$ se
    dice que $V$ es \textbf{suma directa} de los $S_i$ si $V=S_1+\cdots+S_n$ y
    la igualdad $s_1+\cdots+s_n=0$, donde $s_i\in S_i$ para cada
    $i\in\{1,\dots,n\}$, implica que $s_i=0$ para todo $i\in\{1,\dots,n\}$.
\end{block}

\begin{xca}
    Sea $V$ un espacio vectorial y sean $S_1,\dots,S_n$ subespacios de $V$.
    Demuestre que las siguientes afirmaciones son equivalentes:
    \begin{enumerate}
        \item $S_1+\cdots+S_n=S_1\oplus\cdots\oplus S_n$.
        \item Todo $v\in S_1+\cdots+S_n$ puede escribirse en forma única como
            $v=s_1+\cdots+s_n$ con $s_i\in S_i$ para todo $i\in\{1,\dots,n\}$. 
        \item Para cada $i\in\{1,\dots,n\}$, $S_i\cap \sum_{j\ne i}S_j=\{0\}$.
    \end{enumerate}
\end{xca}

\begin{xca}
	Demuestre que $\R^{2\times 2}=U'\oplus H\oplus L'$. 
\end{xca}

\section{Sistemas de generadores y dependencia lineal}

\begin{block}
    Sean $V$ un espacio vectorial y $v_1,\dots,v_n\in V$. Diremos que un vector
    $v\in V$ es \textbf{combinación lineal} de $v_1,\dots,v_n$ si existen
    $\alpha_1,\dots,\alpha_n\in\K$ tales que $v = \alpha_1v_1+\cdots+\alpha_n
    v_n$.  El conjunto de combinaciones lineales de $v_1,\dots,v_n$ será
    denotado por $\langle v_1,\dots,v_n\rangle$.
\end{block}

\begin{example}
    Consideremos los vectores $v_1=(2,3,4)$ y $v_2=(0,1,1)$ de $\R^3$.  Como
    $v=2v_1-v_2$, el vector $v=(4,5,7)$ es combinación lineal de $v_1$ y $v_2$.
\end{example}

\begin{example}\
	$\R[X]=\{f\in\R[X]:f(0)=0\}\oplus\langle 1\rangle$.
\end{example}

\begin{block}
    Sean $V$ un espacio vectorial y $v_1,\dots,v_n\in V$. Entonces $\langle
    v_1,\dots,v_n\rangle$ es el menor subespacio de $V$ que contiene a los
    vectores $v_1,\dots,v_n$ y se denomina el \textbf{subespacio generado} por
	$v_1,\dots,v_n$. Como consecuencia se obtienen fácilmente las siguientes
	propiedades:
	\begin{enumerate}
		\item $\langle v_1,\dots,v_n\rangle=\langle v_1\rangle+\cdots+\langle
			v_n\rangle$.
        \item Si $S=\langle v_1,\dots,v_n\rangle$ y $T=\langle
            w_1,\dots,w_m\rangle$ son subespacios de $V$ entonces
            $S+T=\langle v_1,\dots,v_n,w_1,\dots,w_m\rangle$.
	\end{enumerate}
\end{block}

\begin{xca}
    Demuestre las siguientes propiedades:
    \begin{enumerate}
        \item $\langle v_1,\dots,v_i,\dots,v_j,\dots,v_n\rangle= 
            \langle v_1,\dots,v_j,\dots,v_i,\dots,v_n\rangle$.
        \item $\langle v_1,\dots,v_i,\dots,v_n\rangle=\langle v_1,\dots,\lambda
            v_i,\dots,v_n\rangle$ para todo $\lambda\in\K\setminus\{0\}$.
        \item $\langle v_1,\dots,v_i,\dots,v_j,\dots,v_n\rangle=
            \langle v_1,\dots,v_i+\lambda v_j,\dots,v_j,\dots,v_n\rangle$
            para todo $\lambda\in\K$.
    \end{enumerate}
\end{xca}

\begin{block}
	Sea $V$ un espacio vectorial y $X\subseteq V$ un subconjunto no vacío.
	Diremos que $X$ es un \textbf{conjunto de generadores} para $V$ (o que $X$
	genera a $V$) si para cada $v\in V$ existen $x_1,\dots,x_n\in X$ tales que
	$v\in\langle x_1,\dots,x_n\rangle$.
\end{block}


\begin{examples}\
	\label{exa:generadores}
	\begin{enumerate}
		\item El conjunto $\{(2,3),(1,2)\}$ es un conjunto de generadores de
			$\R^2$ pues 
			\[
				(x,y)=(2x-y)(2,3)+(-3x+2y)(1,2).
			\]
		\item $\{e_1,\dots,e_n\}$ es un conjunto de generadores de $\K^n$.
		\item $\{E^{ij}:1\leq i\leq m,\;1\leq j\leq n\}$ es un conjunto de
			generadores de $\K^{m\times n}$.
		\item $\{1,X,X^2,\dots,\}$ es un conjunto de generadores de $\K[X]$.
		\item Sean $V=\R^{2\times 2}$ y $S=\{A\in V:A^T=A,\;\tr(A)=0\}$.
			Entonces 
			\[
			\left\{\begin{pmatrix}
				1 & 0\\
				0 & -1
			\end{pmatrix},
			\begin{pmatrix}
				0 & 1\\
				1 & 0
			\end{pmatrix}\right\}
			\]
			es un conjunto de generadores de $S$. ¿Por qué?
	\end{enumerate}
\end{examples}

\begin{xca}
	\label{xca:f(1)=0}
	Sean $V=\R[X]$ y $S=\{f\in\R[X]:f(1)=0\}$. Demuestre que 
	\[
	\{X^{m+1}-X^m:m\in\N\}
	\]
	es un conjunto de generadores de $S$.
\end{xca}

\begin{xca}
	\label{xca:X^2-3X+2|f}
	Sea $S=\{f\in\R[X]:f(1)=f(2)=0\}$. Pruebe que 
	\[
		\{(X^2-3X+2)X^m:m\geq 0\}
	\]
	es un conjunto de generadores para $S$.
\end{xca}

\begin{xca}
	Sea $V$ un espacio vectorial y $X\subseteq V$ un subconjunto no vacío.
	Pruebe que
	\[
	\bigcap_{S:X\subseteq S}S,
	\]
	donde la intersección se toma sobre todo los subespacios de $V$ que
	contienen a $X$ es el menor subespacio de $V$ que contiene a $X$. Concluya
	que este subespacio de $V$ es igual al subespacio generado por $X$.
\end{xca}

\begin{example}
	$\R[X]=\R_n[X]\oplus \langle X^{n+1},X^{n+2},\dots\rangle$.
\end{example}

\begin{block}
	Un espacio vectorial $V$ es \textbf{finitamente generado} si admite un
	conjunto finito de generadores. 
\end{block}

\begin{examples}
	Vimos en el ejemplo~\ref{exa:generadores} que los espacios vectoriales
	$\K^n$ y $\K^{m\times n}$ son finitamente generados. 
\end{examples}

\begin{block}
    El conjunto $\R$ de los números reales, visto como espacio vectorial sobre
    $\R$, es finitamente generado. Con un argumento de
    cardinalidad puede demostrarse que $\R$, como espacio vectorial sobre $\Q$, no es
    finitamente generado.
\end{block}

\begin{xca}
	\label{xca:K[X]_no_fg}
	Demuestre que $\K[X]$ no es finitamente generado.	
\end{xca}

\begin{block}
    Sea $V$ un espacio vectorial sobre un cuerpo $\K$.  Un subconjunto finito y
    no vacío $\{v_1,\dots,v_n\}\subseteq V$ se dice \textbf{linealmente
    dependiente} si existen $\alpha_1,\dots,\alpha_n\in\K$, no todos cero,
    tales que $\alpha_1v_1+\cdots+\alpha_nv_n=0$. Observemos que
    $\{v_1,\dots,v_n\}$ es linealmente dependiente si y sólo si existe
    $j\in\{1,\dots,n\}$ tal que $v_j\in\langle
    v_1,\dots,\widehat{v_j},\dots,v_n\rangle$, donde 
	\[
	\langle v_1,\dots,\widehat{v_j},\dots,v_n\rangle=
	\langle v_1,\dots,v_{j-1},v_{j+1},\dots,v_n\rangle
	\]
\end{block}

\begin{examples}\
	\begin{enumerate}
		\item El conjunto $\{(2,3),(1,2),(3,5)\}\subseteq\R^2$ es linealmente
			dependiente. ¿Por qué?
		\item El conjunto $\{(1,1,1),(0,1,1),(0,0,1),(1,2,3)\}\subseteq\R^3$
			es linealmente dependiente. ¿Por qué?
	\end{enumerate}
\end{examples}

\begin{xca}
	Demuestre las siguientes propiedades:
	\begin{enumerate}
		\item Sea $\lambda\in\K\setminus\{0\}$. Entonces
			$\{v_1,\dots,v_i,\dots,v_n\}$ es linealmente dependiente si y sólo
			si $\{v_1,\dots,\lambda v_i,\dots,v_n\}$ es linealmente
			dependiente. 
		\item Sea $\lambda\in\K$. Entonces
			$\{v_1,\dots,v_i,\dots,v_j,\dots,v_n\}$ es linealmente dependiente
			si y sólo si $\{v_1,\dots,v_i+\lambda v_j,\dots,v_j,\dots,v_n\}$ es
			linealmente dependiente.
	\end{enumerate}
\end{xca}

\begin{block}
    Sea $V$ un espacio vectorial sobre $\K$.  Un subconjunto finito se dice
    \textbf{linealmente independiente} si no es linealmente dependiente.
    Por definición, el conjunto vacío $\emptyset$ es linealmente independiente. Un subconjunto
    finito y no vacío $\{v_1,\dots,v_n\}\subseteq V$ es linealmente
    independiente si y sólo si $\alpha_1v_1+\cdots+\alpha_nv_n=0$ implica que
    $\alpha_1=\cdots=\alpha_n=0$. 
\end{block}

\begin{example}
	El conjunto $\{(1,2,1),(2,1,2),(1,0,3)\}\subseteq\R^3$ es linealmente
	independiente.  ¿Por qué?
\end{example}

\begin{block}
    Un conjunto infinito de vectores es linealmente dependiente si tiene un
    subconjunto finito y no vacío linealmente dependiente. Un conjunto infinito
    de vectores es linealmente independiente si todo subconjunto finito es
    linealmente independiente. 
\end{block}

\begin{example}
    Sea $V$ el espacio vectorial de funciones $\R\to\R$. Demostremos que el
    conjunto $\{f_a\colon\R\to\R:a\in\R\}$, donde 
    \[
        f_a\colon\R\to\R,\quad
        f_a(x)=|x-a|,
    \]
    es linealmente independiente.  En caso contrario, existirían
    $\lambda_2,\dots,\lambda_n\in\R$ y $a_1,\dots,a_n\in\R$, con $a_i\ne a_j$
    para $i\ne j$, tales que
    \[
        f_{a_1}=\lambda_2f_{a_2}+\cdots+\lambda_nf_{a_{n}}.
    \]
    Observemos que la función
    $f_{a_1}$ no es derivable en $x=a_1$, y, en cambio, la función
    $\lambda_2f_{a_2}+\cdots+\lambda_nf_{a_{n}}$ sí lo es. Como esto es una
    contradicción, las $f_a$ son linealmente independientes.
\end{example}

\begin{xca}
    Sea $V$ el espacio vectorial de funciones $\R\to\R$. Demuestre las
    siguientes afirmaciones:
    \begin{enumerate}
        \item El conjunto $\{e^{ax}:a\in\R\}$ es linealmente independiente.
        \item El conjunto $\{2^{x-1}x^{i-1}:i\in\N\}$ es linealmente independiente.
    \end{enumerate}
\end{xca}

\begin{example}
    Consideremos a $\R$ como espacio vectorial sobre $\Q$ y demostremos que el
    conjunto $\{\log p:p\text{ primo positivo}\}$ es linealmente independiente.
    Si no lo fuera, existirían $n\in\N$, $\alpha_1,\dots,\alpha_n\in\Q$, no
    todos cero, y primos positivos $p_{1},\dots,p_{n}$ tales que $\alpha_1\log
    p_{1}+\cdots+\alpha_n\log p_{n}=0$. Sin pérdida de generalidad podemos
    suponer que $\alpha_i\in\Z$ para todo $i$. Luego
    \[
    1=e^0=e^{\sum_{j=1}^n \alpha_j\log p_{j}}=\prod_{j=1}^ne^{\alpha_j\log p_{j}}=\prod_{j=1}^n p_j^{\alpha_j}, 
    \]
    una contradicción. 
\end{example}


\begin{xca}
    Pruebe que si $\{v_1,\dots,v_n\}\subseteq V$ es linealmente independiente
    entonces todo subconjunto de $\{v_1,\dots,v_n\}$ es también linealmente
    independiente.
\end{xca}

\begin{xca}
	Pruebe que si $\{v_1,\dots,v_n\}$ es linealmente independiente entonces 
	$\langle v_1,\dots,v_n\rangle=\langle v_1\rangle\oplus\cdots\oplus\langle
	v_n\rangle$.
\end{xca}

\section{Bases y dimensión}

\begin{block}
    Una \textbf{base} de $V$ es un conjunto de generadores de $V$ linealmente
    independiente.  Si $V$ es finitamente generado, una \textbf{base ordenada}
    de $V$ es una sucesión de vectores $v_1,\dots,v_n$ de $V$ tal que
    $\{v_1,\dots,v_n\}$ es linealmente independiente y genera $V$.
\end{block}

\begin{examples}\
	\begin{enumerate}
        \item $\{(2,3),(1,2)\}$ y $\{(1,2),(2,3)\}$ son dos bases ordenadas
            distintas de $\R^2$. 
        \item $\{e_1,\dots,e_n\}$ es base de $\K^n$ y se llama \text{base
            canónica} de $\K^n$.  
		\item $\{E^{ij}:1\leq i\leq m,\;1\leq j\leq n\}$, donde 
			\[
				(E^{ij})_{kl}=\delta_{ik}\delta_{jl}
			\]
			para todo $i,j,k,l$, es base de $\K^{m\times n}$ y se llama
			\textbf{base canónica} de $\K^{m\times n}$.  
		\item $\{1,X,X^2,\dots\}$ es base de $\K[X]$.
	\end{enumerate}
\end{examples}

\begin{prop}
    Sea $V$ un espacio vectorial. El conjunto $\{v_1,\dots,v_n\}$ es una base
    de $V$ si y sólo si para cada $v\in V$ existen únicos
    $\alpha_1,\dots,\alpha_n\in\K$ tales que $v=\alpha_1v_1+\cdots+\alpha_n
    v_n$.

	\begin{proof}
		Supongamos que $\{v_1,\dots,v_n\}$ es una base. Entonces para cada
		$v\in V$ existen $\alpha_1,\dots,\alpha_n\in\K$ tales que
		$v=\alpha_1v_1+\cdots+\alpha_n v_n$. Veamos que los $\alpha_i$ son
		únicos: si suponemos que \[
            v=\alpha_1v_1+\cdots+\alpha_nv_n=\beta_1v_1+\cdots+\beta_n v_n,
        \]
		donde los $\alpha_i,\beta_i\in\K$, entonces
		\[
			(\alpha_1-\beta_1)v_1+\cdots+(\alpha_n-\beta_n)v_n=0.
		\]
		Luego $\alpha_i=\beta_i$ para todo $i\in\{1,\dots,n\}$ por la
		independencia lineal del conjunto $\{v_1,\dots,v_n\}$.

		Recíprocamente, si todo $v\in V$ se escribe como combinación lineal de
		los $v_i$ entonces $\{v_1,\dots,v_n\}$ es un conjunto de generadores de
		$V$. Veamos que es linealmente independiente: si
		$\alpha_1v_1+\cdots+\alpha_nv_n=0$ entonces, como también
		$0=0v_1+\cdots+0v_n$, la unicidad de la escritura implica que
		$\alpha_i=0$ para todo $i\in\{1,\dots,n\}$.
	\end{proof}
\end{prop}

\begin{prop}
	\label{pro:extraer_una_base}
    Sea $V$ un espacio vectorial finitamente generado y sea $\{v_1,\dots,v_n\}$
    un conjunto de generadores de $V$.  Entonces existe un subconjunto de
    $\{v_{1},\dots,v_{n}\}$ que es base de $V$.

	\begin{proof}
		Sean $X_0=\{v_1,\dots,v_n\}$ y 
		\[
			X_1=\begin{cases}
				X_0\setminus\{v_1\} & \text{si $v_1=0$},\\
				X_0 & \text{si $v_1\ne0$}.
			\end{cases}
		\]
		Para cada $j\geq2$ definimos inductivamente
		\[
			X_j=\begin{cases}
				X_{j-1}\setminus\{v_j\} & \text{si $v_j\in\langle X_{j-1}\rangle$},\\
				X_{j-1} & \text{si $v_j\not\in\langle X_{j-1}\rangle$},\\
			\end{cases}
		\]
		Como $V$ está generado por $X_0$, el proceso termina con $X_n$ después
		de $n$ pasos. Por construcción, $V$ está generado por $X_n$, pues 
		$V$ está generado por $X_0$ y si $V$ está generado por $X_j$, también $X_{j+1}$ es un conjunto de generadores
		de $V$, ya que si $v_{j+1}\not\in \langle X_j\rangle$, entonces $X_j=X_{j+1}$ y si $v_{j+1}\in \langle X_{j}$, 
		entonces $\langle X_j\rangle=\langle X_{j+1}\rangle$. Además 
		el conjunto $X_n$ es linealmente
		independiente..\framebox{FIXME}
	\end{proof}
\end{prop}

\begin{cor}
	Todo espacio vectorial no nulo finitamente generado admite una base. 		

	\begin{proof}
		El corolario es consecuencia de la
		proposición~\ref{pro:extraer_una_base}, que afirma que de todo conjunto
		de generadores puede extraerse una base.
	\end{proof}
\end{cor}

\begin{thm}
	\label{thm:>n_es_LD}
	Sea $V$ un espacio vectorial finitamente generado por los vectores $v_1,\dots,v_n$.
	Entonces todo conjunto con más de $n$ elementos es linealmente dependiente. 

	\begin{proof}
		Sea $\{w_1,\dots,w_m\}\subseteq V$ con $m>n$. Como por hipótesis $V=\langle
		v_1,\dots,v_n\rangle$, para cada $j\in\{1,\dots,m\}$ existen escalares
		$a_{ij}$ tales que $w_j=\sum_{i=1}^n a_{ij}v_i$. Como $m>n$, el sistema lineal 
		\[
		\sum_{j=1}^m a_{ij}v_j=0,\quad i\in\{1,\dots,n\},
		\]
		admite una solución no trivial, digamos $(\lambda_1,\dots,\lambda_m)$. Entonces
		\[
		\sum_{j=1}^m \lambda_jw_j=\sum_{j=1}^m\lambda_{j}\left( \sum_{i=1}^na_{ij}v_i\right)=\sum_{i=1}^n\left(\sum_{j=1}^ma_{ij}\lambda_j\right)v_i=0,
		\]
		y luego $\{w_1,\dots,w_m\}$ es un conjunto linealmente dependiente. 
	\end{proof}
\end{thm}

%\begin{block}
%    La demostración que vimos del teorema~\ref{thm:>n_es_LD} utiliza que todo
%    sistema lineal homogéneo con más incógnitas que ecuaciones siempre tiene
%    una solución no trivial. A continuación, daremos una demostración
%    alternativa que no utiliza ese resultado. 
%
%    \begin{proof}[Otra demostración del teorema~\ref{thm:>n_es_LD}]\framebox{FIXME}
%        Supongamos que $m>n$.  Como los $w_i$ son linealmente independientes,
%        $w_1\ne0$.  Sin pérdida de generalidad podemos suponer que
%        $w_1=\sum_{i=1}^n \alpha_iv_i$, donde $\alpha_1\ne0$, y entonces
%        \[
%            v_1=\frac{1}{\alpha_1}(w_1-\alpha_2v_2-\cdots-\alpha_nv_n).
%        \]
%        Luego $\{w_1,\dots,w_m\}\subseteq \langle w_1,v_2,\dots,v_n\rangle$ y
%        $\langle w_1,v_2,\dots,v_n\rangle=V$. Supongamos ahora que $V=\langle w_1,\dots,w_k,v_{k+1},\dots,v_n\rangle$. Entonces
%        existen $\beta_1,\dots,\beta_n\in\K$, no todos cero, tales que 
%        \[
%            w_{k+1}=\sum_{i=1}^k\beta_iw_i+\sum_{i=k+1}^n\beta_iv_i.
%        \]
%        Como $\{w_1,\dots,w_m\}$ es linealmente independiente, existe
%        $j\in\{k+1,\dots,n\}$ tal que $\beta_j\ne0$. Sin pérdida de
%        generalidad, supongamos que $\beta_{k+1}\ne0$. Entonces
%        \[
%        v_{k+1}=\frac{1}{\beta_{k+1}}\left(w_{k+1}-\sum_{i=1}^k\beta_iw_i-\sum_{i=k+2}^n\beta_iv_i\right),
%        \]
%        y luego $V=\langle w_1,\dots,w_{k+1},v_{k+2},\dots,v_n\rangle$. Este
%        procedimiento, después de $n$ pasos, implica que $V=\langle
%        w_1,\dots,w_n\rangle$. Luego el conjunto $\{w_1,\dots,w_{n+1}\}$ es
%        linealmente dependiente. 
%    \end{proof}
%\end{block}

%\begin{example}
%	Vamos a demostrar que $\gamma=\sqrt{2}+\sqrt{3}$ es un número algebraico,
%	lo que significa que $\gamma$ es raíz de un polinomio con coeficientes en
%	$\Q$.  Consideremos a $\R$ como espacio vectorial sobre $\Q$. El conjunto
%	$X=\left\{1,\sqrt{2},\sqrt{3},\sqrt{6}\right\}$ es linealmente
%	independiente y entonces, como el conjunto
%	$\{1,\alpha,\alpha^2,\alpha^3,\alpha^4\}\subseteq X$ tiene cinco elementos,
%	tiene que ser linealmente dependiente. Entonces existen
%	$\alpha_1,\dots,\alpha_5\in\Q$ no todos cero tales que
%	$\sum_{i=1}^5\alpha_i\gamma^i=0$, tal como queríamos demostrar.
%\end{example}

\begin{cor}
	Sea $V$ un espacio vectorial finitamente generado. Entonces dos bases
	cualesquiera tienen la misma cantidad de elementos. 

	\begin{proof}
		Es consecuencia inmediata del teorema~\ref{thm:>n_es_LD}.
	\end{proof}
\end{cor}

\begin{block}
	La \textbf{dimensión} de un espacio vectorial finitamente generado $V$ es
	el cardinal de una base cualquiera de $V$ y se denotará con $\dim V$. Por
	convención: $\dim\{0\}=0$. 
\end{block}

\begin{examples}\
	\begin{enumerate}
		\item $\dim\K^n=n$.
		\item $\dim\K^{m\times n}=mn$.
		\item $\dim\K_n[X]=n+1$.
	\end{enumerate}
\end{examples}

\begin{block}
	\label{block:agregar_v}
	Sean $V$ un espacio vectorial y $v,v_1,\dots,v_n\in V$. Si
	$\{v_1,\dots,v_n\}$ es linealmente independiente y $v\not\in\langle
	v_1,\dots,v_n\rangle$ entonces $\{v,v_1,\dots,v_n\}$ es linealmente
	independiente.  En efecto, si $\alpha v+\sum \alpha_iv_i=0$ entonces, como
	$v\not\in\langle v_1,\dots,v_n\rangle$, se tiene que $\alpha=0$. Luego
	$\alpha_i=0$ para todo $i\in\{1,\dots,n\}$ pues $\{v_1,\dots,v_n\}$ es
	linealmente independiente.
\end{block}

\begin{cor}
	Sea $V$ un espacio vectorial sobre $\K$ de dimensión $n$ y sean
	$v_1,\dots,v_n\in V$. Son equivalentes:
	\begin{enumerate}
		\item $\{v_1,\dots,v_n\}$ es base de $V$.
		\item $\{v_1,\dots,v_n\}$ es linealmente independiente.
		\item $\{v_1,\dots,v_n\}$ genera a $V$.
	\end{enumerate}

	\begin{proof}
		La implicación $(1)\Rightarrow(2)$ es trivial. Para demostrar que $(2)$ implica 
		$(3)$ observemos que si existiera $v\in V\setminus\langle
		v_1,\dots,v_n\rangle$ entonces $\{v,v_1,\dots,v_n\}$ sería linealmente
		independiente por~\ref{block:agregar_v}, y esto contradice el
		teorema~\ref{thm:>n_es_LD}. Finalmente, para demostrar que $(3)$ implica $(1)$ basta
		ver que $\{v_1,\dots,v_n\}$ es linealmente independiente. Para eso, extraemos un
		subconjunto $\{v_{i_1},\dots,v_{i_k}\} \subseteq \{v_1,\dots,v_n\}$
		linealmente independiente tal que $V=\langle
		v_{i_1},\dots,v_{i_k}\rangle$. Luego $k=n$ y entonces
		$\{v_1,\dots,v_n\}$ es base de $V$.
	\end{proof}
\end{cor}

\begin{prop}
	\label{pro:extender_a_una_base}
	Sea $V$ un espacio vectorial sobre $\K$ de dimensión $n$ y para $m<n$ sea
	$\{v_1,\dots,v_m\}\subseteq V$ un conjunto linealmente independiente.
	Entonces existen $v_{m+1},\dots,v_n\in V$ tales que
	$\{v_1,\dots,v_m,v_{m+1},\dots,v_n\}$ es una base de $V$.

	\begin{proof}
		Sea $\{w_1,\dots,w_n\}$ un conjunto de generadores de $V$.
		Procederemos inductivamente. Para eso, sea $X_0=\{v_1,\dots,v_m\}$. 
		En el paso $j$-ésimo, donde $j\geq1$, se define 
		\[
			X_j=\begin{cases}
				X_{j-1} & \text{si $w_j\in\langle X_{j-1}\rangle$},\\
				X_{j-1}\cup\{w_j\} & \text{si $w_j\not\in\langle X_{j-1}\rangle$}.
			\end{cases}
		\]

		El conjunto $X_j$ es linealmente independiente para todo $j$.  Además,
		por construcción, todos los $w_j$ están en el subespacio de $V$
		generado por $X_n$.  Hemos probado entonces que
		$X_n=\{v_1,\dots,v_n,w_{i_1},\dots,w_{i_k}\}$ es una base de $V$.
	\end{proof}
\end{prop}

\begin{prop}
	Sea $V$ un espacio vectorial de dimensión finita y sea $S\subseteq V$ un
	subespacio. Entonces $S$ es de dimensión finita y $\dim S\leq\dim V$.

	\begin{proof}
		Sea $n=\dim V$. Si $S=\{0\}$ entonces $S$ es de dimensión
		finita. Supongamos entonces que $S\ne\{0\}$ y sea $s_1\in
		S\setminus\{0\}$. Si $S=\langle s_1\rangle$ entonces $S$ es de
		dimensión finita. En caso contrario, existe $s_2\in S\setminus\{s_1\}$.
		Si procedemos inductivamente, en el paso $j$-ésimo, donde $j\geq1$, se
		tiene un conjunto $\{s_1,\dots,s_{j-1}\}$. Si $S=\langle
		s_1,\dots,s_{j-1}\rangle$ entonces $S$ es de dimensión finita. En caso
		contrario, existe $s_j\in S\setminus\langle s_1,\dots,s_{j-1}\rangle$.
		Como, por construcción, el conjunto de los $s_j$ es linealmente
		independiente, el teorema~\ref{thm:>n_es_LD} implica que este proceso
		tiene que terminar en a lo sumo $n$ pasos. 
	\end{proof}
\end{prop}

\begin{cor}
    Sean $V$ un espacio vectorial de dimensión finita y $S\subseteq V$ un
    subespacio. Entonces existe un \textbf{complemento} de $S$, es decir:
    existe un subespacio $T\subseteq V$ tal que $V=S\oplus T$. 

	\begin{proof}
		Como $V$ es de dimensión finita, $S$ es de dimensión finita. Sea
		$\{s_1,\dots,s_n\}$ una base de $S$.  Por la
		proposición~\ref{pro:extender_a_una_base}, podemos extender la base de
		$S$ a una base de $V$, es decir: existen $t_1,\dots,t_m\in V$ tales que
		$\{s_1,\dots,s_n,t_1,\dots,t_m\}$ es base de $V$.  Sea $T=\langle
		t_1,\dots,t_m\rangle$. Es evidente que $V=S+T$. Demostremos entonces que $S\cap
		T=\{0\}$. Si $v\in S\cap T$ entonces $v=\sum_{i=1}^n
		\alpha_is_i=\sum_{j=1}^m \beta_jt_j$. Como 
		\[
			\alpha_1s_1+\cdots+\alpha_ns_n+(-\beta_1)t_1+\cdots+(-\beta_m)t_m=0
		\]
		y $\{s_1,\dots,s_n,t_1,\dots,t_m\}$ es linealmente independiente, los
		$\alpha_i$ y los $\beta_j$ son cero y por lo tanto $v=0$. 
	\end{proof}
\end{cor}


\begin{xca}
	Sea $V$ un espacio vectorial de dimensión $n$ y sean $S$ y $T$ dos
	subespacios. Pruebe las siguientes afirmaciones:
	\begin{enumerate}
		\item Si $\dim S=n$ entonces $S=V$.
		\item Si $S\subseteq T$ entonces $\dim S\leq \dim T$.
		\item Si $S\subseteq T$ y $\dim S=\dim T$ entonces $S=T$.
	\end{enumerate}
\end{xca}

\begin{thm}[teorema de la dimensión]
	\label{thm:de_la_dimension}
	Sea $V$ un espacio vectorial de dimensión finita y sean $S$ y $T$ dos
	subespacios de $V$. Entonces
	\[
		\dim(S+T)=\dim S+\dim T-\dim(S\cap T).
	\]

	\begin{proof}
		Supongamos que $\dim S=s$ y que $\dim T=t$ y sea entonces $\{u_1,\dots,u_m\}$ una
		base de $S\cap T$. Completamos a una base
		\[
            \{u_1,\dots,u_m,v_{m+1},\dots,v_s\}
        \]
        de $S$ y a una base
		\[
            \{u_1,\dots,u_m,w_{m+1},\dots,w_t\}
        \]
        de $T$. Entonces $S+T$ está
		generado por \[
            \{u_1,\dots,u_m,w_{m+1},\dots,w_t,v_{m+1},\dots,v_s\}.
        \]
		Para demostrar el teorema basta probar que este conjunto es linealmente
		independiente. Si 
		\[
			0=\sum_{i=1}^m \alpha_i u_i + \sum_{j=m+1}^s \beta_jv_j + \sum_{k=m+1}^t\gamma_kw_k
		\]
		entonces, como $\sum_{k=m+1}^t\gamma_jw_j=-\sum_{i=1}^m \alpha_i u_i -
		\sum_{j=m+1}^s \beta_jv_j\in S\cap T$, existen $\lambda_1,\dots,\lambda_m\in\K$ tales que 
		\[
		\sum_{k=1}^m \gamma_k w_k=\sum_{l=1}^m \lambda_l u_l.
		\]
		Luego, como $\{u_1,\dots,u_m,w_{m+1},\dots,w_t\}$ es
		una base de $T$, concluimos que $\gamma_k=0$ para todo
		$k\in\{m+1,\dots,t\}$ y entonces $\alpha_i=0$  para todo
		$i\in\{1,\dots,m\}$ y $\beta_j=0$ para todo $j\in\{m+1,\dots,s\}$. Luego
        \begin{align*}
			\dim(S+T)&=m+(t-m)+(s-m)=s+t-m\\
            &=\dim S+\dim T-\dim(S\cap T),
        \end{align*}
		que es lo que se quería demostrar.
	\end{proof}
\end{thm}


\section{Rango de matrices}

\begin{block}
	Sea $A\in\K^{m\times n}$. El \textbf{espacio fila} de $A$ es el subespacio
	de $\K^n$ generado por las filas de $A$,
	\[
		E_F(A)=\langle (a_{i1},a_{i2},\dots,a_{in}):i\in\{1,\dots,m\}\rangle,
	\]
	y se denota por $E_F(A)$. El \textbf{espacio columna} de $A$ es el subespacio
	de $\K^n$ generado por las columnas de $A$ 
	\[
		E_C(A)=\langle (a_{1j},a_{2j},\dots,a_{mj}):j\in\{1,\dots,n\}\rangle.
	\]
	y se denota por $E_C(A)$. 
\end{block}

\begin{example}
	Sea $A=\begin{pmatrix}
		1 & 2 & 1\\
		0 & 1 & 5
	\end{pmatrix}$. Entonces $E_F(A)=\langle (1,2,1),(0,1,5)\rangle$ y
	$E_C(A)=\langle (1,0),(2,1),(1,5)\rangle$. Observemos que $\dim E_F(A)=\dim E_C(A)=2$.
\end{example}

\begin{block}
	Sea $A\in\K^{m\times n}$. La dimensión de $E_F(A)$ se denomina el
	\textbf{rango fila} de $A$ y se denota por $\rg_F(A)$.  La dimensión de
	$E_C(A)$ se denomina el \textbf{rango columna} de $A$ y se denota por
	$\rg_C(A)$. 
\end{block}

\begin{remark}
	\label{rem:producto}
    Sean $A\in\K^{m\times n}$, $B\in\K^{m\times r}$ y $C=\K^{r\times n}$ tales
    que $A=BC$. Entonces cada columna de $A$ puede escribirse como combinación
    lineal de las columnas de $B$, es decir:
	\[
		\colvec{4}{a_{1j}}{a_{2j}}{\vdots}{a_{mj}}=c_{1j}\colvec{4}{b_{11}}{b_{21}}{\vdots}{b_{m1}}+c_{2j}
		\colvec{4}{b_{12}}{b_{22}}{\vdots}{b_{m2}}+\cdots+c_{rj}\colvec{4}{b_{1r}}{b_{2r}}{\vdots}{b_{mr}}
	\]
	para todo $j\in\{1,\dots,n\}$. Similarmente, toda fila de $A$ puede
	escribirse como combinación lineal de las filas de $C$:
	\[
		(a_{i1}\;a_{i2}\;\cdots\;a_{in})=b_{i1}(c_{11}\;c_{12}\;\cdots\;c_{in})+\cdots+b_{ir}(c_{r1}\;\cdots\;c_{rn})
	\]
	para todo $i\in\{1,\dots,m\}$.
\end{remark}

\begin{thm}
	\label{thm:rgC=rgF}
	Sea $A\in\K^{m\times n}$. Entonces $\rg_F(A)=\rg_C(A)$.

	\begin{proof}
		\framebox{FIXME}
        Si $A=0$ el resultado es trivial. Supongamos entonces que $A\ne0$.  Sea
        $r$ el menor entero positivo tal que existen $B\in\K^{m\times r}$ y
        $C\in\K^{r\times n}$ tales que $A=BC$. (Observemos que si $I$ es la
        identidad de $n\times n$ entonces $A=AI$ y entonces la existencia de un
        tal $r$ está garantizada.) Por lo visto en la
        observación~\ref{rem:producto}, las filas de $C$ forman un conjunto
        minimal de generadores de $E_F(A)$ y las $r$ columnas de $B$ forman un
        conjunto minimal de generadores de $E_C(A)$.  Luego, $\dim E_F(A)=\dim
        E_C(A)=r$. 
	\end{proof}
\end{thm}

\begin{block}
	El \textbf{rango} de una matriz $A\in\K^{m\times n}$ se define como el
	rango fila (o columna) de $A$ y se denota por $\rg(A)$.
\end{block}

\begin{cor}
	Si $A\in\K^{m\times n}$ entonces $\rg(A)=\rg(A^T)$.

	\begin{proof}
		Es consecuencia inmediata del teorema~\ref{thm:rgC=rgF}.
	\end{proof}
\end{cor}

\begin{prop}
    Sean $A\in\K^{m\times n}$ y $B\in\K^{n\times p}$. Entonces
    \[
        \rg(AB)\leq\min\{\rg A,\rg B\}.
    \]

    \begin{proof}
        La proposición es consecuencia de la observación~\ref{rem:producto}. 
        Como el espacio generado por las columnas de $AB$ es un subespacio del
        espacio generado por las columnas de $A$, $\rg(AB)\leq\rg(A)$.
        Similarmente, como el espacio generado por las filas de $AB$ es un
        subespacio del espacio generado por las filas de $B$, se tiene que
        $\rg(AB)\leq\rg(B)$.
    \end{proof}
\end{prop}

\begin{example}
    Sean $x_1,\dots,x_n$ números reales distintos. 
    Veamos que el conjunto 
    \[
    \left\{
        \colvec{4}{1}{x_1}{\vdots}{x_1^{n-1}},\dots,
        \colvec{4}{1}{x_n}{\vdots}{x_n^{n-1}}
    \right\}
    \]
    es una base de $\R^{n\times1}$. Para esto veamos que la matriz 
    \[
        \begin{pmatrix}
            1 & \cdots & 1\\
            x_1 & \cdots & x_n\\
            \vdots & \ddots & \vdots\\
            x_1^{n-1} & \cdots & x_n^{n-1}
        \end{pmatrix},
    \]
    de tamaño $n\times n$, 
    tiene rango fila igual a $n$. En efecto, si las filas fueran linealmente
    dependientes, existirían $\alpha_0,\dots,\alpha_{n-1}\in\K$, no todos cero,
    tales que
    \[
    0=\alpha_0(1,1,\dots,1)+\alpha_1(x_1,x_2,\dots,x_n)+\cdots+\alpha_{n-1}(x_1^{n-1},x_2^{n-1},\dots,x_n^{n-1}).
    %0=\alpha_0\colvec{4}{1}{1}{\vdots}{1}+\alpha_1\colvec{4}{x_1}{x_2}{\vdots}{x_n}+\cdots+\alpha_{n-1}\colvec{4}{x_1^{n-1}}{x_2^{n-1}}{\vdots}{x_n^{n-1}}.
    \]
    Esto implicaría que los $x_1,\dots,x_n$ son raíces del polinomio
    $f=\sum_{i=0}^{n-1}\alpha_iX^i$. Esto es una contradicción pues $f$ tiene
    grado $\leq n-1$. 
\end{example}

\section{Coordenadas y matriz de cambio de base}

\begin{block}
    Sea $V$ un espacio vectorial de dimensión $n$ y sea $\cB=\{v_1,\dots,v_n\}$
    una base (ordenada) de $V$. Todo elemento $v\in V$ puede escribirse como
    combinación lineal de los elementos de $\cB$, digamos
    $v=\alpha_1v_1+\cdots+\alpha_n v_n$, donde los $\alpha_i\in\K$. La
    $n$-tupla $(\alpha_1,\dots,\alpha_n)\in\K^n$ es la tupla de
    \textbf{coordenadas} del vector $v$ en la base $\cB$. El vector de
    coordenadas de $v$ en la base $\cB$ se denota por $(v)_{\cB}$.
\end{block}

\begin{remark}
	Se tienen las siguientes propiedades:
	\begin{enumerate}
		\item $(v)_{\cB}+(w)_{\cB}=(v+w)_{\cB}$.
		\item $(\lambda v)_{\cB}=\lambda(v)_{\cB}$.
	\end{enumerate}
\end{remark}

\begin{examples}\
	\begin{enumerate}
		\item Sea $V=\C$ como espacio vectorial sobre $\R$. El conjunto
			$\{1,i\}$ es base de $V$.  Las coordenadas de $a+bi$ en la base
			$\{1,i\}$ son $(a,b)$.
		\item Sean $V=\R^2$. Las coordenadas de un vector $(x,y)\in V$ en la
			base $\{(2,3),(1,2)\}$ son $(2x-y,-3x+2y)$.
		\item Sea $V=\{f\in\R[X]:\deg f\leq2\}$. El vector de coordenadas
			del elemento $f=aX^2+bX+c$ en la base $\{X^2,X,1\}$ es $(a,b,c)$.
	\end{enumerate}
\end{examples}

\begin{block}
	Sea $V$ un espacio vectorial de dimensión $n$ y sean
	$\cB=\{v_1,\dots,v_n\}$ y $\cB'=\{v_1',\dots,v_n'\}$ dos bases de
	$V$. Escribamos a cada elemento de $\cB$ en la base $\cB'$:
	\begin{align*}
		v_j=\sum_{i=1}^n a_{ij}v_i'=a_{1j}v_1'+\cdots+a_{nj}v_n',\quad j\in\{1,\dots,n\}.
	\end{align*}
	La matriz $A=(a_{ij})$ es la matriz de \textbf{cambio de base} de $\cB$ a
	$\cB'$ y se denota por $C(\cB,\cB')$. 
	Observemos que las columnas de la matriz $C(\cB,\cB')$ son las coordenadas
	de los $v_i$ en la base $\cB'$, es decir:
	\[
		C(\cB,\cB')=( (v_1)_{\cB'}\mid\dots\mid (v_n)_{\cB'}).
	\]
	Es evidente que $C(\cB,\cB)=I$.
\end{block}

\begin{example}
	Sean $\{e_1,e_2\}$ la base canónica de $\R^2$ y $\cB=\{(1,1),(1,-1)\}$.  Un
	cálculo sencillo muestra que 
	\[
		C(\cB,\{e_1,e_2\})=
		\begin{pmatrix} 
			1 & 1\\
			1 & -1
		\end{pmatrix},
		\quad
		C(\{e_1,e_2\},\cB)=\begin{pmatrix}
			1/2 & 1/2\\
			1/2 & -1/2
		\end{pmatrix}.
	\]
\end{example}

\begin{prop}
	\label{pro:coordenadas}
	Sea $V$ un espacio vectorial de dimensión $n$ y supongamos que 
	$\cB=\{v_1,\dots,v_n\}$ y $\cB'=\{v_1',\dots,v_n'\}$ son dos bases de
	$V$. Si los $\alpha_i$ son las coordenadas de $v$ en la base $\cB$ y los
	$\beta_j$ son las coordenadas de $v$ en la base $\cB'$ entonces 
	\begin{align}
		\label{eq:cambio_de_coordenadas}
		C(\cB,\cB')
		\begin{pmatrix}
			\alpha_1\\
			\vdots\\
			\alpha_n
		\end{pmatrix}
		=
		\begin{pmatrix}
			\beta_1\\
			\vdots\\
			\beta_n
		\end{pmatrix}.
	\end{align}

	\begin{proof}
		Un cálculo directo muestra que 
		\[
		v=\sum_{j=1}^n \alpha_jv_j=\sum_{j=1}^n\alpha_j\left(\sum_{i=1}^na_{ij}v_i'\right)=\sum_{i=1}^n\left(\sum_{j=1}^na_{ij}\alpha_j\right)v_i',
		\]
		tal como se quería demostrar.
	\end{proof}
\end{prop}

\begin{thm}
	\label{thm:cambio_de_base}
	Sea $V$ un espacio vectorial de dimensión $n$ y supongamos que
	$\cB=\{v_1,\dots,v_n\}$ y $\cB'=\{v_1',\dots,v_n'\}$ y
	$\cB''=\{v_1'',\dots,v_n''\}$ son tres bases de $V$. Entonces
	\[
		C(\cB,\cB'')=C(\cB',\cB'')C(\cB,\cB').
	\]

	\begin{proof}
		Supongamos que $C(\cB,\cB'')=(c_{ij})$, $C(\cB',\cB'')=(b_{ij})$ y que
		$C(\cB,\cB')=(a_{ij})$, es decir:
		\begin{align*}
			v_j=\sum_{i=1}^na_{ij}v_i',
			&&
			v_j'=\sum_{i=1}^n b_{ij}v_i'',
			&&
			v_j=\sum_{i=1}^nc_{ij}v_i''.
		\end{align*}
		Como $v_j=\sum_{i=1}^nc_{ij}v_i''$, 
		\begin{align*}
			v_j = \sum_{k=1}^n a_{kj}v_k'
			=\sum_{k=1}^n a_{kj}\left(\sum_{i=1}^n b_{ik}v_i''\right)
			=\sum_{i=1}^n\left(\sum_{k=1}^n b_{ik}a_{kj}\right)v_i''
		\end{align*}
		y los $v_i''$ forman una base de $V$, se tiene que $c_{ij}=\sum_{k=1}^n
		b_{ik}a_{kj}$ que es lo que se quería demostrar.
	\end{proof}
\end{thm}

\begin{cor}
	\label{cor:C_inversible}
	La matriz $C(\cB,\cB')$ es inversible y $C(\cB,\cB')^{-1}=C(\cB',\cB)$. 

	\begin{proof}
		Por el teorema~\ref{thm:cambio_de_base} sabemos que $C(\cB,\cB)=C(\cB',\cB)C(\cB,\cB')$ y 
		$C(\cB',\cB')=C(\cB,\cB')C(\cB',\cB)$. El corolario queda demostrado pues $C(\cB,\cB)=I$ y
		$C(\cB',\cB')=I$.
	\end{proof}
\end{cor}

%\begin{example}\framebox{FIXME}
%	En el espacio vectorial de polinomios en $\R[X]$ de grado $\leq2$
%	consideremos las bases $\cB=\{X^2,X,1\}$ y $\cB'=\{(X+1)^2, X+1, 1\}$. La
%	matriz de cambio de base de $\cB$ a $\cB'$ es
%	\[
%		C(\cB,\cB')=
%		\begin{pmatrix}
%			1 & -1 & 1\\
%			0 & 1 & 2\\
%			0 & 0 & 1
%		\end{pmatrix}.
%	\]
%	La proposición~\ref{pro:coordenadas} nos da la fórmula
%	\[
%		aX^2+bX+c=a(X+1)^2+(b-2a)(X+1)+(c-b+a),
%	\]
%	que puede escribirse como
%	\[
%		aX^2+bX+c=\frac{f''(-1)}{2}(X+1)^2+f'(-1)(X+1)+f(-1).
%	\]
%\end{example}

\begin{prop}
	\label{pro:C(-,B)}
	Sea $A\in\K^{n\times n}$ una matriz inversible y $\cB$ una base de $\K^n$.
	Entonces existe una base $\cB'$ de $\K^n$ tal que $A=C(\cB',\cB)$.

	\begin{proof}
		Supongamos que $A=(a_{ij})$. Para cada $j\in\{1,\dots,n\}$ sea
		$v_j'=\sum_{i=1}^n a_{ij}v_i$.  Veamos que el conjunto
		$\{v_1',\dots,v_n'\}$ es linealmente independiente. Como
		\begin{align*}
			0=\sum_{j=1}^n\alpha_jv_j'=\sum_{j=1}^n\alpha_j\left(\sum_{i=1}^n a_{ij}v_i\right)=\sum_{i=1}^n\left(\sum_{j=1}^n a_{ij}\alpha_j\right)v_i
		\end{align*}
		y los $v_i$ son linealmente independientes, $\sum_{j=1}^n
		a_{ij}\alpha_j=0$ para todo $i$. Entonces 
		\[
			A\begin{pmatrix}
				\alpha_1\\
				\vdots\\
				\alpha_n
			\end{pmatrix}=0
		\]
		y luego, como $A$ es inversible, $\alpha_j=0$ para todo
		$j\in\{1,\dots,n\}$. Como tenemos $n$ de los $v_j'$, 
%		Veamos ahora que los $v_i'$ generan a $\K^n$.
%		Dado $v\in V$ existen $\beta_1,\dots,\beta_n\in\K$ tales que
%		$v=\sum_{i=1}^n\beta_i v_i$.  Como $A$ es inversible, existen
%		$\alpha_1,\dots,\alpha_n$ tales que
%		\[
%		A\begin{pmatrix}
%			\alpha_1\\
%			\vdots\\
%			\alpha_n
%		\end{pmatrix}
%		=
%		\begin{pmatrix}
%			\sum_{j=1}^n a_{1j}\alpha_j\\
%			\vdots\\
%			\sum_{j=1}^n a_{nj}\alpha_j
%		\end{pmatrix}
%		=
%		\begin{pmatrix}
%			\beta_1\\
%			\vdots\\
%			\beta_n
%		\end{pmatrix}.
%		\]
%		Entonces
%		\begin{align*}
%			v=\sum_{i=1}^n \beta_i v_i=\sum_{i=1}^n\left(\sum_{j=1}^n a_{ij}\alpha_j\right)v_i=\sum_{j=1}^n\alpha_j\left(\sum_{i=1}^n a_{ij}v_i\right)=\sum_{j=1}^n \alpha_jv_j'.
%		\end{align*}
		los $v_j'$ son entonces una base de $\K^n$. Luego basta tomar
		$\cB=\{v_1',\dots,v_n'\}$ y la proposición queda demostrada. 
	\end{proof}
\end{prop}

\begin{cor}
	\label{cor:C(B,-)}
	Sea $A\in\K^{n\times n}$ una matriz inversible y $\cB$ una base de $\K^n$.
	Entonces existe una base $\cB'$ de $\K^n$ tal que $A=C(\cB,\cB')$.

	\begin{proof}
		Basta aplicar la proposición~\ref{pro:C(-,B)} a la matriz $A^{-1}$ y
		utilizar el corolario~\ref{cor:C_inversible} que afirma que
		$C(\cB',\cB)^{-1}=C(\cB,\cB')$.
	\end{proof}
\end{cor}

\begin{example}
    Consideremos en el espacio vectorial $\R[X]$ los primeros cuatro
    \textbf{polinomios de Hermite}:
    \begin{align}
        \label{eq:Hermite}
        H_0 = 1,
        &&
        H_1 = 2X,
        && 
        H_2 = 4X^2-2,
        &&
        H_3 = 8X^3-12X.
    \end{align}
    Dejamos como ejercicio demostrar que $\cB=\{H_0,H_1,H_2,H_3\}$ es un
    conjunto linealmente independiente. Escribamos cada polinomio de Hermite en
    la base canónica $\cE=\{1,X,X^2,X^3\}$ de $\R_3[X]$ y formemos las matrices
    de cambio de base: 
    \begin{align*}
        C(\cB,\cE)=\begin{pmatrix}
            1 & 0 & -2 & 0\\
            0 & 2 & 0 & -12\\
            0 & 0 & 4 & 0\\
            0 & 0 & 0 & 8
        \end{pmatrix},
        &&
        C(\cB,\cE)^{-1}=C(\cE,\cB)=\frac18
        \begin{pmatrix}
            8 & 0 & 4 & 0\\
            0 & 4 & 0 & 6\\
            0 & 0 & 2 & 0\\
            0 & 0 & 0 & 1
        \end{pmatrix}.
    \end{align*}
    Luego, para escribir a $f=a_0+a_1X+a_2X^2+a_3X^3$ en la base de los
    polinomios de Hermite, hacemos
    \[
        C(\cE,\cB)\colvec{4}{a_0}{a_1}{a_2}{a_3}=\frac18\colvec{4}{8a_0+4a_2}{4a_1+6a_3}{2a_2}{a_3},
    \]
    es decir:
    \[
        f=\frac18(8a_0+4a_2)H_0+\frac18(4a_1+6a_3)H_1+\frac14a_2H_2+\frac18a_3H_3.
    \]
\end{example}

\section{Una aplicación del rango}

\begin{problem}
    Una ciudad tiene dos reglas que controlan la creación de clubes: 1) Cada
    club debe tener un número impar de miembros, y 2) Dos clubes cualesquiera
    tienen a lo sumo un número par de miembros en común.  Demuestre que la
    cantidad de clubes que pueden crearse con estas reglas es menor o igual que
    la cantidad de habitantes de la ciudad.

    \begin{solution}
        Supongamos que la ciudad tiene $n$ habitantes y sean $C_1,\dots,C_m$
        los clubes. Queremos demostrar que $m\leq n$.  Sea
        $A=(a_{ij})\in\Z_2^{m\times n}$ la matriz dada por 
        \[
        a_{ij}=\begin{cases}
            1 & \text{ si $j\in C_i$},\\
            0 & \text{ si $j\not\in C_i$}.
        \end{cases}
        \]
        Entonces $AA^T\in\Z_2^{m\times m}$ y $(AA^T)_{ij}=\sum_{k=1}^n
        a_{ik}a_{jk}$ es la cantidad de personas que hay en $C_i\cap C_j$. En
        particular, 
        \[
            \begin{cases}
                1 & \text{si $|C_i\cap C_j|$ es impar},\\
                0 & \text{si $|C_i\cap C_j|$ es par}.
            \end{cases}
        \]
        La condiciones para la creación de clubes nos dicen que $AA^T=I$. Luego
        \[
        m=\rg(I)=\rg(AA^T)\leq\rg(A)\leq\min\{n,m\}
        \]
        y entonces $m\leq n$. 
    \end{solution}

    
\end{problem}

\section{Una aplicación a las fracciones simples}

\begin{block}
	Sean $r_1,\dots,r_n\in\R$ y sea 
	$g=(X-r_1)\cdots(X-r_n)\in\R[X]$. 
	Como aplicación de la teoría de espacios vectoriales vamos a demostrar que
	si $f\in\R[X]$ con $\deg f<\deg g$ entonces existen
	$\alpha_1,\dots,\alpha_n\in\R$ tales que 
	\[
		\frac{f}{g}=\sum_{i=1}^n\alpha_i\frac{1}{X-r_i}.
	\]
	En efecto, primero observamos que 
	el conjunto
	\[
	S=\left\{\frac{f}{g}:\deg f<\deg g\right\}
	\]
	con las operaciones 
	\[
		\frac{f_1}{g}+\frac{f_2}{g}=\frac{f_1+f_2}{g},\quad
		\lambda\frac{f}{g}=\frac{\lambda f}{g},
	\]
	es un espacio vectorial sobre $\R$. Además tiene dimensión finita pues el
	conjunto
	$
	\left\{\frac{1}{g},\frac{X}{g},\dots,\frac{X^{n-1}}{g}\right\}\subseteq S
	$
	es base de $S$.  Por otro lado, como 
	\[
	\left\{\frac{1}{X-r_1},\dots,\frac{1}{X-r_n}\right\}\subseteq S
	\]
	es un conjunto linealmente independiente, es también base de $S$ y entonces 
	existen escalares $\alpha_1,\dots,\alpha_n\in\R$ tales que
	\[
		\frac{f}{g}=\sum_{i=1}^n\alpha_i\frac{1}{X-r_i},
	\]
	tal como queríamos demostrar. 
\end{block}
%\section{Existencia de bases de Hamel}
%
%\begin{block}
%    Sea $P$ un conjunto no vacío \textbf{parcialmente ordenado}, es decir:
%    existe una relación $\leq$ en $P$ tal que $x\leq y$ e $y\leq x$ implica que
%    $x=y$, $x\leq y$ e $y\leq z$ implica que $x\leq z$, y además dados $x,y\in
%    P$ con $x\leq y$ e $y\leq x$ entonces $x=y$. Un subconjunto no vacío $Q\subseteq P$ 
%    es una \textbf{cadena} si dados $x,y\in Q$ entonces $x\leq y$ o bien $y\leq x$. 
%
%    \begin{lemma}[lema de Zorn]
%                                
%    \end{lemma}
%\end{block}


\chapter{Transformaciones lineales}

\section{Transformaciones lineales}

\begin{block}
    Sean $V$ y $W$ dos espacios vectoriales sobre $\K$. Una función $f\colon V\to W$ 
    es una \textbf{transformación lineal} si 
    \[
        f(x+y)=f(x)+f(y),\quad
        f(\lambda x)=\lambda f(x),
    \]
	para todo $x,y\in V$ y $\lambda\in\K$. Denotaremos por $\hom(V,W)$ al
	conjunto de transformaciones lineales $V\to W$.  En $\hom(V,W)$ se tiene
	una multiplicación y está dada por la composición: si $f\in\hom(U,V)$ y
	$g\in\hom(V,W)$ entonces la composición $g\circ f\colon U\to W$, dada por
	$x\mapsto g(f(x))$, es un elemento de $\hom(U,W)$. 
    
    Utilizaremos la notación $gf=g\circ f$. 
    
    La composición de transformaciones lineales tiene las siguientes
    propiedades:
    \begin{enumerate}
        \item Es asociativa.
        \item La identidad $\id_V\colon V\to V$, dada por $\id_V(x)=x$, cumple
            $f\id_V=\id_V$ para toda $f\in\hom(V,W)$. 
        \item La identidad $\id_W\colon W\to W$ cumple $\id_Wf=\id_W$ para toda
            $f\in\hom(V,W)$. 
        \item Valen las propiedades distributivas:
            \[
            (f_1+f_2)g=f_1g+f_2g,\quad
            f(g_1+g_2)=fg_1+fg_2,
            \]
            para toda $f_1,f_2,f\in\hom(U,V)$ y $g_1,g_2,g\in\hom(V,W)$.
    \end{enumerate}
    Tal como sucede con las matrices, la composición de aplicaciones lineales
    en general no es conmutativa. 
\end{block}

\begin{xca}
	Una función $f\colon V\to W$ es una transformación lineal si y sólo si
	$f(v+\lambda w)=f(v)+\lambda f(w)$ para todo $v,w\in V$ y $\lambda\in\K$.
\end{xca}

\begin{block}
    Si $V$ es un espacio vectorial, la mayoría de las funciones, digamos $V\to
    V$, no son transformaciones lineales.  De hecho, supongamos por ejemplo que
    $V=\Z_2^4$ como espacio vectorial sobre el cuerpo $\Z_2$. Entonces $V$
    tiene $2^4=16$ elementos y hay $16^{16}$ funciones $V\to V$. Por otro lado, cada
    transformación lineal queda determinada por su valor en una base, y como cada base de $V$ 
    tiene cuatro elementos, hay, como mucho, $16^4$ transformaciones lineales $V\to V$. Luego
    la probabilidad de que una función $V\to V$ elegida al azar sea una transformación lineal
    es menor o igual a $1/16^{12}$.
\end{block}

\begin{examples}\
    \begin{enumerate}
        \item Sea $A\in\K^{m\times n}$. La función $f\colon\K^{n\times1}\to\K^m$ dada
            por $x\mapsto Ax$ es una transformación lineal pues
            \begin{align*}
                &f(x+y)=A(x+y)=Ax+Ay=f(x)+f(y),\\
                &f(\lambda x)=A(\lambda x)=\lambda Ax=\lambda f(x).
            \end{align*}
        \item Sea $A\in\K^{m\times n}$. La función $f\colon\K^{m}\to\K^n$ dada por
            \[
                (x_1,\dots,x_m)\mapsto \left(\sum_{i=1}^m a_{i1}x_i,\dots,\sum_{i=1}^n a_{in}x_i\right)
            \]
            es una transformación lineal. 
        \item La traza $\tr\colon\K^{n\times n}\to\K$ es una transformación
            lineal.
        \item La derivada 
            \[
            K[X]\to\K[X],\quad
                f\mapsto f'
            \]
            donde $f'$ denota el polinomio derivado de $f$, 
            es una transformación lineal. 
        \item La derivada 
            \[
                C^{\infty}(\R)\to C^{\infty}(\R), \quad
                f\mapsto f',
            \]
            donde $f'$ denota la función derivada de $f$, es una transformación
            lineal.
        \item Sean $a,b\in\R$ con $a<b$. La aplicación 
            \[
                C(\R)\to C(\R), \quad 
                f\mapsto\int_{a}^{b}f(x)dx
            \]
            es una transformación lineal.
		\item Las funciones $f\colon\K^{\infty}\to\K^{\infty}$ y $g\colon\K^{\infty}\to\K^{\infty}$, 
			\begin{align*}
            &f(x_1,x_2,x_3\dots)=(x_2,x_3,\dots),\\
			&g(x_1,x_2,x_3,\dots)=(0,x_1,x_2,\dots),
			\end{align*}
            son transformaciones lineales.
    \end{enumerate}
\end{examples}

\begin{thm}[Propiedad fundamental de las transformaciones lineales]
    Sea $V$ un espacio vectorial sobre $\K$ de dimensión finita y sea
    $\{v_1,\dots,v_n\}$ una base de $V$.  Sea $W$ un espacio vectorial sobre
    $\K$ y sean $w_1,\dots,w_n\in W$. Entonces existe una única 
    $f\in\hom(V,W)$ tal que $f(v_i)=w_i$ para todo $i\in\{1,\dots,n\}$. 

    \begin{proof}
        Demostremos la existencia: todo $v\in V$ se escribe unívocamente como
        $v=\sum_{i=1}^n\alpha_iv_i$. Definimos entonces a $f$ como
        $f(v)=\sum_{i=1}^n\alpha_iw_i$. Es evidente que $f$ es una
        transformación lineal.
        
        Demostremos ahora la unicidad: si 
        $g\in\hom(V,W)$ y $g(v_i)=w_i=f(v_i)$ para todo $i\in\{1,\dots,n\}$ entonces 
        \[
        g(v)=g\left(\sum_{i=1}^n\alpha_iv_i\right)=\sum_{i=1}^n\alpha_ig(v_i)=\sum_{i=1}^n\alpha_if(v_i)=f\left(\sum_{i=1}^n\alpha_iv_i\right)=f(v),
        \] y luego $f=g$.
    \end{proof}
\end{thm}

\begin{example}
    Encontraremos explícitamente las transformaciones lineales $f\colon\R^2\to\R^3$ tales que $f(1,2)=(1,0,2)$ y $f(-1,3)=(0,0,1)$. Como $\{(1,2),(-1,3)\}$ es base de $\R^2$,
    el teorema anterior nos dice que existe una única $f$. Si $(x,y)\in\R^2$ entonces
    \[
        (x,y)=\frac{3x+y}{5}(1,2)+\frac{-2x+y}{5}(-1,3)
    \]
    y luego 
    \[
        f(x,y)=\frac{3x+y}{5}(1,0,2)+\frac{-2x+y}{5}(0,0,1).
    \]
\end{example}

%\section{Una aplicación: baldosas cuadradas}
%
%\begin{problem}
%	Demuestre que no es posible cubrir el piso de una habitación rectangular de
%	$1\times\sqrt{2}$ con un número finito de bandosas cuadradas (Las baldosas
%	no tienen que ser todas necesariamente del mismo tamaño.)
%
%	\begin{solution}
%		Supongamos que se tienen $B_1,\dots,B_k$ baldosas cuadradas, donde la
%		baldosa $B_i$ tiene lado $b_i$.  Consideremos a $\R$ como espacio
%		vectorial sobre $\Q$ y sea $V$ el espacio vectorial sobre $\Q$ generado
%		por $\{1,\sqrt{2},b_1,\dots,b_k\}$.  Definimos la transformación lineal
%		$f\colon V\to\Q$, $f(1)=1$,  $f(\sqrt{2})=-1$ y $f(b_j)=0$ para todo
%		$b_j\not\in\langle 1,\sqrt{2}\rangle$.  Al comparar las áreas, tenemos
%		\begin{align}
%			\label{eq:sqrt2}
%			\sqrt{2}=b_1^2+\cdots+b_k^2.
%		\end{align}
%		Si escribimos a cada $b_i\in\langle 1,\sqrt{2}\rangle$ como
%		$b_i=x_i+y_i\sqrt{2}$ con $x_i,y_i\in\Q$, entonces
%		\[
%			b_i^2=(x_i+y_i\sqrt{2})^2=x_i^2+2x_iy_i\sqrt{2}+y_i^2. 
%		\]
%		Luego $f(b_i^2)\geq0$ para todo $b_i$ tal que $b_i\not\in\langle1,\sqrt{2}\rangle$ pues 
%		\[
%			f(b_i^2)=f(x_i^2+2x_iy_i\sqrt{2}+y_i^2)=x_i^2-2x_iy_i+y_i^2=(x_i-y_i)^2\geq0.
%		\]
%		Al aplicar $f$ en la ecuación~\eqref{eq:sqrt2} obtenemos 
%		\[
%			-1=f(b_1^2)+\cdots+f(b_k^2)\geq0,
%		\]
%		una contradicción.
%	\end{solution}
%\end{problem}

\section{Subespacios asociados a transformaciones lineales}

\begin{xca}
	\label{xca:imagen_y_preimagen}
	Sean $V$ y $W$ dos espacios vectoriales sobre $\K$ y $f\in\hom(V,W)$.
	Demuestre las siguientes afirmaciones:
	\begin{enumerate}
		\item Si $S\subseteq V$ es subespacio entonces $f(S)\subseteq W$ es
			subespacio. 
		\item Si $T\subseteq W$ es subespacio entonces $f^{-1}(T)\subseteq V$ es
			subespacio. 
	\end{enumerate}
\end{xca}

\begin{block}
	Sean $V$ y $W$ dos espacios vectoriales sobre $\K$ y $f\in\hom(V,W)$. 
	Se define el \textbf{núcleo} de $f$ como el conjunto
	\[
		\ker f=\{v\in V: f(v)=0\}
	\]
	y la \textbf{imagen} de $f$ como 
	\[
    \im f=f(V)=\{f(v):v\in V\}.
	\]
	Como consecuencia del ejercicio~\ref{xca:imagen_y_preimagen} se tiene que
	$\ker f\subseteq V$ e $\im f\subseteq W$ son subespacios.
\end{block}

\begin{example}
	Sea $A\in\K^{m\times n}$ y sea $f$ la transformación $x\mapsto Ax$. Más precisamente:
	\[
	f\colon\K^{n\times1}\to\K^{m\times1},
	\quad
	\begin{pmatrix}
		x_1\\
		\vdots\\
		x_n
	\end{pmatrix}
	\mapsto
	\begin{pmatrix}
		\sum_{j=1}^n a_{1j}x_j\\
		\vdots\\
		\sum_{j=1}^n a_{mj}x_j
	\end{pmatrix}.
	\]
	Entonces $\ker f$ es el conjunto de soluciones del sistema homogéneo $Ax=0$
	y la imagen $\im f$ es el conjunto de matrices $b$ de $m\times1$ tales que
	existe $x$ de tamaño $n\times1$ con $Ax=b$. 
\end{example}

\begin{block}
	Sea $f\in\hom(V,W)$. Entonces $f$ es inyectiva si y sólo si $\ker f=0$. En
	efecto, si $f$ es inyectiva y $v\in\ker f$ entonces $f(0)=0=f(v)$ y luego
	$v=0$. Recíprocamente, si $\ker f=0$ y $f(v)=f(v')$ entonces $f(v-v')=0$ y
	por lo tanto $v-v'=0$.
\end{block}

\begin{block}
	Sea $f\in\hom(V,W)$. Diremos que $f$ es \textbf{monomorfismo} si $\ker f=0$, $f$
	es \textbf{epimorfismo} si $f$ es sobreyectiva, $f$ es \textbf{isomorfismo}
	si $f$ es monomorfismo y epimorfismo. Una transformación lineal $f\in\hom(V,V)$
	se denomina \textbf{endomorfismo}.  Un endomorfismo biyectivo se denomina
	\textbf{automorfismo}.
\end{block}

\begin{xca}
	\label{xca:fg=id}
	Sean $f\in\hom(V,W)$ y $g\in\hom(W,V)$ tales que $fg=\id$. Pruebe que $f$ es
	epimorfismo y $g$ es monomorfismo. 

	\begin{solution}
		Si $w\in W$ entonces $g(w)\in V$ y $w=f(g(w))$. Si $w\in\ker g$ entonces
		$g(w)=0$ y luego $w=(fg)(w)=f(0)=0$. 
	\end{solution}
\end{xca}

\begin{examples}
	Veamos algunos ejemplos de transformaciones lineales:
	\begin{enumerate}
		\item $\R^2\to\R^3$, $(x,y)\mapsto(x,y,x)$, es monomorfismo y no es epimorfismo.
		\item $\R^3\to\R$, $(x,y,y)\mapsto x$, es epimorfismo y no es monomorfismo. 
		\item $\K^{n+1}\to\K_n[X]$, $(a_0,\dots,a_n)\mapsto\sum_{i=0}^na_0X^i$, es isomorfismo.
		\item Si $A\in\K^{2\times2}$ es inversible entonces
			\[
				\K^{2\times1}\to\K^{2\times1},
				\quad			
				\begin{pmatrix}x\\y\end{pmatrix}\mapsto
				A\begin{pmatrix}x\\y\end{pmatrix},
			\]
			es un automorfismo de $\R^{2\times1}$.
	\end{enumerate}
\end{examples}

\begin{block}
	Si $f\in\hom(V,W)$ es un isomorfismo y $g$ es la inversa de $f$ entonces
	$g\in\hom(W,V)$. Sean $w,w'\in W$. Entonces existen $v,v'\in V$ tales que
	$f(v)=w$ y $f(v')=w'$. Luego
	$g(w+w')=g(f(v)+f(v'))=g(f(v+v'))=v+v'=g(w)+g(w')$. Además, si
	$\lambda\in\K$, entonces $g(\lambda w)=g(\lambda f(v))=g(f(\lambda
	v))=\lambda v=\lambda g(w)$.
\end{block}

\begin{block}
	Sean $V$ y $W$ dos espacios vectoriales sobre $\K$. Si existe un
	isomorfismo $V\to W$ entonces $V$ y $W$ se dicen \textbf{isomorfos}. La
	notación es: $V\simeq W$.
\end{block}

\begin{lem}
	\label{lem:mono}
	Sean $V$ y $W$ dos espacios vectoriales sobre $\K$, y $f\in\hom(V,W)$ un
	monomorfismo.  Si $\{v_1,\dots,v_n\}\subseteq V$ es linealmente
	independiente entonces $\{f(v_1),\dots,f(v_n)\}$ es linealmente
	independiente.

	\begin{proof}
		Si $\sum_{i=1}^n\alpha_if(v_i)=0$ entonces
		$f\left(\sum_{i=1}^n\alpha_iv_i\right)=0$ y luego, como $f$ es
		monomorfismo, $\sum_{i=1}^n\alpha_iv_i=0$. Como los $v_i$ son
		linealmente independientes, $\alpha_i=0$ para todo $i$.
	\end{proof}
\end{lem}

\begin{lem}
	\label{lem:epi}
	Sean $V$ y $W$ dos espacios vectoriales sobre $\K$, y $f\in\hom(V,W)$ un
	epimorfismo.  Si 	$\{v_1,\dots,v_n\}\subseteq V$ es un conjunto 
	de generadores de $V$ entonces $\{f(v_1),\dots,f(v_n)\}$ genera $W$.

	\begin{proof}
		Sea $w\in W$. Como $f$ es epimorfismo, existe $v\in V$ tal que
		$w=f(v)$. Si escribimos
		$v=\sum_{i=1}^n\alpha_iv_i$ entonces 
		$w=f(v)=\sum_{i=1}^n\alpha_if(v_i)$. 
	\end{proof}
\end{lem}

\begin{prop}
	\label{pro:iso}
	Sean $V$ y $W$ dos espacios vectoriales sobre $\K$ y supongamos que $V$
	tiene dimensión finita y que $\{v_1,\dots,v_n\}$ es una base de $V$. Sea
	$f\in\hom(V,W)$.  Entonces $f$ es un isomorfismo si y sólo si
	$\{f(v_1),\dots,f(v_n)\}$ es base de $W$.

	\begin{proof}
		Si $f$ es un isomorfismo, es monomorfismo y entonces, por el
		lema~\ref{lem:mono}, el conjunto $\{f(v_1),\dots,f(v_n)\}$ es
		linealmente independiente. Como $f$ es también un epimorfismo,
		$\{f(v_1),\dots,f(v_n)\}$ genera a $W$ por el lema~\ref{lem:epi}.

		Supongamos ahora que los $f(v_i)$ son base de $W$.  Probemos que $f$ es
		epimorfismo: si $w\in W$ entonces $w\in\im f$ pues $w=\sum_{i=1}^n\alpha_if(v_i)=f(\sum_{i=1}^n \alpha_i
		v_i)$. Probemos ahora que $f$ es monomorfismo: sea $v$ tal que
		$f(v)=0$. Entonces $v=\sum_{i=1}^n\alpha_iv_i$ y luego
		$0=f(v)=f(\sum_{i=1}^n\alpha_iv_i)=\sum_{i=1}^n\alpha_if(v_i)$. Como los $f(v_i)$ son
		base, $\alpha_i=0$ para todo $i$. Luego $v=0$.
	\end{proof}
\end{prop}

%\begin{prop}
%	\label{pro:transformacion_lineal}
%	Sean $V$ y $W$ dos espacios vectoriales sobre $\K$.
%	\begin{enumerate}
%		\item Si $V=\langle v_1,\dots,v_n\rangle$ entonces $\im f=\langle
%			f(v_1),\dots,f(v_n)\rangle$. 
%		\item Sean $\{w_1,\dots,w_m\}\in\im f$ linealmente independiente y
%			$v_1,\dots,v_m\in V$ tales que $f(v_i)=w_i$ para todo $i$. Entonces
%			$\{v_1,\dots,v_m\}$ es linealmente independiente. 
%		\item Sea $\{v_1,\dots v_n\}\subseteq V$ linealmente independiente. Si
%			$f$ es monomorfismo entonces $\{f(v_1),\dots,f(v_n)\}$ es
%			linealmente independiente.
%		\item Supongamos que $\dim V=n$ y sea $\{v_1,\dots,v_n\}$ una base de
%			$V$. Entonces $f$ es isomorfismo si y sólo si
%			$\{f(v_1),\dots,f(v_n)\}$ es base de $W$.
%	\end{enumerate}
%
%	\begin{proof}\
%		\begin{enumerate}
%			\item Si $w\in \im f$ entonces existe $v\in V$ tal que $f(v)=w$.
%				Como $v=\sum\alpha_i v_i$, entonces $f(v)=f(\sum\alpha_i
%				v_i)=\sum\alpha_if(v_i)$. 
%			\item Si $0=\sum\alpha_i v_i$ entonces
%				$0=f(\sum\alpha_iv_i)=\sum\alpha_iw_i$. Como los $w_i$ son
%				independientes, $\alpha_i=0$ para todo $i$.
%			\item Si $0=\sum\alpha_if(v_i)$ entonces $0=f(\sum\alpha_iv_i)$.
%				Como $f$ es monomorfismo, $\sum\alpha_iv_i=0$. Como los $v_i$
%				son linealmente independientes, $\alpha_i=0$ para todo $i$.
%			\item Supongamos que $f$ es isomorfismo. Entonces, como $f$ es
%				monomorfismo, $\{f(v_1),\dots,f(v_n)\}$ es linealmente independiente
%				por el tercer ítem. Por el primer ítem, $\{f(v_1),\dots,f(v_n)\}$
%				genera a $\im f=W$. Supongamos ahora que los $f(v_i)$ son base de $W$.
%				Probemos que $f$ es epimorfismo: si $w\in W$ entonces
%				$w=\sum\alpha_if(v_i)=f(\sum\alpha v_i)$. Probemos ahora que $f$ es
%				monomorfismo: sea $v$ tal que $f(v)=0$. Entonces $v=\sum\alpha_iv_i$ y
%				luego $0=f(v)=f(\sum\alpha_iv_i)=\sum\alpha_if(v_i)$. Como los $f(v_i)$
%				son base, $\alpha_i=0$ para todo $i$. Luego $v=0$.
%		\end{enumerate}
%	\end{proof}
%\end{prop}

\begin{thm}
	Sean $V$ y $W$ dos espacios vectoriales sobre $\K$ de dimensión finita.
	Entonces $V\simeq W$ si y sólo si $\dim V=\dim W$. 

	\begin{proof}
		Supongamos que $V\simeq W$ y sea $f\colon V\to W$ un isomorfismo. Si
		$\{v_1,\dots,v_n\}$ es una base de $V$ entonces
		$\{f(v_1),\dots,f(v_n)\}$ es una base de $W$ por la
		proposición~\ref{pro:iso}. Luego $\dim V=\dim W$.

		Recíprocamente, supongamos que $\dim V=\dim W$ y sean
		$\{v_1,\dots,v_n\}$ un base de $V$ y $\{w_1,\dots,w_n\}$ una base de
		$W$. Entonces la función $f\colon V\to W$ que cumple $f(v_i)=w_i$ para
		todo $i$, es un isomorfismo por la 
		proposición~\ref{pro:iso}.
	\end{proof}
\end{thm}

\begin{thm}[teorema de la dimensión]
	\label{thm:dimension}
	Sean $V$ y $W$ dos espacios vectoriales sobre $\K$ y $f\in\hom(V,W)$.
	Supongamos que $V$ es de dimensión finita.  Entonces
	\[
		\dim V=\dim\ker f+\dim\im f.
	\]

    \begin{proof} 
        Sea $\{v_1,\dots,v_k\}$ una base de $\ker f$. Consideremos la
        extensión a una base $\{v_1,\dots,v_k,v_{k+1},\dots,v_n\}$ de $V$.
        Afirmamos que $\{f(v_{k+1}),\dots,f(v_n)\}$ es una base de $\im f$. 
        Veamos que es un conjunto linealmente independiente: si 
        \[
            0=\sum_{i=k+1}^n\alpha_if(v_i)=f\left(\sum_{i=k+1}^n\alpha_iv_i\right)
        \]
        entonces, como $\{v_1,\dots,v_k\}$ es base de $\ker f$, 
        existen $\beta_1,\dots,\beta_k\in\K$ tales que 
        \[
            \sum_{i=k+1}^n \alpha_i v_i-\sum_{j=1}^k \beta_jv_j=0.
        \]
        Como los $v_i$ son linealmente independientes, $\alpha_i=0$ para todo $i\in\{k+1,\dots,n\}$ y $\beta_j=0$ para
        todo $j\in\{1,\dots,k\}$. Veamos ahora que $\im f$ está generado por $\{f(v_{k+1}),\dots,f(v_n)\}$: si $v\in V$ 
        entonces $v=\sum_{i=1}^n\alpha_iv_i$. Luego 
        \[
            f(v)=\sum_{i=1}^n\alpha_if(v_i)=\sum_{i=k+1}^n\alpha_if(v_i)
        \]
        pues $f(v_j)=0$ para todo $j\in\{1,\dots,k\}$. Esto demuestra el teorema.
    \end{proof}
%	\begin{proof}
%		Si $f=0$ el resultado es trivialmente válido. Supongamos entonces que
%		$f\ne0$.  Sea $\{w_1,\dots,w_l\}$ una base de $\im f$ y sean
%		$v_1',\dots,v_l'\in V$ tales que $f(v_i')=w_i$ para todo $i$. Como los
%		$w_i$ son linealmente independientes, los $v_i'$ son linealmente
%		independientes pues si $0=\sum_{i=1}^n\alpha_iv_i'$ entonces 
%		\[
%		0=f(0)=f\left(\sum_{i=1}^n\alpha_iv_i'\right)=\sum_{i=1}^n\alpha_if(v_i')=\sum_{i=1}^n\alpha_iw_i
%		\]
%		y luego $\alpha_i=0$ para todo $i$.
%		Sea $\{v_1,\dots,v_k\}$ una base de $\ker f$. Vamos a demostrar que 
%		$\{v_1,\dots,v_k,v_1',\dots,v_l'\}$ 
%		es una base de $V$. Veamos primero que es un conjunto de generadores de
%		$V$: si $v\in V$ entonces $f(v)\in\im f$ y luego \[
%		f(v)=\sum_{i=1}^l\alpha_i w_i=\sum_{i=1}^l\alpha_if(v_i')=f\left(\sum_{i=1}^l\alpha_iv_i'\right). 
%		\]
%		Entonces $v-\sum_{i=1}^l\alpha_iv_i'\in\ker f$ y, como los $v_j$ son base de
%		$\ker f$, tenemos 
%		\[
%		v=\sum_{i=1}^l\alpha_iv_i'+\sum_{j=1}^k\beta_jv_j. 
%		\]
%
%		Veamos ahora que 
%		$\{v_1,\dots,v_k,v_1',\dots,v_l'\}$ 
%		es
%		linealmente independiente: si 
%		\[
%		\sum_{i=1}^l\alpha_iv_i'+\sum_{j=1}^k\beta_jv_j=0
%		\]
%		entonces $0=\sum_{i=1}^l\alpha_if(v_i')=\sum_{i=1}^l\alpha_iw_i$. Como los $w_i$ son
%		linealmente independientes, $\alpha_i=0$ para todo $i$. Luego
%		$\sum\beta_jv_j=0$ y entonces $\beta_j=0$ para todo $j$ pues los $v_j$
%		son linealmente independientes.
%	\end{proof}
\end{thm}

\begin{cor}
	\label{cor:no_monomorfismo}
	Sean $V$ y $W$ dos espacios vectoriales sobre $\K$ de dimensión finita. 
	Sea $f$ un monomorfismo. Entonces $\dim V\leq \dim W$.

	\begin{proof}
        Si $f$ es monomorfismo entonces $\ker f=\{0\}$. Luego,  
		por el teorema de la dimensión,
		\[
			0=\dim\ker f=\dim V-\dim\im f\geq \dim V-\dim W,
		\]
		como se quería demostrar.
	\end{proof}
\end{cor}

\begin{cor}
	\label{cor:no_epimorfismo}
	Sean $V$ y $W$ dos espacios vectoriales sobre $\K$ de dimensión finita. Sea $f$ un epimorfismo. Entonces
    $\dim V\geq\dim W$.

	\begin{proof}
        Si $f$ es un epimorfismo entonces $\im f=W$. Luego, por el teorema de la dimensión,
		\[
			\dim W=\dim\im f=\dim V-\dim\ker f\leq \dim V,
		\]
		tal como se quería demostrar.
	\end{proof}
\end{cor}

\begin{cor}
    \label{cor:mono<=>epi<=>iso}
	Sean $V$ y $W$ dos espacios vectoriales sobre $\K$ de dimensión finita tales que $\dim V=\dim
	W$ y sea $f\in\hom(V,W)$. Son equivalentes:
	\begin{enumerate}
		\item $f$ es isomorfismo.
		\item $f$ es monomorfismo.
		\item $f$ es epimorfismo.
	\end{enumerate}

	\begin{proof}
		La implicación $(1)\Rightarrow(2)$ es trivial. Para demostrar
		$(2)\Rightarrow(3)$ observemos que, si $f$ es monomorfismo, $\ker
		f=\{0\}$ y luego $\im f=W$, ya que por el teorema de la dimensión se
		tiene $\dim V=\dim\im f$. Queda demostrar que $(3)\Rightarrow(1)$. Si
		$f$ es epimorfismo entonces $\im f=W$ y luego, por el teorema de la
		dimensión, $\dim V=\dim\ker f+\dim W$. Como $\dim V=\dim W$, $\ker
		f=\{0\}$ y entonces $f$ es isomorfismo.
	\end{proof}
\end{cor}

\begin{example}
	El corolario anterior no vale en dimensión infinita. Por ejemplo, si
	$f\colon\K^{\infty}\to\K^{\infty}$ es la transformación dada por
	$f(x_1,x_2,x_3,\dots)=(x_2,x_3,\dots)$, $g\colon\K^{\infty}\to\K^{\infty}$
	está dada por $g(x_1,x_2,\dots)=(0,x_1,x_2,\dots)$ entonces $fg=\id$ pues 
	\begin{align*}
		(fg)(x_1,x_2,\dots)=f(0,x_1,x_2,\dots)=(x_1,x_2,\dots).
	\end{align*}
	Por el ejercicio~\ref{xca:fg=id} sabemos que entonces $f$ es epimorfismo y
	$g$ es monomorfismo.  Sin embargo, $f$ no es monomorfismo y $g$ no es
	epimorfismo.
\end{example}

\begin{xca}
    \label{xca:Silvester}
    Sean $U,V,W$ espacios vectoriales de dimensión finita y sean
    $f\in\hom(U,V)$ y $g\in\hom(V,W)$. Demuestre las siguientes afirmaciones:
    \begin{enumerate}
        \item $\dim\ker(gf)\leq\dim\ker f+\dim\ker g$.
        \item $\dim\im(gf)\leq\min\{\dim\im f,\dim\im g\}$.
        \item $\dim\im f+\dim\im g-\dim V\leq \dim\im(gf)$.
    \end{enumerate}
\end{xca}

%\begin{prop}
%	\label{pro:inversa}
%	Sean $A,B,C\in\K^{n\times n}$. 
%	\begin{enumerate}
%		\item Si $CA=I$ entonces $A$ es inversible.
%		\item Si $AB=I$ entonces $A$ es inversible.
%	\end{enumerate}
%
%	\begin{proof}
%		Supongamos primero que $CA=I$. Consideremos el sistema lineal $Ax=0$. Como 
%		\[
%			x=Ix=(CA)x=C(Ax)=C0=0,
%		\]
%		el sistema $Ax=0$ tiene solución única, y entonces $Ax=b$ también tiene
%		solución única para cualquier $b\in\K^{n\times1}$ por el
%		teorema~\ref{thm:sistemas:n=m}. En particular, al tomar como $b$ a los
%		$e_1,\dots,e_n$, donde $(e_i)_j=\delta_{ij}$, tenemos la existencia de
%		una matriz $B\in\K^{n\times n}$ tal que $AB=I$.  Por~\ref{rem:inversa}
%		sabemos que entonces $B=C$ y luego $A$ es inversible.
%
%		Supongamos ahora que $AB=I$. Por el primer ítem, $B$ es inversible y
%		entonces $A=B^{-1}$ es también inversible. 
%	\end{proof}
%\end{prop}
%
%\begin{cor}
%	Sean $A,B\in\K^{n\times n}$. Entonces $AB$ es inversible si y sólo si $A$ y
%	$B$ son inversibles.
%
%	\begin{proof}
%		Si $A$ y $B$ son inversibles entonces $AB$ es inversible por lo visto
%		en~\ref{rem:inversa}. Supongamos entonces que $AB$ es inversible.
%		Entonces existe $C\in\K^{n\times n}$ tal que $(AB)C=C(AB)=I$. En
%		particular, como $A(BC)=I$ y $(CA)B=I$, la matrices $A$ y $B$ son
%		inversibles por la proposición~\ref{pro:inversa}.
%	\end{proof}
%\end{cor}
%
%\begin{prop}
%	Sea $A\in\K^{n\times n}$. Son equivalentes:
%	\begin{enumerate}
%		\item $A$ es inversible.
%		\item $Ax=0$ tiene una única solución.
%		\item Para cada $b\in\K^{n\times 1}$ el sistema $Ax=b$ tiene una única solución.
%		\item Para cada $b\in\K^{n\times 1}$ el sistema $Ax=b$ tiene al menos una solución.
%	\end{enumerate}
%
%	\begin{proof}
%		Demostremos primero $(1)\Rightarrow(2)$. Si $Ax=0$ entonces, como $A$
%		es inversible, $x=Ix=(A^{-1}A)x=A^{-1}(Ax)=A^{-1}0=0$ y $Ax=0$ tiene
%		una única solución. La implicación $(2)\Rightarrow(3)$ es el
%		teorema~\ref{thm:sistemas:n=m}.  La implicación $(3)\Rightarrow(4)$ es
%		trivial. Finalmente, para demostrar $(4)\Rightarrow(1)$ basta tomar $b$
%		igual a los $e_j$ para construir una matriz $B$ tal que $AB=I$ y
%		utilizar la proposición~\ref{pro:inversa}.
%	\end{proof}
%\end{prop}

\section{Proyectores}

\begin{block}
    Una transformación lineal $f\colon V\to V$ es un \textbf{proyector} si
    $f^2=f$. 
\end{block}

\begin{example}
	La transformación lineal $f\colon \R^3\to\R^3$ dada por
	\[
		f(x,y,z)=(3x-2z,-x+y+z,3x-2z)
	\]
	es un proyector.
\end{example}

\begin{xca}
	\label{xca:proyector}
    Sea $f\in\hom(V,V)$.  Pruebe que $f$ es un proyector si y sólo si $f(w)=w$
    para todo $w\in\im f$. 
\end{xca}

\begin{xca}
    \label{xca:proyector(2f-1)^2}
    Sea $V$ un espacio vectorial sobre $\R$. Demuestre que $f\in\hom(V,V)$ es
    un proyector si y sólo si $(2f-\id_V)^2=\id_V$. 
\end{xca}

\begin{prop}
    Sea $f\in\hom(V,V)$ un proyector. Entonces 
    \[
        V=\ker f\oplus \im f.
    \]

    \begin{proof}
        Veamos primero que $V=\ker f+\im f$. En efecto, si $v\in V$ entonces
        $v=v-f(v)+f(v)\in \ker f+\im f$ pues $f(v-f(v))=f(v)-f^2(v)=0$. Si
        $v\in\ker f\cap \im f$ entonces, como $v\in\im f$, existe $x\in V$ tal
        que $v=f(x)$. Luego, como $v\in\ker f$, $0=f(v)=f(f(x))=f(x)$ y
        entonces $v=0$. 
    \end{proof}
\end{prop}

\begin{xca}
	\label{xca:pfp=fp}
	Sean $V$ un espacio vectorial y $f\in\hom(V,V)$. Pruebe que $S\subseteq V$
	es un \textbf{subespacio invariante} por $f$ (es decir: $f(S)\subseteq S$)
	si y sólo si $pfp=fp$ para todo proyector $p\colon V\to V$ con $\im p=S$.
\end{xca}

\begin{prop}
	Sea $V$ un espacio vectorial y sean $S$ y $T$ dos subespacios de $V$ tales que 
	$V=S\oplus T$. Entonces existe un único proyector $f\colon
	V\to V$ tal que $\im f=T$ y $\ker f=S$.

		\begin{proof}
			Demostremos primero la existencia: sea $f\colon V\to V$ dado por
			$f(s+t)=t$ para todo $s\in S$ y $t\in T$.  Como $V=S\oplus T$, $f$
			está bien definida. Es claro que $\ker f=S$ y que $\im f=T$.
			Además $f$ es un proyector pues si $v\in V$ entonces $v$ se escribe
			unívocamente como $v=s+t\in S+T$ y entonces $f(f(v))=t=f(v)$. 

			Demostremos la unicidad: si $g\in\hom(V,V)$ es un proyector tal que
			$\im g=T$ y $\ker f=S$ entonces, como todo $v\in V$ se escribe
			unívocamente como $v=s+t$ con $s\in S$ y $t\in T$, se tiene que
			$g(v)=g(s)+g(t)=g(t)$. Como $g$ es proyector, $g(y)=y$ para todo
			$y\in\im g$ por el ejercicio~\ref{xca:proyector}. Luego
			$g(v)=t=f(v)$. 
		\end{proof}
\end{prop}

\begin{xca}
	\label{xca:oplus_proyectores}
	Sea $V$ un espacio vectorial y sean $S_1,\dots,S_n\subseteq V$ subespacios.
	Pruebe que las siguientes afirmaciones son equivalentes:
	\begin{enumerate}
		\item $V=S_1\oplus\cdots\oplus S_n$.
		\item Existen proyectores $p_1,\dots,p_n\in\hom(V,V)$ con $S_i=\im p_i$
			para todo $i$ tales que $p_ip_j=0$ si $i\ne j$ y
			$p_1+\cdots+p_n=\id_V$.
	\end{enumerate}
\end{xca}

\begin{xca}
	\label{xca:proyector_matriz}
	Sean $V$ un espacio vectorial de dimensión finita sobre $\K$ y
	$f\in\hom(V,V)$. Pruebe que $f$ es un proyector si y sólo si existe una
	base $\cB$ de $V$ tal que 
	\[
	\|f\|_{\cB}=
	\left(
	\begin{array}{c|c}
		\id_r & 0\\ \hline
		0 & 0
	\end{array}
	\right)
	\]
	para algún $r\leq\dim V$.
\end{xca}

\section{Matriz de una transformación lineal}

\begin{block}
    Sean $V$ y $W$ dos espacios vectoriales sobre $\K$ de dimensión finita y
    sean $\cB_V=\{v_1,\dots,v_n\}$ y $\cB_W=\{w_1,\dots,w_m\}$ bases ordenadas
    de $V$ y $W$ respectivamente.  Sea $f\in\hom(V,W)$. Si escribimos cada
    $f(v_j)$ en la base $\cB_W$
    \[
        f(v_j)=\sum_{i=1}^m a_{ij}w_i
    \]
    se define la \textbf{matriz de la transformación lineal} $f$ con respecto a
    las bases $\cB_V$ y $\cB_W$ como la matriz
    $\|f\|_{\cB_V,\cB_W}\in\K^{m\times n}$ dada por
    \[
        \|f\|_{\cB_V,\cB_W}
        =
        \begin{pmatrix}
            a_{11} & \cdots & a_{1n}\\
            a_{21} & \cdots & a_{2n}\\
            \vdots & \ddots & \vdots\\
            a_{m1} & \cdots & a_{mn}
        \end{pmatrix}.
    \]
\end{block}

\begin{examples}\
    \begin{enumerate}
        \item Si $V=W$ y $\cB$ y $\cB'$ son dos bases de $V$ 
            entonces la matriz de la identidad $\id_V\colon V\to V$,
            $v\mapsto v$, es $\|\id_V\|_{\cB,\cB'}=C(\cB,\cB')$.
%        \item Si $A=(a_{ij})\in\K^{m\times n}$ y $f\colon\K^n\to\K^m$ 
%            es la transformación lineal dada por
%            \[
%                f(x_1,\dots,x_n)=(x_1\;\dots\;x_n)A^T,
%            \]
%            entonces la matriz de $f$ con respecto a las bases canónicas de
%            $\K^n$ y $\K^m$ es $\|f\|=A$.
		\item Si $A\in\K^{m\times n}$ y $f\colon\K^{n\times1}\to\K^{m\times1}$ 
            es la transformación lineal dada por
            \[
			f\colvec{3}{x_1}{\vdots}{x_n}=A\colvec{3}{x_1}{\vdots}{x_n},
            \]
            entonces la matriz de $f$ con respecto a las bases canónicas de
			$\K^{n\times1}$ y $\K^{m\times1}$ es $\|f\|=A$.
    \end{enumerate}
\end{examples}

\begin{example}
    Sea $v=(a,b,c)\in\R^3$ y sea $f_v\colon\R^3\to\R^3$ la transformación
    lineal dada por $w\mapsto v\times w$, donde $v\times w$ denota el producto
    vectorial entre $v$ y $w$. Como $f_v(1,0,0)=(0,c,-b)$,
    $f_v(0,1,0)=(-c,0,a)$ y $f_v(0,0,1)=(b,-a,0)$, la matriz de $f_v$ con
    respecto a las bases canónicas es
    \[
        \|f_v\|=
        \begin{pmatrix}
            0 & -c & b\\
            c & 0 & -a\\
            -b & a & 0
        \end{pmatrix}.
    \]
\end{example}

\begin{example}
    Consideremos la transformación lineal $\partial\colon\K_n[X]\to\K_{n}[X]$
    dada por 
    \[
        \partial(a_0+a_1X+\cdots+a_nX^n)=a_1+a_2X+2a_3X^2+\cdots+na_nX^{n-1}.
    \]
    Como $\partial(1)=0$ y $\partial(X^j)=jX^{j-1}$ para todo
    $j\in\{1,\dots,n\}$, la matriz de $\partial$ con respecto a las bases
    canónicas es
    \[
    \|\partial\|
    =
    \begin{pmatrix}
        0 & 1 & 0 & \cdots & 0\\
        0 & 0 & 2 & \cdots & 0\\
        \vdots & \vdots & \vdots & \ddots & \vdots\\
        0 & 0 & 0 & \cdots & n\\
        0 & 0 & 0 & \cdots & 0\\
    \end{pmatrix}\in\K^{(n+1)\times{(n+1)}}.
    \]
\end{example}

\begin{block}
    Como toda transformación lineal queda unívocamente determinada por su valor
    en una base, se obtiene el siguiente resultado: Sean $f,g\in\hom(V,W)$ y
    sean $\cB_V$ y $\cB_W$ bases de $V$ y $W$ respectivamente. Entonces $f=g$
    si y sólo si $\|f\|_{\cB_V,\cB_W}=\|g\|_{\cB_V,\cB_W}$. 
\end{block}

\begin{prop}
    \label{pro:|f|a=b}
	Sean $\cB_V=\{v_1,\dots,v_n\}$ y $\cB_W=\{w_1,\dots,w_m\}$ bases ordenadas de $V$ y
	$W$ respectivamente.  Sea $f\in\hom(V,W)$, sea $v\in V$ y supongamos que 
    \[
        v=\sum_{i=j}^n\alpha_jv_j,\quad
        f(v)=\sum_{i=1}^m\beta_iw_i,
    \]
    es decir: $(v)_{\cB_V}=(\alpha_1,\dots,\alpha_n)$ y
    $\left(f(v)\right)_{\cB_W}=(\beta_1,\dots,\beta_m)$.  Entonces
    \[
        \|f\|_{\cB_V,\cB_W}\colvec{3}{\alpha_1}{\vdots}{\alpha_n}=\colvec{3}{\beta_1}{\vdots}{\beta_m}.
    \]

    \begin{proof}
        La matriz de $f$ con respecto a $\cB_V$ y $\cB_W$ es $(a_{ij})\in\K^{m\times n}$ donde
        $f(v_j)=\sum_{i=1}^m a_{ij}w_i$ para todo $j$. 
        Como $v=\sum_{j=1}^n\alpha_jv_j$ entonces 
        \begin{align*}
            f(v)&=\sum_{j=1}^n\alpha_jf(v_j)=\sum_{j=1}^n\alpha_j\left(\sum_{i=1}^ma_{ij}w_i\right)
            =\sum_{i=1}^m\left(\sum_{j=1}^na_{ij}\alpha_j\right)w_i.
        \end{align*}
        Como los $w_i$ son base de $W$ entonces
        $\sum_{j=1}^na_{ij}\alpha_j=\beta_i$ para todo $i$, que es lo que
        queríamos demostrar.
    \end{proof}
\end{prop}

\begin{thm}
    \label{thm:|gf|=|g||f|}
    Sean $U$, $V$ y $W$ espacios vectoriales sobre $\K$ de dimensión finita, y
    sean $\cB_U=\{u_1,\dots,u_r\}$, $\cB_V=\{v_1,\dots,v_n\}$ y
    $\cB_W=\{w_1,\dots,w_m\}$ son bases ordenadas de $U$, $V$ y $W$,
    respectivamente.  Si $f\in\hom(U,V)$ y $g\in\hom(V,W)$ entonces
    \[
        \|gf\|_{\cB_U,\cB_W}=\|g\|_{\cB_V,\cB_W}\|f\|_{\cB_U,\cB_V}.
    \]

    \begin{proof}
        Supongamos que 
        \begin{align*}
        f(u_j)=\sum_{k=1}^n b_{kj}v_k,&&
        g(v_k)=\sum_{i=1}^ma_{ik}w_i,&&
        (gf)(u_j)=\sum_{i=1}^m c_{ij}w_i,
        \end{align*}
        para todo $j\in\{1,\dots,r\}$ y $k\in\{1,\dots,n\}$, es decir: 
        \begin{align*}
            \|f\|_{\cB_U,\cB_V}=(b_{ij}),
            && 
            \|g\|_{\cB_V,\cB_W}=(a_{ij}),
            &&
            \|gf\|_{\cB_U,\cB_W}=(c_{ij}).
        \end{align*}
        Para cada $j\in\{1,\dots,r\}$ se tiene entonces
        \begin{align*}
            (gf)(u_j)&=g(f(u_j))=g\left(\sum_{k=1}^n b_{kj}v_k\right)=\sum_{k=1}^n b_{kj}g(v_k)\\
            &=\sum_{k=1}^n b_{kj}\left(\sum_{i=1}^ma_{ik}w_i\right)
            =\sum_{i=1}^m\left(\sum_{k=1}^na_{ik}b_{kj}\right)w_i.
        \end{align*}
        Como los $w_k$ son base de $W$ y $(gf)(u_j)=\sum_{i=1}^mc_{ij}w_i$ para todo $j$, se tiene
        que $c_{ij}=\sum_{k=1}^na_{ik}b_{kj}$, tal como se quería demostrar.
    \end{proof}
\end{thm}

\begin{cor}
    \label{cor:iso<=>|f|inversible}
    Sean $V$ y $W$ dos espacios vectoriales sobre $\K$ de dimensión $n$, y
    sean $\cB_V$ y $\cB_W$ bases ordenadas de $V$ y $W$ respectivamente.  Si
    $f\in\hom(V,W)$ entonces $f$ es un isomorfismo si y sólo si
    $\|f\|_{\cB_V,\cB_W}$ es inversible.

    \begin{proof}
        Supongamos que $\cB_V=\{v_1,\dots,v_n\}$, $\cB_W=\{w_1,\dots,w_n\}$.
        Si $\|f\|_{\cB_V,\cB_W}$ es inversible, sean
        $v=\sum_{i=1}^n\alpha_iv_i\in\ker f$ y $f(v)=\sum_{j=1}^n\beta_jw_j$.
        Entonces $\beta_j=0$ para todo $j$. Por la
        proposición~\ref{pro:|f|a=b},
        \[
        \|f\|_{\cB_V,\cB_W}\colvec{3}{\alpha_1}{\vdots}{\alpha_n}=\colvec{3}{0}{\vdots}{0}
        \]
        y entonces $\alpha_i=0$ para todo $i$ pues $\|f\|_{\cB_V,\cB_W}$ es
        inversible. Luego $f$ es monomorfismo y entonces $f$ es isomorfismo por
        el corolario~\ref{cor:mono<=>epi<=>iso}. 

        Recíprocamente, supongamos que $f$ es un isomorfismo y sea $g\in\hom(W,V)$
        la inversa de $f$. Entonces $\id_V=gf$ y $\id_W=fg$. Por el
        teorema~\ref{thm:|gf|=|g||f|},
        \begin{align*}
            &I=\|\id_V\|_{\cB_V,\cB_W}=\|gf\|_{\cB_V,\cB_W}=\|g\|_{\cB_W,\cB_V}\|f\|_{\cB_V,\cB_W}.
        \end{align*}
        Análogamente se obtiene que $\|f\|_{\cB_V,\cB_W}\|g\|_{\cB_W,\cB_V}=I$
        y entonces la matriz $\|f\|_{\cB_V,\cB_W}$ es inversible.
    \end{proof}
\end{cor}

\begin{cor}
    \label{cor:semejanza:tls}
    Sea $V$ un espacio vectorial de dimensión $n$ y sean $\cB_V$ y $\cB_V'$
    bases ordenadas de $V$. Sea $W$ un espacio vectorial de dimensión $m$ y sean $\cB_W$
    y $\cB_W'$ bases ordenadas de $W$. Si $f\in\hom(V,W)$ entonces
    \[
        \|f\|_{\cB_V',\cB_W'}=C(\cB_W,\cB_W')\|f\|_{\cB_V,\cB_W}C(\cB_V,\cB_V')^{-1}.
    \]
    
    \begin{proof}
		El teorema~\ref{thm:|gf|=|g||f|} implica que
		\begin{align*}
			\|f\|_{\cB_V',\cB_W'}&=\|\id_W\circ f\|_{\cB_V',\cB_W'}\\
			&=\|\id_W\|_{\cB_W,\cB_W'}\|f\|_{\cB_V',\cB_W}\\
			&=C(\cB_W,\cB_W')\|f\circ\id_V\|_{\cB_V',\cB_W}\\
			&=C(\cB_W,\cB_W')\|f\|_{\cB_V,\cB_W}\|\id_V\|_{\cB_V',\cB_V}\\
			&=C(\cB_W,\cB_W')\|f\|_{\cB_V,\cB_W}C(\cB_V',\cB_V),
		\end{align*}
		de donde se deduce el corolario.
%        Vamos a demostrar que 
%        \[
%            \|f\|_{\cB_V',\cB_W'}C(\cB_V,\cB_V')=C(\cB_w,\cB_W')\|f\|_{\cB_V,\cB_W}.
%        \]
%        Sea $v\in V$ y sea $(v)_{\cB_V}$ el vector de coordenadas de $v$ en la
%        base $\cB_V$. Con la proposición~\ref{pro:coordenadas} se obtiene:
%        \[
%        C(\cB_W,\cB_W')\|f\|_{\cB_V,\cB_W}(v)_{\cB_V}=C(\cB_W,\cB_W')(f(v))_{\cB_W}=(f(v))_{\cB_W'}.
%        \]
%        Por otro lado, también con la proposición~\ref{pro:coordenadas}: 
%        \[
%        \|f\|_{\cB_V',\cB_W'}C(\cB_V,\cB_V')(v)_{\cB_V}=\|f\|_{\cB_V',\cB_W'}(v)_{\cB_V'}=(f(v))_{\cB_W'}.
%        \]
%        De esto se deduce el corolario.
    \end{proof}
\end{cor}

\begin{block}
    Diremos que dos matrices cuadradas $A\in\K^{n\times n}$ y $B\in\K^{n\times
    n}$ son \textbf{semejantes} si existe una matriz inversible
    $C\in\K^{n\times n}$ tal que $B=CAC^{-1}$. 
\end{block}

\begin{cor}
    \label{cor:semejanza}
    Dos matrices $A,B\in\K^{n\times n}$ son semejantes si y sólo si existe
    $f\in\hom(\K^{n\times1},\K^{n\times1})$ y existen $\cB$ y $\cB'$ bases ordenadas de 
    $\K^{n\times1}$ tales que $\|f\|_{\cB,\cB}=A$ y $\|f\|_{\cB',\cB'}=B$.

    \begin{proof}
        Supongamos que $\|f\|_{\cB,\cB}=A$ y que $\|f\|_{\cB',\cB'}=B$. Para
        demostrar que las matrices $A$ y $B$ son semejantes hay que aplicar el
        corolario~\ref{cor:semejanza:tls} con $V=W=\K^{n\times1}$,
        $\cB_V=\cB_W=\cB$ y $\cB_V'=\cB_W'=\cB'$. 

		Recíprocamente, supongamos que $B=CAC^{-1}$. Sea
		$f\colon\K^{n\times1}\to\K^{n\times1}$ la transformación lineal
		definida por $f(x)=Ax$. Entonces $\|f\|=A$.  Como $C$ es una matriz
		inversible, la proposición~\ref{pro:C(-,B)} nos dice que existe una
		base $\cB$ de $\K^{n\times1}$ tal que $C$ es la matriz de cambio de
		base entre $\cB$ y la base canónica $\{e_1,\dots,e_n\}$, es decir
		$C=C(\cB,\{e_1,\dots,e_n\})$. 
		Entonces, por el corolario~\ref{cor:semejanza:tls},
		\[
		\|f\|_{\cB,\cB}=C\|f\|C^{-1}=CAC^{-1}=B,
		\]
		tal como queríamos demostrar.
		%por $g(x)=C^{-1}x$. Como $C^{-1}$ es inversible, $g$ es un isomorfismo.
		%Si $\cE=\{e_1,\dots,e_n\}$ es la base canónica de $\K^{n\times1}$
		%entonces $\cB=\{g(e_1),\dots,g(e_n)\}=\{C^{-1}e_1,\dots,C^{-1}e_n\}$ es
		%base de $\K^{n\times1}$.  Las columnas de la matriz $C^{-1}$ son los
		%$C^{-1}e_j$ y entonces $C(\cB,\cE)=C^{-1}$. Luego $C(\cE,\cB)=C$ y
		%entonces, el corolario \ref{cor:semejanza:tls} con $V=W=\K^{n\times1}$,
		%$\cB_V=\cB_W=\cE$ y $\cB_V'=\cB_W'=\cB$ implica que $B=\|f\|_{\cB,\cB}$.
    \end{proof}
\end{cor}

\begin{remark}
    Sean $A\in\K^{m\times n}$ y \[
	f\colon\K^{n\times1}\to\K^{m\times1},
	\quad
	x\mapsto Ax.
	\]
	Si $\{e_1,\dots,e_n\}$ es la base
    canónica de $\K^{n\times1}$. Entonces $\im f=\langle
	Ae_1,\dots,Ae_n\rangle$ y luego, como los $Ae_j$ son las columnas de 
	$A$, obtenemos $\dim\im f=\rg(A)$. 
\end{remark}

\section{El espacio vectorial $\hom(V,W)$}

\begin{block}
	Sean $V$ y $W$ dos espacios vectoriales. El conjunto $\hom(V,W)$ de
	transformaciones lineales $V\to W$ con las operaciones 
    \begin{align*}
        & (f+g)(x)=f(x)+g(x),\\
        & (\lambda f)(x)=\lambda f(x),
    \end{align*}
    es un espacio vectorial sobre $\K$. 
\end{block}

\begin{prop}
    \label{pro:hom(V,W)}
    Sean $V$ y $W$ dos espacios vectoriales de dimensión finita y supongamos
    que $\dim V=n$ y $\dim W=m$. Entonces $\hom(V,W)\simeq\K^{m\times n}$. En
    particular, $\dim\hom(V,W)=(\dim V)(\dim W)$.

    \begin{proof}
        Sean $\cB_V=\{v_1,\dots,v_n\}$ y $\cB_W=\{w_1,\dots,w_m\}$ bases de $V$
        y $W$ respectivamente.  La función
        \[
            T\colon\hom(V,W)\to\K^{m\times n},\quad f\mapsto\|f\|_{\cB_V,\cB_W}.
        \]
		es una transformación lineal.  Primero observemos que $T$ es
		monomorfismo pues si $Tf=0$ entonces $f(v_j)=0$ para todo $j$ y luego
		$f=0$. Además $T$ es epimorfismo pues si $(a_{ij})\in\K^{m\times n}$
		entonces la función $f\colon V\to W$, $f(v_j)=\sum_{i=1}^ma_{ij}w_i$
		para todo $j$, es una función lineal que cumple
		$\|f\|_{\cB_V,\cB_W}=(a_{ij})$.    Luego $\hom(V,W)\simeq\K^{m\times
		n}$ y entonces $\dim\hom(V,W)=mn$.
    \end{proof}
\end{prop}

\section{Aplicaciones a sistemas lineales}

%\begin{block}
%	El corolario~\ref{cor:no_monomorfismo} da una demostración sencilla y
%	elegante de un resultado importante en la teoría de sistemas lineales, ver
%	corolario~\ref{cor:homogeneo}.
%
%	\begin{cor*}
%		\label{cor:mas_X_que_ecuaciones}
%		Un sistema lineal homogéneo que tiene más incógnitas que ecuaciones
%		tiene al menos una solución no trivial.		
%
%		\begin{proof}
%			Sean $A=(a_{ij})\in\K^{m\times n}$ y $f\colon\K^n\to\K^m$ la
%			transformación lineal dada por 
%			\[
%				(x_1,\dots,x_n)\mapsto\left(\sum_{i=1}^na_{1i}x_i,\dots,\sum_{i=1}^na_{mi}x_i\right).
%			\]
%
%			Observemos que $\ker f=0$ es un sistema homogéneo de $m$ ecuaciones
%			y $n$ incógnitas cuya matriz asociada es $A$.  El
%			corolario~\ref{cor:no_monomorfismo} nos dice que si $n>m$ entonces
%			$\ker f\ne0$, es decir: existe una solución no trivial del sistema
%			lineal homogéneo asociado a la matriz $A$.
%		\end{proof}
%	\end{cor*}
%\end{block}
%
%\begin{block}
%	El corolario~\ref{cor:no_epimorfismo} puede usarse para demostrar que un
%	sistema lineal no homogéneo que tiene más ecuaciones que incógnitas puede
%	no tener solución.
%\end{block}

\begin{block}
	\label{cor:Ax=0<=>Ax=b}
	El corolario~\ref{cor:mono<=>epi<=>iso} nos permite demostrar el siguiente
	resultado.

	\begin{cor*}
		Un sistema lineal no homogéneo de $n\times n$ tiene solución única si y
		sólo si el sistema lineal homogéneo asociado tiene solución única.

		\begin{proof}
			Sean $A=(a_{ij})\in\K^{n\times n}$ y
			$f\in\hom(\K^{n\times1},\K^{n\times1})$ dada por $x\mapsto Ax$.  El
			resultado se deduce de la proposición~\ref{cor:mono<=>epi<=>iso}
			que afirma que $f$ es monomorfismo si y sólo si $f$ es epimorfismo.
		%	$f\colon\K^n\to\K^n$ la
		%	transformación lineal dada por 
		%	\[
		%		(x_1,\dots,x_n)\mapsto\left(\sum_{i=1}^na_{1i}x_i,\dots,\sum_{i=1}^na_{ni}x_i\right).
		%	\]
		%	Luego el resultado es consecuencia directa de la
		%	proposición~\ref{cor:mono<=>epi<=>iso}.
		\end{proof}
	\end{cor*}
\end{block}

\begin{prop}
    Sea $A=(a_{ij})$ una matriz de tamaño $m\times n$. Entonces 
    \[
        \dim\{x\in\K^{n\times1}:Ax=0\}=n-\rg(A).
    \]

    \begin{proof}
        Sea $f\colon\K^{n\times1}\to\K^{m\times1}$ la transformación lineal
        definida por $x\mapsto Ax$. Entonces $\ker
        f=\{x\in\K^{n\times1}:Ax=0\}$ y luego \[
            \dim\ker f=n-\dim\im f=n-\rg(A),
        \]
        tal como se quería demostrar.
    \end{proof}
\end{prop}

\begin{prop}
    Sean $A=(a_{ij})\in\K^{m\times n}$ y $b\in\K^{m\times1}$. Entonces el
    sistema lineal $Ax=b$ tiene solución si y sólo si $\rg(A|b)=\rg(A)$,
    donde $(A|b)$ es la matriz ampliada del sistema $Ax=b$. 

    \begin{proof}
        Primero observemos que el sistema lineal $Ax=b$ puede rescribirse como
        \begin{equation}
            \label{eq:rg(A|b)=rg(A)}
            \colvec{3}{b_1}{\vdots}{b_m}=\sum_{i=1}^m x_i\colvec{3}{a_{1i}}{\vdots}{a_{mi}}.
        \end{equation}
        Ahora, $x\in\K^{n\times1}$ es solución de $Ax=b$ si y sólo si $x$ 
        satisface la ecuación~\eqref{eq:rg(A|b)=rg(A)}, y esto es
        equivalente a decir que $b$ es combinación lineal de las columnas de
        $A$, que es equivalente a $\rg(A|b)=\rg(A)$.
    \end{proof}
\end{prop}

\section{Aplicaciones a matrices inversibles}

\begin{prop}
	\label{pro:inversa}
	Sea $A\in\K^{n\times n}$. Entonces $A$ es inversible si y sólo si
	existe $B\in\K^{n\times n}$ tal que $AB=I$.

	\begin{proof}
		Sea $B\in\K^{n\times n}$ tal que $AB=I$. Sea $f\in\K^{n\times
		n}\to\K^{n\times n}$ la transformación lineal dada por $X\mapsto
		AX$. Veamos que $f$ es epimorfismo: si $C\in\K^{n\times n}$ entonces
		$f(BC)=A(BC)=(AB)C=IC=C$. Como
		\[
		f(BA-I)=A(BA-I)=A(BA)-AI=(AB)A-A=IA-A=A-A=0,
		\]
		y $f$ es monomorfismo por el corolario~\ref{cor:mono<=>epi<=>iso}, se
		tiene que $BA-I=0$. Luego $A$ es inversible y $A^{-1}=B$.
	\end{proof}
\end{prop}

\begin{cor}
	\label{cor:inversa}
	Sea $A\in\K^{n\times n}$. Entonces $A$ es inversible si y sólo si
	existe $C\in\K^{n\times n}$ tal que $CA=I$.

	\begin{proof}
		Si existe $C\in\K^{n\times n}$ tal que $CA=I$ entonces $C$ es
		inversible por la proposición anterior.  Luego $C^{-1}=A$ y
		entonces $A$ es inversible.  
	\end{proof}
\end{cor}

\begin{cor}
	Sean $A,B\in\K^{n\times n}$. Entonces $AB$ es inversible si y sólo si $A$ y
	$B$ son inversibles.

	\begin{proof}
		Si $A$ y $B$ son inversibles entonces $AB$ es inversible con inversa
		$B^{-1}A^{-1}$.  Supongamos que $AB$ es inversible.  Entonces existe
		$C\in\K^{n\times n}$ tal que $(AB)C=C(AB)=I$. En particular, como
		$A(BC)=I$ y $(CA)B=I$, la matrices $A$ y $B$ son inversibles por la
		proposición~\ref{pro:inversa}.
	\end{proof}
\end{cor}

\begin{prop}
	Sea $A\in\K^{n\times n}$. Son equivalentes:
	\begin{enumerate}
		\item $A$ es inversible.
		\item El sistema lineal $Ax=0$ tiene una única solución.
		\item Para cada $b\in\K^{n\times 1}$ el sistema $Ax=b$ tiene una única solución.
		\item Para cada $b\in\K^{n\times 1}$ el sistema $Ax=b$ tiene al menos una solución.
        \item $\rg(A)=n$. 
	\end{enumerate}

	\begin{proof}
        Demostremos primero $(1)\Rightarrow(2)$. Si $Ax=0$ entonces, como $A$
        es inversible, $x=Ix=(A^{-1}A)x=A^{-1}(Ax)=A^{-1}0=0$ y $Ax=0$ tiene
        una única solución. La implicación $(2)\Rightarrow(3)$ es el
        corolario~\ref{cor:Ax=0<=>Ax=b}.  La implicación $(3)\Rightarrow(4)$ es
        trivial. Para demostrar $(4)\Rightarrow(5)$ sea
        $f\colon\K^{n\times1}\to\K^{n\times1}$ dada por $x\mapsto Ax$. Como $f$
        es epimorfismo por hipótesis, $\dim\im f=\rg(A)=n$. Finalmente, para
        demostrar que $(5)\Rightarrow(1)$ basta observar que
        $f\colon\K^{n\times1}\to\K^{n\times1}$ dada por $x\mapsto Ax$ es un
        epimorfismo y entonces es un isomorfismo.
        %Para demostrar
		%$(4)\Rightarrow(1)$ basta tomar $b$ igual a los $e_j$ para construir
		%una matriz $B$ tal que $AB=I$ y utilizar la
		%proposición~\ref{cor:inversa}. 
	\end{proof}
\end{prop}

%\begin{prop}
%    Una matriz $A\in\K^{n\times n}$ es inversible si y sólo si $\rg(A)=n$. 
%    
%    \begin{proof}
%        Sea $f\colon\K^{n\times1}\to\K^{n\times1}$ la transformación lineal
%        definida por $x\mapsto Ax$. Entonces la matriz de $f$ con respecto a la
%        base canónica es $\|f\|=A$ y además vale que $\rg(A)=\dim\im f$. Si $A$
%        es inversible entonces $f$ es isomorfismo y $\rg(A)=\dim\im f=n$.
%        Recíprocamente, si $\rg(A)=n$ entonces $f$ es epimorfismo y luego $A$
%        es inversible por los corolarios~\ref{cor:mono<=>epi<=>iso}
%        y~\ref{cor:iso<=>|f|inversible}.
%    \end{proof}
%\end{prop}


\chapter{Espacio dual}

\section{Espacio dual y base dual}

\begin{block}
    Sea $V$ un espacio vectorial sobre $\K$. Se define el \textbf{espacio dual}
    $V^*$ como el espacio vectorial $\hom(V,\K^*)$. Los elementos de $V^*$ se
    denominal \textbf{funcionales lineales} de $V$.  Observemos que si $V$ es
    un espacio vectorial de dimensión finita $n$ entonces, por lo visto en la
    proposición~\ref{pro:hom(V,W)}, se tiene que 
    \[ 
        \dim V^*=\dim\hom(V,\K)=\dim V.
    \]
\end{block}

\begin{examples}
	En el espacio de matrices de $n\times n$ la traza $\tr\colon \K^{n\times
	n}\to\K$ es una funcional lineal.  En el espacio de polinomios, la
	evaluación en $\lambda\in\K$,
	\[
	\K[X]\to\K,\quad
	\sum_{i=1}^n a_iX^i\mapsto\sum_{i=1}^n a_i\lambda^i,
	\]
    es una funcional lineal. En el espacio vectorial real $C[a,b]$ de funciones
    continuas $[a,b]\to\R$ la aplicación $g\mapsto\int_{a}^{b}g(x)dx$ es una
    funcional lineal.
\end{examples}

\begin{example}
    Sea $V$ el subespacio de $C^\infty(\R)$ formado por las funciones $f$ tales
    que $f(x)=0$ para todo $x$ fuera de algún intervalo cerrado y acotado.  La
    aplicación $f\mapsto\int_{-\infty}^{+\infty}f(x)dx$ es una funcional lineal
    de $V$.
\end{example}

\begin{example}
    \label{exa:funcional_lineal}
    La función $f\colon\K^n\to\K$ dada por
    \[
        f(x_1,\dots,x_n)=\alpha_1x_1+\cdots+\alpha_nx_n
    \]
    es una funcional lineal.  Más aún, la matriz de $f$ con respecto a las
    bases canónicas de $\K^n$ y $\K$ es $\|f\|=(\alpha_1\cdots\alpha_n)$.
    Veamos que toda funcional lineal de $\K^n$ es de esta forma. En efecto, si
    $\{e_1,\dots,e_n\}$ es la base canónica de $\K^n$ basta con definir
    $f(e_j)=\alpha_j$ para todo $j$ pues entonces 
    \begin{align*}
        f(x_1,\dots,x_n) &= f(x_1e_1+\cdots+x_ne_n)\\
        &=x_1f(e_1)+\cdots+x_nf(e_n)\\
        &=\alpha_1x_1+\cdots+\alpha_nx_n.
    \end{align*}
\end{example}

\begin{example}
    Para $i\in\{1,2,3\}$ sean $\delta_i\colon\R^3\to\R$ dada por
    $(x_1,x_2,x_3)\mapsto x_i$. Entonces por lo visto en el
    ejemplo~\ref{exa:funcional_lineal} se tiene que 
    $(\R^3)^*=\langle\delta_1,\delta_2,\delta_3\rangle$.
%	\begin{align*}
%		\left(\R^3\right)^* &= \{f\colon\R^3\to\R:\text{ $f$ es transformación lineal}\}\\
%		&=\{f\colon\R^3\to\R:f(x_1,x_2,x_3)=\alpha_1 x_1+\alpha_2x_2+\alpha_3x_3\}\\
%		&=\{f\colon\R^3\to\R:f=\alpha_1\delta_1+\alpha_2\delta_2+\alpha_3\delta_3\}.
%	\end{align*}
\end{example}

\begin{xca}
    \label{xca:traza}
    Sea $V=\K^{n\times n}$ y sea $\delta\in V^*$ tal que
    $\delta(AB)=\delta(BA)$ para todo $A,B\in V$. Pruebe que existe
    $\lambda\in\K$ tal que $\delta(A)=\lambda\tr(A)$ para todo $A\in V$. 
\end{xca}

\begin{prop}
	Sea $V$ un espacio vectorial de dimensión finita y sea $\{v_1,\dots,v_n\}$
	una base de $V$. Entonces existe una única base $\{f_1,\dots,f_n\}$ de
	$V^*$ tal que $f_i(v_j)=\delta_{ij}$ para todo $i,j$. La base
	$\{f_1,\dots,f_n\}$ se denomina \textbf{base dual} a la base
	$\{v_1,\dots,v_n\}$.

	\begin{proof}
        Como los $v_j$ son una base de $V$, para cada $i$ las funcionales
        lineales $f_i$ quedan bien definidas por $f_i(v_j)=\delta_{ij}$. Veamos
        que el conjunto $\{f_1,\dots,f_n\}$ es linealmente independientes: si
        $\sum_{i=1}^n f_i=0$ entonces para todo $v_j$ se tiene
		\[
			0=\left(\sum_{i=1}^n \alpha_if_i\right)(v_j)=\sum_{i=1}^n \alpha_if_i(v_j)=\alpha_j
		\]
		y luego $\alpha_j=0$ para todo $j$. Como $\dim V^*=n$, esto demuestra
		además que los $f_j$ forman una base de $V^*$.

		Veamos la unicidad: si las $g_i$ son funcionales lineales tales que
		$g_i(v_j)=f_i(v_j)$ para todo $i,j$ entonces 
		\[
		g_i(v)
		=g_i\left(\sum_{j=1}^n \alpha_jv_j\right)
		=\sum_{j=1}^n\alpha_jg_i(v_j)
		=\sum_{j=1}^n \alpha_jf_i(v_j)=f_i(v)
		\]
		y luego $g_i=f_i$ para todo $i$.
	\end{proof}
\end{prop}

\begin{examples}\
	\label{exa:base_dual}
	\begin{enumerate}
		\item La base dual de $\{(1,0),(0,1)\}\subseteq\R^2$ es $\{f_1,f_2\}$
			donde $f_1(x,y)=x$ y $f_2(x,y)=y$.
		\item La base dual de $\{(1,1),(1,-1)\}\subseteq\R^2$ es $\{f_1,f_2\}$
			donde $f_1(x,y)=\frac{x+y}{2}$ y $f_2(x,y)=\frac{x-y}{2}$.
	\end{enumerate}
\end{examples}

\begin{example}
	Sea $V=\K_n[X]$ y consideremos la base 
	\[
		\cB=\{1,X-a,(X-a)^2,\dots,(X-a)^n\}.
	\]
	La base dual a $\cB$ es $\{f_0,f_1,\dots,f_n\}$, donde
	\[
		f_k(p)=\frac{p^{(k)}(a)}{k!}
	\]
	para todo $k\in\{0,\dots,n\}$ y $p\in\K_n[X]$. 
\end{example}

\begin{prop}
	Sean $\cB=\{v_1,\dots,v_n\}$ una base de $V$ y $\cB^*=\{f_1,\dots,f_n\}$ su
	base dual. Si $v\in V$ y $f\in V^*$ entonces 
	\begin{align}
		(v)_{\cB}=(f_1(v),\dots,f_n(v)), && 
		(f)_{\cB^*}=(f(v_1),\dots,f_n(v)).
	\end{align}

	\begin{proof}
		Si $v=\sum_{i=1}^n\alpha_iv_i$ entonces
		$f_j(v)=\sum_{i=1}^n\alpha_if_j(v_i)=\alpha_j$ y luego $v=\sum_{i=1}^n
		f_i(v)v_i$. Por otro lado, si escribimos a $f$ como
		$f=\sum_{i=1}^n\alpha_if_i$ entonces $f(v_j)=\alpha_j$ y luego
		$f=\sum_{i=1}^n f(v_i)f_i$. 
	\end{proof}
\end{prop}

\begin{xca}
    \label{xca:v=0<=>f(v)=0}
    Sea $V$ un espacio vectorial de dimensión finita. Sean $f\in V^*$ y $v\in
    V$.  Pruebe que $v=0$ si y sólo si $f(v)=0$ para todo $f\in V^*$. 
\end{xca}

\begin{xca}
    \label{xca:V=<v>+kerf}
    Sea $V$ un espacio vectorial de dimensión finita.  Sean $f\in V^*$ y $v\in
    V$ con $f(v)\ne0$. Pruebe que $V=\langle v\rangle\oplus \ker f$.
\end{xca}

\begin{xca}
    \label{xca:kerf=kerg}
    Sea $V$ un espacio vectorial de dimensión finita. Si $f,g\in V^*$ pruebe
    que $\ker f=\ker g$ si y sólo si existe $\lambda\in\K\setminus\{0\}$ tal
    que $f=\lambda g$.
\end{xca}

%\begin{prop}
%    Si $V$ es un espacio vectorial de dimensión finita $n$ y
%    $\{f_1,\dots,f_n\}$ es base de $V^*$ entonces existe una única base
%    $\{v_1,\dots,v_n\}$ de $V$ tal que tiene a $\{f_1,\dots,f_n\}$ como su base
%    dual.
%
%	\begin{proof}
%		Demostremos la unicidad. Para eso, supongamos que existen bases
%		$\{v_1,\dots,v_n\}$ y $\{v_1',\dots,v_n'\}$ de $V$ tales que ambas
%		tienen a $\{f_1,\dots,f_n\}$ como base dual. Para cada $j$ escribimos
%		\[
%			v_j'=\sum_{k=1}^n a_{kj}v_k
%		\]
%		y entonces $\delta_{ij}=f_i(v_j')=\sum_{k=1}^n a_{kj}f_i(v_k)=a_{ij}$
%		para todo $i,j\in\{1,\dots,n\}$. Luego $v_j'=v_j$ para todo
%		$j\in\{1,\dots,n\}$.
%
%		Demostremos la existencia. Sea $\{w_1,\dots,w_n\}$ una base de $V$ y
%		sea $\{g_1,\dots,g_n\}$ su base dual.  La matriz $A=(f_{i}(w_j))$ es
%		inversible pues es la traspuesta de la matriz de cambio de base entre
%		$\{g_1,\dots,g_n\}$ y $\{f_1,\dots,f_n\}$: si $f_i = \sum_{k=1}^n
%		a_{ki}g_k$ entonces $f_i(w_j)=\sum_{k=1}^n a_{ki}g_k(w_j)=a_{ji}$. Sea
%		$B=(b_{ij})$ la inversa de $A$ y para cada $j\in\{1,\dots,n\}$ sea 
%		\[
%			v_j=\sum_{k=1}^n b_{kj}w_k.
%		\]
%		Entonces $f_i(v_j)=\delta_{ij}$ pues 
%		\[
%			f_i(v_j)
%			=f_i\left(\sum_{k=1}^n b_{kj}w_k\right)
%			=\sum_{k=1}^n f_{i}(w_k)b_{kj}
%			=\delta_{ij}.
%		\]
%		
%		Por último, para ver que $\{v_1,\dots,v_n\}$ es base, basta ver que es un
%		conjunto linealmente independiente. Si $\sum_{i=j}^n\alpha_j v_j=0$
%		entonces, al aplicar $f_i$, se tiene que 
%		$\alpha_i=\sum_{j=1}^n\alpha_jf_i(v_j)=0$ para todo $i$. 
%	\end{proof}
%\end{prop}

\begin{example}
	Sea $V=\R_2[X]$ y sean $\varphi_0,\varphi_1,\varphi_2\in V^*$ dadas por
	$\varphi_i(p)=p(i)$ para todo $p\in V$. Vamos a demostrar que
	$\{\varphi_0,\varphi_1,\varphi_2\}$ es base de $V^*$. Veamos primero que es
	un conjunto linealmente independiente. Si 
	\[
		\alpha_0\varphi_0+\alpha_1\varphi_1+\alpha_2\varphi_2=0, 
	\]
	al evaluar en el polinomio $(X-1)(X-2)$ obtenemos $\alpha_0=0$.
	Al evaluar en $X(X-1)$ obtenemos $\alpha_2=0$. Finalmente, al evaluar en $X(X-2)$
	obtenemos $\alpha_1=0$.  Luego $\{\varphi_0,\varphi_1,\varphi_2\}$ es una
	base de $V^*$ porque es un conjunto linealmente independiente y 
	$\dim V=\dim V^*=3$.
\end{example}

\section{Matriz de cambio de base}

\begin{prop}
	\label{pro:dual:cambio_de_base}
	Sean $\cB=\{v_1,\dots,v_n\}$ y $\cB'=\{v_1',\dots,v_n'\}$ bases de $V$. Entonces
	\[
		C(\cB^*,\cB'^*)=C(\cB',\cB)^T.
	\]

	\begin{proof}
		Supongamos que $C(\cB^*,\cB'^*)=(a_{ij})$ y que $C(\cB',\cB)=(b_{ij})$.
		Para cada $i,j$ escribimos
		\begin{align*}
			v_i'=\sum_{k=1}^n b_{ki}v_k,
			&&
			f_j=\sum_{k=1}^n a_{kj}f_k'.
		\end{align*}
		Luego, si calculamos $f_j(v_i')$ de dos formas, obtenemos 
		\begin{align*}
			b_{ji}=\sum_{k=1}^n b_{ki}\delta_{jk}
			=\sum_{k=1}^n b_{ki}f_j(v_k)
			=f_j(v_i')
			=\sum_{k=1}^n a_{kj}f_k'(v_i')
			=\sum_{k=1}^n a_{kj}\delta_{ki}=a_{ij},
		\end{align*}
		que es lo que queríamos demostrar.
	\end{proof}
\end{prop}

\begin{example}
	Vamos a utilizar la proposición~\ref{pro:dual:cambio_de_base} para calcular
	la base dual a la base $\cB=\{(1,1),(1,-1)\}$ de $\R^2$ vista en el
	ejemplo~\ref{exa:base_dual}. 
	Si $\{e_1,e_2\}$ es la base canónica de $\R^2$ entonces 
	\[
	C(\cB,\{e_1,e_2\})=
	\begin{pmatrix} 
		1 & 1\\
		1 & -1
	\end{pmatrix},
	\quad
	C(\{e_1,e_2\},\cB)=
		\frac12\begin{pmatrix}
			1 & 1\\
			1 & -1
		\end{pmatrix},
	\]
	Sea $\{f_1,f_2\}$ la base dual a $\{e_1,e_2\}$ y sea $\cB^*=\{g_1,g_2\}$ la
	base dual a $\cB$.  Por la proposición~\ref{pro:dual:cambio_de_base}
	sabemos que 
	\[
		\begin{pmatrix}
			1 & 1\\
			1 & -1
		\end{pmatrix}
		=C(\cB,\{e_1,e_2\})^T=C(\{f_1,f_2\},\cB^*).
	\]
	Entonces, como 
	\[
	C(\cB^*,\{f_1,f_2\})=
		\frac12\begin{pmatrix}
			1 & 1\\
			1 & -1
		\end{pmatrix},
	\]
	podemos calcular las coordenadas de las $g_i$ en la base $\{f_1,f_2\}$:
	\begin{align*}
		&\frac12\begin{pmatrix}
			1 & 1\\
			1 & -1
		\end{pmatrix}
		\colvec{2}{1}{0}=\frac12\colvec{2}{1}{1},
		&&
		\frac12\begin{pmatrix}
			1 & 1\\
			1 & -1
		\end{pmatrix}
		\colvec{2}{0}{1}=\frac12\colvec{2}{1}{-1}.
	\end{align*}
	Luego $g_1(x,y)=\frac{1}{2}(x+y)$ y $g_2(x,y)=\frac{1}{2}(x-y)$.
\end{example}

\section{El anulador de un subespacio}

\begin{block}
	Sean $V$ un espacio vectorial y $S\subseteq V$ un subespacio. Se define el
	\textbf{anulador} de $S$ en $V$ como el subespacio
	\[
	\ann S=\{f\in V^*:f(s)=0\text{ para todo $s\in S$}\}=\{f\in V^*: S\subseteq\ker f\}
	\]
    Observemos que el anulador $\ann S$ puede definirse si $S$ es un
    subconjunto no vacío de $V$. Por ejemplo, el anulador en $\R^2$ de
    $\{(1,1)\}$ es el subespacio de $\hom(\R^2,\R)$ generado por la función
    $(x,y)\mapsto x-y$.  De la definición es evidente que $\ann S$ es un
    subespacio de $V^*$ y que $\ann V=\{0\}$ y $\ann\{0\}=V^*$. 
\end{block}

\begin{xca}
    \label{xca:annX=<X>}
	Sea $V$ un espacio vectorial y $X$ un subconjunto de $V$. Pruebe que $\ann
	X=\ann\langle X\rangle$.
\end{xca}

\begin{example}
	Sea $S=\langle(1,1,1,1),(1,1,0,0),(1,0,1,0)\rangle\subseteq\R^4$. Entonces
	el anulador de $S$ está generado por la funcional lineal 
	\[
		(x_1,x_2,x_3,x_4)\mapsto -x_1+x_2+x_3-x_4.
	\]
\end{example}

\begin{lem}
	\label{lem:dual:fundamental}
	Sean $V$ un espacio vectorial de dimensión finita, $S\subseteq V$ un
	subespacio y $\{v_1,\dots,v_k\}$ una base de $S$. Supongamos que
	$\{v_1,\dots,v_k,v_{k+1},\dots,v_n\}$ es una base de $V$ y sea
	$\{f_1,\dots,f_n\}$ su base dual. Entonces $\{f_{k+1},\dots,f_n\}$ es una
	base de $\ann S$.

	\begin{proof}
		Sea $f\in\ann S$ y escribamos a $f$ en la base $\{f_1,\dots,f_n\}$.
		Entonces, como $f(v_j)=0$ para todo $j\in\{1,\dots,k\}$, 
		\[
		f=\sum_{i=1}^n\alpha_if_i=\sum_{i=1}^n f(v_i)f_i=\sum_{i=k+1}^n\alpha_if_i.
		\]
		Luego $\{f_{k+1},\dots,f_n\}$ es un conjunto de generadores de $\ann S$. Como es
		un conjunto linealmente independiente, es también una base de $\ann S$.
	\end{proof}
\end{lem}

\begin{prop}
	Sean $V$ un espacio vectorial de dimensión finita y $f\in V^*$. Entonces
	$\ann\ker f=\langle f\rangle$.

	\begin{proof}
        Vamos a demostrar que $\ann\ker f\subseteq\langle f\rangle$ ya que la
        otra inclusión es trivial.  Como el resultado es trivialmente válido para $f=0$ vamos a suponer que $f\ne0$. 
        Si $g\in\ann\ker f$ entonces $\ker f\subseteq \ker g$. Si $g=0$
        el resultado es trivial. Si $g\ne 0$ entonces $\ker f=\ker g$ pues
        ambos tienen la misma dimensión. Luego $g\in\langle f\rangle$ por el
        ejercicio~\ref{xca:v=0<=>f(v)=0}.
%		
%        Si $f=0$ el resultado es trivialmente
%		válido. Supongamos entonces que $f\ne0$.  Sea $g\in\ann \ker f$.
%		Entonces $\dim\ker f=n-1$. Sea $\{v_1,\dots,v_{n-1}\}$ una base de
%		$\ker f$ y sea $v_n\in V$ tal que $f(v_n)\ne0$.  El conjunto
%		$\{v_1,\dots,v_{n-1},v_n\}$ es base de $V$ y
%		$\{f_1,\dots,f_{n-1},f_n\}$ es su base dual. Escribimos
%		\[
%			g=\sum_{i=1}^n g(v_i)f_i\in\ann\ker f.
%		\]
%		Como $g\in\ann \ker f$ entonces $g(v)=0$ para todo $v\in\ker f$ y luego
%		$g(v_j)=0$ para todo $j\in\{1,\dots,n-1\}$. Por lo tanto $g=g(v_n)f_n$.
%		Como $f=\sum_{i=1}^n f(v_i)f_i$ entonces $f=f(v_n)f_n$, donde
%		$f(v_n)\ne0$. Luego $g\in\langle f_n\rangle=\langle f\rangle$.
	\end{proof}
\end{prop}

\begin{thm}[teorema de la dimensión del anulador]
	\label{thm:dimension_anulador}
	Si $V$ es un espacio vectorial de dimensión $n$ y $S\subseteq V$ es un
	subespacio entonces
	\[
		\dim\ann S=n-\dim S.
	\]
	\begin{proof}
		Supongamos que $\dim S=k$ y sea $\{v_1,\dots,v_k\}$ una base de $S$.
		Sean $\{v_1,\dots,v_k,v_{k+1},\dots,v_n\}$ una base de $V$ y
		$\{f_1,\dots,f_n\}$ su base dual. El lema~\ref{lem:dual:fundamental}
		nos dice que $\{f_{k+1},\dots,f_n\}$ es base de $\ann S$. Luego
		$\dim\ann S=n-k$ y el teorema queda demostrado.
	\end{proof}
\end{thm}

\begin{example}
	Consideremos los siguientes vectores de $\R^5$:
	\begin{align*}
		&v_1=(2,-2,3,4,-1), 
		&& v_2=(-1,1,2,5,2),\\
		& v_3=(0,0,-1,-2,3),
		&& v_4=(1,-1,2,3,0),
	\end{align*}
	 y sea $S=\langle v_1,v_2,v_3,v_4\rangle$. 
	 Vamos a calcular $\ann S$.  Como $\dim S=3$, por el teorema de la
	 dimensión del anulador sabemos que $\dim\ann S=2$. Si $f\in(\R^5)^*$ entonces
	 \[
	 	f(x_1,\dots,x_5)=\sum_{i=1}^5\alpha_ix_i. 
  	 \]
	 Luego $f\in\ann S$ si y sólo si $f(v_i)=0$ para todo $i$, es decir si y sólo si
	 \[
		\begin{cases}
			\begin{aligned}
			2\alpha_1-2\alpha_2+3\alpha_3+4\alpha_4-\alpha_5 &=0,\\
			-\alpha_1+\alpha_2+2\alpha_3+5\alpha_4+2\alpha_5 &=0,\\
			-\alpha_3-2\alpha_4+3\alpha_5 &=0,\\
			\alpha_1-\alpha_2+2\alpha_3+3\alpha_4 &=0.
			\end{aligned}
		\end{cases}
	 \]
	 Como este sistema es equivalente a
	 \[
		\begin{cases}
			\begin{aligned}
				\alpha_1-\alpha_2-\alpha_4 &=0, \\
				\alpha_3+2\alpha_4 &= 0,\\
				\alpha_5 &=0,
			\end{aligned}
		\end{cases}
	 \]
	 se deduce que
	 $f(x_1,\dots,x_5)=(\alpha_2+\alpha_4)x_1+\alpha_2x_2-2\alpha_4x_3+\alpha_4x_4$.
	 Luego una base de $\ann S$ es $\{f_1,f_2\}$, donde 
	 \[
	 	f_1(x_1,\dots,x_5)=x_1+x_2,\quad 
		f_2(x_1,\dots,x_5)=x_1-2x_3+x_4. 
	 \]
\end{example}

\begin{cor}
	Sea $V$ un espacio vectorial de dimensión finita $n$ y sea $S\subseteq V$
	un subespacio de dimensión $m<n$.  Entonces existen \textbf{hiperplanos}
	(es decir: subespacios de dimensión $n-1$) $H_1,\dots,H_{n-m}\subseteq V$
	tales que $S=H_1\cap\cdots\cap H_{n-m}$. 

	\begin{proof}
		Sea $\{v_1,\dots,v_m\}$ una base de $S$. Extendemos esta base de $S$ a
		una base $\{v_1,\dots,v_m,v_{m+1},\dots,v_n\}$ de $V$ y sea
		$\{f_1,\dots,f_m,f_{m+1},\dots,f_n\}$ su base dual.  Para cada
		$i\in\{1,\dots,n-m\}$ sea $H_i=\ker f_{m+i}$. Si $v\in V$ entonces
		$v=\sum_{i=1}^n f_i(v)v_i$. Tenemos entonces que $v\in S$ si y sólo si
		$v\in\cap_{i=1}^{n-m}\ker f_i$ y por lo tanto $S=\cap_{i=1}^{n-m}H_i$.
	\end{proof}
\end{cor}

\begin{cor}
	\label{cor:annS=annT}
    Sea $V$ un espacio vectorial de dimensión finita y sean $S$ y $T$ dos
    subespacios de $V$. Entonces $S=T$ si y sólo si $\ann S=\ann T$.

	\begin{proof}
		Si $S=T$ entonces trivialmente $\ann S=\ann T$. Recíprocamente,
		supongamos que $S\ne T$. Sin pérdida de generalidad podemos suponer
		entonces que existe $v\in T\setminus S$. Vamos a demostrar que existe
		$f\in V^*$ tal que $f(s)=0$ para todo $s\in S$ y $f(v)\ne0$, lo que
		significa que $\ann S\ne\ann T$ pues $f\not\in\ann T$ y $f\in\ann S$.
		En efecto, sea $\{v_1,\dots,v_k\}$ una base de $S$. Como $v\not\in S$,
		el conjunto $\{v,v_1,\dots,v_k\}$ es linealmente independiente y
		entonces puede extenderse a una base $\{v,v_1,\dots,v_n\}$ de $V$. Sea
		$\{f,f_1,\dots,f_n\}$ la base dual a $\{v,v_1,\dots,v_n\}$. Luego
		$f(v)=1$ y $f(v_j)=0$ para todo $j$. En particular $f(s)=0$ para todo
		$s\in S$. 
%		Como una de las implicaciones es trivial, basta con 
%		demostrar lo siguiente: si $\ann T\subseteq \ann S$ entonces $S\subseteq
%		T$. Sea $\{f_1,\dots,f_k\}$ una base de $\ann T$. Sabemos que esta base
%		puede extenderse a una base $\{f_1,\dots,f_k,f_{k+1},\dots,f_l\}$ de $\ann
%		S$ y esta última, a su vez, a una base $\{f_1,\dots,f_n\}$ de $V^*$. Sea
%		$\{v_1,\dots,v_n\}$ una base de $V$ cuya base dual es $\{f_1,\dots,f_n\}$.
%		Afirmamos que $\{v_{k+1},\dots,v_n\}$ es base de $T$, y que
%		$\{v_{l+1},\dots,v_n\}$ es base de $S$.  En efecto, si $v\in T$ entonces
%		\[
%			v=\sum_{i=1}^n \alpha_iv_i=\sum_{i=1}^n f_i(v)v_i=\sum_{i=k+1}^n f_i(v)v_i.
%		\]
%		Luego, como los $v_i$ son linealmente independientes,
%		$\{v_{k+1},\dots,v_n\}$ es una base de $T$.  Similarmente, si $v\in S$
%		entonces 
%		\[
%			v=\sum_{i=1}^n \alpha_iv_i=\sum_{i=1}^n f_i(v)v_i=\sum_{i=l+1}^n f_i(v)v_i,
%		\]
%		y así se obtiene que $\{v_{l+1},\dots,v_n\}$ es una base de $S$. Como, por
%		construcción, $\{v_{l+1},\dots,v_n\}\subseteq\{v_{k+1},\dots,v_n\}$
%		entonces $T\subseteq S$.
	\end{proof}
\end{cor}

\begin{block}
	Observemos que en el corolario~\ref{cor:annS=annT} es necesario asumir que
	$S$ y $T$ son subespacios de $V$. En efecto, si $V=\R$, $S=\{1\}$ y $T=\R$
	entonces $S\subsetneq T$ pero $\ann S=\ann T=\{0\}$. 
\end{block}

\begin{cor}
	\label{cor:anuladores}
	Sean $V$ un espacio vectorial de dimensión finita y $S\subseteq V$ un
	subespacio. Entonces
	\[
		\{v\in V:f(v)=0\text{ para todo $f\in\ann S$}\}=S.
	\]

	\begin{proof}
		Si $v\in S$ entonces trivialmente $f(v)=0$ para todo $f\in \ann S$.
		Recíprocamente, sea $v\in V$ tal que $f(v)=0$ para todo $f\in\ann S$.  Sea
		$\{v_1,\dots,v_k\}$ una base de $S$. Si $v\not\in S$ entonces
		$\{v,v_1,\dots,v_k,\}$ es linealmente independiente. Extendamos este
		conjunto a una base $\{v,v_1,\dots,v_n\}$ de $V$ y sea
		$\{f,f_1,\dots,f_n\}$ su base dual. Por construcción,
		$f(v_1)=\cdots=f(v_n)=0$ y entonces $S\subseteq\ker(
		f)$, es decir: $f\in\ann S$. Sin embargo, por construcción,
		$f(v)=1$, una contradicción.
	\end{proof}
\end{cor}

\begin{cor}
	\label{cor:base_annS}
	Sean $V$ un espacio vectorial de dimensión finita, $S\subseteq V$ un
	subespacio, y $\{f_1,\dots,f_k\}$ una base de $\ann S$. Entonces
	\[
		S=\bigcap_{i=1}^k \ker(f_i).
	\]

	\begin{proof}
		Si $\{f_1,\dots,f_k\}$ es una base de $\ann S$ entonces, por el
		corolario~\ref{cor:anuladores},
		\begin{align*}
			S &=\{v\in V: f(v)=0\text{ para todo $f\in\ann S$}\}\\
			&=\{v\in V: f_i(v)=0\text{ para todo $i\in\{1,\dots,k\}$}\}
			=\bigcap_{i=1}^k\ker f_i,
		\end{align*}
		tal como queríamos demostrar.
	\end{proof}
\end{cor}

\begin{prop}
    \label{pro:ann(S+T)}
	Sean $S$ y $T$ dos subespacios de un espacio vectorial $V$. Entonces
	\begin{equation}
		\label{eq:ann(S+T)}
		\ann (S+T)=\ann S\cap \ann T.
	\end{equation}

    \begin{proof}
        Si $f\in\ann S\cap\ann T$ entonces $S\subset\ker f$ y
        $T\subseteq\ker f$ y luego $S+T\subseteq\ker f$, es decir: $f\in\ann(S+T)$.
        Recíprocamente, si $S+T\subseteq\ker f$ entonces $S$ y $T$ están contenidos
        en $\ker f$, es decir: $f\in\ann S\cap\ann T$. 
    \end{proof}
\end{prop}

\begin{cor}
    Si $S_1,\dots,S_k$ son subespacios de $V$ entonces 
	\begin{equation}
		\label{eq:ann(S1+...+S_k)}
		\ann (S_1+\cdots+S_k)=\ann(S_1)\cap\cdots\cap\ann(S_k).
	\end{equation}

    \begin{proof}
        Es consecuencia de la proposición~\ref{pro:ann(S+T)} y del principio de
        inducción.
    \end{proof}
\end{cor}

\begin{prop}
	\label{pro:ann(ScapT)=annS+annT}
	Sea $V$ un espacio vectorial de dimensión finita y sean $S$ y $T$ subespacios
	de $V$.  Entonces:
	\[
	\ann (S\cap T)=\ann S+\ann T.
	\]

	\begin{proof}
		Sea $f\in\ann S +\ann T$ y escribamos $f=g+h$, donde $S\subseteq\ker g$
		y $T\subseteq\ker h$. Si $v\in S\cap T$ entonces $v\in\ker f$ pues
		$f(v)=g(v)+h(v)=0$, es decir: $f\in\ann(S\cap T)$. Para ver que $\ann
		(S\cap T)=\ann S+\ann T$ calculemos la dimensión de $\ann S+\ann T$ con
		la proposición~\ref{pro:ann(S+T)}:
		\begin{align*}
			\dim(\ann S&+\ann T) = \dim\ann S+\dim\ann T-\dim(\ann S\cap\ann T)\\
			&=n-\dim S+n-\dim T-\dim\ann(S+T)\\
			&=n-\dim S+n-\dim T-(n-\dim(S+T))\\
			&=n-\dim S-\dim T+\dim(S+T)\\
			&=n-\dim(S\cap T)\\
			&=\dim\ann(S\cap T),
		\end{align*}
		y entonces la proposición queda demostrada.
	\end{proof}
\end{prop}

\begin{cor}
	Sea $V$ un espacio vectorial de dimensión finita y sean $S_1,\dots,S_k$
	subespacios de $V$.  Entonces:
	\[
	\ann (S_1\cap\cdots\cap S_k)=\ann S_1+\cdots+\ann S_k.
	\]

    \begin{proof}
        Es consecuencia de la proposición~\ref{pro:ann(ScapT)=annS+annT} y del
        principio de inducción.
    \end{proof}
\end{cor}

\begin{cor}
    \label{cor:basis_of_V*}
	Sea $V$ un espacio vectorial de dimensión finita $n$ y sea
	$\{f_1,\dots,f_n\}$ un subconjunto de $V^*$. Entonces $\{f_1,\dots,f_n\}$
	es base de $V^*$ si y sólo si  
	\begin{equation}
		\label{eq:capker(f_i)}
		\bigcap_{i=1}^n\ker f_i=\{0\}.
	\end{equation}

	\begin{proof}
		Si $\{f_1,\dots,f_n\}$ es base, entonces el resultado se sigue del
		corolario~\ref{cor:base_annS} con $S=\{0\}$. 
%		\begin{align*}
%			V^*&=\langle f_1,\dots,f_n\rangle 
%			=\langle f_1\rangle+\cdots+\langle f_n\rangle
%			=\sum_{i=1}^n\ann\ker(f_i)
%			=\ann\left(\bigcap_{i=1}^n\ker f_i\right).
%		\end{align*} 
%		Como $V^*=\ann\{0\}$, se concluye~\eqref{eq:capker(f_i)}. 
		Recíprocamente, si se asume~\eqref{eq:capker(f_i)} entonces 
		\begin{align*}
			V^* &= \ann\left(\bigcap_{i=1}^n\ker(f_i)\right)\\
			&=\sum_{i=1}^n \ann\left(\ker(f_i)\right)=\langle f_1\rangle+\cdots+\langle f_n\rangle=\langle f_1,\dots,f_n\rangle.
		\end{align*}
		Como $\dim V^*=n$, entonces $\{f_1,\dots,f_n\}$ es una base de $V^*$.
	\end{proof}
\end{cor}

\begin{example}
	Vamos a usar el corolario~\ref{cor:anuladores} para calcular las ecuaciones
	de un subespacio. Sea $S=\langle(1,1,1),(1,2,1)\rangle\subseteq\R^3$. Como
	$\dim S=2$ e teorema de la dimensión del anulador nos dice que $\dim\ann
	S=1$. Sea $f\in\ann S$ y supongamos que $f(x,y,z)=\alpha x+\beta y+\gamma z$. Entonces
	$f$ puede escribirse como $f(x,y,z)=\alpha(x-z)$ y luego $\ann S$ está
	generado por la funcional $(x,y,z)\mapsto x-z$. El corolario~\ref{cor:anuladores} nos dice
	que 
	\[
		S=\{v\in V:f(v)=0\text{ para todo $f\in\ann S$}\}=\{(x,y,z)\in\R^3:x-z=0\}.
	\]
\end{example}

\section{El doble dual}

\begin{block}
    Como $V^*$ es un espacio vectorial es posible considerar el \textbf{doble
    dual} $\left(V^*\right)^*=V^{**}$. Si $V$ es de dimensión finita entonces
    \[
        \dim V=\dim V^*=\dim V^{**}.
    \]
\end{block}

\begin{block}
	Todo $v\in V$ induce una funcional lineal en $V^*$. En efecto, la función
	$L_v\colon V^*\to\K$ dada por $f\mapsto \langle L_v|f\rangle=f(v)$, es
	lineal pues  
	\[
	\langle L_v|f+\lambda g\rangle=(f+\lambda g)(v)=f(v)+(\lambda g)(v)=f(v)+\lambda g(v)=\langle L_v|f\rangle+\lambda\langle L_v|f\rangle
    \]
    para todo $f,g\in V^*$ y $\lambda\in\K$.
\end{block}

\begin{thm}
    \label{thm:doble_dual}
    Sea $V$ un espacio vectorial de dimensión finita. Entonces $L\colon V\to
    V^{**}$, $v\mapsto L_v$, es un isomorfismo.

    \begin{proof}
		Veamos que $L$ es lineal: si $v,w\in V$, $\lambda\in\K$ y $\varphi\in
		V^*$ entonces
        \begin{align*}
			\langle L_{v+\lambda w}|f\rangle=f(v+\lambda w)=f(v)+\lambda f(w)=\langle L_v|f\rangle+\lambda\langle L_w|f\rangle.
        \end{align*}
        Veamos que $L$ es monomorfismo: si $v\in V$ tal que $L_v=0$ entonces
		$L_v(\varphi)=\varphi(v)=0$ para todo $\varphi\in V^*$. Luego $v=0$ por
		el ejercicio~\ref{xca:v=0<=>f(v)=0}.
        
		Como $L$ es monomorfismo y $\dim V^{**}=\dim V$ por ser $V$ de
		dimensión finita, $L$ es un isomorfismo por el
		corolario~\ref{cor:mono<=>epi<=>iso}. 
    \end{proof}
\end{thm}

\begin{cor}[teorema de representación]
    \label{cor:representacion}
    Sea $V$ un espacio vectorial de dimensión finita sobre $\K$. Si
    $T\in\hom(V^*,\K)$ entonces existe un único vector $v\in V$ tal que
    $T(f)=f(v)$ para todo $f\in V^*$
    
    \begin{proof}
		Como $T\in V^{**}$ y $L$ es biyectiva por el
		teorema~\ref{thm:doble_dual}, existe un único $v\in V$ tal que $T=L_v$.          
    \end{proof}
\end{cor}

\begin{cor}
    Sea $V$ un espacio vectorial de dimensión finita sobre $\K$. Entonces toda
    base de $V^*$ es dual de una única base de $V$.

    \begin{proof}
        Demostremos la existencia. Sea $\{f_1,\dots,f_n\}$ una base de $V^*$ y
        sea $\{T_1,\dots,T_n\}\subseteq V^{**}$ su base dual. Por el corolario
        anterior,~\ref{cor:representacion}, existen $v_1,\dots,v_n\in V$ tales
		que $T_i(f)=f(v_i)$ para todo $f\in V^*$. Luego, como $L$ es un
		isomorfismo, $\{v_1,\dots,v_n\}$ es base de $V$ y $\{f_1,\dots,f_n\}$
		es su base dual pues
		\[
			f_j(v_i)=\langle L_{v_i}|f_j\rangle=\langle T_i|f_j\rangle=\delta_{ij}.
		\]

		Demostremos ahora la unicidad. Supongamos que existen bases
		$\{v_1,\dots,v_n\}$ y $\{v_1',\dots,v_n'\}$ de $V$ tales que ambas
		tienen a $\{f_1,\dots,f_n\}$ como base dual. Para cada $j$ escribimos
		\[
			v_j'=\sum_{k=1}^n a_{kj}v_k
		\]
		y entonces $\delta_{ij}=f_i(v_j')=\sum_{k=1}^n a_{kj}f_i(v_k)=a_{ij}$
		para todo $i,j\in\{1,\dots,n\}$. Luego $v_j'=v_j$ para todo
		$j\in\{1,\dots,n\}$.
    \end{proof}
\end{cor}

%\begin{example}
%    \framebox{FIXME}
%	Sea $V=\K^{\infty}$ y sea $\{e_1,e_2,\dots\}$ la base canónica.
%	Supongamos que $\{\varphi_1,\varphi_2,\dots\}$ es su base dual, es decir:
%	$\varphi_i(e_j)=\delta_{ij}$. Sea $v=(v_1,v_2,\dots)\in V$ con finitas coordenadas no nulas, digamos 
%	$v_j=0$ para todo $j>N$. Entonces
%	$L_v(\varphi_k)=\varphi_k(v)=0$ si $k>N$. Sin embargo existe $f\in V^{**}$ tal que
%	$f(\varphi_k)\ne1$. 
%\end{example}


\begin{xca}
	\label{xca:dual:LI}
	Sea $V$ un espacio vectorial de dimensión $n$ y sea
	$\{f_1,\dots,f_n\}$ un subconjunto de $V^*$. Pruebe que si existe $v\in
	V\setminus\{0\}$ tal que $f_i(v)=0$ para todo $i\in\{1,\dots,n\}$
	entonces el conjunto $\{f_1,\dots,f_n\}$ es linealmente dependiente. 
\end{xca}



\section{La traspuesta de una transformación lineal}

\begin{block}
    Sean $V$ y $W$ dos espacios vectoriales sobre $\K$ y sea $f\in\hom(V,W)$.
    Se define la \textbf{traspuesta} de la transformación $f$ como la función
    $f^T\colon W^* \to V^*$ dada por 
    $(f^T\varphi)(v)=\varphi(f(v))$, o equivalentemente 
    \[
        \langle f^T\varphi|v\rangle=\langle\varphi|f(v)\rangle,
    \]
    para todo $\varphi\in W^*$ y $v\in V$. Veamos que
    $f^T\in\hom(W^*,V^*)$: si $\varphi,\psi\in W^*$, $\lambda\in\K$ y $v\in V$ entonces
	\begin{align*}
        \langle f^T(\varphi+\lambda\psi)|v\rangle&=
        \langle \varphi+\lambda\psi|f(v)\rangle=
        \left(\varphi+\lambda\psi\right)(f(v))\\
        &=\varphi(f(v))+\lambda\psi(f(v))
		=\langle f^T(\varphi)|v\rangle+\langle f^T(\psi)|v\rangle.
	\end{align*}
\end{block}

\begin{prop}
    Sean $V$ y $W$ espacios vectoriales sobre $\K$ y supongamos que $\dim V=n$
    y $\dim W=m$. Sea $f\in\hom(V,W)$. Entonces $L_W\circ f=(f^T)^T\circ L_V$,
    es decir: el siguiente diagrama es conmutativo:
    \[
    \xymatrix{
    V
    \ar[d]_{L_V}
    \ar[r]^-{f}
    & W
    \ar[d]^{L_W}
    \\
    V^{**}
    \ar[r]^-{(f^T)^T}
    & W^{**}
    }
    \]

    \begin{proof}
        Si $v\in V$ y $\varphi\in W^*$ entonces 
        \[
            \langle (f^T)^TL_v|\varphi\rangle=\langle L_v|f^T\varphi\rangle=\langle L_v|\varphi\circ f\rangle
			=(\varphi\circ f)(v)=\varphi(f(v))=\langle L_{f(v)}|\varphi\rangle,
        \]
        tal como queríamos demostrar.
    \end{proof}
\end{prop}

\begin{prop}
    \label{pro:ker(fT)=ann(imf)}
    Sean $V$ y $W$ dos espacios vectoriales. Sea 
    $f\in\hom(V,W)$. Entonces
    \[
        \ker\left(f^T\right)=\ann\im f.
    \]

    \begin{proof}
        Tenemos:
        \begin{align*}
            \varphi\in\ker\left(f^T\right) & \Leftrightarrow \varphi\circ f=f^T\circ\varphi=0\\
            & \Leftrightarrow \varphi(f(v))=0\text{ para todo $v\in V$}\\
            & \Leftrightarrow \varphi(w)=0\text{ para todo $w\in\im f$}.
        \end{align*}
        Esto demuestra la proposición.
    \end{proof}
\end{prop}

\begin{prop}
    \label{pro:dimimf=dimimfT}
    Sean $V$ y $W$ espacios vectoriales sobre $\K$ de dimensión finita y sea
    $f\in\hom(V,W)$. Entonces
    \[
        \dim\im f=\dim\im\left(f^T\right).
    \]

    \begin{proof}
        Supongamos que $\dim W^*=m$. Entonces, por el teorema de la dimensión y
        el teorema de la dimensión del anulador:
        \[
            m=\dim\ker\left(f^T\right)+\dim\im\left(f^T\right)
            =\dim\ann\im f+\dim\im f.
        \]
        Por la proposición~\ref{pro:ker(fT)=ann(imf)} sabemos que
        $\dim\ker\left(f^T\right)=\dim\ann\im f$. Entonces 
        \[
            \dim\im\left(f^T\right)=\dim\im f,
        \]
        que es lo que queríamos demostrar.
    \end{proof}
\end{prop}

\begin{prop}
    \label{pro:imf^T=annkerf}
    Sean $V$ y $W$ espacios vectoriales sobre $\K$ de dimensión finita y sea
    $f\in\hom(V,W)$. Entonces
    \[
        \im\left(f^T\right)=\ann\ker f.
    \]

    \begin{proof}
        Demostremos que $\im\left(f^T\right)\subseteq\ann\ker f$.  Sea
        $\varphi\in\im\left(f^T\right)$. Para ver que $\ker
        f\subseteq\ker\varphi$ escribamos $f^T\psi=\varphi$ con $\psi\in W^*$.
        Entonces, si $v\in\ker f$, $\varphi(v)=(\psi\circ f)(v)=0$ y luego
        $\varphi\in\ann\ker f$.

        Por otro lado, como
        \[
            \dim V=\dim\ker f+\dim\ann\ker f=\dim\ker f+\dim\im f,
        \]
        entonces $\dim\ann\ker f=\dim\im f$. Además 
        $\dim\im\left(f^T\right)=\dim\im f$ por la proposición~\ref{pro:dimimf=dimimfT}. Luego 
        \[
            \dim\ann\ker f=\dim\im\left(f^T\right),
        \]
        que implica lo que queríamos demostrar.
    \end{proof}
\end{prop}

\begin{prop}
    \label{pro:|f^T|=|f|^T}
    Sean $V$ y $W$ espacios vectoriales sobre $\K$ de dimensión finita y sea
    $f\in\hom(V,W)$. Si $\cB_V=\{v_1,\dots,v_n\}$ es una base ordenada de $V$ y
    $\cB_V^*=\{f_1,\dots,f_n\}$ es su base dual, $\cB_W=\{w_1,\dots,w_m\}$ es
    una base ordenada de $W$ y $\cB_W^*=\{g_1,\dots,g_m\}$  es su base dual,
    entonces 
    \[
        \left\|f^T\right\|_{\cB_W^*,\cB_V^*}=\|f\|_{\cB_V,\cB_W}^T.           
    \]

    \begin{proof}
        Supongamos que 
        \begin{align*}
            \|f\|_{\cB_V,\cB_W}=(a_{ij})\in\K^{m\times n},
            &&
            \left\|f^T\right\|_{\cB_W^*,\cB_V^*}=(b_{ij})\in\K^{n\times m}. 
        \end{align*}
        Entonces $f(v_j)=\sum_{i=1}^m a_{ij}w_i$ para todo $j\in\{1,\dots,n\}$ y
        $\left(f^T\right)(g_j)=\sum_{i=1}^n b_{ij}f_i$ para todo $j\in\{1,\dots,m\}$. 
        Para cada $k\in\{1,\dots,n\}$ por un lado tenemos
        \[
        f^T(g_j)(v_k)=\sum_{i=1}^n b_{ij}f_i(v_k)=\sum_{i=1}^n b_{ij}\delta_{ik}=b_{kj}
        \]  
        y por otro lado
        \[
        f^T(g_j)(v_k)=g_j(f(v_k))=g_j\left(\sum_{i=1}^m a_{ik}w_i\right)=\sum_{i=1}^m a_{ik}g_j(w_i)=\sum_{i=1}^m a_{ik}\delta_{ji}=a_{jk}.
        \]
        Luego $a_{ij}=b_{ji}$ para todo $i,j$.
    \end{proof}
\end{prop}

\begin{prop}
    \label{pro:f_iso<=>f^T_iso}
    Sean $V$ y $W$ espacios vectoriales sobre $\K$ de dimensión finita y
    supongamos que $\dim V=\dim W$. Sea $f\in\hom(V,W)$. Entonces $f$ es un
    isomorfismo si y sólo si $f^T$ es un isomorfismo.

    \begin{proof}
        Sea $\cB_V$ una base ordenada de $V$ y $\cB_V^*$ su base dual. Sea
        $\cB_W$ una base ordenada de $W$ y $\cB_W^*$ su base dual. Entonces 
        \begin{align*}
            f\text{ es isomorfismo} &\Leftrightarrow \|f\|_{\cB_V,\cB_W}\text{ es inversible}
            \Leftrightarrow \|f\|_{\cB_V,\cB_W}^T\text{ es inversible}. 
        \end{align*}
        Como $\left\|f^T\right\|_{\cB_W^*,\cB_V^*}=\|f\|_{\cB_V,\cB_W}^T$ por
        la proposición~\ref{pro:|f^T|=|f|^T}, concluimos que $f$ es isomorfismo
        si y sólo si $f^T$ es un isomorfismo.
    \end{proof}
\end{prop}

\section{Una aplicación al rango de matrices}

\begin{block}
    Vamos a demostrar que el rango fila y el rango columna son iguales. 
    
    Sea
    \[
        f\colon\K^{n\times1}\to\K^{m\times1},\quad x\mapsto Ax
    \]
    Con respecto a las bases canónicas de $\K^{n\times1}$ y $\K^{m\times1}$ la
    matriz de $f$ es $\|f\|=A$. En la proposición~\ref{pro:dimimf=dimimfT}
    vimos que $\dim\im f=\dim\im f^T$. Además $\|f\|^T=\|f^T\|$ por la
    proposición~\ref{pro:|f^T|=|f|^T}. Entonces
    \begin{align*}
        \rg_C(A)&=\dim\im f=\dim\im f^T\\
        &=\rg_C\left(\left\|f^T\right\|\right)
        =\rg_C\left(\|f\|^T\right)=\rg_F(\|f\|)=\rg_F(A).
    \end{align*}
\end{block}

\section{Una aplicación a los cuadrados mágicos}

\begin{block}
	\label{block:magic_squares}
    Vamos a utilizar el teorema de la dimensión del anulador para estudiar
    cuadrados mágicos. Un \textbf{cuadrado mágico} es, por definición, una
    matriz racional $A$ de $n\times n$ tal que cada una de las sumas de sus filas
    es igual a la traza $\tr(A)$ de $A$, cada una de la suma de columnas es
    igual a $\tr(A)$, y la suma de su antidiagonal es igual a $\tr(A)$, es
    decir: $A$ es un cuadrado mágico si y sólo si para todo $k$ se tiene que
    \begin{align*}
        \sum_{i=1}^n a_{ik}=\tr(A), &&
        \sum_{j=1}^n a_{kj}=\tr(A), &&
        \sum_{i+j=n+1}^n a_{ij}=\tr(A).
    \end{align*}

    Sea $M(n)$ el conjunto de cuadrados mágicos. 

	Por ejemplo, la matrices
    \begin{align}
        \label{eq:magic}
		\begin{pmatrix}
			1 & 1 & 1\\
			1 & 1 & 1\\
			1 & 1 & 1  
		\end{pmatrix},
         &&
		\begin{pmatrix}
			8 & 3 & 4\\
			1 & 5 & 9\\
			6 & 7 & 2  
		\end{pmatrix},
        &&
		\begin{pmatrix}
			4 & 9 & 2\\
			3 & 5 & 7\\
			8 & 1 & 6  
		\end{pmatrix},
    \end{align}
	son cuadrados mágicos de $3\times 3$. 
    
    \begin{xca*}
        Demuestre que el conjunto $M(n)$ de cuadrados mágicos de $n\times n$ es
        un subespacio vectorial de $\Q^{n\times n}$. 
    \end{xca*}

    Ahora que sabemos que $M(n)$ es un subespacio vectorial, calculemos su
    dimensión. Por el teorema de la dimensión del anulador,
    teorema~\ref{thm:dimension_anulador}, sabemos que
	\[
		\dim M(n)+\dim\ann M(n)=n^2.
	\]

    Para cada $i,j\in\{1,\dots,n\}$ consideremos las funcionales lineales
    $c_i\colon\Q^{n\times n}\to\Q$, $r_i\colon\Q^{n\times n}\to\Q$ y
    $\mathrm{antitr}\colon\Q^{n\times n}\to\Q$ dadas por
    \[
    c_j(A)=\sum_{k=1}^n a_{kj},\quad
    r_i(A)=\sum_{k=1}^n a_{ik},\quad
    \mathrm{antitr}(A)=\sum_{i+j=n+1}a_{ij}.
    \]

	El conjunto
	$\{r_1-\tr,\dots,r_n-\tr,c_1-\tr,\dots,c_{n-1}-\tr,\mathrm{antitr}-\tr\}$
	es un conjunto de generadores para $\ann M(n)$. Puede demostrarse además
	que es un conjunto linealmente independiente. Luego, por el teorema de la
	dimensión del anulador, $\dim M(n)+2n=n^2$ y entonces
    \[
        \dim M(n)=n(n-2). 
    \]

    Demostremos la independencia lineal en el caso $n=3$. Sean los escalares
	$\alpha_1,\alpha_2,\alpha_3\,\beta_1,\beta_2\in\Q$ tales que
	\[
        \sum_{i=1}^3 \alpha_i(r_i-\tr)+\sum_{j=1}^{2}\beta_j(c_j-\tr)+\gamma(\mathrm{antitr}-\tr)=0.
	\]
	
	Veamos que $\alpha_1=\alpha_2=\alpha_3=\beta_1=\beta_2=\gamma=0$. Al
	evaluar esta expresión en la matriz $E_{13}+E_{22}+E_{31}$ se obtiene que
	$\gamma=0$.  Después, al evaluar en la matrices canónicas $E_{13}$ y
	$E_{23}$ se obtiene $\alpha_1=\alpha_2=0$.  Evaluar en $E_{21}$ y $E_{12}$
	nos da $\beta_1=\beta_2=0$. Por último, al evaluar en $E_{31}$ obtenemos
	que $\alpha_3=0$.
\end{block}

\begin{xca}
	Utilice~\ref{block:magic_squares} para demostrar que la dimensión del
	espacio de cuadrados mágicos de $n\times n$ es $n(n-2)$. 
\end{xca}


% TODO:
% Considerar K como un anillo conmutativo (pensar en K cuerpo, Z, o K[X])
% En la definición de función n-lineal tomar d:(K^n)^n->K y después considerar funciones en las filas de las matrices
% Ejercicio: Desarrollo por fila. 
% Agregar dos o tres ejemplos 
% Mostrar que una matriz A es inversible si y sólo si detA es una unidad. Como corolario, el caso en que K sea cuerpo
% Aplicación de Cramer en Z
% Determinantes especiales: Matriz compañera

\chapter{Determinantes}

\section{Permutaciones}

\begin{block}
	Sea $n\in\N$. Denotaremos por $\Sym_n$ al conjunto de
	\textbf{permutaciones} del conjunto $\{1,\dots,n\}$, es decir:
	\[
		\Sym_n=\left\{\sigma\colon\{1,\dots,n\}\to\{1,\dots,n\}:\sigma\text{ es biyectiva}\right\}.	
	\]
	Es fácil demostrar que el conjunto $\Sym_n$ tiene $n!=n(n-1)\cdots2$ elementos. 
	Denotaremos a una permutación $\sigma\in\Sym_n$ de la siguiente forma:
	\[
	\begin{pmatrix}
		1 & 2 & \cdots & n\\
		\sigma(1) & \sigma(2) & \cdots & \sigma(n)
	\end{pmatrix}
	\]
	Por ejemplo, las permutaciones del conjunto $\{1,2\}$ son:
	\begin{gather*}
		\Sym_2=\left\{
		\binom{12}{12},
		\binom{12}{21}
		\right\}
	\end{gather*}
	Las permutaciones del conjunto $\{1,2,3\}$ son:
	\begin{gather*}
		\Sym_3=\left\{
		\binom{123}{123},
		\binom{123}{213},
		\binom{123}{321},
		\binom{123}{132},
		\binom{123}{231},
		\binom{123}{312}
		\right\}.
	\end{gather*}
\end{block}

\begin{block}
	Las permutaciones son funciones biyectivas, y la composición de funciones
	biyectivas es una función biyectiva. Luego la composición de dos
	permutaciones es una permutación. Por ejemplo: 
	\[
	\sigma=\binom{123}{132},\quad
	\tau=\binom{123}{312},\quad 
	\sigma\tau=\binom{123}{213}.
	\]
	Como sabemos, la composición de funciones es asociativa. La identidad $\id$
	es el neutro para la composición, y cada permutación, por ser una función
	biyectiva, es inversible.
\end{block}

\begin{block}
    Una permutación $\sigma\in\mathbb{S}_{n}$ es un $r$\textbf{-ciclo} si
    existen $a_{1},\dots,a_{r}\in\{1,\dots,n\}$ todos distintos tales que
    $\sigma(j)=j$ para todo $j\not\in\{a_{1},\dots,a_{r}\}$ y además 
    \[
    \sigma(a_{i})=\begin{cases}
        a_{i+1} & \text{si }i<r,\\
        a_{1} & \text{si \ensuremath{i=r}.}
    \end{cases}
    \]
	Si $a_1,\dots,a_r\in\{1,\dots,n\}$ son todos distintos, denotaremos por
	$(a_1\,a_2\,\dots,a_r)$ al ciclo $\sigma$ que mueve únicamente a los $a_i$
	y tal que 
	\begin{align*}
	\sigma(a_1)=a_2,
	&&
	\sigma(a_2)=a_3,
	&&
	\cdots
	&&
	\sigma(a_{n-1})=a_n,
	&&
	\sigma(a_n)=a_1.
	\end{align*}
\end{block}

\begin{example}
	En general omitiremos los $1$-ciclos al escribir una permutación como
	producto de ciclos disjuntos.  Escribamos algunas permutaciones de $\Sym_3$:
	\begin{align*}
		&\binom{123}{213}=(12)(3)=(12),
		&&
		\binom{123}{321}=(13)(2)=(13),
		&&
		\binom{123}{231}=(123). 
	\end{align*}
	Veamos como ejemplo algunas permutaciones de $\Sym_4$:
	\begin{align*}
		&\binom{1234}{2314}=(123)(4)=(123).
		&&
		\binom{1234}{2143}=(12)(34).
	\end{align*}
\end{example}

\begin{block}
	Las permutaciones $\sigma$ y $\tau$ son \textbf{disjuntas} si para cada
	$j\in\{1,\dots,n\}$ se tiene que $\sigma(j)=j$ o bien $\tau(j)=j$.
\end{block}

\begin{xca}
	\label{xca:disjuntas_conmutan}
	Pruebe que las permutaciones disjuntas conmutan.
\end{xca}

\begin{example}
    Las permutaciones $(123)\in\Sym_6$ y $(56)\in\Sym_6$ son disjuntas. Las
    permutaciones $(123)\in\Sym_6$ y $(25)\in\Sym_6$ no son disjuntas. 
\end{example}

\begin{xca}
    \label{xca:permutaciones}
	Sea $\sigma=\alpha\beta\in\mathbb{S}_{n}$, donde $\alpha$ y $\beta$ son
	disjuntas. Si $\alpha(i)\ne i$ entonces $\sigma^{k}(i)=\alpha^{k}(i)$ para
	todo $k\geq0$. 
\end{xca}

\begin{prop}
	Sea $\sigma\ne\id$ una permutacion de $\Sym_n$.  Entonces $\sigma$ es
	producto de ciclos disjuntos de longitud $\geq2$.  La descomposición es
	única salvo el orden de los factores.

    \begin{proof}
		Hacemos inducción en la cantidad $k$ de elementos del conjunto
		$\{1,\dots,n\}$ que mueve $\sigma$. El caso base es $k=2$, que es
		trivial pues $\sigma$ es una trasposición. Supongamos que el resultado
		es válido para permutaciones que mueven menos de $k$ elementos, donde
		$k>0$. Sea $i_{1}\in\{1,\dots,n\}$ tal que $\sigma(i_{1})\ne i_{1}$.
		Definimos $i_{2}=\sigma(i_{1})$, $i_{3}=\sigma(i_{2})$... y sabemos que
		existe $r\in\mathbb{N}$ tal que $\sigma(i_{r})=i_{1}$ (si existe
		$j\in\{2,\dots,n\}$ tal que $\sigma(i_{r})=i_{j}$ entonces
		$\sigma(i_{j-1})=i_{j}$, una contradicción).  Tomemos
		$\sigma_{1}=(i_{1}i_{2}\cdots i_{r})$. Por hipótesis inductiva, como
		$\sigma_{1}^{-1}\sigma$ es una permutación que mueve menos de $k$
		elementos, $\sigma_{1}^{-1}\sigma=\sigma_{2}\cdots\sigma_{s}$, donde
		$\sigma_{2},\dots,\sigma_{s}$ son ciclos disjuntos. 

		Probemos ahora la unicidad. Para eso, supongamos que
		$\sigma=\sigma_{1}\dots\sigma_{s}=\tau_{1}\cdots\tau_{t}$ con $s>0$.
		Sea $i_{1}\in\{1,\dots,n\}$ tal que $\sigma_{1}(i_{1})\ne i_{1}$.  Por
		lo que hicimos en el ejercicio~\ref{xca:permutaciones},
		$\sigma^{k}(i_{1})=\sigma_{1}^{k}(i_{1})$ para todo $k\geq0$. Existe
		$j\in\{1,\dots,t\}$ tal que $\tau_{j}$ mueve a $i_{1}$. Sin pérdida de
		generalidad (pues los $\tau_{k}$ conmutan) podemos suponer que $j=1$, y
		luego $\sigma_{1}^{k}(i_{1})=\tau_{1}^{k}(i_{1})$ para todo $k\geq0.$
		De acá se deduce que $\sigma_{1}=\tau_{1}$ y por lo tanto
		$\sigma_{2}\cdots\sigma_{s}=\tau_{2}\cdots\tau_{t}$.  Por inducción en
		$\max\{s,t\}$ obtenemos que $s=t$ y por lo tanto $\sigma_{k}=\tau_{k}$
		para todo $k$. 
    \end{proof}
\end{prop}

\begin{cor}
    \label{cor:trasposiciones}
    Toda permutación es producto de trasposiciones. 

    \begin{proof}
        Como toda permutación $\ne\id$ es producto de ciclos disjuntos, alcanza
        con demostrar que cada ciclo es producto de trasposiciones. Luego
        \[
            (a_{1}\dots a_{r})=(a_{1}a_{r})(a_{1}a_{r-1})\cdots(a_{1}a_{3})(a_{1}a_{2})
        \]
        demuestra el corolario.
    \end{proof}
\end{cor}

\begin{cor}
	\label{cor:trasposiciones_(1k)}
    Toda permutación es producto de trasposiciones de la forma $(1k)$.

    \begin{proof}
		Por el corolario~\ref{cor:trasposiciones} sabemos que toda permutación
		es producto de trasposiciones. Como cada trasposición puede escribirse
		como
        \[
            (ij)=(1i)(1j)(1i),
        \]
        entonces toda permutación puede escribirse como producto de
        trasposiciones de la forma $(1k)$. 
	\end{proof}
\end{cor}


\begin{cor}
    \label{cor:trasposiciones_ady}
	Toda permutación es producto de \textbf{trasposiciones adyacentes}, es
	decir: trasposiciones de la forma $(k\,k+1)$. 

    \begin{proof}
		Por el corolario~\ref{cor:trasposiciones_(1k)} sabemos que toda
		permutación es producto de trasposiciones de la forma $(1k)$.  Probemos
		por inducción que toda permutación de la forma $(1k)$ puede escribirse
		como producto de trasposiciones adyacentes. El primer caso es
		$(13)=(12)(23)(12)$. Si suponemos que la afirmación es válida para la
		permutación $(1k)$ entonces el corolario queda demostrado al utilizar
		la hipótesis inductiva en la descomposición
		$(1\,k+1)=(k\,k+1)(1k)(k\,k+1)$.
    \end{proof}
\end{cor}

\begin{block}
    El corolario~\ref{cor:trasposiciones} nos dice que toda $\sigma\in\Sym_n$
    puede escribirse como producto de trasposiciones.  Esta escritura no es
    única, por ejemplo:
    \[
        (123)=(13)(12)=(23)(12)(23)(12).
    \]
	Sin embargo, puede probarse que si
	$\sigma=\sigma_1\cdots\sigma_k=\tau_1\cdots\tau_l$ como producto de
	trasposiciones entonces $k$ y $l$ tienen la misma paridad. Probaremos esta
	afirmación como aplicación de la teoría de determinantes.
\end{block}

\section{Funciones multilineales alternadas}

\begin{block}
    Sea $n\in\N$. Una función $d\colon\K^{n\times n}\to\K$ es
    \textbf{$n$-lineal} (por filas) si para cada $i\in\{1,\dots,n\}$ la función
    $d$ es lineal en la $i$-ésima fila cuando las otras $n-1$ se dejan fijas,
    es decir: 
	\begin{multline*}
		d(A_1,\dots,A_{i-1},x+\lambda y,A_{i+1},\dots,A_n)
		\\=d(A_1,\dots,A_{i-1},x,A_{i+1},\dots,A_n)+
		\lambda d(A_1,\dots,A_{i-1},y,A_{i+1},\dots,A_n).
	\end{multline*}
\end{block}

\begin{xca}
    \label{xca:combinacion_nlineal}
	Toda combinación lineal de funciones $n$-lineales es $n$-lineal.
\end{xca}

\begin{block}
    Una función $n$-lineal $d$ es \textbf{alternada} si $d(A)=0$ para toda
    matriz $A$ que tiene dos filas iguales. Una función $d$ es un
    \textbf{determinante} si es $n$-lineal, alternada y $d(I)=1$.
\end{block}

\begin{examples}
    \label{xca:2x2}
    La función $d\colon\K\to\K$ dada por $d(a)=a$ es una función determinante.
    Similarmente la función $d\colon\K^{2\times2}\to\K$ definida por 
    \[
        d\begin{pmatrix}
            a_{11} & a_{12}\\
            a_{21} & a_{22}
        \end{pmatrix}
        =a_{11}a_{22}-a_{12}a_{21}
    \]
    es una función determinante.
\end{examples}

\begin{example}
    Supongamos que $d\colon\K^{2\times2}\to\K$ es una función determinante. Si
    $e_1=(1,0)$ y $e_2=(0,1)$ entonces la primera fila de la matriz
    $A=\begin{pmatrix}a&b\\c&d\end{pmatrix}$ puede escribirse como $ae_1+be_2$
        y la segunda fila de $A$ puede escribirse como $ce_1+de_2$. Si
        usamos la $2$-linealidad de $d$, 
    \begin{align*}
        d(A)&=d(ae_1+be_2,ce_1+de_2)\\
        &=acd(e_1,e_1)+add(e_1,e_2)+bcd(e_2,e_1)+bdd(e_2,e_2).
    \end{align*}
    Como $d$ es alternada, 
    \[
        d(e_1,e_1)=d(e_2,e_2)=0,\quad
        d(e_2,e_1)=-d(e_1,e_2).
    \]
    Entonces 
    \[
        d(A)=(ad-bc)d(e_1,e_2)=(ad-bc)d(I).
    \]
    Como $d$ es una función determinante, $d(I)=1$ y luego $d(A)=ad-bc$. 
\end{example}

\begin{lem}
    \label{lem:alternada}
    Si $d$ es alternada entonces 
    \[
        d(A_1,\dots,A_i,\dots,A_j,\dots,A_n)=-d(A_1,\dots,A_j,\dots,A_i,\dots,A_n).
    \]

	\begin{proof}
		Como $d$ es alternada,
		\[
		d(A_1,\dots,A_{i-1},A_i+A_j,A_{i+1},\dots,A_{j-1},A_i+A_j,A_{j+1},\dots,A_{n})=0.
		\]
		Al usar la $n$-linealidad obtenemos
		\begin{multline*}
			0=d(A_1,\dots,A_i,\dots,A_i,\dots,A_n)+d(A_1,\dots,A_i,\dots,A_j,\dots,A_n)\\
			+d(A_1,\dots,A_j,\dots,A_i,\dots,A_n)+d(A_1,\dots,A_j,\dots,A_j,\dots,A_n).
		\end{multline*}
		Como $d$ es alternada, 
		\[
		d(A_1,\dots,A_i,\dots,A_i,\dots,A_{n})=d(A_1,\dots,A_j,\dots,A_j,\dots,A_n)=0,
		\]
		y entonces el lema queda demostrado.
	\end{proof}
\end{lem}

\begin{block}
	Sea $A\in\K^{n\times n}$ y sean $i,j\in\{1,\dots,n\}$. Si $d$ es una
	función $n$-lineal y $A\in\K^{n\times n}$ entonces
	\[
		d_{ij}(A)=d(A[i|j]),
	\]
	donde $A[i|j]$ es la matriz de tamaño $(n-1)\times(n-1)$ que se obtiene al
	eliminar la $i$-ésima fila y la $j$-ésima columna de $A$. Por ejemplo:
	\begin{align*}
		&
		A=\begin{pmatrix}
			1 & 2 & 3\\
			4 & 5 & 6\\
			7 & 8 & 9
		\end{pmatrix},
		&&
		A[1|1]=\begin{pmatrix}
			5 & 6\\
			8 & 9
		\end{pmatrix},
		&&
		A[1|2]=\begin{pmatrix}
			4 & 6\\
			7 & 9
		\end{pmatrix},
		&&
		A[3|2]=\begin{pmatrix}
			1 & 3\\
			4 & 6
		\end{pmatrix}.
	\end{align*}
\end{block}

\begin{thm}
    \label{thm:desarrollo_cols}
    Si $d$ es una función $(n-1)$-lineal, alternada y $d(I)=1$ entonces para
    cada $j\in\{1,\dots,n\}$ la función
    \[
        E_j(A)=\sum_{i=1}^n(-1)^{i+j}a_{ij}d_{ij}(A)
    \]
    es $n$-lineal, alternada y cumple que $E_j(I)=1$.

    \begin{proof}
		Para demostrar que $E_j$ es $n$-lineal basta observar que la función
		$A\mapsto a_{ij}d_{ij}(A)$ es $n$-lineal y utilizar el
		ejercicio~\ref{xca:combinacion_nlineal} que afirma que toda
		combinaciones lineal de funciones $n$-lineales es $n$-lineal.

		Demostremos que $E_j$ es alternada. Para eso, supongamos que $A$ tiene
		dos filas iguales, digamos que son las filas $k$ y $l$, donde $k<l$. Si
		$i\not\in\{k,l\}$ entonces $d_{ij}(A)=0$ pues la matriz $A[i|j]$ tiene
		dos filas iguales. Entonces
		\begin{align*}
			E_j(A)&=(-1)^{k+j}a_{kj}d_{kj}(A)+(-1)^{l+j}a_{lj}d_{lj}(A).
		\end{align*}
		La $(l-1)$-ésima fila de $A[k|j]$ y la $k$-ésima fila de $A[l|j]$ son
		iguales a $A_k$ y entonces $A[k|j]$ y $A[l|j]$ difieren en $l-k-1$
		trasposiciones. Luego 
		\[
			d_{kj}(A)=(-1)^{l-k-1}d_{lj}(A).
		\]
		Como $a_{kj}=a_{lj}$, 
		\begin{align*}
			E_j(A)
			=(-1)^{k+j+l-k-1}a_{lj}d_{lj}(A)+(-1)^{l+j}a_{lj}d_{lj}(A)
			=0.
		\end{align*}
		
		Para ver que $E_j(I)=1$ basta observar que 
        \[
            E_j(I)=\sum_{i=1}^n(-1)^{i+j}\delta_{ij}d_{ij}(I)=d_{jj}(I)=1.
        \]
        Esto completa la demostración.
    \end{proof}
\end{thm}

\begin{cor}
    \label{cor:determinante:existencia}
    Existe al menos una función determinante. 

    \begin{proof}
		Vimos en el ejemplo~\ref{xca:2x2} que existen determinantes cuando
		$n\in\{1,2\}$. El caso general es consecuencia directa del teorema
		anterior y el principio de inducción.
    \end{proof}
\end{cor}

\section{Aplicación a la teoría de permutaciones}

\begin{block}
    Sea $\sigma\in\Sym_n$. El corolario~\ref{cor:trasposiciones} nos dice que
    $\sigma$ puede escribirse como producto de traposiciones. Esta escritura no
    es única, por ejemplo:
    \[
        (123)=(13)(12)=(23)(12)(23)(12).
    \]
	Sin embargo, puede probarse que si
	$\sigma=\sigma_1\cdots\sigma_k=\tau_1\cdots\tau_l$ como producto de
	trasposiciones entonces $k$ y $l$ tienen la misma paridad.  
\end{block}

\begin{cor}
    \label{cor:signo}
    Sea $\sigma\in\mathbb{S}_{n}$ tal que
    $\sigma=\sigma_{1}\cdots\sigma_{k}=\tau_{1}\cdots\tau_{l}$ como producto de
    trasposiciones. Entonces $(-1)^{k}=(-1)^{l}$.

    \begin{proof}
		Por el corolario~\ref{cor:determinante:existencia} sabemos que existe
		una función determinante $d\colon\K^{n\times n}\to\K$. Si
		$\{e_1,e_2,\dots,e_n\}$ es la base canónica de $\K^{n\times 1}$
		entonces $d(e_1,\dots,e_n)=d(I)=1$ pues $d$ es una función
		determinante.  Como, por hipótesis,
		$\sigma=\sigma_1\cdots\sigma_k=\tau_1\cdots\tau_l$ y $d$ es una función
		alternada, entonces 
        \begin{align*}
            (-1)^k&=(-1)^kd(e_1,\dots,e_n)\\
            &=d(e_{\sigma(1)},e_{\sigma(2)},\dots,e_{\sigma(n)})=(-1)^ld(e_1,\dots,e_n)=(-1)^l,
        \end{align*}
        que es lo que queríamos demostrar.
    \end{proof}
\end{cor}

\begin{block}
	El corolario~\ref{cor:signo} nos permite definir el signo de una
	permutación.  Sea $\sigma\in\Sym_n$. Se define el \textbf{signo} de
	$\sigma$ como el número $\sgn(\sigma)=(-1)^k$ si
	$\sigma=\sigma_1\cdots\sigma_k$ como producto de trasposiciones. Una
	permutación $\sigma$ es \textbf{par} si $\sgn(\sigma)=1$ y es
	\textbf{impar} si $\sgn(\sigma)=-1$. 
\end{block}

\begin{examples}
	La identidad es una permutación par. Las trasposiciones son permutaciones
	impares.  Un $r$-ciclo tiene signo $(-1)^{r-1}$. Otros ejemplos: 
	\begin{align*}
	&\sgn((123))=
	\sgn((12)(34))=1,
	&&
	\sgn((1234))= 
	\sgn((123)(45))=-1.
\end{align*}
\end{examples}

\begin{cor}
    \label{cor:sgn_es_morfismo}
    Sean $\sigma,\tau\in\Sym_n$. Entonces
    $\sgn(\sigma\tau)=\sgn(\sigma)\sgn(\tau)$.

    \begin{proof}
		Escribamos $\sigma=\sigma_1\cdots\sigma_k$ y $\tau=\tau_1\cdots\tau_l$,
		ambas como producto de trasposiciones. Entonces, como $\sigma\tau$
		puede escribirse como producto de $k+l$ trasposiciones, 
       	\[
			(\sgn\sigma)(\sgn\tau)
			=(-1)^k(-1)^l
			=(-1)^{k+l}
			=\sgn(\sigma\tau),
        \]
        tal como queríamos demostrar.
    \end{proof}
\end{cor}

\begin{cor}
	Sea $\sigma\in\Sym_n$. Entonces $\sgn(\sigma)=\sgn(\sigma^{-1})$. 

	\begin{proof}
		Si escribimos a $\sigma$ como producto de trasposiciones, digamos
		$\sigma=\sigma_1\cdots\sigma_k$, entonces, como $\sigma_j^2=\id$ para
		todo $j$, $\sigma^{-1}=\sigma_k\cdots\sigma_1$. Luego
		$\sgn(\sigma)=\sgn(\sigma^{-1})$. 
	\end{proof}
\end{cor}

\begin{cor}
    Supongamos que $\sigma\in\Sym_n$ se descompone como producto de $m$ ciclos
    disjuntos de longitud $l_1,\dots,l_m$. Entonces
    \[
        \sgn(\sigma)=(-1)^{\sum_{k=1}^m(l_k-1)}.
    \]

    \begin{proof}
        Supongamos que $\sigma$ se descompone como producto de ciclos disjuntos
        $\sigma=c_1\dots c_m$ y que cada ciclo $c_k$ tiene longitud $l_k$. Como
        cada $c_k$  puede escribirse como producto de $l_k-1$ trasposiciones,
        $\sgn(c_k)=(-1)^{l_k-1}$. Por el corolario~\ref{cor:sgn_es_morfismo}, 
        \[
            \sgn(\sigma)=\prod_{k=1}^m\sgn(c_k)=\prod_{k=1}^m(-1)^{l_k-1}=(-1)^{\sum_{k=1}^m(l_k-1)},
        \]
        tal como queríamos demostrar.
    \end{proof}
\end{cor}

\section{Unicidad y propiedades básicas}

\begin{lem}
	\label{lem:sigma}
    Sean $d\colon\K^{n\times n}\to\K$ una función alternada y
    $\sigma\in\Sym_n$. Entonces, para todo $A\in\K^{n\times n}$, 
   \[
        d(A_{\sigma(1)},\dots,A_{\sigma(n)})=(\sgn\sigma)d(A).
   \]

   \begin{proof}
		Supongamos que $\sigma$ se escribe como producto de $k$ trasposiciones.
		Entonces 
        \[
            d(A_{\sigma(1)},\dots,A_{\sigma(n)})=(-1)^k\det A=(\sgn\sigma)d(A),
        \]
        tal como queríamos demostrar. 
   \end{proof}
\end{lem}

\begin{thm}
    \label{thm:determinante:unicidad}
    Existe una única función determinante $\det\colon\K^{n\times n}\to\K$. Más
    aún, 
    \begin{equation}
        \label{eq:det:permutaciones}
        \det(A)=\sum_{\sigma\in\Sym_n}\sgn(\sigma)a_{1\sigma(1)}\cdots a_{n\sigma(n)}
    \end{equation}
    para todo $A\in\K^{n\times n}$. 

    \begin{proof}
        Supongamos que $d$ es una función determinante y sea $A\in\K^{n\times
        n}$. Escribamos a cada fila de $A$ como $A_i=\sum_{j=1}^n a_{ij}e_j$
        donde $\{e_1,\dots,e_n\}$ es la base canónica de $\K^{1\times n}$.
        Entonces
		\begin{align*}
			d(A)&=d(A_1,\dots,A_n)
			=\sum_{j_1=1}^n\cdots\sum_{j_n=1}^na_{1j_1}\cdots a_{nj_n}d(e_{j_1},\dots,e_{j_n}).
		\end{align*}
        Como $d$ es alternada, la suma es no nula únicamente cuando todos los
        $j_k$ son distintos, es decir cuando $|\{j_1,\dots,j_n\}|=n$. Entonces,
        la suma anterior se hace sobre todas las $n$-uplas $(j_1,\dots,j_n)$ de
        elementos distintos del conjunto $\{1,\dots,n\}$. Luego, por el
        lema~\ref{lem:sigma},  
		\begin{align*} 
			d(A)&=\sum_{\sigma\in\Sym_n}a_{1\sigma(1)}\cdots a_{n\sigma(n)}d(e_{\sigma(i)},\dots,e_{\sigma(n)})\\
			&=\sum_{\sigma\in\Sym_n}\sgn(\sigma)a_{1\sigma(1)}\cdots a_{n\sigma(n)}d(e_1,\dots,e_n).
		\end{align*}
        Como $d$ es un determinante, $d(e_1,\dots,e_n)=d(I)=1$ y entonces 
        \[
            d(A)=\sum_{\sigma\in\Sym_n}\sgn(\sigma)a_{1\sigma(1)}\cdots a_{n\sigma(n)}.
        \]
		En particular, existe una única función determinante. 
	\end{proof}
\end{thm}

\begin{xca}
    \label{xca:dA=(detA)dI}
    Utilice lo hecho en la demostración del teorema~\ref{thm:determinante:unicidad}
    y pruebe que si $d$ es $n$-lineal y alternada entonces
    \[
        d(A)=(\det A)d(I)
    \]
    para todo $A\in\K^{n\times n}$. 
\end{xca}

\begin{cor}[Desarrollo por columnas]
    Sea $A\in\K^{n\times n}$. Entonces, para cada $j\in\{1,\dots,n\}$, 
    \[
    \det(A)=\sum_{i=1}^n(-1)^{i+j}a_{ij}M_{ij}(A),
    \]
    donde $M_{ij}(A)=\det A[i|j]$. 

	\begin{proof}
		Es consecuencia directa del 
		teorema~\ref{thm:determinante:unicidad}. 
	\end{proof}
\end{cor}

\begin{example}
    Los números $917$, $854$ y $994$ son divisibles por $7$.
    Vamos a utilizar esta información para demostrar que 
    \[
        \det\begin{pmatrix}
            9 & 1 & 7\\
            8 & 5 & 4\\
            9 & 9 & 4
        \end{pmatrix}
    \]
    es divisible por $7$. Observemos que al efectuar 
    la operación de columnas $100C_1+10C_2+C_3\to C_3$ 
    se tiene que 
    \[
        \det\begin{pmatrix}
            9 & 1 & 7\\
            8 & 5 & 4\\
            9 & 9 & 4
        \end{pmatrix}       
        =
        \det\begin{pmatrix}
            9 & 1 & 917\\
            8 & 5 & 854\\
            9 & 9 & 994
        \end{pmatrix}.
    \]
    Si desarrollamos este último determinante por la última columna, el valor
    del determinante es divisible por siete.
\end{example}

\begin{thm}
    \label{thm:det(AB)=(detA)(detB)}
	Sean $A,B\in\K^{n\times n}$. Entonces 
	\[
		\det(AB)=(\det A)(\det B).
	\]

	\begin{proof}
		Sea $d\colon\K^{n\times n}\to\K$ dada por
		\[
			d(X)=\det(XB)
		\]
        Si las filas de $X$ son los vectores $X_1,\dots,X_n$ entonces las filas
        de la matriz $XB$ son $X_1B,\dots,X_nB$.  Demostremos que $d$ es
        alternada: si $X_i=X_j$ entonces $X_iB=X_jB$ y usamos que $\det$ es
        alternada. Demostremos que $d$ es $n$-lineal: si la fila $i$-ésima de
        $X$ es $X_i+\lambda Y_i$ entonces la $i$-ésima fila de $XB$ es
        $(X_i+\lambda Y_i)B=X_iB+\lambda Y_iB$ y usamos que $\det$ es
        $n$-lineal. Por el ejercicio~\ref{xca:dA=(detA)dI}, 
		\[
		\det(AB)=d(A)=(\det A)d(I)=(\det A)(\det B),
		\]
		tal como queríamos demostrar.
	\end{proof}
\end{thm}

\begin{block}
	\label{block:determinante:semejanza} 
	El teorema anterior implica que si $A,B\in\K^{n\times n}$ son semejantes
	entonces, como $B=CAC^{-1}$ para alguna matriz inversible $C\in\K^{n\times
	n}$, se tiene que 
	\begin{align*}
		\det B &= \det(CAC^{-1})
		=(\det C)(\det A)(\det C^{-1})\\
		&=(\det C)(\det A)(\det C)^{-1}
		=\det A. 
	\end{align*}
\end{block}

\begin{block}
	Sean $V$ un espacio vectorial de dimensión finita y $f\in\hom(V,V)$. Como
	las matrices de $f$ con respecto a bases distintas son semejantes, la
	observación hecha en~\ref{block:determinante:semejanza} implica que tiene
	sentido definir el \textbf{determinante} de $f$ como
	\[
		\det f=\det [f]_{\cB,\cB},
	\]
	donde $\cB$ es alguna base de $V$. 
\end{block}

\begin{example}
    \label{exa:det:fibonacci}
	La \textbf{sucesión de Fibonacci} $F_n$
    es la sucesión 
	\[
        1,1,2,3,5,8,13,21,34,55,89,144,233,377,\dots
    \]
	definida recursivamente por $F_0=F_1=1$ y
    $F_{n+1}=F_n+F_{n-1}$ para $n\geq1$. Por inducción puede demostrarse que 
    \[
        \begin{pmatrix}
            0 & 1\\
            1 & 1
        \end{pmatrix}^n
        =
        \begin{pmatrix}
            F_{n-1} & F_n\\
            F_n & F_{n+1}
        \end{pmatrix}
    \]
	para todo $n\geq1$. Si aplicamos la función determinante a la fórmula
	anterior y usamos que el determinante es una función multiplicativa,
	obtenemos
    \[
        F_{n+1}F_{n-1}-F_n^2=(-1)^n
    \]
    para todo $n\geq1$.
\end{example}

\begin{prop}
	Si $A=(a_{ij})$ es triangular superior (es decir $a_{ij}=0$ si $i>j$)
	entonces $\det A=a_{11}\cdots a_{nn}$. 	

	\begin{proof}
		Procedemos por inducción en $n$. Los casos $n\in\{1,2\}$ son triviales.
		Supongamos entonces que el resultado es válido para matrices de tamaño
		$(n-1)\times(n-1)$. Entonces, al desarrollar el determinante por la
		primera columna y utilizar la hipótesis inductiva en la matriz $A[1|1]$, 
		\[
		\det A=a_{11}\det A[1|1]=a_{11}a_{22}\cdots a_{nn},
		\]
		que es lo que queríamos demostrar.
	\end{proof}
\end{prop}

\begin{prop}
	\label{pro:detA=det(AT)}
    Si $A\in\K^{n\times n}$ entonces $\det A=\det A^T$.    

    \begin{proof}
%		Procederemos por inducción en $n$. Los casos $n\in\{1,2\}$ son
%		triviales. Supongamos entonces que el resultado es válido en matrices
%		de tamaño $(n-1)\times(n-1)$. Si llamamos $B=A^T$ entonces, al
%		desarrollar por la primera columna, 
%		\begin{align*}
%			\det B&=\sum_{i=1}^n(-1)^{i+1}b_{i1}\det B[i|1].
%		\end{align*}
%		Por hipótesis inductiva,
%		\[
%			\det B[i|1]=\det B[1|i]=\det\left(A^T[1|i]\right)=\det\left(A[i|1]^T\right)=\det A[i|1]=\det A[1|i].
%		\]
%		Luego, como $b_{i1}=a_{1i}$ para todo $i$, 
%		\[
%			\det B=\sum_{i=1}^n(-1)^{i+1}a_{1i}\det A[i|1]=\det A,
%		\]
%		que es lo que queríamos demostrar.
        Si $\sigma\in\Sym_n$ y $\sigma(i)=j$ entonces
        $\sigma^{-1}(j)=i$ y además $a_{\sigma(i)i}=a_{j\sigma^{-1}(j)}$. Entonces
        \begin{align*}
            \det A&=\sum_{\sigma\in\Sym_n}\sgn(\sigma)a_{1\sigma(1)}\cdots a_{n\sigma(n)}\\
            &=\sum_{\sigma\in\Sym_n}\sgn(\sigma)a_{\sigma^{-1}(1)1}\cdots a_{\sigma^{-1}(n)n}\\
            &=\sum_{\sigma^{-1}\in\Sym_n}\sgn(\sigma^{-1})a_{\sigma^{-1}(1)1}\cdots a_{\sigma^{-1}(n)n}\\
			&=\det A^T,
        \end{align*}
		pues $\sgn(\sigma)=\sgn(\sigma^{-1})$ y si $\sigma$ recorre todo $\Sym_n$
		entonces $\sigma^{-1}$ también recorre todo $\Sym_n$. 
    \end{proof}
\end{prop}

\begin{xca}
	Demuestre que si $A=(a_{ij})\in\K^{n\times n}$ es triangular inferior
	entonces $\det A=a_{11}\cdots a_{nn}$.
\end{xca}

\begin{block}
	En vista de la forma en que construimos la función determinante	y de la
	proposición~\ref{pro:detA=det(AT)} se tiene la siguiente propiedad: si $B$
	se obtiene de $A$ al sumar en una fila un múltiplo de otra fila
	(o al sumar a una columna un múltiplo de otra columna) entonces $\det B=\det A$.
\end{block}

\begin{xca}
	\label{xca:bloques_2x2}
	Sean $A\in\K^{r\times r}$, $B\in\K^{r\times s}$, $C\in\K^{s\times s}$.
	Pruebe que 
	\begin{align*}
		&\det\begin{pmatrix}
		A & B\\
		0 & C
	\end{pmatrix}
	=(\det A)(\det C).
	\end{align*}
\end{xca}

\begin{cor}[Desarrollo por filas]
	Sea $A\in\K^{n\times n}$. Entonces, para cada $i\in\{1,\dots,n\}$, 
    \[
    \det(A)=\sum_{j=1}^n(-1)^{i+j}a_{ij}M_{ij}(A),
    \]
    donde $M_{ij}(A)=\det A[i|j]$. 
	\begin{proof}
		Es consecuencia directa de la proposición~\ref{pro:detA=det(AT)},
		donde vimos que $\det A=\det A^T$, y el
		teorema~\ref{thm:determinante:unicidad}. 
	\end{proof}
\end{cor}

%\begin{block}[Desarrollo por filas del determinante]
%	En la proposición~\ref{pro:detA=det(AT)} vimos que $\det A=\det A^T$ para
%	toda matriz $A\in\K^{n\times n}$. 
%	El teorema~\ref{thm:determinante:unicidad} nos dice entonces que 
%	para cada $i\in\{1,\dots,n\}$, 
%    \[
%    \det(A)=\sum_{j=1}^n(-1)^{i+j}a_{ij}M_{ij}(A),
%    \]
%    donde $M_{ij}(A)=\det A[i|j]$. 
%\end{block}

\begin{example}
    Como ejemplo, vamos a demostrar que
    \[
    \det\begin{pmatrix}
        1 & x & x^2\\
        1 & y & y^2\\
        1 & z & z^2
    \end{pmatrix}
    =(y-x)(z-x)(z-y).
    \]

    Dado que el determinante en matrices de $3\times3$ es una función
    $3$-lineal, efectuar las operaciones elementales de filas $F_2-F_1\to F_2$
    y $F_3-F_1\to F_3$ no cambia el valor del determinante. Esto implica que 
    \[
    \det\begin{pmatrix}
        1 & x & x^2\\
        1 & y & y^2\\
        1 & z & z^2
    \end{pmatrix}
    =
    \det\begin{pmatrix}
        1 & x & x^2\\
        0 & y-x & y^2-x^2\\
        0 & z-x & z^2-x^2
    \end{pmatrix}
    =(y-x)(z-x)(z-y),
    \]
    después de desarrollar este último determinante por la primera columna.
\end{example}

\begin{block}
    Dada una matriz $A\in\K^{n\times n}$ se define la matriz de los \textbf{cofactores}
    de $A$ como 
    \[
        C_{ij}=(-1)^{i+j}\det A[i|j]
    \]
    para todo $i,j$. Se define además 
    la \textbf{adjunta} $\adj A$ de $A$ como 
    la traspuesta de la matriz de cofactores de $A$, es decir:
    \[
    (\adj A)_{ij}=(-1)^{i+j}\det A[j|i]=C_{ji}
    \]
    para todo $i,j$.
\end{block}

\begin{example}
	Observemos que 
	\[
		A=\begin{pmatrix}
			1 & 2 & 1\\
			0 & 1 & 2\\
			1 & 0 & 0
		\end{pmatrix}
		,\quad
		\adj A=\begin{pmatrix}
			0 & 0 & 3\\
			2 & -1 & -2\\
			-1 & 2 & 1
		\end{pmatrix}.
	\]
%	Luego $A(\adj A)=3I$. 
\end{example}

\begin{example}
		
\end{example}

\begin{thm}
    \label{thm:AadjA=(detA)I}
    Sea $A\in\K^{n\times n}$. Entonces 
    \[
        A(\adj A)=(\adj A)A=(\det A)I.
    \]

	\begin{proof}\framebox{FIXME}
		Si $i\ne j$ y $B$ es la matriz $A$ después de haber reemplazado su
		$i$-ésima columna por su $j$-ésima columna, entonces $A[k|i]=B[k|i]$
		para todo $k$. Luego, como $B$ tiene dos columnas iguales, al
		desarollar el determinante por la columna $i$-ésima, 
        \[
        0=\det B=\sum_{k=1}^n(-1)^{k+i}b_{ki}\det B[k|i]=\sum_{k=1}^n(-1)^{k+i}a_{kj}\det A[k|i].
        \]

        Por otro lado, para cada $i,j$ tenemos que 
        \begin{align*}
            \left((\adj A)A\right)_{ij}=\sum_{k=1}^n (\adj A)_{ik}a_{kj}=\sum_{k=1}^n (-1)^{k+i}a_{kj}\det A[k|i].
        \end{align*}
        Observemos que si $i\ne j$ entonces $((\adj A)A)_{ij}$ es el determinante de la matriz $B$, que 
        tiene dos columnas iguales. En cambio, si $i=j$, entonces $(A\adj A)=\det A$ por
		la fórmula~\ref{eq:det:permutaciones}. 
        Luego
        \[
            (A\adj A)_{ij}=\begin{cases}
                \det A & \text{si $i=j$},\\
                0 & \text{si $i\ne j$},
            \end{cases}
        \]
        tal como queríamos demostrar.
    \end{proof}
\end{thm}

\begin{example}
	Si $A=\begin{pmatrix}a&b\\c&d\end{pmatrix}\in\K^{2\times2}$ entonces 
	$\adj A=\begin{pmatrix}d&-b\\-c&a\end{pmatrix}$. Entonces, si $\det
	A=ad-bc\ne0$, 
	\[
		A^{-1}=\frac{1}{ad-bc}\begin{pmatrix}d&-b\\-c&a\end{pmatrix}.
	\]
\end{example}

\begin{xca}
	\label{xca:adjadjA}
	Sean $A\in\K^{n\times n}$ matrices inversibles. Demuestre que 
	$\adj(BAB^{-1})=B\adj(A)B^{-1}$. 
\end{xca}

\begin{xca}
	\label{xca:adj(BAB^(-1))}
    Sean $A,B\in\K^{n\times n}$. Demuestre que si $A$ es inversible entonces
    $\adj(\adj A)=(\det A)^{n-2}A$.
\end{xca}

\begin{xca}
    Sea $A\in\R^{3\times3}$ una matriz de rango $1$. Calcule el rango de la
    adjunta de $A$.
\end{xca}

\begin{cor}
	\label{cor:no_inversible<=>detA=0}
    Sea $A\in\K^{n\times n}$. Entonces $A$ es inversible si y sólo si $\det
    A\ne0$. 

    \begin{proof}
        Si $A$ es inversible entonces existe $B\in\K^{n\times n}$ tal que
        $AB=I$ y luego $(\det A)(\det B)=\det(AB)=\det(I)=1$, que implica que
        $\det A\ne0$.  Recíprocamente, si $\det A\ne0$ entonces $A^{-1}=(\det
        A)^{-1}\adj A$ por el teorema~\ref{thm:AadjA=(detA)I}.
    \end{proof}
\end{cor}

\begin{xca}[Regla de Cramer]
	\label{block:Cramer}
	Supongamos que $A\in\K^{n\times n}$ es inversible y sea
    $b\in\K^{n\times1}$. Utilice el 
    teorema~\ref{thm:AadjA=(detA)I}.
    para demostrar que la solución $x=(x_1,\dots,x_n)^T$ del sistema lineal
    $Ax=b$ puede calcularse de la siguiente forma: 
   % Para resolver el sistema lineal $Ax=b$ procedemos de
   % la siguiente forma. Si $x\in\K^{n\times 1}$ es una 
   % solución de $Ax=b$ entonces
   % \[
   %     (\det A)x=(\adj A)Ax=(\adj A)b.
   % \]
   % Para cada $i\in\{1,\dots,n\}$ se tiene
   % \[
   %     (\det A)x_i=\sum_{j=1}^n (\adj A)_{ij}b_j=\sum_{j=1}^n(-1)^{i+j}b_j\det A[j|i].
   % \]
   % En resumen, cada $x_i$ puede calcularse como
    \[
    x_i=\frac{\det B_i}{\det A},
    \]
	donde $B_i$ es la matriz que se obtiene de $A$ después de reemplazar su
	$i$-ésima columna por el vector columna $b$. 
\end{xca}

\begin{block}
    Vamos a dar una demostración elemental de la regla de Cramer vista
    en~\ref{block:Cramer}.  Sea $\{e_1,\dots,e_n\}$ la base canónica de
    $\K^{n\times1}$ y para cada $i\in\{1,\dots,n\}$ sea $X_i$ la que se obtiene
    después de reemplazar la $i$-ésima columna de la matriz identidad por el
    vector columna $x=(x_1\dots x_n)^T$. Por ejemplo:
	\[
	X_1=\begin{pmatrix}
		x_1 & 0 & 0 & \cdots & 0\\
		x_2 & 1 & 0 & \cdots & 0\\
        x_3 & 0 & 1 & \cdots & 0\\
		\vdots & \vdots & \vdots & \ddots & \vdots\\
		x_n & 0 & 0 & \cdots & 1
	\end{pmatrix}
	\]
	Entonces 
	\begin{align*}
		AX_1&=(Ax,Ae_2,\dots,Ae_n)
		=(b,A_2,\dots,A_n).
	\end{align*}
	donde $A_i$ es la $i$-ésima columna de $A$. Luego
	\begin{align*}
		(\det A)x_1&=(\det A)(\det X_1)=\det(AX_1)\\
		&=\det(b,A_2,\dots,A_n).
	\end{align*}
	De forma similar puede calcularse el valor de cada $x_i$. 
\end{block}

\begin{xca}
	\label{xca:rango_submatriz}
	Sea $A\in\K^{n\times n}$. Pruebe que $\rg A\geq r$ si y sólo si existe una
	submatriz de $A$ inversible y de tamaño $r\times r$. Concluya que $\rg A=r$
	si y sólo si existe una submatriz de $A$ inversible y de tamaño $r\times r$
	y toda submatriz de $A$ de tamaño $(r+1)\times(r+1)$ es no inversible. 
\end{xca}

\section{Algunos determinantes especiales}

\begin{block}[Matriz de Vandermonde]
	Sean $x_1,\dots,x_n\in\K$. Vamos a demostrar que si 
	\[
		V(x_1,\dots,x_n)=\det\begin{pmatrix}
			1 & 1 & \cdots & 1\\
			x_1 & x_2 & \cdots & x_n\\
			\vdots & \vdots & \ddots & \vdots\\
			x_1^{n-1} & x_2^{n-1} & \cdots & x_n^{n-1}
		\end{pmatrix}
	\]
	entonces
	\[
		V(x_1,\dots,x_n)=\prod_{1\leq i<j\leq n}(x_j-x_i).
	\]

	Sin pérdida de generalidad podemos suponer que todos los $x_j$ son
	distintos ya que si $x_i=x_j$ para $i\ne j$ entonces el determinate es cero
	y la fórmula es válida. Podemos suponer entonces que todos los $x_j$ son
	distintos.  Procederemos por inducción en $n$. El caso $n=2$ es sencillo y
	se deja como ejercicio. Supongamos que el resultado es válido para algún
	$n\geq2$. Sea
	\[
	f=V(x_1,\dots,x_{n},X)=\det\begin{pmatrix}
			1 & 1 & \cdots & 1 & 1\\
			x_1 & x_2 & \cdots & x_n & X\\
			\vdots & \vdots & \ddots & \vdots & \vdots \\
			x_1^{n-1} & x_2^{n-1} & \cdots & x_n^{n-1} & X^{n-1}\\
			x_1^{n} & x_2^{n} & \cdots & x_n^{n} & X^{n}\\
		\end{pmatrix}.
	\]
	Por hipótesis inductiva, $V(x_1,\dots,x_n)\ne0$. Al desarrollar entonces el
	determinante por la última columna, se ve claramente que el polinomio
	$V(x_1,\dots,x_{n},X)\in\K[X]$ tiene grado $n$ y que su coeficiente
	principal es $V(x_1,\dots,x_{n})$. 
	Como $f(x_j)=0$ para todo $j\in\{1,\dots,n\}$, entonces 
	\[
		f=V(x_1,\dots,x_n)\prod_{i=1}^n(X-x_j).
	\]
	Luego 
    \begin{align*}
        f(x_{n+1})&=V(x_1,\dots,x_{n+1})\\
        &=V(x_1,\dots,x_n)\prod_{i=1}^n(x_{n+1}-x_j)\\
        &=\prod_{1\leq i<j\leq n+1}(x_j-x_i).
    \end{align*}
	tal como queríamos demostrar.
\end{block}

\begin{xca}
	Sea $V$ un espacio vectorial de dimensión finita $n$ y sea
	$\{v_1,\dots,v_n\}$ una base de $V$. Sean $x_1,\dots,x_n\in\K$ con $x_i\ne
	x_j$ si $i\ne j$. Demuestre que si 
	\begin{align*}
		&w_1 = v_1+v_2+\cdots+v_n,\\
		&w_2 = x_1v_1+x_2v_2+\cdots+x_nv_n,\\
		&\quad\vdots\\
		&w_n=x_1^{n-1}v_1+x_2^{n-1}v_2+\cdots+x_n^{n-1}v_n,
	\end{align*}
	entonces $\{w_1,\dots,w_n\}$ es una base de $V$.
\end{xca}

\begin{xca}[Polinomio interpolador]
	\label{xca:lagrange}
	Sean $x_1,\dots,x_{n+1}\in\K$ tales que $x_i\ne x_j$ si $i\ne j$ y sean
	$y_1,\dots,y_{n+1}\in\K$. Demuestre que el polinomio
	\begin{align*}
		f=\sum_{i=1}^{n+1}y_i\prod_{\substack{j=1\\j\ne i}}^{n+1}\frac{X-x_j}{x_i-x_j}
	\end{align*}
	es el único polinomio
	$f\in\K[X]$ de grado $\leq n$ tal que $f(x_i)=y_i$ para todo
	$i\in\{1,\dots,n+1\}$. 
\end{xca}

\begin{block}[Matriz compañera]
    \framebox{mover!}
	Vamos a demostrar que si 
	\[
		f=X^n+a_{n-1}X^{n-1}+\cdots+a_2X^2+a_1X+a_0\in\K[X]
	\]
	entonces 
	\begin{equation}
		\label{eq:C}
		f=
		\det\begin{pmatrix}
			X &  0 & 0 & 0 & \cdots & a_0\\
			-1 &  X & 0 & 0 & \cdots & a_1\\
			0 & -1 & X & 0 & \cdots & a_2\\
			\vdots & \vdots & \ddots & \ddots & \ddots & \vdots\\
			0 & 0 & 0 & -1 & X & a_{n-2} \\
			0 & 0 & 0 & 0 & -1 & X+a_{n-1} 
		\end{pmatrix}.
	\end{equation}

	La \textbf{matriz compañera} del polinomio $f$ es la matriz $C=(c_{ij})$
	que está dada por 
	\[
	c_{ij}=\begin{cases}
		1 & \text{si $i=j+1$ y $j<n$},\\
		-a_{i-1} & \text{si $j=n$},\\
		0 & \text{en otro caso}.
	\end{cases}
	\]
	La igualdad~\eqref{eq:C} puede escribirse entonces como $f=\det(XI-C)$,
	donde $C$ es la matriz compañera de $f$. 
	Para demostrar esta igualdad procederemos por inducción en $n$. El caso $n=2$ es fácil y se deja como
	ejercicio. Supongamos que la afirmación es válida para $n\geq2$. Si desarrollamos 
	el determinante de la matriz
	\[
	d(a_0,\dots,a_n)=
	\begin{pmatrix}
  		 X &  0 & 0 & 0 & \cdots & a_0\\
		-1 &  X & 0 & 0 & \cdots & a_1\\
		 0 & -1 & X & 0 & \cdots & a_2\\
		 \vdots & \vdots & \ddots & \ddots & \ddots & \vdots\\
		 0 & 0 & 0 & -1 & X & a_{n-1} \\
		 0 & 0 & 0 & 0 & -1 & X+a_{n} 
	 \end{pmatrix}
	\]
	por la primera fila, se tiene que
    \begin{align*}
        d(a_0,\dots,a_n)&=Xd(a_1,\dots,a_{n-1})+a_0(-1)^{n+1}(-1)^{n-1}\\
        &=X(X^{n-1}+a_{n-1}X^{n-2}+\cdots+a_1)+a_0\\
        &=X^n+a_{n-1}X^{n-1}+\cdots+a_1X+a_0.
    \end{align*}
\end{block}

\section{Anillos conmutativos con unidad}

\framebox{dar definición y un par de ejemplos básicos}

\framebox{matrices sobre anillos conmutativos}

\framebox{determinantes de matrices cuadradas (como ejercicio)}
\section{Ejercicios}

\begin{xca}
    \label{xca:determinante:A1000}
    Sea $A\in\K^{4\times4}$ una matriz tal que $\det A=-1$. Demuestre que el
    conjunto $\{A^{1000},A^{1001}\}$ es linealmente independiente. 
\end{xca}


\chapter{Diagonalización}

%\framebox{$f\in\K[X]$ y $A,B\in\K^{n\times n}$ tal que $f(AB)=0$ entonces $g(BA)=0$ para $g=Xf$}

\section{Autovalores y autovectores}

\begin{block}
    Sean $V$ un espacio vectorial y $f\in\hom(V,V)$. Un escalar $\lambda\in\K$
    es un \textbf{autovalor} de $f$ si existe $v\in V\setminus\{0\}$ tal que
    $f(v)=\lambda v$. El vector $v$ se denomina \textbf{autovector} de
    autovalor $\lambda$. 
\end{block}

\begin{examples}\
    \begin{enumerate}
        \item Si $\ker f\ne\{0\}$ entonces $\lambda=0$ es un autovalor pues
            para cada $v\in\ker f\setminus\{0\}$ se tiene $f(v)=0v$. 
        \item Si $f$ es un proyector no nulo entonces $\lambda=1$ es autovalor
            pues para cada $v\im f\setminus\{0\}$ se tiene que $f(v)=1v$. 
        \item Si $V=C^{\infty}(\R)$ y $\partial\colon V\to V$ es la aplicación $f\mapsto f'$ entonces
            todo $\lambda\in\R$ es autovalor pues $e^{\lambda x}\ne0$ para todo $x\in\R$ y 
            $\partial(e^{\lambda x})=\lambda e^{\lambda x}$. 
    \end{enumerate}
\end{examples}

\begin{prop}
	\label{pro:autovalores}
    Sean $V$ un espacio vectorial de dimensión finita sobre $\K$, $f\in\hom(V,V)$ y $\lambda\in\K$. 
    Las siguientes afirmaciones son equivalentes:
    \begin{enumerate}
        \item $\lambda$ es autovalor de $f$.
        \item $\lambda\id_V-f$ no es un isomorfismo.
		\item $\det(\lambda\id_V-f)=0$.
        %\item Si $\cB$ es una base de $V$ y $A=[f]_{\cB,\cB}$ entonces
        %    $\det(\lambda I-A)=0$.
    \end{enumerate} 

    \begin{proof}
		Para demostrar que $(1)$ implica $(2)$ basta observar que, por
		definición, como $\lambda$ es autovalor, existe $v\in V\setminus\{0\}$
		tal que $f(v)=\lambda v$. Luego
		$v\in\ker(\lambda\id_V-f)\setminus\{0\}$ y entonces $\lambda\id_V-f$ no
		es un isomorfismo.

		Demostremos que $(2)$ implica $(3)$. Sean $\cB$ una base de $V$ y
		$A=[f]_{\cB,\cB}$. Si $\lambda\id_V-f$ no es un isomorfismo entonces la
		matriz 
		\[
			\lambda I-A=[\lambda\id_V-f]_{\cB,\cB}
		\]
		es no inversible por el corolario~\ref{cor:iso<=>|f|inversible}.  Luego
		$\det(\lambda I-A)=0$ por el
		corolario~\ref{cor:no_inversible<=>detA=0}. 

		Demostremos que $(3)$ implica $(1)$. Supongamos que
		$\cB=\{v_1,\dots,v_n\}$. Como $\lambda I-A$ no es inversible, existen
		$\alpha_1,\dots,\alpha_n\in\K$ no todos cero tales que 
		\[
			(\lambda I-A)\colvec{3}{\alpha_1}{\vdots}{\alpha_n}=\colvec{3}{0}{\vdots}{0}.
		\]
		Luego $v=\sum_{i=1}^n\alpha_iv_i$ es no nulo y $f(v)=\lambda v$. 
    \end{proof}
\end{prop}

\begin{block}
	Sea $V$ un espacio vectorial y sea $f\in\hom(V,V)$. 
     El \textbf{espectro} $\spec f$ de $f$ es el conjunto 
    formado por los autovalores de $f$. 
\end{block}

%\begin{examples}
%	Sea $A=\begin{pmatrix}0&-1\\1&0\end{pmatrix}$
%\end{examples}
%
%\begin{xca}
%	Demuestre que $\spec A=\spec A^T$. 
%\end{xca}

\begin{xca}
    \label{xca:spec(fg)=spec(gf)}
    Sea $V$ un espacio vectorial de dimensión finita. Demuestre que si
    $f,g\in\hom(V,V)$ entonces $\spec(fg)=\spec(gf)$.
\end{xca}

\begin{example}
    Retomemos el ejemplo~\ref{exa:det:fibonacci} y 
    calculemos los autovalores de la matriz
    $A=\begin{pmatrix}0&1\\1&1\end{pmatrix}$. El polinomio característico es 
    \[
        \det(XI-A)=\det\begin{pmatrix}
        X & -1\\
        -1 & X-1
        \end{pmatrix}
        =X^2-X-1.
    \]
    Las raíces del polinomio $X^2-X-1$, son \[
        \lambda_1=(1+\sqrt{5})/2,
        \quad
        \lambda_2=(1-\sqrt{5})/2.
    \]
    Tenemos entonces que $\spec
    A=\{\lambda_1,\lambda_2\}$. Calculemos ahora el autoespacio del
    autovalor $\lambda_i$. Si a la matriz $\lambda_iI-A$ le aplicamos la
    operación de filas $F_1+\lambda_iF_2\to F_2$ vemos 
    que el conjunto
    \[
        \left\{\colvec{2}{1}{\lambda_1},\colvec{2}{1}{\lambda_2}\right\}
    \]
    es una base de $\K^{2\times1}$ formada por autovectores de $A$. Entonces
    $A=CDC^{-1}$, donde 
    \begin{align*}
        C=\begin{pmatrix}
            1 & 1\\
            \lambda_1 & \lambda_2
        \end{pmatrix},
        &&
        D=\begin{pmatrix}
            \lambda_1 & 0\\
            0 & \lambda_2
        \end{pmatrix},
        &&
        C^{-1}=\frac{1}{\sqrt{5}}\begin{pmatrix}
            -\lambda_2 & 1\\
            \lambda_1 & -1
        \end{pmatrix}.
    \end{align*}
    En particular, como $A^n=(CDC^{-1})^n=CD^nC^{-1}$, se obtiene una fórmula
    cerrada para la sucesión de Fibonacci:
    \[
    F_n=\frac{1}{\sqrt{5}}\left(\lambda_1^n-\lambda_2^n\right)
    \quad
    \text{para todo $n\geq1$}.
    \]
\end{example}

\section{El polinomio característico}

%\begin{block}
%    Sea $V$ un espacio vectorial de dimensión finita y sea $f\in\hom(V,V)$.
%    Sean $\cB$ y $\cB'$ bases ordenadas de $V$ y sean $A=[f]_{\cB,\cB}$ y
%    $B=[f]_{\cB',\cB'}$. Sea $C=C(\cB,\cB')$. Como $B=CAC^{-1}$,
%    \[
%    \det(XI-A)=\det(C(XI-A)C^{-1})=\det(XI-B).
%    \]
%    Tiene sentido entonces definir el \textbf{polinomio característico} de $f$
%    como el polinomio $\chi_f=\det(XI-A)\in\K[X]$, donde $A=[f]_{\cB,\cB}$ y
%    $\cB$ es alguna base de $V$. Luego $\lambda\in\K$ es autovalor de $f$ si y
%    sólo si $\lambda$ es raíz de $\chi_f$.
%\end{block}
%

\begin{block}
    Se define el \textbf{polinomio característico} de una matriz
    $A\in\K^{n\times n}$ como
    \[
        \chi_A=\det(XI-A)\in\K[X].
    \]
    De la definición es evidente que $\chi_f$ es un polinomio mónico de grado
    $n$ y que $\chi_A(0)=(-1)^n\det A$. 
\end{block}

\begin{lem}
    \label{lem:chiA=chiB}
    Sean $A,B\in\K^{n\times n}$. Si $A$ y $B$ son semejantes entonces
    \[
		\chi_A=\chi_B. 
	\]

    \begin{proof}
		Si existe $C\in\K^{n\times n}$ inversible tal que $B=CAC^{-1}$ entonces
		$\chi_A=\det(XI-A)=\det(C(XI-A)C^{-1})=\det(XI-B)=\chi_B$.
    \end{proof}
\end{lem}

\begin{block}
    Sean $V$ un espacio vectorial de dimensión finita y $f\in\hom(V,V)$.  El
    lema~\ref{lem:chiA=chiB} afirma que matrices semejantes tienen el mismo
    polinomio característico. Tiene sentido entonces definir el
    \textbf{polinomio característico} de $f$ como el polinomio característico
    de la matriz $[f]_{\cB,\cB}$, donde $\cB$ es alguna base ordenada de $V$. 
\end{block}

\begin{remark}
	En la proposición~\ref{pro:autovalores} se demostró entonces lo siguiente:
	si $V$ es un espacio vectorial de dimensión finita y $f\in\hom(V,V)$
	entonces $\lambda\in\K$ es autovalor de $f$ si y sólo si $\lambda$ es raíz
	del polinomio característico de $f$, es decir: $\chi_f(\lambda)=0$.
\end{remark}

\begin{example}
	Sea $f\colon\R^2\to\R^2$ dada por $f(x,y)=(-y,x)$. Entonces el polinomio
	característico $\chi_f=1+X^2$ no tiene raíces reales y $\spec f=\emptyset$. 
\end{example}

\begin{example}
	Sea $V$ un espacio vectorial complejo de dimensión finita $n$ y sea
	$f\in\hom(V,V)$.  Entonces, por el teorema fundamental del álgebra, $f$
	tiene exactamente $n$ autovalores (contados con multiplicidad).
\end{example}

\begin{example}
	Sea $V$ un espacio vectorial real de dimensión finita $n=2k+1$ y sea
	$f\in\hom(V,V)$.  Entonces $f$ tiene al menos un autovalor.
\end{example}

\begin{lem}
	Sean $V$ un espacio vectorial finita. Sea $f\in\hom(V,V)$ y sean
	$\lambda_1,\dots,\lambda_r\in\K$ autovalores de $f$ tales que
	$\lambda_i\ne\lambda_j$ si $i\ne j$. Sean $v_1,\dots,v_r$ tales que
	$f(v_i)=\lambda_i v_i$ para todo $i\in\{1,\dots,r\}$. Entonces
	$\{v_1,\dots,v_r\}$ es linealmente independiente.

	\begin{proof}
		Procederemos por inducción en $r$. Si $r=1$ el resultado es evidente
		pues $v_1\ne0$. Si suponemos que el resultado es válido para un cierto
		$r\geq1$ demostremos que es válido para $r+1$. Sean
		$\lambda_1,\dots,\lambda_{r+1}\in\K$ autovalores distintos entre sí y
		sean $v_1,\dots,v_{r+1}\in V$ tales que $f(v_i)=\lambda_i v_i$ para
		todo $i$. Supongamos que $\alpha_1v_1+\cdots+\alpha_{r+1}v_{r+1}=0$. Si
		$\alpha_{r+1}=0$ entonces $\alpha_1v_1+\cdots+\alpha_rv_r=0$ y luego
		$\alpha_i=0$ para todo $i$ por hipótesis inductiva. En cambio, si
		$\alpha_{r+1}\ne0$, entonces $v_{r+1}\in\langle v_1,\dots,v_r\rangle$
		pues 
		\[
		v_{r+1}=\beta_1v_1+\cdots+\beta_rv_r=\left(-\frac{\alpha_1}{\alpha_{r+1}}\right)v_1+\cdots+\left(-\frac{\alpha_r}{\alpha_{r+1}}\right)v_r.
		\]
		si $\beta_i=-\alpha_i/\alpha_{r+1}$.
		Como $f(v_{r+1})=\lambda_{r+1}v_{r+1}$, se tiene que 
		\[
		\beta_1(\alpha_1-\alpha_{r+1})v_1+\cdots+\beta_r(\alpha_r-\alpha_{r+1})v_r=0,
		\]
		y luego $\beta_i=0$ para todo $i$. Luego $\alpha_i=0$ para todo
		$i\in\{1,\dots,r\}$ y por lo tanto $\alpha_{r+1}v_{r+1}=0$, una
		contradicción.
	\end{proof}
\end{lem}

\begin{block}
	Si $V$ es de dimensión finita, $f\in\hom(V,V)$ y $\lambda\in\K$ es
	autovalor de $f$ entonces se define el \textbf{autoespacio} de $f$ asociado
	al autovalor $\lambda$ como 
    \[
        S(\lambda)=\{v\in V:f(v)=\lambda v\}.
    \]
    
	Queda como ejercicio demostrar que $S(\lambda)$ es un subespacio de $V$.
	Se define la
    \textbf{multiplicidad algebraica} de $\lambda$ como el mayor entero
    positivo $k$ tal que $(X-\lambda)^k$ divide a $\chi_f$.  La
    \textbf{multiplicidad geométrica} es el número $\dim S(\lambda)$. 
\end{block}

\begin{example}
	Sea $f\colon\R^3\to\R^3$ definida por $f(x,y,z)=(x,x+y,2z)$.  El polinomio
	característico de $f$ es $\chi_f=(X-1)^2(X-2)$. Queda como ejercicio
	demostrar que $S(1)=\langle(0,1,0)\rangle$ y $S(2)=\langle(0,0,1)\rangle$.
	La multiplicidad algebraica del autovalor $1$ es $2$ y la
	multiplicidad geométrica es $\dim S(1)=1$. 
\end{example}

\begin{lem}
	Sean $V$ un espacio vectorial finita. Sea $f\in\hom(V,V)$ y sean
	$\lambda_1,\dots,\lambda_r\in\K$ autovalores de $f$ tales que
	$\lambda_i\ne\lambda_j$ si $i\ne j$. Entonces
	\[
		S(\lambda_1)+\cdots+S(\lambda_r)=S(\lambda_1)\oplus\cdots\oplus S(\lambda_r).
	\]
\end{lem}
\begin{lem}
	Sean $V$ un espacio vectorial de dimensión finita, $f\in\hom(V,V)$ y
	$\lambda$ un autovalor de $f$. Sea $m$ la multiplicidad algebraica de
	$\lambda$. Entonces 
	\[
		\dim S(\lambda)\leq m.
	\]
	\begin{proof}
		Supongamos que $x_f=(X-\lambda)^mq$, donde $q\in\K[X]$ y
		$q(\lambda)\ne0$. Si $\dim S(\lambda)=r\geq m+1$, sea
		$\{v_1,\dots,v_r\}$ una base de $S(\lambda)$. La extendemos a una base
		$\cB=\{v_1,\dots,v_r,v_{r+1},\dots,v_n\}$ de $V$ y consideramos la
		matriz de $f$ con respecto a $\cB$:
		\[
		[f]_{\cB,\cB}=
		\left(
			\begin{array}{c|c}
			\lambda I & \star\\
			\hline
			0 & B
		\end{array}
		\right),
		\]
		donde $\lambda I$ es una matriz de $r\times r$ y $B$ es una matriz de
		$(n-r)\times(n-r)$. Entonces
		\[
		\chi_f=\det(X-\lambda)^r\det(XI-B)=(X-\lambda)^m(X-\lambda)^{r-m}\det(XI-B),
		\]
		donde $r-m\geq1$. Luego $(X-\lambda)^{r-m}\det(XI-B)$ es un polionimo
		que se anula en $\lambda$, una contradicción.
	\end{proof}
\end{lem}

\begin{example}
    Sea $f\colon\R^{3\times1}\to\R^{3\times1}$ dada por 
    \[
        \colvec{3}{x}{y}{z}\mapsto\begin{pmatrix}
            1 & 0 & 0\\
            1 & 1 & 0\\
            0 & 0 & 2
        \end{pmatrix}
        \colvec{3}{x}{y}{z}=\colvec{3}{x}{x+y}{2z}.
    \]
    Entonces $X_f=(X-1)^2(X-2)$ y además
    \begin{align*}
        & S(1)=\{(x,y,z)^T\in\R^{3\times1}:x=z=0\}=\langle (0,1,0)^T\rangle,\\
        & S(2)=\{(x,y,z)^T\in\R^{3\times1}:x=y=0\}=\langle (0,0,1)^T\rangle.
    \end{align*}
    Luego $\dim S(1)=\dim S(2)=1$. 
    \framebox{VER}
\end{example}

\begin{example}
    \label{exa:f_diagonalizable}
    Sea $f\colon\R^{3\times1}\to\R^{3\times1}$ dada por     
    \[
        \colvec{3}{x}{y}{z}\mapsto\begin{pmatrix}
            1 & 0 & 0\\
            0 & 0 & 1\\
            0 & 0 & -1
        \end{pmatrix}
        \colvec{3}{x}{y}{z}=\colvec{3}{x}{z}{-z}.
    \]
    entonces $x_f=X(X-1)(X+1)$ y 
    \begin{align*}
        &S(0)=\langle (0,1,-1)^T\rangle,
        &&S(1)=\langle (0,1,0)^T\rangle,
        &&S(-1)=\langle (1,0,0)^T\rangle.
    \end{align*}
    \framebox{VER}
\end{example}

\begin{block}
    Sean $V$ un espacio vectorial de dimensión finita y $f\in\hom(V,V)$. Se
    dice que $f$ es \textbf{diagonalizable} si existe una base $\cB$ de $V$ tal
    que $[f]_{\cB,\cB}$ es una matriz diagonal. 
\end{block}

\begin{example}
    La transformación lineal del ejemplo~\ref{exa:f_diagonalizable} es
    diagonalizable ya que si $\cB=\{(0,1,-1)^T,(0,1,0)^T,(1,0,0)^T\}$ entonces
    \[
    [f]_{\cB}=\begin{pmatrix}
        0 & 0 & 0\\
        0 & 1 & 0\\
        0 & 0 & -1
    \end{pmatrix}
    \]
\end{example}

\begin{thm}
    Sea $V$ un espacio vectorial de dimensión finita sobre $\K$, sea
    $f\in\hom(V,V)$ y sean $\lambda_1,\dots,\lambda_k$ autovalores distintos de
    $f$. Entonces $f$ es diagonalizable si y sólo si $\dim S(\lambda_i)=m_i$
    para todo $i\in\{1,\dots,k\}$ y
    \[
        \chi_f=(X-\lambda_1)^{m_1}\cdots(X-\lambda_k)^{m_k}.
    \]

	\begin{proof}
		Supongamos primero que $\dim S(\lambda_i)=m_i$ para todo
		$i\in\{1,\dots,k\}$ y que 
		\[
		\chi_f=(X-\lambda_1)^{m_1}\cdots(X-\lambda_k)^{m_k}.
		\]
		Sean $\cB_1=\{u_{i},\dots,u_{m_1}\}$ una base de $S(\lambda_1)$,
		$\cB_2=\{v_1,\dots,v_{m_2}\}$ una base de $S(\lambda_2)$\ldots y
		$\cB_k=\{w_1,\dots,w_{m_k}\}$ una base de $S(\lambda_k)$. Vamos a
		demostrar que $\cB_1\cup\cdots\cup\cB_k$ es base de $V$, y para esto,
		como $m_1+\cdots+m_k=n$, basta ver que $\cB_1\cup\cdots\cup\cB_k$ es un
		conjunto linealmente independiente. Supongamos que 
		\[
		\alpha_1u_1+\cdots+\alpha_{m_1}u_{m_1}+\beta_1v_1+\cdots+\beta_{m_2}v_{m_2}+\cdots+\gamma_1 w_1+\cdots+\gamma_{m_k}w_{m_k}=0.
		\]
		Observemos que 
		\begin{align*}
		&\alpha_1u_1+\cdots+\alpha_{m_1}u_{m_1}\in S(\lambda_1),\\
		&\beta_1v_1+\cdots+\beta_{m_2}v_{m_2}\in S(\lambda_2),\\
		&\quad\vdots\\
		&\gamma_1 w_1+\cdots+\gamma_{m_k}w_{m_k}\in S(\lambda_k).
		\end{align*}
		Luego, como los $S(\lambda_i)$ están en suma directa, $\alpha_i=0$ para todo $i$, 
		$\beta_j=0$ para todo $j$\ldots, y $\gamma_k=0$ para todo $k$.

		Recíprocamente, si $f$ es diagonalizable, existe una base
		$\{v_1,\dots,v_n\}$ de $V$ tal que la matriz de $f$ en esa base es
		diagonal. Al agrupar los autovalores iguales  se obtiene
		$X_f=\prod_{i=1}^r (X-\lambda_i)^{m_i}$. Es claro que $\dim
		S(\lambda_i)\geq m_i$. Por otro lado, siempre vale que $\dim
		S(\lambda_i)\leq m_i$. Luego $\dim S(\lambda_i)=m_i$. 
	\end{proof}
\end{thm}

\begin{block}
	Una matriz $A\in\K^{n\times n}$ es \textbf{diagonalizable} si la
	transformación lineal $f\colon\K^{n\times 1}\to\K^{n\times1}$ dada por
	$x\mapsto Ax$ es diagonalizable.
\end{block}

\begin{xca}
	\label{xca:matriz_diagonalizable}
	Sea $A\in\K^{n\times n}$. Demuestre que las siguientes afirmaciones son equivalentes:
	\begin{enumerate}
		\item $A$ es diagonalizable.
		\item $\chi_A=\prod_{i=1}^r(X-\lambda_i)^{m_i}$ y $m_i=n-\rg(\lambda_iI-A)$. 
		\item $A$ es semejante a una matriz diagonal.
	\end{enumerate}
\end{xca}

\section{Polinomios e ideales}

\begin{thm}[Algoritmo de división]
	Sean $f,d\in\K[X]$ con $d\ne0$. Entonces existen únicos $q,r\in\K[X]$ tales
	que $f=dq+r$ y además $r=0$ o bien $\deg r<\deg d$. 
%
%	\begin{proof}
%		Para la demostración referimos por ejemplo a la página 315 del libro de
%		Gentile.
%	\end{proof}
\end{thm}

\begin{block}
	Sean $f,g\in\K[X]$. Recordemos que $g$ \textbf{divide} a $f$ (o que $f$ es
	\textbf{divisible} por $g$, o que $g$ es un \textbf{múltiplo} de $f$) si
	existe $h\in\K[X]$ tal que $f=gh$. Un escalar $\lambda\in\K$ es
	\textbf{raíz} (o \textbf{cero}) de $f$ si $f(\lambda)=0$. 
\end{block}

\begin{cor}
	Sea $f\in\K[X]$ y sea $\lambda\in\K$. Entonces $f$ es divisible por
	$X-\lambda$ si y sólo si $f(\lambda)=0$.

	\begin{proof}
		Si $f$ es divisible por $X-\lambda$ entonces $f=(X-\lambda)q$ para
		algún $q\in\K[X]$. Al evaluar en $\lambda$ se obtiene entonces
		$f(\lambda)=0$. Recíprocamente, si utilizamos el algoritmo de división,
		existen únicos $q,r\in\K[X]$ tales que $f=(X-\lambda)q+r$. Como
		$f(\lambda)=0$, entonces $r(\lambda)=0$ y luego $r=0$. 
	\end{proof}
\end{cor}

\begin{cor}
	Sea $f\in\K[X]$ de grado $n$. Entonces $f$ tiene a lo sumo $n$ raíces en
	$\K$.

	\begin{proof}
		Procedemos por inducción en $n$. Si $n=1$ entonces el resultado es
		trivialmente válido. Suponemos entonces que el resultado vale para
		algún $n\geq1$ y sea $f$ es un polinomio de grado $n+1$. Si $f$ no
		tiene raíces en $\K$ entonces no hay nada que demostrar.  Si
		$x_0\in\K$ es raiz de $f$, existe $g\in\K[X]$ con $\deg g=n$ tal
		que $f=(X-x_0)g$. Por hipótesis inductiva $g$ tiene a lo sumo $n$
		raíces y entonces $f$ tiene a lo sumo $n+1$ raíces en $\K$. 
	\end{proof}
\end{cor}

\begin{block}
	Un \textbf{ideal} de $\K[X]$ es un subespacio $I\subseteq\K[X]$ tal que
	$fg\in I$ si $f\in I$ y $g\in\K[X]$.
\end{block}

\begin{examples}
    El subespacio $\{0\}$ es un ideal de $\K[X]$. Si $g\in\K[X]$ entonces
    el conjunto 
    \[
        \{fg:f\in\K[X]\}
    \]
    es un ideal de $\K[X]$ que se denota por $(g)$ y se denomina el ideal
    generado por $g$.  Más generalmente, si $g_1,\dots,g_n\in\K[X]$, el conjunto
    \[
    (g_1,\dots,g_n)=\left\{\sum_{i=1}^n f_ig_i:f_1,\dots,f_n\in\K[X]\right\}
    \]
    es un ideal de $\K[X]$ que se denomina \textbf{ideal generado} por
    $g_1,\dots,g_n$.
\end{examples}

\begin{example}
	Demostremos que el ideal 
    \[
        I=(X-8,X^2+8X+5)\subseteq\R[X]
    \]
	es igual $\R[X]$.  Como $I$ es un ideal, $X^2-8X=X(X-8)\in I$. De la misma
	forma, como los polinomios $X^2-8X$ y $X^2+8X+5$ son elementos de $I$, se
	tiene que $5=X^2-8X-(X^2+8X+5)\in I$. Luego, como $1=(1/5)5\in I$, se tiene
	que $I=\R[X]$ ya que para todo $f\in\R[X]$ tenemos $f=f1\in I$. 
\end{example}

\begin{thm}
	Sea $I\subseteq\K[X]$ un ideal no nulo. Entonces existe un único polinomio
	mónico $g\in\K[X]$ tal que $I=(g)$. 

    \begin{proof}
		Demostremos la existencia.  Como $I\ne\{0\}$, existe un polinomio no
		nulo que pertenece a $I$. Sea entonces $g$ el polinomio mónico de grado
		mínimo tal que $g\in I$. Para demostrar que $I=(g)$ sea $f\in I$. Como
		$g$ tiene grado mínimo entre los polinomios de $I$, $\deg f\geq \deg g$
		y entonces, por el algoritmo de división, existen $h,r\in\K[X]$ tal que
		$f=gh+r$ donde $r=0$ o $\deg r<\deg g$. Como $I$ es un ideal,
		$r=f-gh\in I$.  La minimalidad del grado de $g$ implica que $r=0$ y
		entonces $f=gh\in(g)$. 

		Demostremos la unicidad. Si $I=(g_1)=(g_2)$, existen $h_1,h_2\in\K[X]$
		tales que $g_1=h_1g_2$ y $g_2=h_2g_1$. Entonces $g_1=(h_1h_2)g_2$. Como
		$g_1$ y $g_2$ tienen el mismo grado, $0=\deg(h_1h_2)=\deg h_1+\deg h_2$
		y entonces $\deg h_1=\deg h_2=0$.  Como $g_1$ y $g_2$ son mónicos,
		$h_1=h_2=1$.
    \end{proof}
\end{thm}

\begin{block}
	Recordemos que si $f,g\in\K[X]$ entonces se define el \textbf{máximo común
	divisor} entre $f$ y $g$ como el único polinomio mónico $h\in\K[X]$ tal que 
	\begin{enumerate}
		\item $h$ divide a $f$, $h$ divide a $g$,
		\item $h$ es múltiplo de cada polinomio que divida a $f$ y a $g$. 
	\end{enumerate}

	El máximo común divisor entre $f$ y $g$ se denota por $(f:g)$. 

	Similarmente se define el \textbf{mínimo común múltiplo} de $f$ y $g$ como
	el único polinomio mónico $h\in\K[X]$ tal que: 
	\begin{enumerate}
		\item $f$ divide a $h$, $g$ divide a $h$,
		\item $h$ divide a cada polinomio que es múltiplo de $f$ y de $g$.
	\end{enumerate}

	El mínimo común múltiplo entre $f$ y $g$ se denota por $[f:g]$. 
	
	Observemos
	que vale 
	\[
		fg=(f:g)[f:g].
	\]
\end{block}

\begin{xca}
	Sean $f_1,\dots,f_n\in\K[X]$.  Demuestre que el generador $g$ del ideal
	$(f_1,\dots,f_n)$ es el \textbf{máximo común divisor} de los polinomios
	$f_1,\dots,f_n$, es decir: $g\in\K[X]$ es el único polinomio mónico tal que
	\begin{enumerate}
		\item $g\in(f_1,\dots,f_n)$,
		\item $g$ divide a cada $f_i$,
		\item $g$ es divisible por todo polinomio que divida a cada uno de los $f_i$.
	\end{enumerate}
\end{xca}

%%% FIXME: igualdad del minimal y caracteristico no implica semejanza

\section{El polinomio minimal}

\begin{block}
	Dados un polinomio $p\in\K[X]$, 
	\[
		p=a_0+a_1X+\cdots+a_dX^d=\sum_{i=0}^d a_iX^i, 
	\]
	y una matriz $A\in\K^{n\times n}$ se define 
	\[
		p(A)=a_0I+a_1A+\cdots+a_nA^n\in\K^{n\times n}.
	\]
	
	Es evidente que si $p,q\in\K[X]$ y $A\in\K^{n\times n}$ entonces
	\begin{align*}
		&(p+q)(A)=p(A)+q(A),\\
		&(pq)(A)=p(A)q(A)=q(A)p(A),
	\end{align*}
	para todo $p,q\in\K[X]$ y $A\in\K^{n\times n}$. Más generalmente, puede
	demostrarse que el conjunto
	\[
	\{B\in\K^{n\times n}:B=p(A)\text{ para algún $p\in\K[X]$}\}
	\]
	es un anillo conmutativo con unidad. 
\end{block}

\begin{examples}
	Sea $A\in\K^{n\times n}$. 
	Si $p=X^2+1\in\K[X]$ entonces 
	\[
	p(A)=A^2+I.
	\]
	Si $q=X^3+2X^2-3X+4$ entonces 
	\[
	q(A)=A^3+2A^2-3A+4I.
	\]
\end{examples}

\begin{lem}
    Sea $A\in\K^{n\times n}$. Entonces existe $p\in\K[X]$ no nulo tal que
    $p(A)=0$. 
\end{lem}

\begin{proof}
    El conjunto $\{I,A,A^2,\dots,A^{n^2}\}\subseteq\K^{n\times n}$ tiene
    $n^2+1$ elementos. Como $\dim\K^{n\times n}=n^2$, el conjunto $S$ es
    linealmente dependiente. Luego existen escalares $a_0,\dots,a_{n^2}\in\K$,
    no todos cero, tales que $a_0I+a_1A+\cdots+a_{n^2}A^{n^2}=0$.  En
    conclusión, si $p=\sum_{i=0}^{n^2}a_iX^i\in\K[X]$, entonces $p\ne0$ y
    $p(A)=0$.
\end{proof}

\begin{block}
	Sea $A\in\K^{n\times n}$. El conjunto 
	\[
		\{p:p\in\K[X]\text{ tal que $p(A)=0$}\}
	\]
	es un ideal de $\K[X]$ y entonces existe un único polinomio mónico que lo
	genera, es decir: existe un único polinomio mónico $m_A$ que lo genera. 
	Este polinomio se denomina \textbf{polinomio minimal} de $A$. 
\end{block}

\begin{xca}
	\label{xca:minimal}
	Sea $A\in\K^{n\times n}$ y sea $m\in\{1,\dots,n\}$ el único entero que
	satisface que $\{I,A,\dots,A^{m-1}\}$ es linealmente independiente y que
	$A^m\in\langle I,A,\dots,A^{m-1}\rangle$. Sean $a_0,a_1,\dots,a_{m-1}\in\K$
	los únicos escalares tales que $A^m=a_0I+a_1A+\cdots+a_{m-1}A_{m-1}$.
	Demuestre que el polinomio \[
        X^m-\sum_{i=0}^{m-1}a_iX^i
    \]
    es el polinomio minimal de $A$.
\end{xca}

\begin{examples}
	Es fácil verificar que $m_0=X$ y que $m_I=X-1$. Si $A=\lambda I$ entonces
	$m_A=X-\lambda$. Si $A\not\in\{0,I\}$ satisface que $A^2=A$ entonces
	$m_A=X^2-X$.
\end{examples}

\begin{example}
	Calculemos el polinomio minimal de 
	\[
		A=\begin{pmatrix}-1&0\\1&-1\end{pmatrix}\in\R^{2\times2}.
	\]
	Primero observemos que el conjunto $\{I,A\}$ es linealmente independiente.
	Buscamos entonces $p=X^2+bX+c\in\R[X]$ que anule a la matriz $A$. Un
	cálculo sencillo muestra que $A^2+bA+cI=0$ si y sólo si $b=2$ y $c=1$.
	Luego $m_A=X^2+2X+1$. 
\end{example}

\begin{remark}
	\label{rem:m_A|p}
    De la definición de polinomio minimal se obtiene inmediatamente que si
    $A\in\K^{n\times n}$ y $p\in\K[X]$ entonces $p(A)=0$ si y sólo si $m_A$
    divide a $p$.
%
%	\begin{proof}
%		Si $p(A)=0$ entonces por definición $p\in(m_A)$ y luego existe
%		$q\in\K[X]$ tal que $p=m_Aq$. Luego $m_A$ divide a $p$. Recíprocamente,
%		si $m_A$ divide a $p$, $p=m_Aq$ para algún $q\in\K[X]$. Luego
%		$p(A)=m_A(A)q(A)=0q(A)=0$.
%	\end{proof}
\end{remark}

\begin{lem}
	\label{lem:A_sem_B=>p(A)_sem_p(B)}
	Sean $A,B\in\K^{n\times n}$ y $p\in\K[X]$. Si $A$ y $B$ son semejantes
	entonces $p(A)$ y $p(B)$ son semejantes. En particular, si $A$ y $B$ son
	semejantes entonces $p(A)=0$ si y sólo si $p(B)=0$.

	\begin{proof}
		Supongamos que $p=\sum_{i=0}^d a_iX^i$ y que $A=CBC^{-1}$ para alguna
		matriz inversible $C\in\K^{n\times n}$. Como $(CBC^{-1})^i=CB^iC^{-1}$
		para todo $i\geq0$, 
		\begin{align*}
			p(A)&=p(CBC^{-1})=\sum_{i=0}^d a_i(CBC^{-1})^i\\&=\sum_{i=0}^d a_iCB^iC^{-1}
			=C\left(\sum_{i=0}^d a_iB^i\right)C^{-1}=Cp(B)C^{-1}.
		\end{align*}
		Luego $p(A)$ y $p(B)$ son semejantes. Si $A$ y $B$ son semejantes y
		entonces $p(A)$ y $p(B)$ son semejantes y luego, en particular,
		$p(A)=0$ si y sólo si $p(B)=0$. 
	\end{proof}
\end{lem}

\begin{prop}
	Sean $A,B\in\K^{n\times n}$. Si $A$ y $B$ son semejantes entonces $m_A=m_B$.

	\begin{proof}
        Por el lema~\ref{lem:A_sem_B=>p(A)_sem_p(B)}, las matrices $m_A(A)$ y
        $m_B(A)$ son semejantes y entonces, como $0=m_A(A)$, se obtiene que
        $m_B(A)=0$. La observación~\ref{rem:m_A|p} implica que $m_A$ divide a
        $m_B$.  Similarmente se demuestra que $m_B$ divide a $m_A$ y luego,
        como $m_A$ y $m_B$ son mónicos, $m_A=m_B$.
	\end{proof}
\end{prop}

\begin{block}
	La proposición anterior nos permite definir el \textbf{polinomio minimal}
	de una transformación lineal $f\colon V\to V$, donde $V$ es un espacio
	vectorial de dimensión finita. En efecto, basta tomar $m_f$ como
	$m_{[f]_{\cB}}$ donde $\cB$ es alguna base de $V$.
\end{block}

\begin{prop}
	Sea $A\in\K^{n\times n}$. Entonces $\lambda$ es autovalor de $A$ si y sólo
	si $\lambda$ es raíz de $m_A$.

	\begin{proof}
		Supongamos primero que $\lambda$ es autovalor de $A$. Entonces existe
		$v\ne0$ tal que $Av=\lambda v$. Por el algoritmo de división, existen
		$q,r\in\K[X]$ tales que $m_A=(X-\lambda)q+r$, donde $r=0$ o bien
		$\deg(r)=0$. Al evaluar en la matrix $A$ obtenemos $0=m_A(A)=(A-\lambda
		I)q(A)+rI$ y en particular $0=(A-\lambda I)q(A)v+rv$. Como $A-\lambda
		I$ y $q(A)$ conmutan, $rv=0$. Como $v\ne0$ entonces $r=0$.

		Recíprocamente, si $m_A(\lambda)=0$ entonces $m_A=(X-\lambda)q$ para
		algún $q\in\K[X]$. Como $\deg(q)<\deg(m_A)$, $q(A)\ne0$. Existe
		entonces $w\in\K^n$ tal que $q(A)w\ne0$. Sea $v=q(A)w$. Entonces
		$(A-\lambda I)v=0$.
	\end{proof}
\end{prop}

\section{El polinomio minimal de un vector}

\begin{block}
	Sean $A\in\K^{n\times n}$ y $v\in\K^{n\times1}$. Para cada polinomio
	$p\in\K[X]$ se define $p(v)=p(A)v$. En particular, diremos que el polinomio
	$p$ se anula (con respecto a la matriz $A$) en $v$ si $p(v)=0$.  El
	conjunto 
	\[
		\{p\in\K[X]:p(v)=0\}
	\]
	es un ideal de $\K[X]$ y entonces está generado por un único polinomio
	mónico $m_v$. Este polinomio se denomina el \textbf{polinomio minimal} del
	vector $v$ con respecto a la matriz $A$. Observar que $m_v$ es el polinomio
	mónico de menor grado que especializado en $A$ se anula en $v$.
\end{block}

\begin{example}
	Sean $A\in\K^{n\times n}$ y $v\in\K^{n\times1}$. Entonces $m_v=X-\lambda$ si y sólo si $v$ es
	autovector de $A$ de autovalor $\lambda$. 
\end{example}

\begin{example}
	Sean $v=(1,0)^T$ y 
	\[
		A=\begin{pmatrix}
			-1 & 0\\
			1 & -1
		\end{pmatrix}\in\R^{2\times2}.
	\]
	Calculemos el polinomio minimal de $v$ con respecto a la matriz $A$.
	Primero buscamos $p=X+c\in\R[X]$ tal que $p(v)=0$.  Esto equivale a
	resolver la ecuación $Av^T=-cv^T$, que no tiene solución.  Buscamos
	entonces un polinomio $X^2+bX+c\in\R[X]$ que anula al vector $v$. Un
	cálculo sencillo muestra que $p(v)=0$ si y sólo si $b=2$ y $c=1$.  Luego
	$m_v=X^2+2X+1$. 
\end{example}

\begin{remark}
    \label{rem:m_v|m_A}
	De la definición se obtiene inmediatamente que si $A\in\K^{n\times n}$,
	$v\in\K^{n\times1}$ y $p\in\K[X]$ entonces $p(v)=0$ si y sólo si $m_v$
	divide a $p$.  En particular, $m_v$ divide a $m_A$.
\end{remark}

%\begin{prop}
%	\label{prop:m_v|m_A}
%	Sean $A\in\K^{n\times n}$, $v\in\K^n$ y $p\in\K[X]$. Entonces $p(v)=0$ si y
%	sólo si $m_v$ divide a $p$. En particular, $m_v$ divide a $m_A$.
%
%	\begin{proof}
%		Por el algoritmo de división existen $p,r\in\K[X]$ tales que $p=m_vq+r$,
%		donde $r=0$ o bien $\deg(r)<\deg(m_v)$. Luego, si suponemos que $p(v)=0$,
%		entonces  $0=p(v)=r(A)v=r(v)$, y la minimalidad del grado de $m_v$ implica
%		que $r=0$. Recíprocamente, si $p=m_vq$ para algún $q\in\K[X]$ entonces
%		$p(v)=0$.
%	\end{proof}
%\end{prop}

\begin{prop}
	\label{pro:mA=mcm}
	Sean $A\in\K^{n\times n}$ y $\{v_1,\dots,v_n\}$ una base de
	$\K^{n\times1}$.  Entonces $m_A=\lcm(m_{v_1},\dots,m_{v_n})$. 
	
	\begin{proof}
		Sea $p=\lcm(m_{v_1},\dots,m_{v_n})$. La observación~\ref{rem:m_v|m_A}
		implica que $m_{v_i}$ divide a $m_A$ para todo $i\in\{1,\dots,n\}$ y
		entonces $p$ divide a $m_A$.  Por otro lado, cada $m_{v_i}$ divide a
		$p$ y entonces $p(A)v_i=p(v_i)=0$ para todo $i\in\{1,\dots,n\}$. Como
		$\{v_1,\dots,v_n\}$ es una base de $\K^{n\times1}$, entonces $p(A)=0$ y
		luego $m_A$ divide a $p$.  Como $p$ y $m_A$ son polinomios mónicos,
		$p=m_A$. 
	\end{proof}
\end{prop}

%\begin{xca}
%	\label{xca:m_A=m_v=chi_A}
% 	Sea $A\in\K^{n\times n}$ y sea $v\in\K^{n\times1}$ no nulo. Demuestre que si 
%	$\{v,Av,\dots,A^{n-1}v\}$ es una base de $\K^{n\times1}$ entonces
%	$m_v=m_A=\chi_A$.
%\end{xca}

\begin{xca}
	Sea $\{e_1,e_2,e_3\}$ la base canónica de $\R^{3\times1}$ y sea
	\[
		A=\begin{pmatrix}
			1 & 0 & 0\\
			1 & 1 & 0\\
			0 & 0 & 2
		\end{pmatrix}\in\R^{3\times3}.
	\]
	Demuestre que $m_{e_1}=X^2-2X+1$, $m_{e_2}=X-1$, $m_3=X-2$ y concluya que
	$m_A=(X-1)^2(X-2)$. ¿Es $A$ diagonalizable?
\end{xca}

\begin{prop}
	\label{pro:minimal_diagonalizable}
	Sea $A\in\K^{n\times n}$. Entonces $A$ es diagonalizable si y sólo si 
	%$V$ un espacio vectorial de dimensión finita y sea $f\in\hom(V,V)$.
	%Entonces $f$ es diagonalizable si y sólo si 
	\[
		m_A=(X-\lambda_1)\cdots(X-\lambda_r) 
	\]
	donde $\lambda_i\ne\lambda_j$ si $i\ne j$. 

	\begin{proof}
		Supongamos que $A$ es diagonalizable. Entonces existe una base
		$\{v_1,\dots,v_n\}$ de $\K^{n\times1}$ y existen
		$\lambda_1,\dots,\lambda_n\in\K$ tales que $Av_i=\lambda_i v_i$ para
		todo $i$. Luego $m_{v_i}=X-\lambda_i$ para todo $i$. La
		proposición~\ref{pro:mA=mcm} implica que $m_A$ tiene la forma deseada. 

		Supongamos ahora que $m_A=(X-\lambda_1)\cdots(X-\lambda_r)$ donde los
		$\lambda_i$ son todos distintos entre sí.  Entonces
		$\lambda_1,\dots,\lambda_r$ son raíces distintas de $\chi_f$ y por lo
		tanto son autovalores de $f$. Luego 
		\[
			S(\lambda_1)+\cdots+S(\lambda_r)=S(\lambda_1)\oplus\cdots\oplus S(\lambda_r).
		\]
		Vamos a demostrar que $V=\oplus_{i=1}^r S(\lambda_i)$. Para cada $j\in\{1,\dots,r\}$ definimos
		\[
			p_j=\frac{1}{X-\lambda_j}\prod_{i=1}^r(X-\lambda_i).
		\]
		Como $(p_1:\cdots:p_r)=1$, existen $h_1,\dots,h_r\in\K[X]$ tales que
		$1=\sum_{i=1}^rh_ip_i$. Si evaluamos en la matriz $A$ obtenemos
		\[
			I=\sum_{i=1}^r h_i(A) p_i(A).
		\]
		Al multiplicar a izquierda por $v$ obtenemos $v=\sum_{i=1}^r h_i(A)p_i(A)v$.
		Veamos que $h_j(A)p_j(A)v\in\ker(A-\lambda_jI)$. Como $(X-\lambda_j)p_j=m_A$, entonces 
		$(X-\lambda_j)h_jp_j=h_jm_A$. Al evaluar en $A$ y utilizar que $m_A(A)=0$, 
		\[
			(A-\lambda_jI)h_j(A)p_j(A)=h_j(A)m_A(A)=0.
		\]
		Luego $(A-\lambda_jI)h_j(A)p_j(A)v=0$ y entonces $h_j(A)p_j(A)v\in
		S(\lambda_j)$.
	\end{proof}
\end{prop}

\begin{example}
	Sea $\{e_1,e_2,e_3\}$ la base canónica de $\R^{3\times1}$ y sea
	\[
		A=\begin{pmatrix}
			1 & 1 & 1\\
			0 & 0 & -1\\
			0 & 0 & 0
		\end{pmatrix}.
	\]

	Calculemos $m_{e_1}$, $m_{e_2}$ y $m_{e_3}$. Como $Ae_1=e_1$ entonces
	$m_{e_1}=X-1$. Además $Ae_2=e_1$ y $A^2e_2=Ae_1=e_1=Ae_2$ y entonces
	$m_{e_2}=X^2-X$. Como $Ae_3=e_1-e_2$ entonces $A^2e_3=0$ y luego
	$m_{e_3}=X^2$. Observemos que $m_A=[m_{e_1}:m_{e_2}:m_{e_3}]=X^2(X-1)$ y
	entonces $A$ no es diagonalizable.
\end{example}

\begin{example}
	\label{exa:A^k=I}
	Sea $A\in\C^{n\times n}$ y supongamos que existe $k\in\N$ tal que $A^k=I$.
	Entonces $A$ es diagonalizable. En efecto, como el polinomio $X^k-1$ anula
	a la matriz $A$, sabemos que $m_A$ divide a $X^k-1$. Como todas las raíces
	de $X^k-1$ son simples, todas las raíces de $m_A$ también son simples.
	Luego $A$ es diagonalizable por la
	proposición~\ref{pro:minimal_diagonalizable}.
\end{example}

\begin{example}
	El resultado del ejemplo~\ref{exa:A^k=I} no vale para matrices reales. Sea 
	\[
		A=\begin{pmatrix}
			0 & 0 & 1\\
			1 & 0 & 0\\
			0 & 1 & 0
		\end{pmatrix}\in\R^{3\times3}.
	\]
	Entonces $A^3=I$ pero $A$ no es diagonalizable ya que 
	\[
		m_A=(X-1)(X^2+X+1)\in\R[X]. 
	\]
\end{example}


\begin{xca}
	Sea $V$ un $\C$-espacio vectorial de dimensión finita y sea
	$f\in\hom(V,V)$. Demuestre que $f$ es diagonalizable si y sólo si $m_f$ y
	$m'_{f}$ son coprimos.
\end{xca}

\section{El teorema de Cayley--Hamilton}

\framebox{como ejercicio y para endomorfismos}

\begin{thm}[Cayley--Hamilton]
	\label{thm:Hamilton_Cayley}
    Si $A\in\K^{n\times n}$ entonces $\chi_A(A)=0$, es decir: $m_A$ divide a
    $\chi_A$. 
\end{thm}

\begin{proof}
	Sea $T\colon\K^{n\times1}\to\K^{n\times1}$ la transformación lineal
	definida por $x\mapsto Ax$ y sea $v\in\K^{n\times1}$.  Tomemos $k\in\N$ tal
	que $\{v,Av,\dots,A^kv\}$ sea un conjunto linealmente independiente y
	$A^{k+1}v$ sea combinación lineal de $v,Av,\dots,A^kv$, digamos 
	\[
	A^{k+1}v=-a_0v-a_1Av-\cdots-a_kA^kv, 
	\]
	donde $a_0,\dots,a_k\in\K$.
	Extendemos el conjunto $\{v,Av,\dots,A^kv\}$ a una base 
	\[
	\cB=\{v,Av,\dots,A^kv,w_1,\dots,w_m\}
	\]
	de $V$, donde $k+1+m=n$. La matriz de $T$ con respecto a la base $\cB$ es
	una matriz por bloques
	\[
	[T]_{\cB,\cB}=\left(\begin{array}{c|c}
		C & \star\\
		\hline
		0 & M
	\end{array}\right),\qquad
	C=\left(\begin{array}{ccccc}
		0 &  &  &  & -a_{0}\\
		1 & 0 &  &  & -a_{1}\\
		& 1 & \ddots &  & \vdots\\
		&  & \ddots & 0 & -a_{k-1}\\
		&  &  & 1 & -a_{k}
	\end{array}\right).
	\]
	El polinomio característico de $A$ es 
	\[
	\chi_A=\chi_{[T]_{\cB,\cB}}=\chi_C\chi_M.
	\]
	Pero sabemos que $\chi_C=X^{k+1}+a_{k}X^{k}+\cdots+a_1X+a_0$. Además, por
	definición, tenemos que $\chi_C=m_v$.  Luego $\chi_A=m_v\chi_M$ y entonces
	$m_v$ divide a $\chi_A$ para cualquier $v\in\K^n$. 
	
	Si $\{e_1,\dots,e_n\}$ es la base canónica de $\K^{n\times1}$, en particular, tenemos
	que $m_{e_i}$ divide a $\chi_A$ para cada $i\in\{1,\dots,n\}$. En
	consecuencia, \[
		m_A=\lcm(m_{e_1},\dots,m_{e_n})
	\]
	divide a $\chi_A$, que es lo
	que queríamos probar.
\end{proof}

\begin{block}
	Sea $A\in\K^{n\times n}$. Como corolario del teorema de Hamilton--Cayley 
	se obtienen fácilmente las siguientes afirmaciones:
	\begin{enumerate}
		\item $\deg m_A\leq n$.
		\item Si $\deg m_A=n$ entonces $m_A=\chi_A$.
		\item Si existe $v\in\K^{n\times1}$ tal que $\deg m_v=n$ entonces $m_v=m_A=\chi_A$.
	\end{enumerate}
\end{block}

\begin{example}
	Vamos a utilizar el teorema de Hamilton--Cayley para calcular 
	las potencias de una matriz. Sea 
	\[
		A=\begin{pmatrix}
		0 & -1\\
		1 & 2
	\end{pmatrix}\in\R^{2\times2}.
	\]
	Entonces $\chi_A=(X-1)^2$. Por el algoritmo de división sabemos que existen $q\in\R[X]$ y $a_{n},b_n\in\R$ tales que
	\[
		X^n=(X-1)^2q+(a_nX+b_n)
	\]
	para todo $n\in\N$. Al evaluar en $X=1$ se obtiene $a_n+b_n=1$. Al derivar 
	\[
		nX^{n-1}=2(X-1)q+(X-1)^2q'+a_n
	\]
	y al evaluar esto en $X=1$, $a_n=n$ para todo $n\in\N$. Luego
	\[
		X^n=(X-1)^2q+nX+(1-n)
	\]
	para todo $n\in\N$. Si evaluamos este polinomio en $A$ y utilizamos el teorema de 
	Hamilton--Cayley, que implica que $(A-I)^2=0$, obtenemos 
	\[
		A^n=nA+(1-n)I=\begin{pmatrix}
			1-n & -n\\
			n & n+1
		\end{pmatrix}.
	\]
\end{example}

\begin{cor}
	Sea $A\in\K^{n\times n}$ una matriz inversible. Entonces $A^{-1}$ es
	combinación lineal de las matrices $I,A,A^2,\dots,A^{n-1}$.

	\begin{proof}
		Supongamos que \[
			\chi_A=X^n+a_{n-1}X^{n-1}+\cdots+a_1X+a_0.
		\]
		Sabemos que
		\[
			a_0=\chi_A(0)=\det(0I-A)=\det(-A)=(-1)^n\det A
		\]
		es distinto de cero pues $A$ es inversible. Sea 
		\[
		B=\frac{-1}{a_0}\left(A^{n-1}+a_{n-1}A^{n-2}+\cdots+a_2A+a_1\right).
		\]
		Como $\chi_A(A)=0$, se tiene entonces que $AB=BA=I$. Luego $A^{-1}=B$.
	\end{proof}
\end{cor}

\begin{example}
	Sea $C$ la matriz compañera del polinomio
	\[
		p=X^n+a_{n-1}X^{n-1}+\cdots+a_1X+a_0\in\K[X].
	\]
	Demostremos que $m_C=p$. Sea $\{e_1,\dots,e_{n}\}$ la base canónica de
	$\K^{n\times1}$. Por inducción se demuestra fácilmente que
	\[
		e_k=Ce_{k-1}=C^{k-1}e_1,\quad
		k\in\{2,\dots,n\}.
	\]
	Tenemos entonces que $\{e_1,ce_1,\dots,C^{n-1}e_1\}$ es un conjunto
	linealmente independiente y luego $\{I,A,\dots,A^{n-1}\}$ es también
	linealmente independiente. Esto nos dice que $\deg m_A\geq n$. Como
	$\chi_A$ es un polinomio de grado $n$, el minimal $m_A$ divide al
	característico $\chi_A$ por el teorema de Cayley--Hamilton, y $m_A$ y
	$\chi_A$ son polinomios mónicos, $m_A=\chi_A=p$.
%	
%	Por el teorema de Cayley--Hamilton, $\chi_A=p(A)=0$. Luego
%	\[
%		A^ne_1=-a_{n-1}A^{n-1}-\cdots-a_1A-a_0.
%	\]
%	\begin{align*}
%		p(A)(e_1)&=(A^n+a_{n-1}A^{n-1}+\cdots+a_1A+a_0)e_1\\
%		&=A^ne_1+a_{n-1}e_n+\cdots+a_1e_2+a_0e_1.
%	\end{align*}
\end{example}



%\include{deprecated/CH}
\chapter{Forma de Jordan}

\section{Subespacios invariantes}

\begin{block}
    \index{subespacio invariante}
    Sea $V$ un espacio vectorial y sea $f\in\hom(V,V)$. Un subespacio
    $S\subseteq V$ es \textbf{$f$-invariante} si $f(S)\subseteq S$.
\end{block}

\begin{examples}
	Sea $f\in\hom(V,V)$. Entonces:
	\begin{enumerate}
		\item $\{0\}$ y $V$ entonces son $f$-invariantes. 
		\item $\ker f$ e $\im f$ son subespacios $f$-invariantes. 
		\item $S\subseteq V$ es un subespacio $f$-invariante de dimensión $1$
			si y sólo si $S=\langle v\rangle$ para algún autovector $v$ de
			$f$.
	\end{enumerate}
\end{examples}

\begin{example}
	Sea $f\colon\R^4\to\R^4$ dada por \[
		f(x_1,x_2,x_3,x_4)=(x_1,x_1+x_2,2x_3,x_4),
	\]
	y sea $\{e_1,e_2,e_3,e_4\}$ la base canónica de $\R^4$. Algunos
	subespacios $f$-invariantes son $\langle e_1,e_2\rangle$, $\langle
	e_3\rangle$, $\langle e_4\rangle$ y $\langle e_1,e_2,e_4\rangle$.
\end{example}

\begin{xca}
	Sea $f\colon\R^2\to\R^2$ dada por $f(x,y)=(x,2x+y)$. Demuestre que
	$S=\langle(0,1)\rangle$ es el único subespacio de dimensión $1$ que es
	$f$-invariante.
\end{xca}

\begin{xca}
	Sea $V$ un espacio vectorial y sea $f\in\hom(V,V)$. Demuestre que si
	$S,T\subseteq V$ son subespacios $f$-invariantes entonces $S\cap T$ y $S+T$
	son $f$-invariantes.
\end{xca}

\begin{block}
    \index{transformación lineal!restricción}
	Si $V$ es un espacio vectorial de dimensión finita, $f\in\hom(V,V)$ y
	$S\subseteq V$ es un subespacio $f$-invariante entonces la restricción de
	$f$ a $S$, 
	\[
		f|_S\colon S\to S,\quad
		s\mapsto f(s)
	\]
	es una transformación lineal, es decir: $f|_S\in\hom(S,S)$.
\end{block}

\begin{prop}[diagonalización simultánea]
    \index{Diagonalización simultánea}
    Sea $V$ de dimensión finita y sean $f,g\in\hom(V,V)$ tales que $f$ y $g$
    son diagonalizables y $fg=gf$. Entonces existe una base de $V$ donde las
    matrices de $f$ y $g$ son simultáneamente diagonalizables.

    \begin{proof}
        Como $f$ es diagonalizable, $V=S(\lambda_1)\oplus\cdots\oplus S(\lambda_k)$, donde
        $S(\lambda_i)=\{v\in V:f(v)=\lambda_iv\}$. Cada $S(\lambda_i)$ es $g$-invariante pues
        \[
            f(g(v))=(fg)(v)=(gf)(v)=g(f(v))=g(\lambda v)=\lambda g(v).
        \]
        Luego $g|_{S(\lambda_i)}\in\hom(S(\lambda_i),S(\lambda_i))$ y además
        $g|_{S(\lambda_i)}$ conmuta con $f$. Como $g$ es diagonalizable, para
        cada $i\in\{1,\dots,k\}$ existe una base de $S(\lambda_i)$ formada por 
        autovectores de $g|_{S(\lambda_i)}$. Como estos autovectores son
        también autovectores de $f$, entonces $f|_{S(\lambda_i)}$ y
        $g|_{S(\lambda_i}$ son simultáneamente diagonalizables.
    \end{proof}
\end{prop}

\begin{prop}
    Sea $V$ un espacio vectorial de dimensión finita y sea $f\in\hom(V,V)$. Si
    $S\subseteq V$ es $f$-invariante entonces
	\begin{enumerate}
		\item $m_{f|_S}$ divide a $m_f$.
		\item $\chi_{f|_S}$ divide a $\chi_f$.
	\end{enumerate}

	\begin{proof}
		Supongamos que $\dim V=n$ y sea $\cB_S=\{v_1,\dots,v_s\}$ una base de $S$. Si
		extendemos esta base de $S$ a una base
		\[
			\cB=\{v_1,\dots,v_s,v_{s+1},\dots,v_n\}
		\]
		de $V$ entonces
		$m_{f|_S}=\lcm(m_{v_1},\dots,m_{v_s})$ y
		$m_f=\lcm(m_{v_1},\dots,m_{v_n})$. Como entonces $m_{v_i}$ divide a
		$m_f$ para cada $i\in\{1,\dots,s\}$, se tiene entonces que $m_{f|_S}$
		divide a $m_f$.

		Para demostrar la segunda afirmación, si escribamos a $f$ en la base
		$\cB$ obtenemos la siguiente matriz por bloques:
		\[
			[f]_{\cB,\cB}=\begin{pmatrix}
				A & \star\\
				0 & \star
			\end{pmatrix},
		\]
		donde $A=[f|_S]_{\cB_S,\cB_S}$. Existe entonces un polinomio
		$q\in\K[X]$ tal que $\chi_f=\chi_{f|_S}q$, tal como queríamos
		demostrar.
	\end{proof}
\end{prop}

\begin{block}
	Sea $V$ un espacio vectorial y sea $f\in\hom(V,V)$. Si $S\subseteq V$ es
	$f$-invariante, un subespacio $T\subseteq V$ es un \textbf{complemento
	invariante} para $S$ si $T$ es $f$-invariante y $S\oplus T=V$. 
\end{block}

\begin{example}
	No siempre existen complementos invariantes: si
	\[
		f\colon\R^2\to\R^2,\quad
		(x,y)\mapsto (0,x)
	\]
	entonces $S=\langle(0,1)\rangle$ es $f$-invariante pero no admite
	complemento invariante pues todo autovector de $f$ pertenece a $S$.
\end{example}

\begin{block}
	\label{block:XY}
	Sean $V$ un espacio vectorial de dimensión finita, $f\in\hom(V,V)$ y
	$S,T\subseteq V$ subespacios $f$-invariantes tales que $S\oplus T=V$. Si
	$\cB_S=\{v_1,\dots,v_s\}$ es una base de $S$ y $\cB_T=\{w_1,\dots,w_t\}$ es
	una base de $T$, entonces 
	\[
	\cB=\{v_1,\dots,v_s,w_1,\dots,w_t\}
	\]
	es una base de $V$. La matriz de $f$ en la base $\cB$ es la siguiente
	matriz por bloques:
	\[
		[f]_{\cB,\cB}=\begin{pmatrix}
			X & 0\\
			0 & Y
		\end{pmatrix},
	\]
	donde $X=[f|_S]_{\cB_S,\cB_S}$ y $Y=[f|_T]_{\cB_T,\cB_T}$. 
\end{block}

\begin{xca}
    \label{xca:f_invariante}
    Sea $V$ un espacio vectorial de dimensión finita, sea $f\in\hom(V,V)$ y 
    sean $S$ y $T$ subespacios $f$-invariantes tales que $S\oplus T=V$.
    Demuestre que entonces:
	\begin{enumerate}
		\item $\chi_f=\chi_{f|_S}\chi_{f|_T}$. 
		\item $m_f=\lcm(m_{f|_S},m_{f|_T})$.
	\end{enumerate}
%
%	\begin{proof}
%		La primera afirmación es una consecuencia inmediata de lo visto
%		en~\ref{block:XY}. Demostremos entonces la segunda afirmación. Sea
%		$p=\lcm(m_{f|_S},m_{f|_T})$. Como $m_{f|_S}$ y $m_{f|_T}$ dividen a
%		$f$, entonces $p$ divide a $m_f$. Por otro lado, por definición sabemos
%		que $m_{f|_S}$ y $m_{f|_T}$ dividen a $p$.  Sea utilizamos la expresión
%		por bloques para $f$ dada en~\ref{block:XY}, \framebox{completar}
%	\end{proof}
\end{xca}


\begin{block}
    \index{subespacio cíclico}
    Sean $V$ un espacio vectorial, $v\in V$ y $f\in\hom(V,V)$. El subespacio
    $C(v)=\langle v,f(v),f^2(v),\dots\rangle$ se denomina el \textbf{subespacio
    cíclico} con respecto a $f$ generado por $v$. Queda como ejercicio
    demostrar que $C(v)$ es un subespacio $f$-invariante.
\end{block}

\begin{lem}
    \label{lem:dimC(v)=degm_v}
	Sea $V$ de dimensión finita y sea $f\in\hom(V,V)$. Si $v\in
	V\setminus\{0\}$ entonces $\deg m_v=\dim C(v)$. En particular, si $\deg
	m_v=d$ entonces el conjunto $\{v,f(v),\dots,f^{d-1}(v)\}$ es base de
	$C(v)$. 

	\begin{proof}
		Supongamos que $\deg m_v=d$. 
		Todo elemento de $C(v)$ es de la forma $p(f)(v)$ para algún
		$p\in\K[X]$. Si $p\in\K[X]$, el algoritmo de división implica que
		existen $r,q\in\K[X]$ tales que $p=qm_v+r$, donde $r=0$ o $\deg r<\deg 
		m_v$. Al especializar esta igualdad en $f$ y evaluar en $v$,
		\[
			p(f)(v)=q(f)m_v(f)(v)+r(f)(v)=r(f)(v),
		\]
		y entonces $r(f)(v)\in C(v)$. Si escribimos $r=\sum_{i=0}^{d-1}\gamma_i
		X^i$ entonces hemos demostrado que todo elemento de $C(v)$ puede
		escribirse como combinación lineal de $\{v,\dots,f^{d-1}(v)\}$ Para ver
		que $\{v,f(v),\dots,f^{d-1}(v)\}$ es linealmente independiente basta
		observar que en caso contrario existiría un polinomio no nulo de grado
		$<d$ que anula a $v$ con respecto a $f$, algo que contradice la
		minimalidad de $m_v$. 
	\end{proof}
\end{lem}

\begin{xca}
	\label{xca:auxiliar}
	Sea $V$ de dimensión finita, sea $f\in\hom(V,V)$ y sea $p\in\K[X]$.
	Demuestre las siguientes afirmaciones:
	\begin{enumerate}
		\item $p(C(v))=C(p(v))$.
		\item Si $V=V_1\oplus\cdots\oplus V_k$ donde cada $V_i$ es $p$-invariante, entonces
			$p(V)=p(V_1)\oplus\cdots\oplus p(V_k)$. 
        \item Si $m_v=m_w$ entonces $m_{p(v)}=m_{p(w)}$ y \[
                \dim C(p(v))=\dim C(p(w)). 
              \]
	\end{enumerate}
\end{xca}

%\begin{block}
%	Sea $V$ de dimensión finita y sea $f\in\hom(V,V)$. Si $v\in
%	V\setminus\{0\}$ entonces $m_{f|_C(v)}=m_v$. En efecto, si $p\in\K[X]$ entonces, como
%	\[
%		m_v\left(f|_{C(v)}\right)(p(f)(v))=m_v(f)(p(f)(v))=0,
%	\]
%	$m_v\left(f|_{C(v)}\right)=0$. Si $q\in\K[X]$ con $\deg q<\deg m_v$, entonces $q\left(f|_{C(v)}\right)(v)=q(f)(v)
%\end{block}

\begin{lem}
	Sea $V$ de dimensión finita y sea $f\in\hom(V,V)$. Entonces existe
	$k\leq\dim V$ tal que $\{v,f(v),f^2(v),\dots,f^{k-1}(v)\}$ es base de
	$C(v)$. Más aún, la matriz de $f$ con respecto a esa base es:
	\[
		\begin{pmatrix}
			0 & 0 & \cdots & 0 & a_0\\
			1 & 0 & \cdots & 0 & a_1\\
			0 & 1 & \cdots & 0 & a_2\\
			\vdots & \vdots & \ddots & \vdots & \vdots\\
			0 & 0 & \cdots & 1 & a_{k-1}
		\end{pmatrix},
	\]
	donde $f^k(v)=\sum_{i=0}^{k-1}a_if^i(v)$.

	\begin{proof}
        Sea $k$ el menor entero positivo tal que el conjunto
        $\{v,f(v),\dots,f^{k-1}(v)\}$ es linealmente independiente y 
		\[
			f^k(v)\in\langle v,f(v),\dots,f^{k-1}(v)\rangle.
		\]
		Vamos a demostrar que $f^m(v)\in\langle v,f(v),\dots,f^{k-1}(v)\rangle$
		para todo $m\geq k$. Procederemos por inducción en $m$. Como el caso
		$m=k$ es trivial, suponemos que el resultado es válido para algún
		$m\geq k$. Si
		\[
			f^{m}(v)\in\langle v,f(v),\dots,f^{k-1}(v)\rangle
		\]
		entonces existen
		$\alpha_0,\dots,\alpha_{k-1}\in\K$ tales que 
		\[
		f^m(v)=\sum_{i=0}^{k-1}\alpha_if^i(v).
		\]
		Al aplicar $f$ se obtiene entonces que 
		\begin{align*}
			f^{m+1}(v)&=f\left(\sum_{i=0}^{k-1}\alpha_if^i(v)\right)\\
			&=\sum_{i=0}^{k-2}\alpha_if^{i+1}(v)+\alpha_{k-1}f^k(v)\in\langle v,f(v),\dots,f^{k-1}(v)\rangle.
		\end{align*}
		Luego $k\leq\dim V$. Un cálculo directo muestra que la matriz de $f$ en
		la base $\{v,f(v),\dots,f^{k-1}(v)\}$ tiene la forma deseada. 
	\end{proof}
\end{lem}

\begin{thm}[forma racional de Frobenius]
	\label{thm:subespacios_ciclicos}
    Sea $V$ de dimensión finita y sea $f\in\hom(V,V)$. Entonces existen
    vectores no nulos $v_1,\dots,v_k\in V$ y polinomios mónicos $p_1,\dots,p_k\in\K[X]$
    tales que
    \begin{align*}
        &V=C(v_1)\oplus\cdots\oplus C(v_k),
    \end{align*}
	$\deg p_i=\dim C(v_i)$, $p_i(f)(v_i)=0$ para todo $i\in\{1,\dots,k\}$ y además $p_i$ divide a $p_{i-1}$
    para todo $i\in\{2,\dots,k\}$. Más aún, los polinomios $p_1,\dots,p_k$
    están unívocamente determinados.

	\begin{proof}
		Procederemos por inducción en $\dim V$. Como el caso $\dim V=1$ es
		trivial, vamos a suponer que el resultado es válido para todo
		endomorfismo de un espacio vectorial de dimensión $<\dim V$. Sea $m$ la
		mayor dimensión que puede tener un subespacio cíclico asociado a $f$.
		Entonces existe $v_1\in V$ tal que $\dim C(v_1)=m$ y $\dim C(v)\leq m$
		para todo $v\in V$, es decir: $\{v_1,f(v_1),\dots,f^{m-1}(v_1)\}$ es 
		linealmente independiente y $f^m(v)\in\langle
		v,f(v),\dots,f^{m-1}(v)\rangle$ para todo $v\in V$.  Si $m=\dim V$, no
		hay nada que demostrar. Supongamos entonces que $m<\dim V$. Vamos a
        construir un complemento de $C(v_1)$ que sea $f$-invariante. Sea $T$ la
        transformación lineal definida por 
		\[
			T\colon V\to\K^{m\times 1},\quad
			v\mapsto\colvec{4}{\varphi(v)}{\varphi(f(v))}{\vdots}{\varphi(f^{m-1}(v))},
		\]
		donde $\varphi\colon V\to\K$ es una funcional lineal que satisface 
		\[
			\varphi(f^{i}(v_1))=\begin{cases}
			1 & \text{si $i=m-1$},\\
			0 & \text{si $i\in\{0,\dots,m-2\}$}.
		\end{cases}
		\]
		(La existencia de una $\varphi$ que cumple lo pedido queda garantizada
		por el siguiente argumento: si extendemos
		$\{v_1,f(v_1),\dots,f^{m-1}(v_1)\}$ a una base de $V$ podemos tomar
		como $\varphi$ el vector de la base dual que vale uno en $f^{m-1}(v_1)$
		y cero en el resto de los vectores.)

		La restricción $T|_{C(v_1)}\colon C(v_1)\to\K^{m\times1}$ es un
    	isomorfismo. En efecto, si $\{e_1,\dots,e_m\}$ es la base canónica de
        $\K^{m\times1}$, un cálculo sencillo muestra que
		\[
		T|_{C(v_1)}(f^i(v_1))=e_{m-i}+\sum_{j=0}^{i-1}\gamma_{m-j}e_{m-j}.
		\]
		Luego $T|_{C(v_1)}$ es epimorfismo y como $\dim C(v_1)=m$ entonces
		$T|_{C(v_1)}$ es un isomorfismo. 

		El subespacio $\ker T$ es $f$-invariante. Para demostrar esta
		afirmación hay que ver que $f(\ker T)\subseteq \ker T$. Si $v\in\ker T$
		entonces $T(v)=0$ y luego $T(f^i(v))=0$ para todo
		$i\in\{0,\dots,m-1\}$. Por otro lado,
		\[
		T(f(v))
		=\colvec{5}{\varphi(f(v))}{\varphi(f^2(v))}{\vdots}{\varphi(f^{m-1}(v))}{\varphi(f^m(v))}
		=\colvec{5}{0}{0}{\vdots}{0}{\varphi(f^m(v))}
		\]
		y $\varphi(f^m(v))=0$ pues $f^m(v)\in\langle
		v,f(v),\dots,f^{m-1}(v)\rangle$.

		Afirmamos que $V=C(v_1)\oplus \ker T$.  
        Veamos que $C(v_1)\cap\ker T=\{0\}$.
		Si $v\in C(v_1)\cap\ker T$ entonces $v\in\ker T|_{C(v_1)}=\{0\}$ pues
		la restricción $T|_{C(v_1)}$ es un isomorfismo. 
        Para demostrar que $V=C(v_1)+\ker T$
		basta observar que
		\begin{align*}
			\dim V &= \dim\ker T+\dim\im T = \dim\ker T+m\\
			&=\dim\ker T+\dim C(v_1)=\dim(\ker T+\dim C(v_1)).
		\end{align*}
	
		Afirmamos que $p_1(f)(v)=0$ para todo $v\in V$.  Como $V=C(v_1)\oplus
		\ker T$ y $p_1(f)(f^{i}(v))=0$ para todo $i\geq0$ pues $p(f)\circ
		f=f\circ p(f)$, es necesario demostrar que $p_1(f)(w)=0$ para todo
		$w\in \ker T$. Si $w\in \ker T$ entonces existen
		$\gamma_0,\dots,\gamma_{m-1}\in\K$ tales que 
		\begin{align*}
			f^m(v_1+w)=\sum_{i=0}^{m-1}\gamma_i f^i(v_1+w)=\sum_{i=0}^{m-1}\gamma_if^i(v_1)+\sum_{i=0}^{m-1}\gamma_if^i(w).
		\end{align*}
		Luego, como $\ker T$ es $f$-invariante, 
		\[
		f^m(v_1)-\sum_{i=0}^{m-1}\gamma_if^i(v_1))=\sum_{i=0}^{m-1}\gamma_if^i(w)-f^m(w)\in C(v_1)\cap W=\{0\}.
		\]
		Como
		$\sum_{i=0}^{m-1}\gamma_if^i(v_1)=f^m(v_1)=\sum_{i=0}^{m-1}\alpha_if^i(v_1)$,
		entonces $\alpha_i=\gamma_i$ para todo $i\in\{0,\dots,m-1\}$. Tenemos
		entonces
		\[
			0=p_1(f)(v_1+w)=p_1(f)(v_1)+p_1(f)(w)=p_1(f)(w)
		\]
		para todo $w\in \ker T$.

        Hemos demostrado que $p_1=m_{v_1}$ y que $V=C(v_1)\oplus \ker T$, donde
        $C(v_1)$ y $\ker T$ son subespacios $f$-invariantes.  Esto nos permite
        descomponer a $f$ en una matriz por bloques
        \[
            \begin{pmatrix}
                C_{p_1} \\
                & [f|_{\ker T}]
            \end{pmatrix}.
        \]
        Además, como $\dim\ker T<\dim V$, la hipótesis inductiva implica que
        $\ker T=C(v_2)\oplus\cdots\oplus C(v_k)$ y que la matriz $[f|_{\ker
        T}]$ puede descomponerse como suma directa de matrices compañeras de
		ciertos polinomios mónicos $p_2,\dots,p_k\in\K[X]$ tales que: 
		\begin{enumerate}
			\item $p_i(f|_{ker T})(v_i)=0$ para cada $i\in\{2,\dots,k\}$,
			\item $\deg p_i=\dim C(v_i)$ para cada $i\in\{2,\dots,k\}$, 
			\item $p_j$ divide a $p_{j-1}$ para cada $j\in\{3,\dots,k\}$. 
		\end{enumerate}

		Demostremos que $p_i(f)(v_i)=0$ para cada $i\in\{1,\dots,k\}$. Ya vimos
		que $p_1(f)(v_1)=0$. Sea entonces $i\in\{2,\dots,k\}$.  Como
		$v_i\in\ker T$, 
		\[
		p_i(f)(v_i)=p_i\left(f|_{\ker T}\right)(v_i)=0
		\]
		por hipótesis inductiva. 

		Demostraremos que $p_2$ divide a $p_1$.  Sea $d=\deg p_2\leq\deg p_1$.
		Entonces, como $d=\dim C(v_2)$, el conjunto
		$\{v_2,f(v_2),\dots,f^{d-1}(v_2)\}$ es linealmente independiente.  Por
		el algoritmo de división, existen $q,r\in\K[X]$ tales que $p_1=qp_2+r$,
		donde $\deg r< \deg p_2=d$ si $r\ne0$. Al especializar en $f$ y evaluar
		en $v_2$ se obtiene:
        \[
            0=p_1(f)(v_2)=q(f)p_2(f)(v_2)+r(f)(v_2)=r(f)(v_2).
        \]
		En particular, existen entonces $\beta_0,\dots,\beta_{d-1}\in\K$ tales
		que
        \[
            0=r(f)(v_2)=\sum_{i=0}^{d-1}\beta_if^i(v_2).
        \]
		Como el conjunto $\{v_2,f(v_2),\dots,f^{d-1}(v_2)\}$ es linealmente
		independiente, entonces $\beta_j=0$ para todo $j\in\{0,\dots,d-1\}$ y
		luego $r=0$. 
       
		Veamos la unicidad de los polinomios $p_1,\dots,p_k$.  Supongamos que
		se tiene otra descomposición en subespacios cíclicos
        \[
            V=C(w_1)\oplus\cdots\oplus C(w_l)
        \]
        cuyos polinomios $q_1,\dots,q_l\in\K[X]$ satisfacen:
		\begin{enumerate}
			\item $q_i(f)(w_i)=0$ para cada $i\in\{1,\dots,l\}$,
			\item $\deg q_i=\dim C(w_i)$ para cada $i\in\{1,\dots,l\}$, 
			\item $q_j$ divide a $q_{j-1}$ para cada $j\in\{2,\dots,l\}$. 
		\end{enumerate}
		Vamos a demostrar que $k=l$ y que $p_i=q_i$ para todo
		$i\in\{1,\dots,k\}$.  Como $q_j$ divide a $q_{j-1}$ para todo
		$j\in\{2,\dots,l\}$ entonces $\dim C(w_1)\geq \dim C(w_j)$ para todo
		$j\in\{2,\dots,l\}$. 
		Primero observamos que $p_1=q_1=m_f$. En efecto, como $p_1(f)(v)=0$ para todo
        $v\in V$ entonces $p_1(f)=0$ y luego $\deg m_f\leq p_1$. Por otro lado,
        como $\{v_1,f(v_1),\dots,f^{m-1}(v_1)\}$ es linealmente
        independiente, el conjunto $\{\id_V,f,\dots,f^{m-1}\}$ es también
        linealmente independiente, y luego $\deg m_f\geq m=\deg p_1$.  Como
        $m_f$ y $p_1$ son polinomios mónicos, concluimos que $m_f=p_1$.  Como
        $q_j$ divide a $q_1$ para todo $j\in\{1,\dots,l\}$ entonces
        $q_1(f)(w_j)=0$ para todo $j\in\{1,\dots,l\}$. Además, como
        $q_1(f)\circ f=f\circ q_1(f)$, entonces $q_1(f)(v)=0$ para todo $v\in
        V$. Tal como se demostró que $p_1=m_f$, se demuestra que $q_1=m_f=p_1$. 

        Supongamos que $k\geq2$. Como $p_1=q_1$, 
		\[
			\dim C(w_1)=\dim C(v_1)<\dim V
		\]
        y luego $l\geq2$.  El ejercicio~\ref{xca:auxiliar} implica que de la
        dos descomposiciones que se tienen para $V$ se obtiene que  
        \[
		p_2(V)=\bigoplus_{i=1}^k p_2(C(v_i))=\bigoplus_{i=1}^lp_2(C(w_i)).
        \]
        Como $\dim C(p_2(v_1))=\dim C(p_2(w_1))$ y como $p_2(f)(v_i)=0$ para
        todo $i\in\{2,\dots,k\}$, entonces $\dim C(p_2(w_j))=0$ para todo
        $j\in\{2,\dots,l\}$. Como en particular $p_2(w_2)=0$, se concluye que
        $q_2$ divide a $p_2$. Similarmente se demuestra que $p_2$ divide a
        $q_2$.  Como los polinomios $p_2$ y $q_2$ son mónicos, entonces
        $p_2=q_2$. 
    \end{proof}
\end{thm}

\begin{block}
	La descomposición de $V$ del teorema~\ref{thm:subespacios_ciclicos} implica
	la existencia de una base de $V$ en la que la matriz de $f$ tiene la forma
    \[
    \begin{pmatrix}
        C_{p_1} \\
        & C_{p_2}\\
        &&\ddots \\
        &&& C_{p_k}
    \end{pmatrix},
	\]
	donde las matrices $C_{p_1},\dots,C_{p_k}$ son las matrices compañeras de
	los polinomios $p_1,\dots,p_k$ respectivamente. Recordemos que 
    si $p=X^{m}-\sum_{i=0}^{m-1}a_iX^i$ entonces 
	\[
    C_{p}=\begin{pmatrix}
        0 & 0 & \cdots & 0 & a_0\\
        1 & 0 & \cdots & 0 & a_1\\
        0 & 1 & \cdots & 0 & a_2\\
        \vdots & \vdots & \ddots & \vdots & \vdots\\
        0 & 0 & \cdots & 1 & a_{m}
    \end{pmatrix}.
    \]
	Es evidente además que $\chi_f=p_1\cdots p_k$. Los polinomios
	$p_1,\dots,p_k$ del teorema~\ref{thm:subespacios_ciclicos} son los
	\textbf{factores invariantes} de $f$. 
\end{block}

\begin{example}
	Sea
	\[
		f\colon\R^3\to\R^3,\quad
		f(x,y,z)=(5x-6y-6z,-x+4y+2z,3x-6y-4z).
	\]
	Un cálculo directo muestra que 
	\[
	\chi_f=(X-1)(X-2)^2,\quad
	m_f=(X-1)(X-2).
	\]
	La forma racional aplicada a $f$ implica que $\R^3=C(v_1)\oplus C(v_2)$, donde
	$\dim C(v_1)=2$ y $\dim C(V_2)=1$. Los factores invariantes son entonces
	$p_1=m_f$ y $p_2=X-2$. Luego, con respecto a la base $\{v_1,f(v_1),v_2\}$,
	la transformación lineal $f$ tendrá la forma
	\[
		\begin{pmatrix}
			0 & -2 & 0\\
			1 & 3 & 0\\
			0 & 0 & 2
		\end{pmatrix}.
	\]
	Es fácil encontrar el vector $v_1$. En efecto, basta tomar cualquier $v_1$
	que no sea autovector. Por ejemplo, si $v_1=(1,1,1)$ entonces
	$f(v_1)=(-7,5,-7)$ y $\{v_1,f(v_1)\}$ es linealmente independiente. Para
	encontrar $v_2$ necesitamos un autovector, por ejemplo $f(v_2)=(2,1,0)$. En
	la base \[
		\{(1,1,1),(-7,5,-7),(2,1,0)\}
	\]
	la matriz de $f$ tiene la forma
	deseada. 
\end{example}

\begin{cor}[Cayley--Hamilton]
	\label{thm:CayleyHamilton}
    Sea $V$ de dimensión finita y sea $f\in\hom(V,V)$. Entonces $\chi_f(f)=0$,
    es decir: $m_f$ divide a $\chi_f$. 

    \begin{proof}
        Si utilizamos la forma racional sobre $f$, tenemos que existen
        $p_1,\dots,p_k\in\K[X]$ tales que $\chi_f=p_1\dots p_k$ y además
        $p_1=m_f$. Luego $m_f$ divide a $\chi_f$.
    \end{proof}
\end{cor}

%\begin{block}
%	Como corolario de la forma racional se obtiene una demostración alternativa
%	del teorema de Cayley--Hamilton. En efecto, vimos que $\chi_f=p_1\dots p_k$
%	y además $p_1=m_f$. Luego $m_f$ divide a $\chi_f$.
%\end{block}

%Más aún, existe una base de $V$ tal que la
%    matriz de $f$ con respecto a esa base es de la forma
%    \[
%    \begin{pmatrix}
%        C_{p_1} \\
%        & C_{p_2}\\
%        &&\ddots \\
%        &&& C_{p_k}
%    \end{pmatrix},
%    \quad
%    C_{p_i}=\begin{pmatrix}
%        0 & 0 & \cdots & 0 & a_0^{i}\\
%        1 & 0 & \cdots & 0 & a_1^{i}\\
%        0 & 1 & \cdots & 0 & a_2^{i}\\
%        \vdots & \vdots & \ddots & \vdots & \vdots\\
%        0 & 0 & \cdots & 1 & a_{m_k}^{i}
%    \end{pmatrix}
%    \]
%    si $p_i=X^{m_k}-\sum_{i=0}^{m_k-1}a_iX^i$. 
%    % 
%    %que $p_i(f)(v_i)=0$ para todo $i\in\{1,\dots,k\}$ y $V$ admite la siguiente
%    %descomposición como suma de subespacios $f$-invariantes:
%	%\[
%	%\]
%    %Valen además las
%    %siguientes afirmaciones:
%    %\begin{enumerate}
%    %    \item Existe una base $\cB$ de $V$ tal que 
%    %        \[
%    %        [f]_{\cB,\cB}=\begin{pmatrix}
%    %            C_{p_1} \\
%    %            & C_{p_2}\\
%    %            &&\ddots \\
%    %            &&& C_{p_k}
%    %        \end{pmatrix},
%    %        \]
%    %        donde cada $C_{p_i}$ es la matriz compañera de $p_i$. 
%    %    \item $\chi_f=p_1\cdots p_k$.
%    %    \item $p_i$ divide a $p_{i-1}$ para cada $i\in\{2,\dots,k\}$.
%    %    \item Los polinomios $p_1,\dots,p_k\in\K[X]$ están unívocamente
%    %        determinados. 
%    %\end{enumerate}

%\begin{cor}
%	Sea $V$ de dimensión finita y sea $f\in\hom(V,V)$. Entonces:
%	\begin{enumerate} 
%		\item Existe
%			una base $\cB$ de $V$ tal que
%			\[
%			[f]_{\cB,\cB}=\begin{pmatrix}
%				C_{p_1} \\
%				& C_{p_2}\\
%				&&\ddots \\
%				&&& C_{p_k}
%			\end{pmatrix},
%			\]
%			donde cada $C_{p_i}$ es la matriz compañera de un 
%			$p_i\in\K[X]$. 
%		\item $\chi_f=p_1\cdots p_k$. 
%		\item $\dim\ker(f-\lambda\id_V)=|\{i\in\{1,\dots,k\}:\;
%			p_i(\lambda)=0\}|$.
%		\item $f$ es diagonalizable si y sólo si todas las $C_{p_i}$ tiene
%			autovalores distintos.
%	\end{enumerate}
%
%	\begin{proof}
%		
%	\end{proof}
%\end{cor}

%\begin{cor}[forma canónica de Frobenius]
%	Sea $V$ de dimensión finita y sea $f\in\hom(V,V)$. Entonces existe una base
%	$\cB$ de $V$ y existen únicos polinomios mónicos $p_1,\dots,p_k\in\K[X]$
%	tales que 
%	\[
%	[f]_{\cB,\cB}=\begin{pmatrix}
%		C_{p_1} \\
%		& C_{p_2}\\
%		&&\ddots \\
%		&&& C_{p_k}
%	\end{pmatrix},
%	\]
%	donde cada $C_{p_i}$ es la matriz compañera de $p_i$, y $p_j$ 
%	divide a $p_{j-1}$ para cada $j\in\{2,\dots,k\}$. 
%
%	\begin{proof}
%		Utilizaremos la misma descomposición en subespacios cíclicos que vimos
%		en el teorema~\ref{thm:subespacios_ciclicos}. Sea $m$ la mayor
%		dimensión que puede tener un subespacio cíclico asociado a $f$.
%		Entonces existe $v_1\in V$ tal que $\{v_1,f(v_1),\dots,f^{m-1}(v_1)\}$
%		es un conjunto linealmente independiente y $f^m(v)\in\langle
%		v,f(v),\dots,f^{m-1}(v)\rangle$ para todo $v\in V$. La descomposición
%		ciclica vista en el teorema~\ref{thm:subespacios_ciclicos} afirma que
%		existe $p_1\in\K[X]$, digamos
%		\[
%			p_1 = X^m-\alpha_{n-1}X^{n-1}-\cdots-\alpha_1X-\alpha_0,
%		\]
%		tal que $p_1(f)(v_1)=0$ y además nos permite escribir $V=C(v_1)\oplus
%		W$, donde $W$ es un subespacio $f$-invariante. 
%		
%		Afirmamos que
%		$p_1(f)(v)=0$ para todo $v\in V$. Como $V=C(v_1)\oplus W$ y
%		$p_1(f)(f^{i}(v))=0$ para todo $i\geq0$ pues $p(f)\circ f=f\circ p(f)$,
%		es necesario demostrar que $p_1(f)(w)=0$ para todo $w\in W$. Si $w\in
%		W$ entonces existen $\gamma_0,\dots,\gamma_{m-1}\in\K$ tales que 
%		\begin{align*}
%			f^m(v_1+w)=\sum_{i=0}^{m-1}\gamma_i f^i(v_1+w)=\sum_{i=0}^{m-1}\gamma_if^i(v_1)+\sum_{i=0}^{m-1}\gamma_if^i(w).
%		\end{align*}
%		Luego, como $W$ es $f$-invariante, 
%		\[
%		f^m(v_1)-\sum_{i=0}^{m-1}\gamma_if^i(v_1))=\sum_{i=0}^{m-1}\gamma_if^i(w)-f^m(w)\in C(v_1)\cap W=\{0\}.
%		\]
%		Como
%		$\sum_{i=0}^{m-1}\gamma_if^i(v_1)=f^m(v_1)=\sum_{i=0}^{m-1}\alpha_if^i(v_1)$,
%		entonces $\alpha_i=\gamma_i$ para todo $i\in\{0,\dots,m-1\}$. Tenemos entonces
%		\[
%			0=p_1(f)(v_1+w)=p_1(f)(v_1)+p_1(f)(w)=p_1(f)(w)
%		\]
%		para todo $w\in W$.
%
%		El mismo argumento aplicado a $W$ nos dice que existe $v_2\in W$ y
%		existe $p_2\in\K[X]$ con $d=\deg p_2\leq\deg p_1$ tal que
%		$\{v_2,f(v_2),\dots,f^{d-1}(v_2)\}$ es un conjunto linealmente
%		independiente y $f^d(w)\in\langle w,f(w),\dots,f^{d-1}(w)\rangle$ para
%		todo $w\in V$.  Por el algoritmo de división, existen $q\in\K[X]$ y
%		$r\in\K[X]$ tales que $p_1=p_2q+r$, donde $r=0$ o $\deg r<\deg p_2$.
%		Pero entonces
%		\[
%			0=p_1(f)(v_2)=q(f)p_2(f)(v_2)+r(f)(v_2)=r(f)(v_2).
%		\]
%		Existen entonces $\beta_1,\dots,\beta_{d-1}\in\K$
%		tales que 
%		\[
%			r(f)(v_2)=\beta_0v_2+\beta_1f(v_2)+\cdots+\beta_{d-1}f^{d-1}(v_2)=0.
%		\]
%		Como por construcción $\{v_2,f(v_2),\dots,f^{d-1}(v_2)\}$ es
%		linealmente indendiente, $\beta_j=0$ para todo $j$ y luego $r=0$.
%
%		Veamos que los polinomios $p_1$ y $p_2$ son únicos. Primero observemos
%		que $p_1=m_f$. En efecto, como $p_1(f)(v)=0$ para todo $v\in V$
%		entonces $p_1(f)=0$ y luego $\deg m_f\leq p_1$. Por otro lado, como
%		$\{v_1,f(v_1),\dots,f^{m-1}(v_1)\}$ es linealmente independiente, el
%		conjunto $\{\id_V,f,\dots,f^{m-1}\}$ es también linealmente
%		independiente, y luego $\deg m_f\geq m=\deg p_1$. Como $m_f$ y $p_1$
%		son polinomios mónicos, concluimos que $m_f=p_1$. Para demostrar que
%		$p_2$ es único supongamos que se tienen dos descomposiciones
%		\[
%			V=C(v_1)\oplus W=C(v_1')\oplus W'.
%		\]
%		Las matrices de $f$ con respecto a estas descomposiciones son matrices
%		semejantes y son matrices por bloques de la forma
%		\begin{align*}
%			&
%			\begin{pmatrix}
%				C_{p_1} & 0 \\
%				0 & [f|_W]
%			\end{pmatrix},
%			&&
%			\begin{pmatrix}
%				C_{p_1} & 0\\
%				0 & [f|_{W'}
%			\end{pmatrix}.
%		\end{align*}
%		\framebox{completar}
%	\end{proof}
%\end{cor}

\begin{example}
    Las matrices $E_{21}\in\C^{4\times4}$ y $E_{21}+E_{43}\in\C^{4\times4}$
    tienen a $X^4$ como polinomio característico y a $X^2$ como polinomio
    minimal. Sin embargo, tienen disinta forma racional.
\end{example}



\section{Endomorfismos nilpotentes}

\begin{block}
    \index{nilpotente!endomorfismo}
    \index{nilpotente!matriz}
    Sea $V$ un espacio vectorial y sea $f\in\hom(V,V)$. Entonces $f$ es
    \textbf{nilpotente} si existe $m\in\N$ tal que $f^m=0$.

	Si $f\in\hom(V,V)$ es nilpotente, el
    número
    \[
        r=\min\{m\in\N:f^m=0\}
    \]
    se denomina \textbf{índice de nilpotencia} de $f$.
\end{block}

\begin{example}
	Sea \[
		f\colon\R^3\to\R^3,\quad
		(x,y,z)\mapsto (-x+2y+z,0,-x+2y+z).
	\]
    Entonces $f$ es nilpotente con índice de nilpotencia igual a dos.
\end{example}

\begin{example}
    Sea $V$ un espacio vectorial de dimensión finita $n$ y sea
    $\{v_1,\dots,v_n\}$ una base de $V$. La aplicación
    $f\colon V\to V$ definida por
    \[
    f(v_i)=\begin{cases}
        v_{i+1} & \text{si $i<n$},\\
        0 & \text{si $i=n$},
    \end{cases}
    \]
    es nilpotente de índice $n$ pues $f^n=0$ y $f^{n-1}\ne0$. 
\end{example}


\begin{xca}
    \label{xca:nilpotente_y_diagonalizable=0}
    Sea $V$ un espacio vectorial de dimensión finita y sea $f\in\hom(V,V)$
    nilpotente y diagonalizable. Demuestre que $f=0$.
\end{xca}

\begin{xca}
    \label{xca:nilpotente:autovalores}
    Sea $V$ un espacio vectorial y sea $f\in\hom(V,V)$ nilpotente. Demuestre
    que $\spec f=\{0\}$. 
\end{xca}

\begin{prop}
    Sea $V$ de dimensión finita y $f\in\hom(V,V)$. Entonces $f$ es nilpotente
    de índice $k$ si y sólo si $m_f=X^k$. 

	\begin{proof}
        Si $f^k=0$ y $f^{k-1}\ne0$ entonces $X^k$ anula a $f$. Luego $m_f$
        divide a $X^k$ y entonces $m_f=X^j$ para algún $j\in\{1,\dots,k\}$.
        Como $f^{k-1}\ne0$ entonces $m_f=X^k$.  Recíprocamente, si $m_f=X^k$
        entonces, trivialmente, $f^k=0$ y $f^{k-1}\ne0$. 
	\end{proof}
\end{prop}

\begin{prop}
	Sea $V$ un espacio vectorial de dimensión finita y sea $f\in\hom(V,V)$
	nilpotente de índice $r$. Sea $v\in V$ tal que $f^{r-1}(v)\ne0$. Entonces
	$\{v,f(v),\dots,f^{r-1}(v)\}$ es linealmente independiente. En particular,
	$r\leq\dim V$.

    \begin{proof}
        Si $r=1$ entonces el resultado es válido pues $v\ne0$. Si $r\geq2$
        entonces sean $a_0,\dots,a_{r-1}\in\K$ tales que
        $\sum_{i=0}^{r-1}a_if^i(v)=0$.  Al aplicar $f^{r-1}$ obtenemos
		\[
		0=f^{r-1}\left(\sum_{i=0}^{r-1}a_if^i(v)\right)=\sum_{i=0}^{r-1}a_if^{r-1+i}(v)=a_0f^{r-1}(v).
		\]
		pues $f^{r-1+i}(v)=0$ si $i>0$. Como $f^{r-1}(v)\ne0$ entonces $a_0=0$.
		Queda entonces $a_1f(v)+\cdots+a_{r-1}f^{r-1}(v)=0$. Al aplicar $f^{r-2}$
		en esta igualdad se obtiene, tal como se hizo antes, que $a_1=0$. Este
		proceso nos da entonces que $a_i=0$ para todo $i$.
    \end{proof}
\end{prop}

\begin{example}
	Si $V$ tiene dimensión $n$ y $f\in\hom(V,V)$ es nilpotente de índice $n$
	entonces $\{v,f(v),\dots,f^{n-1}(v)\}$ es una base de $V$. Luego, en esa
	base, la matriz $f$ es de la forma 
	\[
		\begin{pmatrix}
			0 & 0 & \cdots & 0 & 0\\
			1 & 0 & \cdots & 0 & 0\\
			0 & 1 & \cdots & 0 & 0\\
			\vdots & \vdots & \ddots & \vdots & \vdots\\
			0 & 0 & \cdots & 1 & 0
		\end{pmatrix}
	\]
	y se denomina \textbf{bloque de Jordan nilpotente} de $f$.
\end{example}

\begin{thm}
    \label{thm:Jordan:nilpotente}
    \index{Jordan!forma de}
    Sea $V$ de dimensión finita y sea $f\in\hom(V,V)$ nilpotente de índice $r$.
    Entonces existe una base de $V$ tal que $f$ en esa base es de la forma 
	\[
		\begin{pmatrix}
			J_1 & 0 & \cdots & 0\\
			0 & J_2 & \cdots & 0\\
			\vdots & \vdots & \ddots & \vdots \\
			0 & 0 & \cdots & J_k
		\end{pmatrix}
	\]
	donde cada $J_i$ es un bloque de Jordan nilpotente de tamaño $n_{i}\times
	n_{i}$ y además $r=n_1\geq n_2\geq\cdots\geq n_k\geq1$. 

	\begin{proof}
        La forma racional de Frobenius, teorema~\ref{thm:subespacios_ciclicos},
        implica que existen $v_1,\dots,v_k\in V$ y polinomios mónicos 
        $p_1,\dots,p_k\in\K[X]$ tales que $V=C(v_1)\oplus\cdots\oplus C(v_k)$,
        $\deg p_i=\dim C(v_i)$ y $p_i(f)(v_i)=0$ para todo $i\in\{1,\dots,k\}$
        y $p_i$ divide a $p_{i-1}$ para todo $i\in\{2,\dots,k\}$. Como $f$ es
        nilpotente de índice $r$, entonces $p_1=m_f=X^r$. Luego todo $p_i$ es
        de la forma $X^{n_i}$, donde $n_1=r$, $n_1\geq n_2\geq\cdots\geq
        n_k\geq1$.  La matriz de $f$ asociada a esta descomposición en
        subespacios $f$-invariantes es entonces de la forma deseada.
	\end{proof}
\end{thm}

\begin{block}
    \label{block:rg(J^k)}
	Observemos que si $J$ es una matriz de Jordan nilpotente de $n\times n$
	entonces $\rg (J^i)=n-i$ para todo $i\in\{0,\dots,n\}$. En efecto, sea
	$\{v_1,\dots,v_n\}$ una base de $\K^{n\times1}$. Entonces la aplicación
	lineal $J\colon\K^{n\times 1}\to\K^{n\times1}$ dada por $x\mapsto Jx$
	satisface
    \[
    J(v_i)=\begin{cases}
        v_{i+1} & \text{si $i<n$},\\
        0 & \text{si $i=n$}.
    \end{cases}
    \]
    Entonces $J$ tiene rango uno. Por inducción es fácil ver que 
    \[
        J^k(v_i)=\begin{cases}
            v_{i+k} & \text{si $i+k-1<n$},\\
            0 & \text{en otro caso}.
        \end{cases}
    \]
    y luego $\dim\im J^k=n-k$ y $\rg(JP^k)=n-k$. 
\end{block}

\begin{cor}
	\label{cor:Jordan:nilpotente}
    Sea $V$ de dimensión finita y $f\in\hom(V,V)$ nilpotente de índice $r$.
    Valen las siguientes afirmaciones:
    \begin{enumerate}
        \item La cantidad de bloques de Jordan de $f$ es $\dim\ker f=n-\dim\im f$. 
        \item El bloque de Jordan de $f$ de tamaño máximo es de $r$. 
        \item Para cada $i\in\{1,\dots,r-1\}$ la cantidad de bloques de Jordan
            de $f$ de tamaño $\geq i$ es $\dim\im f^i-\dim\im f^{i+1}$. 
        \item Para cada $i\in\{1,\dots,r-1\}$, la cantidad de bloques de Jordan
            de tamaño $i$ es $\dim\im f^{i+1}-2\dim\im f^i+\dim
            f^{i-1}$. 
    \end{enumerate}

    \begin{proof}
		Para demostrar la primera afirmación observemos que si hay $k$ bloques
		de Jordan entonces $n=\dim\im f+k$ pues cada bloque de Jordan de tamaño
		$d\times d$ tiene rango igual a $d-1$. 

		Para demostrar la segunda afirmación obvervamos que $m_f=X^r$ y que el
		primer bloque de Jordan tiene tamaño igual a $\deg m_f$ pues el primer
		factor invariante de $f$ es $m_f$. 

		Para la tercera afirmación, para cada $i\in\{1,\dots,r-1\}$. Sea $n_i$
		la cantidad de bloques de Jordan de tamaño $i$. Entonces, por lo visto
		en~\ref{block:rg(J^k)}, si agrupamos todos los bloques de tamaño $j$
		para cada $j\in\{i,\dots,r-1\}$ obtenemos:
        \begin{align*}
            \dim \im f^i-\dim\im f^{i+1}&=\sum_{j=i+1}^k n_j(j-i)-\sum_{j=i+2}^kn_j(j-i+1)
            =\sum_{j=i+1}^kn_j,
        \end{align*}
        que es la cantidad de bloques de tamaño $\geq i$. 

        Por ultimo, por la fórmula vista en el ítem anteior, la cantidad
        de bloques de Jordan de $f$ de tamaño $i$ es
        \begin{align*}
            n_i&=\sum_{j=i}^kn_j-\sum_{j=i+1}^kn_j\\
            &=\dim\im f^{i-1}-\dim\im f^i-(\dim\im f^i-\dim\im f^{i+1}),
        \end{align*}
        que es lo que queríamos demostrar.
    \end{proof}
\end{cor}

\begin{example}
    Demostremos que no existe una matriz $A\in\R^{15\times15}$ tal que
    $\rg(A)=10$, $\rg(A^4)=3$ y $\rg(A^5)=0$. Por el corolario anterior, como
    $A$ es nilpotente, $A$ tiene una forma de Jordan que contiene $15-\rg(A)=5$
    bloques, donde el bloque más grande es de $5\times 5$. Además $A$ tiene
    \[
        \rg(A^6)-2\rg(A^5)+\rg(A^4)=3
    \]
    bloques de tamaño $5$, una contradicción. 
\end{example}

\begin{example}
	Sea
	\[
		A=\begin{pmatrix}
			0 & 0 & 0 & 0 & 0 & 0\\
			1 & 0 & 0 & 0 & 0 & 0\\
			-1 & -1 & 0 & 0 & 0 & 0\\
			0 & 1 & 0 & 0 & 1 & 0\\
			-1 & 0 & 0 & 0 & 0 & 0\\
			1 & 0 & 0 & 0 & -1 & 0
		\end{pmatrix}\in\R^{6\times6}.
	\]
	Entonces $\chi_A=X^6$, $m_A=X^3$ y $A$ es nilpotente de índice tres. Como
	$\rg(A)=3$ entonces $\dim\ker A=3$ y luego, por el
	corolario~\ref{cor:Jordan:nilpotente}, la forma de Jordan de $A$ tendrá
	tres bloques de Jordan. Como $A$ es nilpotente de índice tres, el mayor de
	estos bloques de Jordan será de $3\times3$. La forma de Jordan de $A$ será
	entonces
	\begin{align*}
		J=
		\begin{pmatrix}
			J_1\\
			& J_2\\
			&& J_3
		\end{pmatrix},
		&&
		J_1=\begin{pmatrix}
			0 & \\  
			1 & 0\\
			0 & 1 & 0
		\end{pmatrix},
		&&
		J_2=\begin{pmatrix}
			0\\
			1 & 0\\
		\end{pmatrix},
		&&
		J_3=\begin{pmatrix}
			0
		\end{pmatrix}.
	\end{align*}

	Sea $\{e_1,\dots,e_6\}$ la base canónica de $\R^{6\times1}$. Un cálculo
	directo muestra que $\dim C(e_1)=3$, $\dim C(e_2)=2$, $\dim C(e_3)=1$ y que
	además $\{e_1,Ae_1,A^2e_1,e_2,Ae_2,a_3\}$ es una base de $\R^{6\times1}$.
	En esa base, la matriz de la transformación lineal $\R^{6\times1}\to\R^{6\times1}$ dada
	por $x\mapsto Ax$, es la matriz $J$. 
\end{example}

%\section{Forma de Jordan para endomorfismos nilpotentes}
%
%\begin{lem}
%	Sea $V$ de dimensión finita y sea $f\in\hom(V,V)$ nilpotente de índice $r$. Entonces
%	\[
%		\{0\}\subsetneq \ker f\subsetneq\ker f^2\subsetneq\cdots\subsetneq\ker f^k=V.
%	\]
%
%	\begin{proof}
%		Como las inclusiones son evidentes, es necesario ver que son estrictas.
%		Como $f$ es nilpotente de índice $r$, existe $v\in V$ tal que
%		$f^r(v)=0$ y $f^{r-1}(v)\ne0$.  Sea $i\in\{0,\dots,r-1\}$ y sea
%		$w=f^{r-(i+1)}(v)\in V$. Entonces $w\in\ker f^{i+1}$ pues
%		$f^{i+1}(w)=f^r(v)=0$ y además $w\not\in\ker f^i$ pues
%		$f^{i}(w)=f^{r-1}(v)$. 
%	\end{proof}
%\end{lem}
%
%\begin{lem}
%	Sea $V$ de dimensión finita y sea $f\in\hom(V,V)$. Sea $i\in\N$ y sea
%	$\{v_1,\dots,v_r\}\subseteq V$ un conjunto linealmente independiente tal
%	que 
%	\[
%		\ker f^i\cap\langle v_1,\dots,v_r\rangle=\{0\}.
%	\]
%	Entonces $\{f(v_1),\dots,f(v_r)\}$ es linealmente independiente y 
%	\[
%		\ker f^{i-1}\cap\langle f(v_1),\dots,f(v_r)\rangle=\{0\}.
%	\]
%
%	\begin{proof}
%		Veamos primero que $\{f(v_1),\dots,f(v_r)\}$ es linealmente
%		independiente: si $\sum_{j=1}^r\alpha_jf(v_j)=0$ entonces al aplicar
%		$f^{i-1}$ obtenemos que $\sum_{j=1}^r\alpha_jv_j\in\ker f^i\cap\langle
%		v_1,\dots,v_r\rangle=\{0\}$. 
%		Demostremos ahora que 
%		\[
%			\ker f^{i-1}\cap\langle f(v_1),\dots,f(v_r)\rangle=\{0\}.
%		\]
%		Si $v\in\ker f^{i-1}\cap\langle f(v_1),\dots,f(v_r)\rangle$ entonces
%		$f^{i-1}(v)=0$ y $v=\sum_{j=1}^r\beta_jf(v_j)$. Luego, al aplicar
%		$f^{i-1}$, se obtiene que
%		\[
%			0=f^{i-1}(v)=f^{i-1}\left(\sum_{j=1}^r\beta_jf(v_j)\right)=f^{i}\left(\sum_{j=1}^r\beta_jv_j\right).
%		\]
%		Esto dice que $\sum_{j=1}^r\beta_jv_j\in\ker f^i\cap\langle
%		v_1,\dots,v_r\rangle=\{0\}$ y luego $v=0$ pues la independencia lineal
%		de los $v_j$ implica que $\beta_j=0$ para todo $j$.
%	\end{proof}
%\end{lem}
%
%\begin{lem}
%	Sea $V$ de dimensión finita y $f\in\hom(V,V)$ nilpotente de índice $r$. Sea
%	$w\in V$ tal que $S=\langle v,f(v),\dots,f^{r-1}(v)\rangle$ es un subespacio de
%	dimensión $r$. Entonces $S$ admite un complemento $f$-invariante. 
%
%	\begin{proof}
%		Procederemos por inducción en $r$. Si $r=1$ entonces $f=0$ y $S=\langle
%		w\rangle$. Si extendemos $\{w\}$ a una base $\{w,v_1,\dots,v_{n-1}\}$
%		de $V$, entonces $T=\langle v_1,\dots,v_{n-1}\rangle$ es $f$-invariante
%		y $S\oplus T=V$. 
%
%		Supongamos entonces que $k>1$ y que el resultado vale para todo $V$ de
%		dimensión finita y todo $f\in\hom(V,V)$ nilpotente de índice $<k$. Como
%		$\im f$ es $f$-invariante, la restricción $f|_{\im f}$ es nilpotente de
%		índice $k-1$. Entonces, como $X=\langle
%		f(w),\dots,f^{k-1}(w)\rangle\subseteq\im f$, por hipótesis inductiva
%		existe un subespacio $f$-invariante $Y\subseteq \im f$ tal que $\im
%		f=X\oplus Y$. Consideremos el subespacio 
%		\[
%			Z=\{v\in V:f(v)\in Y\}\subseteq V.
%		\]
%		Entonces $Y\subseteq Z$ pues $Y$ es $f$-invariante.
%
%		Afirmamos que $V=S+Z$. Si $w\in V$ entonces $f(w)\in\im f=X\oplus Y$.
%		Luego $f(w)=x+y$ con $x\in X$, $y\in Y$. Por definición de $X$, existe
%		$s\in S$ tal que $x=f(s)$. Si escribimos $w=s+(w-s)$ entonces $s\in S$
%		y $w-s\in Z$ pues $f(w-s)=f(w)-f(s)=f(w)-x=y\in Y\subseteq Z$. 
%
%		Afirmamos ahora que $S\cap Z\subseteq X$. En efecto, si $w\in S\cap Z$
%		entonces $f(w)\in X$ y $f(w)\in Y$. Luego $f(w)\in X\cap Y=\{0\}$ y
%		entonces $w\in\ker f$. Como $w\in S$, existen escalares
%		$\alpha_1,\dots,\alpha_{r-1}\in\K$ tales que
%		$w=\sum_{i=1}^{r-1}\alpha_if^i(v)$. Como entonces 
%		$0=f(w)=\sum_{i=0}^{r-2}\alpha_if^{i+1}(v)$, se tiene que $\alpha_i=0$
%		para todo $i\in\{0,\dots,k-2\}$. Esto nos dice que
%		$w=\alpha_{r-1}f^{r-1}(v)\in X$.
%
%		Afirmamos que $Y\cap(S\cap Z)=\{0\}$. En efecto, si $w\in
%		Y\cap(S\cap Z)$ entonces, como $w\in S\cap Z\subseteq X$ y además $w\in Y$, 
%		se tiene que $w=0$. El espacio vectorial $Y\oplus (S\cap Z)$ es entonces un subespacio de
%		$Z$. Este subespacio tiene un complemento $U$ en $Z$. Luego $Z=U\oplus
%		Y\oplus (S\cap Z)$. 
%
%		Afirmamos que $T=U\oplus Y$ es un complemento $f$-invariante para $S$.
%		Es $f$-invariante pues como $U\subseteq Z$ entonces $f(U)\subseteq Y$ y luego
%		\[
%			f(T)=f(U\oplus Y)\subseteq f(U)+f(Y)\subseteq f(Y)\subseteq Y\subseteq T
%		\]
%		pues $Y$ es $f$-invariante. 
%		Además $T\cap S=\{0\}$ pues $T\cap S\subseteq (S\cap Z)\cap T=\{0\}$. 
%		Por último, como $S\cap Z\subseteq S$, 
%		\[
%			V=S+Z=S+T+(S\cap Z)=S+T,
%		\]
%		tal como se quería demostrar.
%	\end{proof}
%\end{lem}

\section{Descomposición primaria}

\begin{lem}
	Sea $V$ de dimensión finita, $f\in\hom(V,V)$ y $p\in\K[X]$. Entonces $\ker
	p(f)$ es un subespacio $f$-invariante. 

	\begin{proof}
		Tenemos que demostrar que $f(\ker p(f))\subseteq\ker p(f)$. Sea
		$v\in\ker p(f)$. Entonces 
		\[
			p(f)\left( f(v)\right)=(p(f)\circ f)(v)=(f\circ p(f))(v)=f\left(p(f)(v)\right)=f(0)=0,
		\]
		tal como se quería demostrar.
	\end{proof}
\end{lem}

\begin{lem}
	\label{lem:V=kerp+kerq}
	Sea $V$ de dimensión finita y sea $f\in\hom(V,V)$. Supongamos que $m_f=pq$
	donde $\gcd(p,q)=1$. Entonces
	\begin{enumerate}
		\item $V=\ker p(f)\oplus \ker q(f)$.
		\item Si $f_p=f|_{\ker p(f)}$ y $f_q=f|_{\ker q(f)}$ entonces
			$m_{f_p}=p$ y $m_{f_q}=q$.
	\end{enumerate}

	\begin{proof}
		Demostremos la primera afirmación. Sean $r,s\in\K[X]$ tales que
		$rp+sq=1$. Al especializar en $f$ obtenemos
		\begin{equation}
			\label{eq:rp+sq=1}
			r(f)p(f)+s(f)q(f)=\id_V.
		\end{equation}
		
		Veamos que $V=\ker p(f)+\ker q(f)$. Sea $v\in V$. La
		fórmula~\eqref{eq:rp+sq=1} nos permite escribir $v=v_1+v_2$, donde
		$v_1=r(f)p(f)(v)$ y $v_2=s(f)q(f)(v)$. Es fácil comprobar que
		$v_1\in\ker q(f)$ y que $v_2\in\ker p(f)$:
		\begin{align*}
			&q(f)(v_1)=q(f)r(f)p(f)(v)=r(f)p(f)q(f)(v)=r(f)m_f(f)(v)=0,\\
			&p(f)(v_2)=p(f)s(f)q(f)(v)=s(f)p(f)q(f)(v)=s(f)m_f(f)(v)=0.
		\end{align*}
		Veamos ahora que $\ker p(f)\cap\ker q(f)=\{0\}$: si $p(f)(v)=q(f)(v)=0$
		entonces~\eqref{eq:rp+sq=1} implica que $v=0$. 

		Para demostrar la segunda afirmación primero observemos que, como
		$p(f_p)=p(f|_{\ker p(f)})=0$, entonces el minimal de $f_p$ divide a
		$p$. Por otro lado, como 
		\[
			V=\ker p(f)\oplus \ker q(f),
		\]
		entonces 
		$m_{f_p}(f)q(f)=0$ pues $q(f)(v)=0$ para todo $v\in\ker q(f)$ y 
		\[
			m_{f_p}(f)(v)=m_{f_p}(f|_{\ker p(f)})(v)=0
		\]
		para todo $v\in\ker p(f)$. 
		Esto implica que $\deg m_{f_p}\geq\deg p$
		y luego, como $p$ y $m_{f_p}$ son mónicos, $p=m_{f_p}$. De la misma
		forma se demuestra que $q=m_{f_q}$. 
	\end{proof}
\end{lem}

\begin{thm}[descomposición primaria]
    \label{thm:descomposicion_primaria}
	\index{Descomposición primaria}
	Sea $V$ de dimensión finita y sea $f\in\hom(V,V)$. Supongamos que
	$m=p_1^{r_1}\cdots p_k^{r_k}$ donde los $p_i$ son polinomios mónicos e
	irreducibles. Entonces
	\[
		V=\ker p_1(f)^{r_1}\oplus\cdots\oplus\ker p_k(f)^{r_k}.
	\]

	\begin{proof}
		Procederemos por inducción en $k$. Como el caso $k=2$ es el lema
		\ref{lem:V=kerp+kerq}, vamos a suponer que el resultado es válido para
		$k-1\geq2$. Sean $p=p_1^{r_1}\cdots p_{k-1}^{r_{k-1}}$ y $q=p_k^{r_k}$. Entonces 
		$m_f=pq$ y $\gcd(p,q)=1$. Además 
		\[
			\ker q(f)=\ker p_k(f)^{r_k}.
		\]
		Por el lema~\ref{lem:V=kerp+kerq}, $V=\ker p(f)\oplus\ker q(f)$ y el
		minimal de $f|_{\ker p(f)}$ es $p$.  Por hipótesis inductiva, 
		\[
			\ker p(f)=\ker p_1(f)^{r_1}\oplus\cdots\oplus\ker p_{k-1}(f)^{r_{k-1}},
		\]
		y esto completa la demostración.
	\end{proof}
\end{thm}

\begin{cor}[descomposición de Jordan--Chevalley]
    \label{cor:JordanChevalley}
	\label{Jordan--Chevalley!descomposición de}
	Sea $V$ de dimensión finita y sea $f\in\hom(V,V)$. Supongamos que
	$m_f=\prod_{i=1}^k(X-\lambda_i)^{m_i}$, donde $\lambda_i\ne\lambda_j$ si
	$i\ne j$. Entonces $f=g+h$, donde $g$ es diagonalizable, $h$ es nilpotente
	y $gh=hg$. 

    \begin{proof}
        La descomposición primaria, teorema~\ref{thm:descomposicion_primaria},
		implica que $V=V_1\oplus V_k$, donde $V_i=\ker(f-\lambda_i\id_V)^{m_i}$. 
        Para definir $g$ y $h$
        basta con definirlas en cada $V_i$. Para 
        $i\in\{1,\dots,k\}$ definimos entonces 
        \begin{align*}
            &g|_{V_i}=\lambda\id_{V_i},\\
            &h|_{V_i}=f|_{V_i}-\lambda_i\id_{V_i}.
        \end{align*}

		Como para cada $i\in\{1,\dots,k\}$ se tiene que $(g+h)(v_i)=f(v_i)$ y
		cada $v$ se escribe únívocamente como $v=v_1+\cdots+v_k$ con $v_i\in
		V_i$, entonces $f=g+h$. Es evidente que $g$ es diagonalizable y que
		$gh=hg$. Para ver que $h$ es nilpotente, vamos a demostrar que si
		$n=\dim V$ entonces $h^n=0$. En efecto, si $v=v_1+\cdots+v_k$ con
		$v_i\in V_i$ para todo $i\in\{1,\dots,k\}$, entonces 
		\[
		h^n(v)=\sum_{i=1}^kh^n(v_i)=\sum_{i=1}^k(h|_{V_i})^n(v_i)=\sum_{i=1}^k(f|_{V_i}-\lambda_i\id_{V_i})^n(v_i)=0
		\]
		pues $m_i\leq n$ para todo $i\in\{1,\dots,k\}$. 
    \end{proof}
\end{cor}

%%% unicidad de Jordan Chevalley (ejercicio)

%\begin{xca}
%    \label{xca:JordanChevalley:unicidad}
%\end{xca}

\section{Forma de Jordan}

\begin{thm}[forma de Jordan]
    \index{Jordan!forma de}
    Sea $V$ un espacio vectorial de dimensión finita y sea $f\in\hom(V,V)$.
    Supongamos que $m_f=\prod_{i=1}^k(X-\lambda_i)^{r_i}$, donde
    $\lambda_i\ne\lambda_j$ si $i\ne j$. Entonces existe una base de $V$ tal que
    en esa base $f$ es diagonal por bloques donde cada bloque es 
    de la forma
    \[
        \begin{pmatrix}
            \lambda_i\\
            1 & \lambda_i\\
            0 & 1 & \lambda_i\\
            \vdots & \vdots & \ddots & \ddots\\
            0&0&\cdots&1&\lambda_i
        \end{pmatrix}.
    \]
    Estos bloques se denominan \textbf{bloques de Jordan} de $f$.

    \begin{proof}
        Por la descomposición de Jordan--Chevalley,
        corolario~\ref{cor:JordanChevalley}, sabemos que existe $g$
        diagonalizable y $h$ nilpotente tales que $f=g+h$ y $gh=hg$. Como $g$
        es diagonalizable, $V=\oplus_{i=1}^kS(\lambda_i)$, donde
        $S(\lambda)=\{v\in V:g(v)=\lambda v\}$.

        Para cada $i\in\{1,\dots,k\}$ consideremos una base de $S(\lambda_i)$
        que corresponda a la descomposición cíclica de $h|_{S(\lambda_i)}$.
        Como $h|_{S(\lambda_i)}$ es nilpotente, en esa base puede escribirse
        como
        \[
       	\begin{pmatrix}
			0 & 0 & \cdots & 0 & 0\\
			1 & 0 & \cdots & 0 & 0\\
			0 & 1 & \cdots & 0 & 0\\
			\vdots & \vdots & \ddots & \vdots & \vdots\\
			0 & 0 & \cdots & 1 & 0
		\end{pmatrix}.
        \]
        Entonces, en esa misma base, $f|_{S(\lambda_i)}=(g+h)|_{S(\lambda_i)}$
        se escribe como
        \[
        \begin{pmatrix}
            \lambda_i\\
            1 & \lambda_i\\
            0 & 1 & \lambda_i\\
            \vdots & \vdots & \ddots & \ddots\\
            0&0&\cdots&1&\lambda_i
        \end{pmatrix}.
        \]
        Como $V=\oplus_{i=1}^kS(\lambda_i)$ y los $S(\lambda_i)$ son
        $h$-invariantes pues
        \[
            g(h(v))=(gh)(v)=(hg)(v)=h(g(v))=h(\lambda_iv)=\lambda_ih(v),
        \]
        el teorema queda demostrado.
    \end{proof}
\end{thm}

%%% unicidad como ejercicio!

\begin{block}
    \index{Jordan!forma de!unicidad}
    Sea $V$ de dimensión finita y sea $f\in\hom(V,V)$. Vamos a demostrar que la
    forma de Jordan de $f$ queda unívocamente determinada. 
    
    Si $V$ tiene una base tal que $f$ en esa base puede escribirse en forma de
    Jordan, digamos con bloques de Jordan $J_1,\dots,J_k$ tales que $J_i$ es
    de tamaño $n_i\times n_i$ y tiene al escalar $\lambda_i$ en su diagonal.
    Supongamos además que $\lambda_i\ne\lambda_j$ si $i\ne j$. Entonces, como
    $\chi_f=\prod_{i=1}^k(X-\lambda_i)^{n_i}$, tenemos que $n_i$ es la
    multiplicidad de $\lambda_i$ como raíz de $\chi_f$. Luego los $\lambda_i$ y
    los $n_i$ quedan unívocamente determinados salvo por el orden con el que
    aparecen. 

    La forma de Jordan de $f$ nos dice que $V=V_1\oplus\cdots\oplus V_k$ donde
    los $V_i$ son $f$-invariantes. Veamos que $V_i=\ker (f-\lambda_i\id_V)^n$,
    donde $n=\dim V$. Sea $i\in\{1,\dots,k\}$. La matriz $J_i-\lambda_iI$ es
    nilpotente y entonces $(J_i-\lambda_iI)^n=0$. Por otro lado, si $j\ne i$,
    entonces $J_j-\lambda_iI$ es inversible pues es triangular y tiene
    elementos no nulos en la diagonal. Luego $V_i=\ker(f-\lambda_i\id_V)^n$,
    donde $n=\dim V$, y entonces los $V_i$ están unívocamente determinados.

    Para cada $i\in\{1,\dots,k\}$ sea $f_i=f|_{V_i}$. Entonces el bloque de
    Jordan $J_i$ queda unívocamente determinado pues es la forma racional del
    endomorfismo $f_i-\lambda_i\id_{V_i}$. 
\end{block}

\begin{cor}
    Sea $A\in\C^{n\times n}$. Entonces existen matrices simétricas
    $B,C\in\C^{n\times n}$ tales que $A=BC$.

    \begin{proof}
        Existe $P\in\C^{n\times n}$ inversible tal que $A=PJP^{-1}$, donde $J$
        es la forma de Jordan de $A$.  Si $J=KL$ con $K$ y $L$ son matrices
        simétricas, entonces, si $Q=P^T$, tenemos que $A$ es producto de
        matrices simétricas pues 
        \[
        A=PJP^{-1}=PKLP^{-1}=(PKQ)(Q^{-1}LP^{-1}),
        \]
        y las matrices $PKQ$ y $Q^{-1}LP^{-1}$ son simétricas. 
        Veamos entonces que toda matriz de Jordan puede escribirse como
        producto de dos matrices simétricas. Supongamos que $J$ tiene bloques
        de Jordan $J_1,\dots,J_k$, donde $J_p$ es de tamaño $n_p\times n_p$.
        Para cada $p\in\{1,\dots,k\}$ definimos $D_p\in\C^{n\times n}$ por
        \[
        (D_p)_{ij}=\begin{cases}
            1 & \text{si $i+j=n_p+1$},\\
            0 & \text{en otro caso}.
        \end{cases}
        \]
        Entonces $D_p$ es simétrica e inversible con $D_p^{-1}=D_p$. Sean
        \[
            C=\begin{pmatrix}
                J_1D_1\\
                & J_2D_2\\
                & & \ddots\\
                && & J_kD_k
            \end{pmatrix},
            \quad
            D=\begin{pmatrix}
                D_1\\
                & D_2\\
                & & \ddots\\
                && & D_k
            \end{pmatrix}.
        \]
        Entonces $C$ y $D$ son simétricas y $J=CD$. 
    \end{proof}
\end{cor}
%\begin{example}
%    Sea
%    \[
%        A=\begin{pmatrix}
%            3 & 4 & -6\\
%            5 & 6 & -10\\
%            2 & 4 & -5
%        \end{pmatrix}.
%    \]
%    Entonces $\chi_A=(X-1)^2(X-2)$. 
%\end{example}

\chapter{Espacios con producto interno}

\section{Definiciones básicas y ejemplos}

\begin{block}
	\index{Producto interno}
	\index{Espacio vectorial!con producto interno}
	Sea $\K$ el cuerpo $\R$ de los números reales o el cuerpo $\C$ de los
	números complejos. Sea $V$ un espacio vectorial sobre $\K$.  Una función
	$\langle\cdot,\cdot\rangle\colon V\times V\to\K$ es un \textbf{producto
	interno} si:
	\begin{enumerate}
		\item $\langle v_1+v_2,w\rangle=\langle v_1,w\rangle+\langle v_2,w\rangle$ para todo $v_1,v_2,v\in V$.
		\item $\langle \lambda v,w\rangle=\lambda\langle v,w\rangle$ para todo $v\in V$ y $\lambda\in\K$.
		\item $\langle v,w\rangle=\overline{\langle w,v\rangle}$ para todo $v,w\in V$.
		\item $\langle v,v\rangle\in\R_{\geq0}$ para todo $v\in V$ y $\langle v,v\rangle=0$ si y sólo si $v=0$. 
	\end{enumerate}
	Un \textbf{espacio vectorial con producto interno} es un par
	$(V,\langle\cdot,\cdot\rangle)$, donde $V$ es un espacio vectorial (real o
	complejo) y $\langle\cdot,\cdot\rangle$ es un producto interno en $V$. 
\end{block}

\begin{block}
    Si $V$ es un espacio vectorial con producto interno entonces 
    \begin{align*}
        &\langle v,w_1+w_2\rangle=\langle v,w_1\rangle+\langle v,w_2\rangle,\\
        &\langle v,\lambda w\rangle=\overline{\lambda}\langle v,w\rangle, 
    \end{align*}
    para todo $v,w,w_1,w_2\in V$ y $\lambda\in\K$. 
\end{block}

\begin{xca}
	Sea $V$ un espacio vectorial real o complejo y sea $p\in V$.  Si
	$\langle\cdot,\cdot\rangle$ es un producto interno en $V$ entonces $\langle
	v,w\rangle_p=\langle v-p,w-p\rangle$ es un producto interno en $V_p$. 
\end{xca}

\begin{xca}
    \framebox{FIXME}
    Demuestre que la función
    \[
    \langle\cdot,\cdot\rangle\colon\R^2\times\R^2\to\R,\quad
        \langle (x,y),(x',y')\rangle=x(y-y')-x'(y-\lambda y')
    \]
    es un producto interno en $\R^2$ si y sólo si $\lambda>1$. 
\end{xca}

\begin{xca}
	\label{xca:<v-w,x>=0}
    Sea $V$ un espacio vectorial con producto interno y sean $v,w\in V$.
    Demuestre que si $\langle v,x\rangle=\langle w,x\rangle$ para todo $x\in V$
    entonces $v=w$. 
\end{xca}

\begin{examples}\
    \begin{enumerate}
        \item La función
            $\langle\cdot,\cdot\rangle\colon\R^n\times\R^n\to\R$ definida por
            \[
                \langle(x_1,\dots,x_n),(y_1,\dots,y_n)\rangle=\sum_{i=1}^n x_iy_i
            \]
            es un producto interno en $\R^n$. 
        \item La función
            $\langle\cdot,\cdot\rangle\colon\C^n\times\C^n\to\C$ definida por
            \[
            \langle(x_1,\dots,x_n),(y_1,\dots,y_n)\rangle=\sum_{i=1}^n x_i\overline{y_i}
            \]
            es un producto interno en $\C^n$. 
        \item Si $V$ y $W$ son espacios vectoriales con producto interno
            entonces
            \[
                \langle (v_1,w_1),(v_2,w_2)\rangle=\langle v_1,v_2\rangle_V+\langle w_1,w_2\rangle_W
            \]
            es un producto interno en $V\times W$.
        \item Si $A,B\in\C^{m\times n}$ entonces $\langle
            A,B\rangle=\tr(B^*A)$, donde $(B^*)_{ij}=\overline{B_{ji}}$, es un
            producto interno en $\C^{m\times n}$. 
        \item La función
            \[
                \langle f,g\rangle=\int_{a}^{b}f(x)g(x)dx
            \]  
            es un producto interno en el espacio vectorial (real) de funciones
            continuas $[a,b]\to\R$. 
        \item Sea $\ell^2(\C)$ el espacio vectorial de sucesiones $(a_n)_{n\in\N}$ tales que
            $\sum_{n\geq0}|a_n|^2<\infty$. Entonces
            \[
                \langle (a_1,a_2,\dots),(b_1,b_2,\dots)\rangle =\sum_{n\geq0}a_n\overline{b_n}
            \]
            es un producto interno de $\ell^2(\C)$. En efecto, como 
            \[
            0\leq(|a_n|-|b_n|)^2=|a_n|^2-2|a_n||b_n|+|b_n|^2,
            \]
            entonces $|a_n||b_n|\leq\frac12(|a_n|^2+|b_n|^2)$.
    \end{enumerate}
\end{examples}

\begin{prop}[desigualdad de Cauchy--Schwarz]
    \label{pro:CauchySchwarz}
	\index{Cauchy--Schwarz!desigualdad de}
    Sea $V$ un espacio vectorial con producto interno. Entonces 
    \[
        \langle v,w\rangle^2\leq\langle v,v\rangle\langle w,w\rangle
%        |\langle v,w\rangle|\leq\|v\|\|w\|
    \]
    para todo $v,w\in V$.

    \begin{proof}
        Si $\lambda=\frac{\langle v,w\rangle}{\langle w,w\rangle}$ entonces
        $\langle w,v\rangle-\overline{\lambda}\langle w,w\rangle=0$. Luego
        \begin{align*}
            0\leq \langle v-\lambda w,v-\lambda w\rangle&=\langle v,v\rangle-\overline{\lambda}\langle v,w\rangle-\lambda(\langle w,v\rangle-\overline{\lambda}\langle w,w\rangle)\\
            &=\langle v,v\rangle-\overline{\lambda}\langle v,w\rangle,
        \end{align*}
        que implica que $\overline{\lambda}\langle v,w\rangle\leq\langle v,v\rangle$. Esto demuestra la proposición.
%        Luego
%        \[
%            |\langle v,w\rangle|^2=\langle v,w\rangle\overline{\langle w,v\rangle}\leq\|v\|^2\|w\|^2,
%        \]
%        que implica la desigualdad que queríamos demostrar.
    \end{proof}
\end{prop}

\begin{examples}
	La desigualdad de Cauchy--Schwarz es una rica fuente de desigualdades. Por
	ejemplo, si $\lambda_1,\dots,\lambda_n\in\R$, la formula 
	\[
		\langle (a_1,\dots,a_n),(b_1,\dots,b_n)\rangle=\sum_{i=1}^n\lambda_ia_ib_i
	\]
	define un producto interno en $\R^n$ y con la desigualdad de
	Cauchy--Schwarz se obtiene 
	\[
	\left|\sum_{i=1}^n\lambda_ia_ib_i\right|\leq\sqrt{\sum_{i=1}^n\lambda_ia_i^2}\sqrt{\sum_{i=1}^n\lambda_ib_i^2}.
	\]
	Similarmente, si se aplica al producto interno en $C[0,1]$ dado por 
	\[
		\langle f,g\rangle=\int_0^1 f(x)g(x)dx
	\]
	se obtiene
	\[
		\left|\int_0^1f(x)g(x)dx\right|\leq\sqrt{\int_0^1f(x)^2dx}\sqrt{\int_0^1g(x)^2dx}.
	\]
\end{examples}

\begin{block}
    Si $V$ es un espacio vectorial con producto interno entonces la función
    $\|\cdot\|\colon V\to\R_{\geq0}$, dada por $v\mapsto \sqrt{\langle
    v,v\rangle}$, satisface las siguientes propiedades:
    \begin{enumerate}
        \item $\|v\|=0$ si y sólo si $v=0$.
        \item $\|\lambda v\|=|\lambda|\|v\|$ para todo $v\in V$.
        \item (desigualdad triangular) $\|v+w\|\leq\|v\|+\|w\|$ para todo $v,w\in V$. 
    \end{enumerate}

	Las primeras dos afirmaciones quedan como ejercicio. Para demostrar la
	desigualdad triangular utilizamos la proposición~\ref{pro:CauchySchwarz}:
    \begin{align*}
        0\leq\|v+w\|^2&=\langle v+w,v+w\rangle=\|v\|^2+\langle v,w\rangle+\langle w,v\rangle+\|w\|^2\\
        &=\|v\|^2+\langle v,w\rangle+\overline{\langle v,w}\rangle+\|w\|^2\\
        &=\|v\|^2+2\re\langle v,w\rangle+\|w\|^2\\
        &\leq\|v\|^2+2|\langle v,w\rangle|+\|w\|^2\\
		&\leq (\|v\|+\|w\|)^2.
    \end{align*}
	Luego $\|v+w\|\leq\|v\|+\|w\|$, tal como queríamos demostrar. 
\end{block}

\begin{xca}[identidades de polarización]\
    \begin{enumerate}
        \item En un espacio vectorial real con producto interno vale 
            \[
                \langle v,w\rangle=\frac14\left(\|v+w\|^2-\|v-w\|^2\right)
            \]
            para todo $v,w\in V$.
        \item En un espacio vectorial complejo con producto interno vale 
            \[
            \langle v,w\rangle=\frac14\sum_{k=1}i^k\|v-i^kw\|^2
            \]
            para todo $v,w\in V$.
    \end{enumerate}
\end{xca}

\begin{block}
	\index{Distancia entre vectores}
	Sea $V$ un espacio vectorial con producto interno y sean $v,w\in V$. Se
	define la \textbf{distancia} entre $v$ y $w$ como $\dist(v,w)=\|v-w\|$. La
	función $\dist\colon V\times V\to\R$ satisface las siguientes propiedades:
	\begin{enumerate}
		\item $\dist(v,w)\geq0$ para todo $v,w\in V$.
		\item $\dist(v,w)=0$ si y sólo si $v=w$.
		\item $\dist(v,w)=\dist(w,v)$ para todo $v,w\in V$.
		\item $\dist(v,w)\leq\dist(v,u)+\dist(u,w)$ para todo $u,v,w\in V$. 
	\end{enumerate}
\end{block}

\section{Ortogonalidad}

\begin{block}
	\index{Vectores!orgotonales}
    Sea $V$ un espacio vectorial con producto interno y sean $v,w\in V$.
    Entonces $v$ y $w$ son \textbf{ortogonales}, $v\perp w$, si $\langle
    v,w\rangle=0$. Observemos que $v\perp w$ si y sólo si $w\perp v$. Si $S$ y
    $T$ son subespacios de $V$ entonces $S$ y $T$ son ortogonales, $S\perp T$,
    si $\langle s,t\rangle=0$ para todo $s\in S$ y $t\in T$. Si $S\perp T$
    entonces $T\perp S$. Si $X\subseteq V$ es un subconjunto, se define
	\[
		X^\perp=\{v\in V:\langle v,x\rangle=0\text{ para todo $x\in X$}\}.
	\]
	Queda como ejercicio demostrar que $X^\perp$ es un subespacio de $V$. 
\end{block}

\begin{example}
    Todo vector es ortogonal a $0\in V$. Recíprocamente, si $v\in V$ satisface
    que $v\perp w$ para todo $w$ entonces $v=0$.
\end{example}

\begin{example}
    Si consideramos a $\R^2$ con el producto interno usual y $S=\langle
    (1,1)\rangle$ entonces $S^\perp=\langle (1,-1)\rangle$. 
\end{example}

\begin{block}
	\index{Vector!unitario}
	\index{Vectores!ortonormales}
	Sea $V$ un espacio vectorial con producto interno. Los vectores 
	$v_1,\dots,v_n\in V$ son \text{ortogonales} si $\langle v_i,v_j\rangle=0$
	si $i\ne j$. Un vector $v\in V$ es \textbf{unitario} si $\|v\|=1$. Los vectores 
	no nulos $v_1,\dots,v_n\in V$ son \textbf{ortonormales} si $\langle
	v_i,v_j\rangle=\delta_{ij}$
\end{block}

\begin{examples}
    Si consideramos a $\R^2$ con el producto interno usual, la base canónica es
    un conjunto ortonormal.  El conjunto $\{(1,1),(1,-1)\}\subseteq\R^2$ es
    ortogonal y no ortonormal. 
\end{examples}

\begin{example}
	En el espacio vectorial real de funciones continuas
	$f\colon(-\pi,\pi)\to\R$ con el producto interno 
	\[
		\langle f,g\rangle=\frac1\pi\int_{-\pi}^\pi f(x)g(x)dx,
	\]
	el conjunto
	\[
	\left\{\frac{1}{\sqrt{2}}\right\}\cup\{\sin(nx):n\in\N\}\cup\{\cos(nx):n\in\N\}
	\]
	es ortonormal.
\end{example}

\begin{xca}[teorema de Pitágoras]
	\label{xca:Pitagoras}
	\index{Pitágoras!teorema de}
	\index{Teorema!de Pitágoras}
	Sea $V$ un espacio vectorial con producto interno y sean 
	$v,w\in V$. Demuestre que si $v\perp w$ entonces \[
		\|v+w\|^2=\|v\|^2+\|w\|^2. 
	\]
\end{xca}

\begin{xca}
    \label{xca:Sperp}
	Sea $V$ un espacio vectorial con producto interno y sean $S,T\subseteq V$.
	Demuestre las siguientes afirmaciones:
	\begin{enumerate}
		\item $\{0\}^\perp=V$ y $V^\perp=\{0\}$. 
		\item Si $S\subseteq T$ entonces $T^\perp\subseteq S^\perp$.
		\item $(S+T)^\perp=S^\perp\cap T^\perp$.
	\end{enumerate}
\end{xca}

\begin{prop}
    \label{pro:autoadjunto}
	Sea $V$ un espacio vectorial con producto interno y sean $v_1,\dots,v_n\in
	V$ vectores no nulos y ortogonales. Valen entonces las siguientes
	afirmaciones:
	\begin{enumerate}
		\item Para cada $v\in\langle v_1,\dots,v_n\rangle$ se tiene
			\[
			v=\frac{\langle v,v_1\rangle}{\|v_1\|^2} v_1+\cdots+\frac{\langle v,v_n\rangle}{\|v_n\|^2} v_n.  
			\]
		\item El conjunto $\{v_1,\dots,v_n\}$ es linealmente independiente.
	\end{enumerate}

	\begin{proof}
		Si $v=\sum_{i=1}^n\alpha_iv_i$ y $j\in\{1,\dots,n\}$ entonces \[
			\langle v,v_j\rangle=\left\langle\sum_{i=1}^n\alpha_iv_i,v_j\right\rangle=\sum_{i=1}^n\alpha_i\langle v_i,v_j\rangle=\alpha_j\|v_j\|^2,
		\]
		lo que demuestra la primera afirmación. Para demostrar la segunda
		afirmación basta tomar $v=0$ en el ítem anterior.
	\end{proof}
\end{prop}

\begin{cor}
	Sea $V$ un espacio vectorial con producto interno y sean $v_1,\dots,v_n\in
	V$ vectores no nulos y ortonormales. Si $v=\sum_{i=1}^n\alpha_i v_i$
	entonces
	\begin{align*}
		&v=\langle v,v_1\rangle v_1+\cdots+\langle v,v_n\rangle v_n,
		&&\|v\|^2=|\langle v,v_1\rangle|^2+\cdots+|\langle v,v_n\rangle|^2.
	\end{align*}

	\begin{proof}
		La primera fórmula se deduce de la proposición anterior pues
		$\|v_i\|=1$ para todo $i\in\{1,\dots,n\}$. Para la segunda afirmación
		calculamos
		\begin{align*}
			\|v\|^2&=\langle v,v\rangle
			=\left\langle \sum_{i=1}^n\alpha_iv_i,\sum_{j=1}^n\alpha_jv_j\right\rangle
			=\sum_{i=1}^n\sum_{j=1}^n\alpha_i\overline{\alpha_j}\langle v_i,v_j\rangle,
			=\sum_{i=1}^n|\alpha_i|^2
		\end{align*}
		y utlizamos que $\alpha_i=\langle v,v_i\rangle$ para todo
		$i\in\{1,\dots,n\}$. 
	\end{proof}
\end{cor}

%\begin{block}
%    Sean $V$ y $W$ espacio vectoriales con producto interno. Una
%    \textbf{isometría} es un isomorfismo $f\colon V\to W$ tal que $\langle
%    f(x),f(y)\rangle=\langle x,y\rangle$. 
%\end{block}

\begin{thm}[proceso de ortonormalización de Gram-Schmidt]
	Sea $V$ un espacio vectorial con producto interno y sea
	$\{v_1,\dots,v_n\}\subseteq V$ un conjunto linealmente independiente.
	Entonces existe $\{e_1,\dots,e_n\}\subseteq V$ ortonormal tal que para cada
	$k\in\{1,\dots,n\}$ se tiene que $\langle v_1,\dots,v_k\rangle=\langle
	e_1,\dots,e_k\rangle$.
\end{thm}

\begin{proof}
	Procederemos por inducción en $n$. 

	El caso $n=1$ es trivial pues, como
	$v_1\ne0$, basta tomar $e_1=\frac1{\|v_1\|}{v_1}$. Supongamos que
	el resultado es válido para $n-1$. Como $\{v_1,\dots,v_{n-1}\}$ es
	linealmente independiente, existen $e_1,\dots,e_{n-1}\in V$ ortonormales y
	tales que $\langle e_1,\dots,e_k\rangle=\langle v_1,\dots,v_k\rangle$ para
	todo $k\in\{1,\dots,n-1\}$. Sea
	\[
		e_n'=v_n-\sum_{i=1}^{n-1}\langle v_n,e_i\rangle e_i.
	\]
	Como $\{v_1,\dots,v_n\}$ es linealmente independiente, entonces $e_n'\ne0$.
	En efecto, si fuera $e_n'=0$ tendríamos que $v_n\in\langle
	e_1,\dots,e_{n-1}\rangle=\langle v_1,\dots,v_{n-1}\rangle$. Si $k\in\{1,\dots,n-1\}$ entonces
	\[
	\langle e_n',e_k\rangle=\left\langle v_n-\sum_{i=1}^{n-1}\langle v_n,e_i\rangle e_i,e_k\right\rangle
	=\langle v_n,e_k\rangle-\sum_{i=1}^n\langle v_n,e_i\rangle\langle e_i,e_k\rangle=0.
	\]
	Luego, si $e_n=\frac1{\|e_n'\|}e_n'$ entonces $\|e_n\|=1$ y el conjunto
	$\{e_1,\dots,e_n\}$ es ortonormal. Además $\langle
	e_1,\dots,e_k\rangle=\langle v_1,\dots,v_k\rangle$ para todo
	$k\in\{1,\dots,n\}$. 
\end{proof}

%\begin{block}
%	Sea $V$ un espacio vectorial con producto interno y de dimensión finita.
%	Si $\cB=\{v_1,\dots,v_n\}$ es una base y $\cE=\{e_1,\dots,e_n\}$ es la base
%	que se obtiene después de aplicar el proceso de ortonormalización de
%	Gram-Schmidt, entonces $C(\cB,\cE)$ es triangular superior con elementos
%	positivos en la diagonal.
%\end{block}

\begin{cor}
	Sea $V$ un espacio vectorial con producto interno y de dimensión finita.
	Entonces $V$ tiene una base ortonormal.
\end{cor}

\begin{cor}
	\label{cor:complemento_ortogonal}
	Sea $V$ un espacio vectorial con producto interno y de dimensión finita. Si
	$S\subset V$ es un subespacio entonces $V=S\oplus S^\perp$. En particular, 
	$\dim V=\dim S+\dim S^\perp$, 

	\begin{proof}
		Sea $\{v_1,\dots,v_m\}$ una base de $S$. Extendemos este conjunto
		linealmente independiente a una base
		$\{v_1,\dots,v_m,v_{m+1},\dots,v_n\}$ de $V$. Si utilizamos el proceso
		de Gram--Schmidt tenemos una base ortornormal
		$\{e_1,\dots,e_m,e_{m+1},\dots,e_n\}$ de $V$. Por construcción,
		$\{e_1,\dots,e_m\}$ es base de $S$. Veamos que
		$\{e_{m+1},\dots,e_{n}\}$ es base de $S^\perp$. Si $v\in S^\perp$, escribimos
		$v=\sum_{i=1}^n\alpha_ie_i$. Como $v\in S^\perp$, se tiene que 
		$0=\langle v,v_j\rangle$ para todo $j\in\{1,\dots,m\}$ y luego
		$S^\perp\subseteq\langle e_{m+1},\dots,e_n\rangle$. 
	\end{proof}
\end{cor}

\begin{example}
	El corolario~\ref{cor:complemento_ortogonal} no vale en dimensión infinita.
	Si $V$ es el espacio vectorial (real) de funciones continuas $[0,1]\to\R$ y
	\[
	S=\{f\in V:f\text{ es derivable en $(0,1)$}\}
	\]
	entonces $S^\perp=\{0\}$. En efecto, si $h\in S^\perp$ entonces
	$\int_{0}^{1}f(x)h(x)dx=0$ para todo $f\in S$. En particular, si tomamos
	la función constantemente igual a uno, se tiene que $\int_0^1h(x)dx=0$.
	Sea $g(s)=\int_0^sh(x)dx$, $s\in[0,1]$. Entonces $g\in S$, $g(0)=g(1)=0$ y
	$g'(x)=h(x)$ para todo $x\in(0,1)$. Entonces, para toda $f\in S$, se tiene 
	\begin{align*}
		0&=(fg)(1)-(fg)(0)=f(1)g(1)-f(0)g(0)=\int_0^1 (fg)'(x)dx\\
		&=\int_0^1f'(x)g(x)dx+\int_0^1f(x)g'(x)dx\\
		&=\int_0^1f'(x)g(x)dx+\int_0^1f(x)h(x)dx\\
		&=\int_0^1f'(x)g(x)dx,
	\end{align*}
	pues $f\in S$ y $h\in S^\perp$. En particular, si $f(s)=\int_0^sg(x)dx$,
	$s\in[0,1]$, entonces $f'(x)=g(x)$ para todo $x\in[0,1]$, 
	\[
		0=\int_0^1f'(x)g(x)dx=\int_0^1g^2(x)dx,
	\]
	y luego $g(x)=0$. Por lo tanto $h(x)=g'(x)=0$ para todo $x\in[0,1]$. 
\end{example}

\begin{xca}
	\label{xca:perpperp}
	Sea $V$ un espacio vectorial con producto interno y de dimensión finita. Si
	$S\subseteq V$ es un subespacio entonces $(S^{\perp})^{\perp}=S$. 
\end{xca}

\section{Proyección ortogonal}

\begin{block}
	Sea $V$ un espacio vectorial con producto interno. Sea $v\in V$ y sea
	$S\subseteq V$ un subespacio. El vector $v_S\in S$ es una \textbf{mejor
	aproximación} a $v$ por vectores de $S$ si $\dist(v,v_S)\leq\dist(v,w)$
	para todo $w\in S$. 
\end{block}

\begin{thm}
	\label{thm:mejor_aproximacion}
	Sean $V$ un espacio vectorial con producto interno y $S$ un subespacio de
	$V$. Sean $v\in V$ y $v_s\in S$.  Si $v-v_S\perp S$ entonces   $v_S$ es una
	mejor aproximación para $v$ por vectores de $S$. 

	\begin{proof}
		Supongamos que $v-v_S\perp S$ y sea $w\in S$. Entonces, como $v_S-w\in
		S$, se tiene que 
		\begin{align*}
			\|v-w\|^2&=\|v-v_S\|^2+2\re\langle v-v_S,v_S-w\rangle+\|v_S-w\|^2\\
			&=\|v-v_S\|^2+\|v_S-w\|^2\\
			&\geq \|v-v_S\|^2,
		\end{align*}
		que implica la desigualdad que se quería demostrar.
	\end{proof}
\end{thm}

\begin{cor}
	Sean $V$ un espacio vectorial con producto interno y $S$ un subespacio de
	$V$ de dimensión finita. Sean $v\in V$ y $\{v_1,\dots,v_m\}$ una base
	ortonormal de $S$. Entonces 
	\[
		\sum_{i=1}^m\langle v,v_i\rangle v_i
	\]
	es la mejor aproximación a $v$ por vectores de $S$.

	\begin{proof}
		Demostremos que $v-\sum_{i=1}^m\langle
		v,v_i\rangle v_i$ es ortogonal a todo
		elemento de $S$. Si $v_s=\sum_{i=1}^m\langle
		v,v_i\rangle v_i$ y 
		$w=\sum_{j=1}^m\langle w,v_j\rangle v_j\in S$, 
		un cálculo directo muestra que 
		\begin{align*}
			&\langle v_S,w\rangle
			=\sum_{i=1}^m\sum_{j=1}^m\overline{\langle w,v_j\rangle}\langle v,v_i\rangle\langle v_i,v_j\rangle
			=\sum_{i=1}^m\overline{\langle w,v_i\rangle}\langle v,v_i\rangle
			=\langle v,w\rangle.
		\end{align*}
		Luego, como $\langle v-v_S,w\rangle=0$, concluimos 
		que $v_S$ es la mejor aproximación a $v$ por vectores de $S$.
	\end{proof}
\end{cor}

\begin{block}
	\index{Proyector ortogonal}
	Un proyector $p\colon V\to V$ es un \textbf{proyector ortogonal} si $\im p$
	y $\ker p$ son ortogonales. 
\end{block}

\begin{xca}
	\label{xca:proyector_ortogonal}
	Sea $V$ un espacio vectorial con producto interno y sea $p\colon V\to V$ un
	proyector ortogonal. Demuestre que valen las siguientes afirmaciones:
	\begin{enumerate}
		\item $(\im p)^\perp=\ker p$.
		\item $(\ker p)^\perp=\im p$.
		\item $(\im p)^{\perp\perp}=\im p$. 
	\end{enumerate}
\end{xca}

\begin{prop}
	Sea $V$ un espacio vectorial con producto interno y de dimensión finita y sea
	$S\subseteq V$ un subespacio. Entonces existe un único proyector ortogonal
	$p\colon V\to V$ tal que $\im p=S$ y $\ker p=S^\perp$. Más aún, si
	$\{v_1,\dots,v_m\}$ es una base ortonormal de $S$ entonces
	\[
		p(v)=\sum_{i=1}^n\langle v,v_i\rangle v_i.
	\]

	\begin{proof}
		Es claro que $p$ es una transformación lineal. Además $p^2=p$ pues,
		como $p(v_i)=v_i$ para todo $i\in\{1,\dots,n\}$, 
		\[
			p^2(v)=\sum_{i=1}^n\langle v,v_i\rangle p(v_i)=\sum_{i=1}^n\langle v,v_i\rangle v_i=p(v).
		\]

		Demostremos que $\im p=S$. Es evidente que $\im p\subseteq S$. Por otro
		lado, si $v\in S$ entonces $v=\sum_{i=1}^n \langle v,v_i\rangle
		v_i=p(v)$. 

		Demostremos ahora que $\ker p=S^\perp$. Para la inclusión $\ker
		p\subseteq S^\perp$ observamos que si $v\in \ker p$ entonces, como los
		$v_i$ son base de $S$, se tiene que $\langle v,v_i\rangle=0$ para todo
		$i$. Por otro lado, si $v\in S^\perp$ entonces $\langle v,v_i\rangle=0$
		para todo $i\in\{1,\dots,n\}$ y luego $p(v)=0$. 

		Por último demostremos la unicidad. Sean dos proyectores 
		ortogonales $p$ y $q$ tales que $S=\ker p=\ker q$ y $S^\perp=\im p=\im q$. 
		Como $V=S\oplus S^\perp$, todo $v\in V$ se escribe unívocamente como
		$v=v_1+v_2$, donde $v_1\in S$ y $v_2\in S^\perp$. Entonces, como
		$p(w)=q(w)=w$ para todo $w\in S^\perp$, 
		\begin{align*}
			p(v)&=p(v_1+v_2)=p(v_1)+p(v_2)\\
			&=p(v_2)=v_2=q(v_2)=q(v_1)+q(v_2)=q(v_1+v_2)=q(v),
		\end{align*}
		tal como se quería demostrar.
	\end{proof}
\end{prop}

\begin{xca}
	\label{xca:pSpT}
	Sea $V$ un espacio vectorial con producto interno y sean $p_S$ $p_T$
	proyectores ortogonales tales que $\im p_S\perp\im p_T$. Demuestre que
	entonces $p_Sp_T=0$. 
\end{xca}

\section{Transformaciones lineales adjuntas}

\begin{thm}[teorema de representación de Riesz]
	\index{Riesz!teorema de}
	\index{Teorema!de Riesz}
	\label{thm:interno:Riesz}
	Sea $V$ un espacio vectorial con producto interno y de dimensión finita.  Si
	$f\in V^*$ entonces existe un único vector $v_f\in V$ tal que $f(v)=\langle
	v,v_f\rangle$ para todo $v\in V$.

	\begin{proof}
		Sea $\{v_1,\dots,v_n\}$ una base ortonormal de $V$. Definimos
		$v_f=\sum_{i=1}^n\overline{f(v_i)}v_i$. y sea $v\in V$. Vamos a
		demostrar que $f(v)=\langle v,v_f\rangle$. En efecto, si escribimos
		$v=\sum_{i=1}^n\langle v,v_i\rangle v_i$, entonces 
		\[
		f(v)=\sum_{i=1}^n\langle v,v_i\rangle f(v_i)=\left\langle v,\sum_{i=1}^n\overline{f(v_i)}v_i\right\rangle=\langle v,v_f\rangle.
		\]

		Veamos la unicidad: sean $v_f,v_f'\in V$ tales que $\langle
		v,v_f\rangle=f(v)=\langle v,v_f'\rangle$ para todo $v\in V$. Entonces, como 
		$\langle v,v_f-v_f'\rangle=0$ para todo $v\in v$, se concluye que $v_f-v_f'=0$. 
	\end{proof}
\end{thm}

\begin{block}
	\index{Adjunta}
    Sean $V$ y $W$ espacios vectoriales con producto interno. Una
    transformación lineal $g\in\hom(W,V)$ es una \textbf{adjunta} para
    $f\in\hom(V,W)$ si $\langle f(v),w\rangle=\langle v,g(w)\rangle$ para todo
    $v\in V$ y $w\in W$. Observemos que, de existir, la adjunta es única. Si
    $g_1,g_2\hom(W,V)$ son adjuntas para $f$ entonces, para cada $v\in V$
    se tiene que 
	\[
		\langle v,g_1(w)-g_2(w)\rangle=\langle v,g_1(w)\rangle-\langle v,g_2(w)\rangle=\langle f(v),w\rangle-\langle f(v),w\rangle=0,
	\]
	y se concluye que $g_1(w)=g_2(w)$ para todo $w\in W$. 
\end{block}

\begin{example}
	Si $V$ es un espacio vectorial con producto interno entonces
	$\id_V^*=\id_V\colon V\to V$ pues entonces $\langle
	v,\id_V(w)\rangle=\langle v,w\rangle=\langle \id_V(v),w\rangle$ para todo
	$v,w\in V$.
\end{example}

\begin{thm}
    \label{thm:adjunta:existencia}
    Sean $V$ y $W$ espacios vectoriales con producto interno y de dimensión
    finita y sea $f\in\hom(V,W)$. Entonces existe una única adjunta de $f$, que
    será denotada por $f^*$. 

	\begin{proof}
		Para cada $w\in W$ definimos 
		\[
			\phi_w\colon V\to\K,
			\quad
			v\mapsto\langle f(v),w\rangle.
		\]
		Como $\phi_w\in V^*$, el teorema de representación de Riesz,
		teorema~\ref{thm:interno:Riesz}, implica que existe un único vector
		$f^*(w)\in V$ tal que $\langle f(v),w\rangle=\langle v,f^*(w)\rangle$ para todo $v\in V$.
		Para completar la demostración del teorema necesitamos ver que
		$f^*\in\hom(W,V)$. Si $v\in V$ y $w_1,w_2\in W$ entonces
		\begin{align*}
			\langle v,f^*(w_1&+w_2)-f^*(w_1)-f^*(w_2)\rangle\\
			&=\langle v,f^*(w_1+w_2)\rangle-\langle v,f^*(w)\rangle-\langle v,f^*(w_2)\rangle\\
			&=\langle f(v),w_1+w_2\rangle-\langle f(v),w_1\rangle-\langle f(v),w_2\rangle=0.
		\end{align*}
		Luego $f^*(w_1+w_2)-f^*(w_1)-f^*(w_2)=0$
        para todo $w_1,w_2\in W$.
        Similarmente, si $v\in V$, $w\in W$ y
		$\lambda\in\K$, entonces 
		\begin{align*}
			\langle v,f^*(\lambda w)-\lambda f^*(w)\rangle=0
		\end{align*}
		y luego $f^*(\lambda w)=\lambda f(w)$ para todo $w\in W$ y $\lambda\in\K$. 
	\end{proof}
\end{thm}

\begin{example}
	Mostraremos que en espacios de dimensión finita no siempre existe la
	adjunta de una transformación lineal. Sea $V=\R[X]$ con el producto interno 
    \[
        \left\langle \sum_{i=0}^na_iX^i,\sum_{j=0}^mb_jX^j\right\rangle=\sum_{i=0}^{r}a_ib_i,
        \quad
        r=\min\{n,m\}.
    \]
    En particular, si $p=\sum_{i=0}^na_iX^i$, entonces 
    \[
        \langle p,X^m\rangle=\langle X^m,p\rangle=\begin{cases}
            0 & \text{si $n< m$,}\\
            a_m & \text{si $n\geq m$}.
        \end{cases}
    \]
    Vamos a demostrar que la transformación lineal $f\colon V\to V$ definida en
    la base canónica de $\R[X]$ por
    \[
        f(X^k)=1+X+\cdots+X^k,\quad k\geq0,
    \]
    no tiene adjunta. En efecto, si existiera $f^*\colon V\to V$, entonces,
    para todo $k,l\geq0$, tendríamos
    \[
        \langle X^k, f^*(X^l)\rangle
        =\langle f(X^k),X^l\rangle
        =\langle 1+X+\cdots+X^k,X^l\rangle
        =\begin{cases}
            1 & \text{si $k\geq l$,}\\
            0 & \text{si $k<l$.}
        \end{cases}
    \]
    Luego $f^*(X^l)=X^l+X^{l+1}+\cdots\not\in\R[X]$, una contradicción.
\end{example}

\begin{prop}
	Sean $V$ y $W$ espacios vectoriales con producto interno y de dimensión
	finita y sea $f\in\hom(V,W)$. Si $\cB_V$ es una base ortonormal de $V$ y
	$\cB_W$ es una base ortonormal de $W$
	\[
		[f^*]_{\cB_W,\cB_V}=[f]_{\cB_V,\cB_W}^*.
	\]

	\begin{proof}
		Supongamos que 
		\[
		\cB_V=\{v_1,\dots,v_n\},
		\quad
		\cB_W=\{w_1,\dots,w_m\}.
		\]
		Supongamos además que 
		$[f]_{\cB_V,\cB_W}=(a_{ij})$ y que 
		$[f^*]_{\cB,\cB}=(b_{ij})$. Entonces
		\begin{align*}
			\langle f(v_i),w_j\rangle
			=\left\langle \sum_{k=1}^m a_{ki}w_k,w_j\right\rangle
			=\sum_{k=1}^ma_{ki}\delta_{kj}
			=a_{ji}.
		\end{align*}
		Por otro lado, 
		\begin{align*}
			\langle v_i,f^*(w_j)\rangle
			=\left\langle v_i,\sum_{k=1}^n b_{kj}v_k\right\rangle
			=\sum_{k=1}^n\overline{b_{kj}}\delta_{ik}=\overline{b_{ij}}.
		\end{align*}
		Luego $a_{ij}=\overline{b_{ji}}$ pues $\langle
		f(v_i),w_j\rangle=\langle v_i,f^*(w_j)\rangle$ para todo $i,j$.
	\end{proof}
\end{prop}

\begin{xca}
	Sea $V$ un espacio vectorial con producto interno. Sean $f,g\in\hom(V,V)$
	y supongamos que existen las transformaciones adjuntas de $f$ y de $g$.
	Demuestre las siguientes afirmaciones:
	\begin{enumerate}
		\item $(f+g)^*=f^*+g^*$.
		\item $(fg)^*=g^*f^*$.
		\item $(\lambda f)^*=\overline{\lambda}f^*$ para todo $\lambda\in\K$. 
		\item $f^{**}=f$. 
	\end{enumerate}
\end{xca}

\begin{xca}
	Sean $V$ y $W$ espacios vectoriales con producto interno y sea
	$f\in\hom(V,W)$ tal que existe $f^*$. Demuestre que valen las siguientes
	afirmaciones:
	\begin{enumerate}
		\item $\ker f^*=(\im f)^\perp$.
		\item $\im f^*\subseteq(\ker f)^\perp$ y vale la igualdad si $V$ y $W$ son de 
			dimensión finita.
	\end{enumerate}
\end{xca}

\begin{xca}
	\label{xca:autoadjunta:iso}
	Sean $V$ y $W$ dos espacios vectoriales con producto interno y de dimensión
	finita. Si $f\in\hom(V,W)$ es un isomorfismo entonces $f^*$ es un
	isomorfismo y vale que $(f^*)^{-1}=(f^{-1})^*$.
\end{xca}

%\begin{solution}[ejercicio~\ref{xca:autoadjunta:iso}]
%	
%\end{solution}

\section{El teorema de Schur}

\begin{thm}[Schur]
	\index{Schur!teorema de}
	\index{Teorema!de Schur}
	Sea $V$ un espacio vectorial con producto interno y de dimensión finita y
	sea $f\in\hom(V,V)$. Si $\chi_f$ tiene todas sus raíces en $\K$, entonces
	existe una base ortonormal de $V$ tal que la matriz de $f$ en esa base es
	triangular superior.

	\begin{proof}
		Procederemos por inducción en $n=\dim V$. 
        
        Como el caso $n=1$ es trivial, suponemos que el resultado es válido
        para todo endomorfismo de un espacio vectorial de dimensión $n-1$. Sean
        $\lambda\in\K$ y $v\in V$ tal que $\|v\|=1$ y $f(v)=\lambda v$. Entonces $f^*(v)=\overline{\lambda}v$. 
        Si $S=\langle v\rangle$ entonces $V=S\oplus
        S^\perp$.  Además $S^\perp$ es $f$-invariante pues si $w\in S^\perp$
        entonces
		\[
			\langle f(w),\mu v\rangle
			=\langle w,f^*(\mu v)\rangle
			=\langle w,\mu f^*(v)\rangle
			=\overline{\mu}\langle w,\lambda v\rangle
			=\overline{\mu\lambda}\langle w,v\rangle
			=0
		\]
		para todo $\mu\in\K$. Sea $g=f|_{S^\perp}$. Como $\chi_{g}$ divide a
		$\chi_f$ entonces $\chi_{g}$ también tiene a todas sus raíces en $\K$. 
		Entonces, como $\dim S^\perp=n-1$, por hipótesis inductiva, existe una
		base ortonormal de $S^\perp$ tal que la matriz de $g$ en esa base es
		triangular superior.
	\end{proof}
\end{thm}

\begin{cor}
    \framebox{P}
	Toda matriz cuadrada compleja es semejante a una matriz triangular
	superior.

	\begin{proof}
		Es consecuencia inmediata del teorema de Schur.
	\end{proof}
\end{cor}

\begin{block}
	\index{Matriz!hermitiana}
	\index{Matriz!unitaria}
    Recordemos que una matriz $A\in\C^{n\times n}$ es \textbf{hermitiana} si
    $A^*=A$, donde $(A^*)_{ij}=\overline{A_{ji}}$ para todo $i,j$. Una matriz
    $A\in\C^{n\times n}$ es \textbf{unitaria} si $AA^*=A^*A=I$. 
\end{block}

\begin{example}
	Las matrices
	$
	\begin{pmatrix}
		1 & -i\\
		i & 1\\
	\end{pmatrix}
	$
	y 
	$
	\begin{pmatrix}
		0 & -i\\
		i & 0
	\end{pmatrix}
	$
	son hermitianas. 
	La matriz $\begin{pmatrix} 1 & 0\\ 0 & i\end{pmatrix}$ 
	es unitaria. 
\end{example}

%%%
%Sea $\{v_1,\dots,v_n\}$ una base ortonormal de $V$. Si $\{e_1,\dots,e_n\}$ es
%la base canónica de $\R^n$, escribimos $v_j=\sum_{i=1}^n a_{ij}e_i$. Luego
%\begin{align*}
%	\delta_{ij}=\langle v_i,v_j\rangle=\sum_{k=1}^n\sum_{j=1}^na_{??}\overline{a_{??}}\langle e_i,e_j\rangle.
%\end{align*}

\begin{cor}
    \label{cor:Schur}
	Sea $A\in\C^{n\times n}$ una matriz hermitiana. Entonces existe una matriz
	unitaria $P$ tal que $PAP^{-1}$ es diagonal. 

	\begin{proof}
		Por el teorema de Schur, existe una matriz unitaria $P$ tal que $PAP^*$
		es triangular superior. Como $PAP^*=(PAP^*)^*$ es también
		triangular inferior, concluimos que $PAP^*$ es diagonal.
	\end{proof}
\end{cor}

%\begin{block}
%    Recordemos que una matriz $A\in\R^{n\times n}$ es \textbf{simétrica} si
%    $A=A^T$.  Una matriz $A\in\R^{n\times n}$ es \textbf{ortogonal} si
%    $AA^T=A^TA=I$. 
%\end{block}
%
%\begin{cor}
%    Sea $A\in\R^{n\times n}$ una matriz simétrica. Entonces existe una matriz
%    ortogonal $P$ tal que $PAP^{-1}$ es diagonal. 
%
%	\begin{proof}
%        Es consecuencia inmediata del corolario~\ref{cor:Schur}.
%	\end{proof}
%\end{cor}

\section{Transformaciones lineales autoadjuntas}

\begin{block}
	\label{Transformación lineal!autoadjunta}
	Sea $V$ un espacio vectorial con producto interno. Una transformación
	lineal $f\in\hom(V,V)$ es \textbf{autoadjunta} si existe la adjunta $f^*$
	de $f$ y vale $f^*=f$. 
\end{block}

\begin{xca}
	\label{xca:autoadjunta}
	Sea $V$ un espacio vectorial con producto interno y de dimensión finita.
	Demuestre que si $f\in\hom(V,V)$ y $\cB$ es una base ortonormal de $V$
	entonces $f$ es autoadjunta si y sólo si $[f]_{\cB}=[f]_{\cB}^*$. En
	particular, si $\K=\R$ (resp. $\K=\C$) entonces $f$ es autoadjunta si y
	sólo si $[f]_{\cB}$ es simétrica (resp. hermitiana).
\end{xca}

\begin{prop}
	Sea $V$ un espacio vectorial con producto interno y de dimensión finita.
	Sea $p\colon V\to V$ un proyector.  Entonces $p$ es un proyector ortogonal
	si y sólo si $p$ es autoadjunto.

	\begin{proof}
		Supongamos que $p$ es un proyector ortogonal. Como $V=\ker p\oplus \im p$, 
		si $v,w\in V$ entonces $v=v_1+v_2$ y $w=w_1+w_2$ con
		$v_1,w_1\in\ker p$ y $v_2,w_2\in\im p$. En particular, $p(v)=v_2$ y
		$p(w)=w_2$. Como $\ker p\perp\im p$, 
		\begin{align*}
			&\langle p(v),w\rangle=\langle v_2,w_1+w_2\rangle=\langle v_2,w_2\rangle
			=\langle v_1+v_2,w_2\rangle=\langle v,p(w)\rangle.
		\end{align*}
		
		Recíprocamente, supongamos que $p$ es un proyector autoadjunto. Si $v\in\ker p$ y
		$w\in\im p$, entonces $p(v)=0$ y $p(w)=w$. Luego
		\[
			0=\langle 0,w\rangle=\langle p(v),w\rangle=\langle v,p(w)\rangle=\langle v,w\rangle,
		\]
		tal como se quería demostrar.
	\end{proof}
\end{prop}

\begin{prop}
	\label{pro:autoadjunta}
	Sea $V$ un espacio vectorial con producto interno y sea $f\in\hom(V,V)$.
	Supongamos que $f$ es autoadjunta.  Entonces:
	\begin{enumerate}
		\item Todo autovalor de $f$ es real.
		\item Si $v,w\in V$ son autovectores de $v$ de autovalores distintos
			entonces $\langle v,w\rangle=0$.
	\end{enumerate}

	\begin{proof}
		Demostremos la primera afirmación. Sean $\lambda\in\K$ y $v\in
		V\setminus\{0\}$ tales que $f(v)=\lambda v$. Como 
		\[
		\lambda\langle v,v\rangle
		=\langle\lambda v,v\rangle
		=\langle f(v),v\rangle
		=\langle v,f^*(v)\rangle
		=\langle v,f(v)\rangle
		=\langle v,\lambda v\rangle
		=\overline{\lambda}\langle v,v\rangle,
		\]
		y $v\ne 0$, entonces $\lambda\in\R$ pues $\lambda=\overline{\lambda}$. 

		Demostremos ahora la segunda afirmación. Sean $\lambda,\mu\in\K$
		autovalores distintos con autovectores $v,w\in V$. Entonces, como
		$\lambda$ y $\mu$ son números reales, 
		\[
			\lambda\langle v,w\rangle
			=\langle \lambda v,w\rangle
			=\langle f(v),w\rangle
			=\langle v,f^*(w)\rangle
			=\langle v,f(w)\rangle
			=\langle v,\mu w\rangle
			=\mu\langle v,w\rangle.
		\]
		Como $\lambda\ne\mu$, entonces $\langle v,w\rangle=0$.
	\end{proof}
\end{prop}

\begin{prop}
    Sea $V$ un espacio vectorial con producto interno y de dimensión finita.
    Toda $f\in\hom(V,V)$ autoadjunta tiene al menos un autovalor. 

	\begin{proof}
		Supongamos que $V$ es un espacio vectorial real, ya que si $V$ fuera un espacio vectorial complejo el resultado es válido
		gracias al teorema fundamental del álgebra. 
		Sea $\cB=\{v_1,\dots,v_n\}$ una base ortonormal de $V$, sea
		$A=[f]_{\cB,\cB}\in\R^{n\times n}$ y sea $g\colon\C^n\to\C^n$ definida por $g(x)=Ax$.
		Como $f$ es autoadjunta, $A$ es simétrica y $g$ es autoadjunta. El polinomio
		característico $\chi_g$ de $g$ tiene grado $n$, se factoriza
		linealmente en $\C$ y luego existe un autovalor $\lambda$ de $g$. Como
		$g$ es autoadjunta, $\spec f=\spec g\subseteq\R$. 
	\end{proof}
\end{prop}

\begin{thm}
	Sea $V$ un espacio vectorial con producto interno y de dimensión finita.
	Sea $f\in\hom(V,V)$ autoadjunta. Entonces existe una base ortonormal $\cB$
	tal que $[f]_{\cB}$ es diagonal.

	\begin{proof}
        Es consecuencia inmediata del corolario~\ref{cor:Schur} y del
        ejercicio~\ref{xca:autoadjunta}.
	\end{proof}
\end{thm}

\begin{xca}
	Sea $V$ un espacio vectorial con producto interno y de dimensión finita.
	Sea $f\in\hom(V,V)$. Demuestre que $f$ es autoadjunta si y sólo si existe
	una base ortonormal de autovectores con autovalores reales.
\end{xca}

\section{Descomposición en valores singulares}

\begin{thm}
    \label{thm:valores_singulares}
    \index{Valores singulares!descomposición}
    Sean $V$ y $W$ espacios vectorial con producto interno y de dimensión
    finita. Sea $f\in\hom(V,W)$ y sean $n=\dim V$ y $m=\dim W$. Entonces
    existen $k\leq\min\{n,m\}$, escalares $\sigma_1,\dots,\sigma_k\in\R_{>0}$ y
    bases ortonormales $\{v_1,\dots,v_n\}$ de $V$ y $\{w_1,\dots,w_m\}$ de $W$
    tales que 
    \[
        f(v_i)=\begin{cases}
            \sigma_i w_i & \text{si $i\in\{1,\dots,k\}$},\\
            0 & \text{en otro caso,}
        \end{cases}
        \quad
        f^*(w_i)=\begin{cases}
            \sigma_i v_i & \text{si $i\in\{1,\dots,k\}$},\\
            0 & \text{en otro caso.}
        \end{cases}
    \]  
    En particular, 
    \begin{enumerate}
        \item $\ker f=\langle v_{k+1},\dots,v_n\rangle$.
        \item $\im f=\langle w_1,\dots,w_k\rangle$.
        \item $\ker f^*=\langle w_{k+1},\dots,w_m\rangle$.
        \item $\im f^*=\langle v_1,\dots,v_k\rangle$.
        \item Para cada $v\in V$ se tiene que 
        $f(v)=\sum_{i=1}^k\sigma_i\langle v,v_i\rangle w_i$.
    \end{enumerate}

    \begin{proof}
        Si $f=0$ no hay nada para demostrar. Supongamos entonces que $f\ne0$.
        Sea $g=f^*f$.  Como $g\in\hom(V,V)$ es autoadjunto, existe una base
        ortonormal $\{v_1,\dots,v_n\}$ formada por autovectores $v_i$ de $g$ de
        autovalor $\lambda_i$. Como $g$ es autoadjunta, los $\lambda_i$ son
        reales y no negativos pues 
        \[
            \lambda_i
            =\lambda_i\langle v_i,v_i\rangle
            =\langle \lambda_iv_i,v_i\rangle
            =\langle g(v_i),v_i\rangle
            =\langle f(v_i),f(v_i)\rangle\geq0
        \]
        para todo $i\in\{1,\dots,n\}$. Sin pérdida de generalidad podemos suponer que
        \[
            \lambda_1\geq\lambda_2\geq\cdots\geq\lambda_k>0,
            \quad
            \lambda_{k+1}=\cdots=\lambda_n=0.
        \]
        Para cada $i\in\{1,\dots,k\}$ sean $\sigma_i=\sqrt\lambda_i$ y
        $w_i=\sigma_i^{-1}f(v_i)\in W$. Si $i\ne j$ entonces 
        \begin{align*}
            \langle w_i,w_j\rangle
            &=(\sigma_i\sigma_j)^{-1}\langle f(v_i),f(v_j)\rangle\\
            &=(\sigma_i\sigma_j)^{-1}\langle g(v_i),v_j\rangle
            =(\sigma_i\sigma_j)^{-1}\lambda_i\langle v_i,v_j\rangle=0.
        \end{align*}
        Además 
        \begin{align*}
            \langle w_i,w_i\rangle
            =\sigma_i^{-2}\langle f(v_i),f(v_i)\rangle
            =\sigma_i^{-2}\langle g(v_i),v_i\rangle
            =\sigma_i^{-2}\lambda_i\langle v_i,v_i\rangle=1.
        \end{align*}
        Luego $\{w_1,\dots,w_k\}$ es un conjunto ortonormal. 

        Por construcción $f(v_i)=\sigma_iw_i$ para todo $i\in\{1,\dots,k\}$. Además 
        \[
        f^*(w_i)=f^*(\sigma_i^{-1}f(v_i))=\sigma_i^{-1}g(v_i)=\sigma_i^{-1}\lambda_iv_i=\sigma_iv_i
        \]
        para todo $i\in\{1,\dots,k\}$. 

        Si $i\in\{k+1,\dots,n\}$ entonces $g(v_i)=0$ y, como 
        \[
            0=\langle g(v_i),v_i\rangle=\langle f(v_i),f(v_i)\rangle,
        \]
        se concluye que $v_i\in\ker f$. Esto demuestra la fórmula que queríamos
        para los $f(v_i)$. Si $k<m$, sea $\{w_{k+1},\dots,w_m\}$ una base
        ortonormal de $\ker f^*$. Entonces, como $\ker f^*=(\im f)^\perp$, el
        conjunto $\{w_1,\dots,w_k,w_{k+1},\dots,w_m\}$ es una base ortonormal
        de $W$. 
    \end{proof}
\end{thm}

\begin{block}
    \index{Valor singular}
    Los escalares $\sigma_1\geq\sigma_2\geq\cdots\geq\sigma_k$ del
    teorema~\ref{thm:valores_singulares} se denominan \textbf{valores
    singulares} de $f$.
\end{block}

\begin{cor}
    Sea $A\in\C^{m\times n}$. Entonces existe $k\in\min\{n,m\}$ y existen
    matrices unitarias $P\in\C^{m\times m}$, $Q\in\C^{n\times n}$ y una matriz
    diagonal real $D=\diag(\sigma_1,\dots,\sigma_k,0,\dots,0)\in\R^{m\times n}$, donde
    $\sigma_1\geq\sigma_2\geq\cdots\geq\sigma_k$ son los valores singulares de
    $A$, tales que 
    \[
    A=PDQ^*.
    \]
    Más aún, valen las siguientes afirmaciones:
    \begin{enumerate}
        \item Las primeras $k$ columnas de $P$ dan una base ortonormal de $\rg
            A$.
        \item Las últimas $m-k$ columnas de $P$ dan una base ortonormal de
            $\ker A^*$.
        \item Las primeras $k$ columnas de $Q$ dan una base ortonormal de $\rg
            A^*$.
        \item Las últimas $n-k$ columnas de $Q$ dan una base ortonormal de
            $\ker A$.
    \end{enumerate}

	\begin{proof}
		Sea $V=\K^{n\times1}$, $W=\K^{m\times 1}$ y $f\colon V\to W$ dado por
		$x\mapsto Ax$. La descomposición en valores singulares aplicada a $f$
		nos dice que existe una base ortonormal $\cB_V=\{v_1,\dots,v_n\}$ de
		$V$ y una base ortonormal $\cB_W=\{w_1,\dots,w_m\}$ de $W$ tal que
		\[
		[f]_{\cB_V,\cB_W}=\diag(\sigma_1,\dots,\sigma_k,0,\dots,0)=D,
		\]
		donde $\sigma_1,\dots,\sigma_k$ son los valores singulares de $f$. Si
		$\cE_V$ y $\cE_W$ son las bases canónicas de $V$ y $W$,
		respectivamente, entonces la matriz $f$ con respecto a estas bases es
		$A$. Las matrices  
		\[
		P=C(\cB_W,\cE_W)=(w_1|\cdots|w_m),
		\quad
		Q=C(\cB_V,\cE_V)=(v_1|\cdots|v_n),
		\]
		son unitarias y vale que 
		\[
		A=PDQ^*,
		\]
		tal como se quería demostrar.
	\end{proof}
\end{cor}

\begin{example}
	\label{exa:valores_singulares}
	Sea 	
	\[
	A=\left(\begin{array}{ccc}
		3 & 1 & 1\\
		-1 & 3 & 1
	\end{array}\right)\in\mathbb{R}^{2\times3}.
	\]
	Vamos a encontrar la descomposición en valores singulares de $A$. 
	Queremos entonces $P\in\mathbb{R}^{2\times2}$ ortogonal ,
	$Q\in\mathbb{R}^{3\times3}$ ortogonal y $D\in\mathbb{R}^{2\times3}$ tales
	que $A=PDQ^T$, donde 
	\[
	D=\left(\begin{array}{ccc}
		\sigma_{1} & 0 & 0\\
		0 & \sigma_{2} & 0
	\end{array}\right),
	\]
	con $\sigma_{1}\geq\sigma_{2}\geq0$, es la matriz de valores singulares
	de $A$. Como $A$ 
	tiene rango dos, tendremos dos valores singulares no nulos, es decir
	$\sigma_{1}\geq\sigma_{2}>0$. Empecemos por calcular la matriz $Q$.
	Calculemos entonces los autovalores de la matriz 
	\[
	A^{T}A=\begin{pmatrix}
		10 & 0 & 2\\
		0 & 10 & 4\\
		2 & 4 & 2
	\end{pmatrix}.
	\]
	Los autovalores de $A^TA$ son las soluciones de la ecuación
	\[
	0=\det(A^{T}A-\lambda I)=-\lambda^{3}+22\lambda^{2}-120\lambda=-(\lambda-12)(\lambda-10)\lambda.
	\]
	Luego $\lambda$ es autovalor de $A^TA$ si y sólo si
	$\lambda\in\{12,10,0\}$.  Los valores singulares no nulos de $A$ serán
	entonces $\sigma_{1}=\sqrt{12}$ y $\sigma_{2}=\sqrt{10}$. Calculemos los
	autovectores asociados a estos autovalores. Por ejemplo, para $\lambda=12$
	tenemos que resolver el sistema 
	\[
	(A^{T}A-4I)\left(\begin{array}{c}
		x_{1}\\
		x_{2}\\
		x_{3}
	\end{array}\right)=\left(\begin{array}{ccc}
		-2 & 0 & 2\\
		0 & -2 & 4\\
		2 & 4 & 0
	\end{array}\right)\left(\begin{array}{c}
		x_{1}\\
		x_{2}\\
		x_{3}
	\end{array}\right)=\left(\begin{array}{c}
		0\\
		0\\
		0
	\end{array}\right).
	\]
	La solución de este sistema es el espacio vectorial generado por el vector
	$(1,2,1)$. Análogamente, el autoespacio asociado al autovalor $\lambda=10$
	es el espacio generado por $(2,-1,0)$ y el autoespacio asociado al
	autovalor $\lambda=0$ está generado por el vector $(1,2,-5)$.  Ahora usamos
	el proceso de ortonormalización de Gram-Schmidt en cada autoespacio de
	$A^{T}A$. Dejamos como ejercicio chequear que de esta forma obtenemos la
	base ortonormal 
	\[
	\left\{ w_{1}=\frac{1}{\sqrt{6}}(1,2,1),w_{2}=\frac{1}{\sqrt{5}}(2,-1,0),w_{3}=\frac{1}{\sqrt{30}}(1,2,-5)\right\} 
	\]
	y la diagonalización ortogonal de la matriz $A^{T}A$:
	\[
	A^{T}A=Q\begin{pmatrix}
		12 & 0 & 0\\
		0 & 10 & 0\\
		0 & 0 & 0
	\end{pmatrix}Q^{T},\quad Q=(w_{1}|w_{2}|w_{3})=\begin{pmatrix}
		\frac{1}{\sqrt{6}} & \frac{2}{\sqrt{5}} & \frac{1}{\sqrt{30}}\\
		\frac{2}{\sqrt{6}} & \frac{-1}{\sqrt{5}} & \frac{2}{\sqrt{30}}\\
		\frac{1}{\sqrt{6}} & 0 & \frac{-5}{\sqrt{30}}
	\end{pmatrix}.
	\]

	Para obtener entonces la descomposición en valores singulares necesitamos
	hacer algo similar para la matriz $AA^{T}$.  Los autovalores de la matriz 
	\[
	AA^{T}=\begin{pmatrix}
		11 & 1\\
		1 & 11
	\end{pmatrix}
	\]
	son $\lambda=12$ y $\lambda=10$. Tal como hicimos antes, calculamos una
	base de autovectores para $AA^{T}$ y usamos Gram-Schmidt en cada
	autoespacio para obtener una matriz ortogonal $P$ que diagonalice
	ortogonalmente a la matriz $AA^{T}$:
	\[
	AA^{T}=P\begin{pmatrix}
		12 & 0\\
		0 & 10
	\end{pmatrix}P^{T},\quad P=\begin{pmatrix}
		\frac{1}{\sqrt{2}} & \frac{1}{\sqrt{2}}\\
		\frac{1}{\sqrt{2}} & -\frac{1}{\sqrt{2}}
	\end{pmatrix}.
	\]
	La descomposición en valores singulares de la matriz $A$ es
	\[
	A=P\left(\begin{array}{ccc}
		\sqrt{12} & 0 & 0\\
		0 & \sqrt{10} & 0
	\end{array}\right)Q^{T}.
	\]
\end{example}

\begin{example}
	La descomposición en valores singulares de una matriz $A$ nos da bases
	ortonormales para los subespacios: $\ker(A)$, $\im(A^{T})$, $\im(A)$ y
	$\ker(A^{T})$.  Por lo visto en el ejemplo~\ref{exa:valores_singulares}, si
	\[
	A=\left(\begin{array}{ccc}
		3 & 1 & 1\\
		-1 & 3 & 1
	\end{array}\right)\in\mathbb{R}^{2\times3}
	\]
	entonces 
	\begin{align*}
		&\left\{ \frac{1}{\sqrt{2}}(1,1),\frac{1}{\sqrt{2}}(1,-1)\right\} \text{ es una base ortonormal de }\im (A),\\
		&\left\{ \frac{1}{\sqrt{6}}(1,2,1),\frac{1}{\sqrt{5}}(2,-1,0)\right\} \text{ es una base ortonormal de }\im (A^{T}),\\
		&\left\{ \frac{1}{\sqrt{30}}(1,2,-5)\right\} \text{ es una base ortonormal de }\ker (A),
	\end{align*}
	y $\ker (A^{T})=\{0\}$. 
\end{example}

\begin{cor}[descomposición polar]
	\index{Descomposición polar}
	Si $A\in\K^{n\times n}$ entonces $A$ se escribe como $A=BC$, donde
	$B\in\K^{n\times n}$ es unitaria y $C\in\K^{n\times n}$ es autoadjunta con
	autovalores no negativos. 
	%Más aún, si $A$ es inversible, las matrices $B$ y
	%$C$ están unívocamente determinadas.

	\begin{proof}
		Por la descomposición en valores singulares,
		teorema~\ref{thm:valores_singulares}, $A=PDQ^*$, donde $P$ y $Q$ son
		matrices unitarias. Si $B=PQ^*$ y $C=QDQ^*$ entonces $B$ es
		unitaria, $C$ es autoadjunta con autovalores no negativos y 
		\[
			A=(PQ^*)(QDQ^*),
		\]
		tal como se quería demostrar.
%
%		Demostremos ahora que si $A$ es inversible y $A=BC=B_1C_1$ con
%		$B,B_1\in\K^{n\times n}$ unitarias y $C,C_1\in\K^{n\times n}$
%		autoadjuntas con autovalores no negativos, entonces $B=B_1$ y $C=C_1$.
%		Como $A$ es inversible, $B_1$ y $C$ son también inversibles. Entonces
%		\[
%			B_1^{-1}B=C_1C^{-1}.
%		\]
%		Como $B_1$ y $B$ son unitarias y $(B_1^*)^{-1}=(B^{-1})^*$, 
%		\[
%		(B_1^{-1}B)(B_1^{-1}B)^*=B_1^{-1}BB^*(B_1^{-1})^*=B_1^{-1}(B_1^*)^{-1}=(B_1^*B_1)^{-1}=I,
%		\]
%		lo que demuestra que $B_1^{-1}B=C_1C^{-1}$ es unitaria. Luego, como
%		\[
%			I=(C_1C^{-1})^*(C_1C^{-1})=C^{-1}C_1^2C^{-1},
%		\]
%		se concluye que $C^2=C_1^2$.\framebox{COMPLETAR}
	\end{proof}
\end{cor}

\section{Transformaciones lineales normales}

\begin{block}
	Sean $V$ un espacio vectorial con producto interno y $f\in\hom(V,V)$ que
	una transformación lineal que admite adjunta. Diremos que $f$ es
	\textbf{normal} si $f^*f=ff^*$.
\end{block}

\begin{example}
    Sea $\theta\in\R$ tal que $\theta$ no es un múltiplo entero de $\pi$. Si
    $f\colon \R^2\to\R^2$ está dada por 
    \[
        f(x,y)=
            \begin{pmatrix}           
            \cos\theta & \sin\theta\\
            -\sin\theta & \cos\theta
        \end{pmatrix}
        \colvec{2}{x}{y}
    \]
    entonces $f$ es normal pues $f^*f=f^*f=\id$ y no es autoadjunta. 
\end{example}

\begin{lem}
	\label{lem:normales}
    Sea $V$ un espacio vectorial con producto interno y de dimensión finita y
    sea $f\in\hom(V,V)$ una transformación lineal normal. Entonces $v$ es
    autovector de $f$ de autovalor $\lambda$ si y sólo si $v$ es autovector de
    $f^*$ de autovalor $\overline{\lambda}$. 

	\begin{proof}
		Sea $\lambda\in\K$ y sea $g=f-\lambda\id_V$. Como $f$ es normal, $g$ es
		normal. Además $g^*=f^*-\overline{\lambda}\id_V$ y si $v\in V$ entonces 
		\begin{align*}
			\|g(v)\|^2&=\langle g(v),g(v)\rangle
			=\langle v,g^*g(v)\rangle\\
			&=\langle v,g^*g(v)\rangle
			=\langle g^*(v),g^*(v)\rangle
			=\|g^*(v)\|^2.
		\end{align*}
		Luego $g(v)=0$ si y sólo si $g^*(v)=0$, es decir $v$ es autovector de
		$f$ de autovalor $\lambda$ sólo si $v$ es autovector de $f^*$ de
		autovalor $\overline{\lambda}$. 
	\end{proof}
\end{lem}

\begin{thm}
	\label{thm:normales:diagonal}
	Sea $V$ un espacio vectorial complejo con producto interno y de dimensión
	finita y sea $f\in\hom(V,V)$ una transformación lineal normal. Entonces
	existe una base ortonormal de $V$ tal que $[f]_{\cB,\cB}$ es diagonal. 

	\begin{proof}
		Procederemos por inducción en $n=\dim V$. 
		
		El caso $n=1$ es trivial, así que supongamos que el resultado es válido
		para todo endomorfismo normal de un espacio vectorial de dimensión
		$n-1$.  Como estamos sobre los números complejos, existe un autovalor
		$\lambda\in\C$. Sea $v\in V$ un autovector de $f$ de autovalor
		$\lambda$. Como $v\ne 0$, sin pérdida de generalidad podemos suponer
		que $\|v\|=1$. Por el lema~\ref{lem:normales} sabemos que
		$f^*(v)=\overline{\lambda}v$. Sea $S=\langle v\rangle^\perp$. Entonces
		$S$ es $f$-invariante pues si $w\in S$ entonces
		\[
		\langle f(w),v\rangle=\langle w,f^*(v)\rangle=\langle w,\overline{\lambda} v\rangle=\lambda\langle w,v\rangle=0.
		\]
		Similarmente demostramos que $f^*$ invariante pues si $w\in S$ entonces 
		\[
		\langle f^*(w),v\rangle=\langle w,f(v)\rangle=\langle w,\lambda v\rangle=\overline{\lambda}\langle w,v\rangle=0.
		\]
		Las restricciones $f|_S$ y $f^*|_S$ son entonces endomorfismos de $S$.
		Además $f|_S$ es normal pues $f^*|_S=(f|_S)^*$.  Como $\dim S=\dim
		V-1$, la hipótesis inductiva en $f|_S$ implica que existe una base
		ortonormal de $S$ tal que la matriz de $f|_S$ en esa base es diagonal.
		De aquí se deduce el teorema.
	\end{proof}
\end{thm}

\begin{cor}
	Sea $V$ un espacio vectorial complejo con producto interno y de dimensión
	finita y sea $f\in\hom(V,V)$. Entonces $f$ es normal si y sólo si existe
	una base ortonormal formada por autovectores de $f$.

	\begin{proof}
		Una de las implicaciones es el teorema anterior. Recíprocamente, si
		existe una base ortonormal $\cB$ tal que $[f]_{\cB,\cB}$ es diagonal,
		la matriz $[f^*]_{\cB,\cB}=[f]_{\cB,\cB}^*$ es también diagonal. Luego
		$[f]_{\cB,\cB}$ y $[f^*]_{\cB,\cB}$ conmutan, lo que implica que $f$ y
		$f^*$ conmutan.
	\end{proof}
\end{cor}

\begin{thm}[teorema espectral]
	\label{thm:espectral}
    Sea $V$ un espacio vectorial complejo con producto interno y de dimensión
    finita y sea $f\in\hom(V,V)$ una transformación lineal normal. Supongamos que
    $\chi_f=\prod_{i=1}^k(X-\lambda_i)^{m_i}$ con $\lambda_i\ne\lambda_j$ si
    $i\ne j$. Para cada $i\in\{1,\dots,k\}$ sean $V_i=\ker (f-\lambda_i\id_V)$
    y $p_i\colon V\to V$ el proyector ortogonal con $\im p_i=V_i$. Entonces:
    \begin{enumerate}
        \item $V=V_1\oplus\cdots\oplus V_k$ con $V_i\perp V_j$ si $i\ne j$. 
		\item $p_1+\cdots+p_k=\id_V$ con $p_ip_j=0$ si $i\ne j$.
        \item $f=\lambda_1p_1+\cdots+\lambda_kp_k$.
    \end{enumerate}

    \begin{proof}
		Veamos la primera afirmación. 
		Sean $i,j\in\{1,\dots,n\}$ con $i\ne j$ y sean $v_i\in
		V_i\setminus\{0\}$ y $v_j\in V_j\setminus\{0\}$. Entonces, por el lema~\ref{lem:normales}, 
		\[
			\lambda_i\langle v_i,v_j\rangle
			=\langle \lambda_iv_i,v_j\rangle
			=\langle f(v_i),v_j\rangle
			=\langle v_i,f^*(v_j)\rangle
			=\langle v_i,\overline{\lambda_j}v_j\rangle
			=\lambda_j\langle v_i,v_j\rangle
		\]
		y luego $\langle v_i,v_j\rangle=0$ pues $\lambda_i\ne\lambda_j$. Por el
		teorema~\ref{thm:normales:diagonal}, existe una base de $V$ formada
		por autovalores, luego $V=V_1\oplus\cdots\oplus V_k$. 

		Para demostrar la segunda afirmación, sea $v\in V$. Como, por el ítem
		anterior,  $v=v_1+\cdots+v_k$ con $v_j\in V_j$ para todo
		$j\in\{1,\dots,k\}$, entonces $p_i(v)=p_i(v_i)=v_i$ para todo
		$i\in\{1,\dots,k\}$. Luego
		\[
		(p_1+\cdots+p_k)(v)=\sum_{i=1}^k\sum_{j=1}^k p_i(v_j)=v_1+\cdots+v_k=v.
		\]
		Para ver que $p_ip_j=0$ si $i\ne j$ se usa el ejercicio~\ref{xca:pSpT}.

		Como $p_i|_{V_j}=0$ si $i\ne j$ y $p_i|_{V_i}=\id_{V_i}$, se tiene que 
		\[
		(\lambda_1p_1+\cdots+\lambda_kp_k)|_{V_{j}}=\lambda_k\id_{V_j}=f|_{V_j}.
		\]
		Como $V=V_1\oplus\cdots\oplus V_k$, esto implica que
		$f=\lambda_1p_1+\cdots+\lambda_kp_k$. 
    \end{proof}
\end{thm}

\begin{lem}
	\label{lem:normal:polinomio}
	Sea $V$ un espacio vectorial y sean $p_1,\dots,p_n\colon V\to V$
	proyectores tales que $p_1+\cdots+p_n=\id_V$ y $p_ip_j=0$ si $i\ne j$. Si 
    \[
    f=\lambda_1p_1+\cdots+\lambda_np_n,
    \] 
    donde $\lambda_1,\dots,\lambda_n\in\K$, entonces, para cada $p\in\K[X]$, se
    tiene que 
    \[
        p(f)=p(\lambda_1)p_1+\cdots+p(\lambda_n)p_n. 
    \]

	\begin{proof}
		Basta demostrar el lema para $p\in\{X^m:m\geq0\}$. Para esto, procederemos por
		inducción en $m$.

		El caso $m=1$ es trivial pues $f=\lambda_1p_1+\cdots+\lambda_np_n$.
		Supongamos entonces que el resultado es válido para $p=X^{m-1}$ y veamos que
		vale para $p=X^m$. La hipótesis inductiva, $p_i^2=p_i$ para todo
		$i\in\{1,\dots,n\}$ y $p_ip_j=0$ si $i\ne j$ implican que 
		\begin{align*}
		f^{m+1}&=f^mf=\left(\sum_{i=1}^n\lambda_i^mp_i\right)\left(\sum_{j=1}^n\lambda_jp_j\right)\\
		&=\sum_{i=1}^n\sum_{j=1}^n\lambda_i^m\lambda_jp_ip_j=\sum_{i=1}^m\lambda_i^{m+1}p_i,
		\end{align*}
		que demuestra el lema.	
	\end{proof}
\end{lem}

\begin{thm}
    Sea $V$ un espacio vectorial complejo con producto interno y de dimensión
    finita. Sea $f\in\hom(V,V)$. Entonces $f$ es normal si y sólo si existe
    $p\in\C[X]$ tal que $f^*=p(f)$. 

	\begin{proof}
		\framebox{FIXME}
		Si $f^*=p(f)$ para algún $p\in\C[X]$ entonces $f$ es normal pues $p(f)$
		conmuta con $f$. Supongamos entonces que $f$ es normal. Por
		el teorema espectral, teorema~\ref{thm:espectral}, existen
		$\lambda_1,\dots,\lambda_k\in\C$ con $\lambda_i\ne\lambda_j$ si $i\ne
		j$ y proyectores $p_1,\dots,p_k\colon V\to V$ tales que
		$f=\lambda_1p_1+\cdots+\lambda_kp_k$ y $p_ip_j=0$ si $i\ne j$. Sea
		$p\in\C[X]$ un polinomio de grado $k$ tal que
		$p(\lambda_i)=\overline{\lambda_i}$ para todo $i\in\{1,\dots,k\}$.
		Por el lema~\ref{lem:normal:polinomio},
		\[
		p(f)=\sum_{i=1}^kp(\lambda_i)p_i=\sum_{i=1}^k\overline{\lambda_i}p_i.
		\]
		Por otro lado, como 
		\[
		f^*=(\lambda_1p_1+\cdots+\lambda_kp_k)^*=\overline{\lambda_1}p_1+\cdots+\overline{\lambda_k}p_k, 
		\]
		se tiene $p(f)=f^*$, tal como quería demostrar.
	\end{proof}
\end{thm}

\section{Clasificación de transformaciones ortogonales}

\begin{block}
    Sea $V$ un espacio vectorial real. Diremos que $f\in\hom(V,V)$ es 
    \textbf{ortogonal} si exite la adjunta de $f$ y además $f^*f=f^*f=\id_V$. 
\end{block}

\begin{xca}
    \label{xca:ortogonal}
    Sea $V$ un espacio vectorial real con producto interno y de dimensión
    finita. Demuestre que son equivalentes:
    \begin{enumerate}
        \item $\|f(v)\|=\|v\|$ para todo $v\in V$.
        \item $\langle f(v),f(w)\rangle=\langle v,w\rangle$ para todo $v,w\in
            V$.
        \item Si $\{v_1,\dots,v_n\}$ es base ortonormal de $V$ entonces
            $\{f(v_1),\dots,f(v_n)\}$ es base ortonormal de $V$.
        \item $f^*f=\id_V$.
    \end{enumerate}
\end{xca}

\begin{lem}
    \label{lem:ortogonal}
    Sea $V$ un espacio vectorial real con producto interno y de dimensión
    finita. Sea $f\in\hom(V,V)$ una transformación ortogonal. Valen las
    siguientes afirmaciones:
    \begin{enumerate}
        \item Si $\lambda$ es autovalor de $f$ entonces $\lambda\in\{-1,1\}$.
        \item Si $S\subseteq V$ es un subespacio $f$-invariante entonces
            $S^\perp$ es $f$-invariante.
    \end{enumerate}

    \begin{proof}
        Para demostrar la primera afirmación observamos que si $v$ es
        autovector de autovalor $\lambda$ entonces, 
        \[
        |\lambda|\langle v,v\rangle=\lambda\overline{\lambda}\langle v,v\rangle=\langle f(v),f(v)\rangle=\langle v,f^*f(v)\rangle=\langle v,v\rangle.
        \]
        Como $v\ne0$, se obtiene que $|\lambda|=1$. Luego, como $\lambda\in\R$,
        se concluye que $\lambda\in\{-1,1\}$. 

        Demostremos ahora la segunda afirmación: si $v\in S^\perp$ y $w\in S$
        entonces $\langle v,w\rangle=0$.  Como $f^*f=ff^*=\id_V$, $f$ es
        sobreyectiva. En particular, la restricción $f|_S\colon S\to S$ es también
        sobreyectiva. Luego $w$ puede escribirse como $w=f(s)$ para algún $s\in
        S$. Tenemos entonces que $f(v)\in S^\perp$ pues 
        \[
        \langle f(v),w\rangle=\langle f(v),f(s)\rangle=\langle v,f^*f(s)\rangle=\langle v,s\rangle=0,
        \]
        que es lo que se quería demostrar.
    \end{proof}
\end{lem}

\begin{block}
    Sea $V$ un espacio vectorial real con producto interno y de dimensión
    finita.  Una \textbf{rotación} de $V$ es una transformación lineal $V\to V$
    tal que $\det f=1$. Si $S\subseteq V$ es un \textbf{hiperplano} (es decir,
    un subespacio de dimensión $\dim V-1$) entonces $f\colon V\to V$ es una
    \textbf{simetría} respecto de $S$ si $f|_S=\id_S$ y
    $f|_{S^\perp}=-\id_{S^\perp}$. 
\end{block}

\begin{prop}
    \label{pro:ortogonales:dim=2}
    Sea $V$ un espacio vectorial real con producto interno y tal que $\dim
    V=2$. Sea $f\colon V\to V$ una transformación ortogonal. Entonces $f$ es
    una simetría o una rotación. 

   \begin{proof}
		Como $f$ es ortogonal, entonces $f^*f=ff^*=\id_V$. Si $\cB=\{v_1,v_2\}$
		es una base ortonormal de $V$ entonces $\{f(v_1),f(v_2)\}$ es una base
		ortonormal de $V$ pues
        \[
        \langle f(v_i),f(v_j)\rangle=\langle v_i,f^*f(v_j)\rangle=\langle v_i,v_j\rangle=\delta_{ij}
        \]
        para todo $i,j$. Sean $a,b,c,d\in\R$ tales que 
        \[
            f(v_1)=av_1+bv_2,\quad
            f(v_2)=cv_1+dv_2.
        \]
        Entonces $\{(a,b),(c,d)\}$ es una base ortonormal de $\R^2$ pues
        \[
            a^2+b^2=c^2+d^2=1,\quad
            ac+bd=0. 
        \]
        Esto implica que $a^2+b^2=1$ y $(c,d)\in\{(-b,a),(b,-a)\}$. En efecto, 
		como $a^2+b^2=1$, entonces, al multiplicar por $c^2$, se tiene que 
		$(ac)^2+(bc)^2=c^2$. Luego, como $ac=-bd$, 
		\[
		b^2=b^2(d^2+c^2)=(bd)^2+(bc)^2=c^2,
		\]
		o bien $c\in\{-b,b\}$. Luego, si $c=\pm b$, entonces
		$(c,d)\in\{(-b,a),(b,-a)\}$ tal como habíamos afirmado.
		
		Se tienen
        entonces dos posibilidades:

        Si $(c,d)=(-b,a)$ entonces 
        \[
        [f]_{\cB,\cB}=\begin{pmatrix}
            a & -b\\
            b & a
        \end{pmatrix},\quad
        \chi_f=X^2-2aX+1. 
        \]
		Si $\chi_f$ tiene raíces reales, entonces las raíces están en
		$\{-1,1\}$. Esto implica que $(a,b)\in\{(1,0),(-1,0)\}$ y luego
		$f=\pm\id_V$. Supongamos entonces que $\chi_f$ no tiene raíces reales. Esto implica que $(2a)^2-4a<0$ y 
		entonces, como $a\in[-1,1]$, se tiene que $a\in(0,1)$. 
		Existe entonces $\theta\in[0,2\pi)$ tal que $a=\cos\theta$ y
		$b=\sin\theta$. Luego
        \[
        [f]_{\cB,\cB}=\begin{pmatrix}
            \cos\theta & -\sin\theta\\
            \sin\theta & \cos\theta
        \end{pmatrix}
        \]
        y $f$ es una rotación de ángulo $\theta$. 

        Si $(c,d)=(b,-a)$ entonces, como 
        \[
        [f]_{\cB,\cB}=\begin{pmatrix}
            a & b\\
            b & -a
        \end{pmatrix}
		\]
		es una matriz simétrica y $\chi_f=X^2-1$, existe una base ortonormal en
		cuya base la matriz de $f$ es $\diag(1,-1)$, es decir, $f$ es una
		simetría.  
   \end{proof}
\end{prop}

\begin{prop}
    \label{pro:ortogonales:dim=3}
    Sea $V$ un espacio vectorial real con producto interno y tal que $\dim
    V=3$. Sea $f\colon V\to V$ una transformación ortogonal. Entonces $f$ es
    una simetría, una rotación, o una composición de una simetría y una
    rotación. 

    \begin{proof}
        Como $\chi_f$ es de grado tres, entonces tiene una raíz real, es decir
        $f$ tiene un autovalor real. Por el lema~\ref{lem:ortogonal}
        sabemos que este autovalor está en $\{-1,1\}$. 

        Si $\lambda=1$ es autovalor de $f$, sea $v_1\in V$ autovector de
        autovalor $\lambda=1$ tal que $\|v_1\|=1$. Sea $S=\langle v_1\rangle$.
        Como $v_1$ es autovector, $S$ es $f$-invariante. Además $S^\perp$ es
        $f$-invariante por el lema~\ref{lem:ortogonal}. La restricción
        $f|_{S^\perp}\colon S^\perp\to S^\perp$ es una transformación ortogonal
        si consideramos a $S^\perp$ con el producto interno inducido por el
        producto interno de $V$. Como $\dim S^\perp=2$, la
        proposición~\ref{pro:ortogonales:dim=2} implica que existe una base
        ortonormal $\{v_2,v_3\}$ de $S^\perp$ tal que, en esa base, la matriz
        de $f|_{S^\perp}$ es 
        \[
        \begin{pmatrix}
            1 & 0\\
            0 & -1
        \end{pmatrix}
        \text{ o bien }
        \begin{pmatrix}
            \cos\theta & -\sin\theta\\
            \sin\theta & \cos\theta
        \end{pmatrix}
        \text{ para algún $\theta\in[0,2\pi)$.}
        \]
        Luego $\cB=\{v_1,v_2,v_3\}$ es una base ortonormal de $V$ y vale que 
         \[
         [f]_{\cB,\cB}=
         \begin{pmatrix}
            1 & 0 & 0\\
            0 & 1 & 0\\
            0 & 0 & -1
        \end{pmatrix},
        \]
        lo que significa que $f$ es una simetría con respecto al subespacio
        $\langle v_1,v_2\rangle$, o bien que 
        \[
         [f]_{\cB,\cB}=
        \begin{pmatrix}
            1 & 0 & 0\\
            0 & \cos\theta & -\sin\theta\\
            0 & \sin\theta & \cos\theta
        \end{pmatrix}
        \]
        para algún $\theta\in[0,2\pi)$.

        Si $1$ no es autovalor de $f$ entonces, por lo visto anteriormente,
        $\lambda=-1$ es autovalor de $f$. Sea $v_1\in V$ un autovector de
        autovalor $\lambda=-1$ tal que $\|v_1\|=1$. Tal como se hizo en el caso
        anterior, se demuestra que existe una base ortonormal
        $\cB=\{v_1,v_2,v_3\}$ de $V$ tal que 
        \[
         [f]_{\cB,\cB}=
        \begin{pmatrix}
            -1 & 0 & 0\\
            0 & \cos\theta & -\sin\theta\\
            0 & \sin\theta & \cos\theta
        \end{pmatrix}
        \]
        para algún $\theta\in[0,2\pi)$. El teorema queda demostrado al observar que
        \[
        [f]_{\cB,\cB}=
        \begin{pmatrix}
            -1 & 0 & 0\\
            0 & \cos\theta & -\sin\theta\\
            0 & \sin\theta & \cos\theta
        \end{pmatrix}
        =
        \begin{pmatrix}
            -1 & 0 & 0\\
            0 & 1 & 0\\
            0 & 0 & 1
        \end{pmatrix}
        \begin{pmatrix}
            1 & 0 & 0\\
            0 & \cos\theta & -\sin\theta\\
            0 & \sin\theta & \cos\theta
        \end{pmatrix},
        \]
        es decir que $[f]_{\cB,\cB}$ es la composición de una simetría y una
        rotación. 
    \end{proof}
\end{prop}

\begin{thm}
    Sea $V$ un espacio vectorial real y de dimensión finita. Sea $f\colon V\to V$
    una transformación ortogonal. Entonces 
       existe una base ortonormal $\cB$ de $V$ y 
    existen $\theta_1,\dots,\theta_k\in[0,\pi)$ tales que 
    \[
    [f]_{\cB,\cB}=\begin{pmatrix}
        I_r\\
        &-I_s\\
        &&R_{\theta_1}\\
        &&&\ddots\\
        &&&&R_{\theta_k}
    \end{pmatrix},
    \]
    donde $r,s\in\N_0$, $I_r$ es la matriz identidad de $r\times r$, $I_s$ es
    la matriz identidad de $s\times s$, $\theta_1,\dots,\theta_k\in[0,2\pi)$ y
    para cada $j\in\{1,\dots,k\}$ 
    \[
    R_{\theta_j}=
    \begin{pmatrix}
        \cos\theta_j & -\sin\theta_j\\
        \sin\theta_j & \cos\theta_j
    \end{pmatrix}.
    \]

    \begin{proof}
        Procederemos por inducción en $n=\dim V$.

        El caso $n=2$ fue demostrado en la
        proposición~\ref{pro:ortogonales:dim=2}. Supongamos entonces que $n>2$
        y que el resultado es válido para transformaciones ortogonales
        definidas en espacios de dimensión $<n$. Sea $V$ un espacio de
        dimensión $n$ y sea $f\colon V\to V$ una transformación ortogonal.  Por
        el lema~\ref{lem:ortogonal} hay tres casos a considerar:

        Supongamos que $\lambda=1$ es autovalor de $f$. Sea $v_1\in V$
        autovector de autovalor $\lambda=1$ tal que $\|v_1\|=1$. Entonces
        $S=\langle v_1\rangle$ es un subespacio $f$-invariante y $S^\perp$ es
        también $f$-invariante por. Como $\dim S^\perp=n-1$, la hipótesis
        inductiva aplicada a la restricción $f|_{S^\perp}\colon S^\perp\to
        S^\perp$ implica que existe una base ortonormal $\{v_2,\dots,v_n\}$ de
        $S^\perp$ tal que, en esa base, la matriz de $f|_{S^\perp}$ es de la
        forma deseada.  Observemos que $\{v_1,v_2,\dots,v_n\}$ es una base
        ortonormal de $V$ y que, en esa base, la matriz de $f$ es de la forma
        buscada.
        
        Si $1$ no es autovalor de $f$ y $-1$ es autovalor de $f$, se tiene un
        autovector $v_1$ de autovalor $-1$ y tal que $\|v_1\|=1$. En este caso
        se procede como se hizo en el caso anterior. 

        Si $f$ no tiene autovalores reales entonces $m_f=p_1\cdots p_k$, donde
        para cada $i\in\{1,\dots,k\}$ se tiene que $p_i\in\R[X]$ es irreducible
        de grado dos. Sea $q=p_2\cdots p_k$. Entonces $q$ divide a $m_f$ y
        además, como $q\ne m_f$, existe $w\in V$ tal que $q(f)(w)\ne0$. Si
        $v=q(f)(w)$ entonces 
        \[
        0=m_f(f)(w)=((p_1q)(f))(w)=p_1(f)(q(f)(w))=p_1(f)(v).
        \]
        Si $S=\langle v,f(v)\rangle$ entonces $S$ es $f$-invariante pues
        $p_1(f)(v)=0$ y $\deg p_1=2$; además, como $v$ no es autovector de $f$,
        se tiene que $\dim S=2$. Luego, como la restricción $f|_S\colon S\to S$ es
        ortogonal, existe una base ortonormal de $S$ tal que, en esa base, la
        matriz de $f|_S$ es de la forma $R_{\theta_1}$ para algún $\theta_1\in[0,2\pi)$.
        Como $S^\perp$ es $f$-invariante y $\dim S^\perp=n-2$, por hipótesis
        inductiva, existe una base ortonormal $\{v_3,\dots,v_n\}$ de $S^\perp$
        tal que, en esa base, la matriz de $f|_{S^\perp}$ es
        \[
        \begin{pmatrix}
            R_{\theta_2}\\
            &\ddots\\
            &&R_{\theta_k}
        \end{pmatrix},
        \]
        donde $\theta_2,\dots,\theta_k\in[0,2\pi)$. (Sabemos que las
        restricción $f|_{S^\perp}$ no tiene autovalores reales pues el
        polinomio característico de $f|_{S^\perp}$ divide $\chi_f$.) 
        El conjunto $\cB=\{v,f(v),v_3,\dots,v_n\}$ es ortonormal y la matriz de
        $f$ en la base $\cB$ es de la forma deseada.
    \end{proof}
\end{thm}
 

%\chapter{Formas bilineales}

%\begin{block}[matriz de un producto interno]
%    \framebox{definicion}
%
%    \framebox{ejemplo}
%
%    \framebox{es hermitiana pero no vale la vuelta}
%
%    \framebox{calcular <v,w> con la matriz}
%\end{block}

\section{Formas bilineales simétricas}

\begin{block}
    Sea $V$ un espacio vectorial. Una \textbf{forma bilineal} sobre $V$ es una
    función $B\colon V\times V\to\K$ tal que
    \begin{enumerate}
        \item $B(u+v,w)=B(u,w)+B(v,w)$ para todo $u,v,w\in V$.
        \item $B(\lambda v,w)=\lambda B(v,w)$ para todo $v,w\in V$ y $\lambda\in\K$. 
        \item $B(u,v+w)=B(u,v)+B(u,w)$ para todo $u,v,w\in V$.
        \item $B(v,\lambda w)=\lambda B(v,w)$ para todo $v,w\in V$ y $\lambda\in\K$. 
    \end{enumerate}
    Una forma bilineal $B\colon V\times V\to\K$ es \textbf{simétrica} si
    $B(v,w)=B(w,v)$ para todo $v,w\in V$.
\end{block}

\begin{example}
    Si $A\in\R^{n\times n}$ es simétrica entonces $B\colon\R^n\times\R^n\to\R$
    dada por $B(x,y)=xAy^T$ es una forma bilineal simétrica. 
\end{example}

\begin{xca}
    \label{xca:simetrica:diagonal}
    Demuestre que si $B$ es una forma bilineal simétrica de un espacio
    vectorial real o complejo $V$ entonces
    \[
        B(v,w)=\frac12\left(B(v+w,v+w)-B(v,v)-B(w,w)\right)
    \]
    para todo $v,w\in V$. 
\end{xca}

\begin{block}
    Sea $\cB=\{v_1,\dots,v_n\}$ una base de $V$ y 
\end{block}






\chapter{Cocientes}

\section{Espacio cociente}

\begin{block}
	Sea $V$ un espacio vectorial sobre $\K$ y sea $S\subseteq V$ un subespacio.
	Se define la relación de equivalencia en $V$ dada por
    \[
        u\equiv v\bmod S\Leftrightarrow u-v\in S.
    \]
    Si $u\equiv v\bmod S$ diremos que $u$ y $v$ son \textbf{equivalentes módulo
    $S$}. Cada clase de equivalencia módulo $S$ de $V$ es de la forma $v+S$ con $v\in V$ pues
    \begin{align*}
        \{u\in V:\; &u\equiv v\bmod S\}\\
        &=\{u\in V:u-v\in S\}\\
        &=\{u\in V:u=v+s\text{ para algún $s\in S$}\}\\
        &=\{v+s:s\in S\}.
    \end{align*}
    Las clases de equivalencia de $V$ módulo $S$ se denominan \textbf{coclases}
    de $S$ en $V$. El conjunto de coclases de $S$ en $V$ se denota por
    $V/S=\{v+S:v\in V\}$.
\end{block}

\begin{block}
    \label{block:V/S}
   Si $S\subseteq V$ es un subespacio entonces en $V/S$ pueden definirse las
   siguientes operaciones:
   \begin{align*}
       &(u+S)+(v+S)=(u+v)+S, && u,v\in V,\\
       &\lambda(v+S)=\lambda v+S, && v\in V,\;\lambda\in\K.
   \end{align*}

   Veamos que las operaciones están bien definidas. 

   Supongamos que $u_1+S=u_2+S$ y sea $\lambda\in\K$. Entonces $u_1-u_2\in
   S$ y, como $S$ es un subespacio, $\lambda u_1-\lambda
   u_2-\lambda(u_1-u_2)\in S$. Luego $\lambda u_1+S=\lambda u_2+S$ y
   entonces la multiplicación por escalares en $V/S$ está bien definida. 

   Supongamos ahora que $u_1+S=u_2+S$ y que $v_1+S=v_2+S$. Entonces
   $u_1-u_2\in S$ y $v_1-v_2\in S$ y, como $S$ es un subespacio, 
   \[
   (u_1-u_2)+(v_1-v_2)=(u_1+v_1)-(u_2+v_2)\in S.
   \]
   Luego $u_1+v_1+S=u_2+v_2+S$ y entonces la suma en $V/S$ está bien
   definida. 
\end{block}

\begin{xca}
    Demostrar que con la suma y el producto por escalares definidos
    en~\ref{block:V/S}, el cociente $V/S$ es un espacio vectorial. 
\end{xca}

\begin{examples}
	Es evidente que $V/\{0\}\simeq V$ y que $V/V\simeq\{0\}$. 
	\begin{enumerate}
		\item $V/0\simeq V$.
		\item $V/V\simeq 0$.
		\item $\R^2/\langle(1,1)\rangle$.
	\end{enumerate}
\end{examples}

\section{Teoremas de isomorfismo}

\begin{prop}
    Sea $S\subseteq V$ un subespacio. La función 
    \[
        p_S\colon V\to V/S,\quad v\mapsto v+S
    \]
    es un epimorfismo y $\ker p_S=S$. La
    transformación lineal $p_S$ se denomina el \textbf{epimorfismo canónico} de
    $V$ en $V/S$.

    \begin{proof}
        La función $p_S$ es una transformación lineal pues 
        \begin{align*}
            &p_S(u+v)=(u+v)+S=(u+S)+(v+S)=p_S(u)+p_S(v),\\
            &p_S(\lambda v)=(\lambda v)+S=\lambda(v+S)=\lambda p_S(v),
        \end{align*}
        para todo $u,v\in V$ y $\lambda\in\K$. Es evidente que $p_S$ es un
        epimorfismo. Calculemos $\ker(p_S)$. Es claro que
        $S\subseteq\ker(p_S)$, y si $v\in\ker(p_S)$ entonces $p_S(v)=v+S=S$ y
        luego $v\in S$. 
    \end{proof}
\end{prop}

\begin{cor}
    Sea $V$ un espacio vectorial de dimensión finita. Entonces $V/S$ es también
    de dimensión finita y 
    \[
    \dim(V/S)=\dim(V)-\dim(S).
    \]
    La dimensión de $V/S$ se denomina \textbf{codimensión} de $S$ en $V$ y se
    denota por $\codim(S)$.

    \begin{proof}
        Si $V$ es de dimensión finita entonces, por el teorema de la dimensión
        aplicado al epimorfismo canónico $p_S$, 
        \[
            \dim(V)=\dim\ker(p_S)+\dim\im(p_S)=\dim(S)+\dim(V/S), 
        \]
        que es equivalente a lo que se quería demostrar.
    \end{proof}
\end{cor}

\begin{cor}
    Sea $V$ un espacio vectorial y sean $S,T\subseteq V$ subespacios tales que
    $V=S\oplus T$. Entonces $T\simeq V/S$, es decir: todo complemento de $S$ en
    $V$ es isomorfo a $V/S$.

    \begin{proof}
        Sea $f\colon T\to V/S$ la transformación lineal dada por $t\mapsto
        t+S$. Veamos que $f$ es monomorfismo:
        \[
        \ker f=\{t\in T:f(t)=S\}=\{t\in T:t\in S\}=S\cap T=\{0\}
        \]
        pues $V=S\oplus T$. Veamos que $f$ es epimorfismo: 
        \[
        \im f=\{f(t):t\in T\}=\{t+S:t\in T\}=\{v+S:v\in V\}
        \]
        pues todo $v\in V$ se escribe unívocamente como $v=s+t$ con $s\in S$ y
        $t\in T$. Luego $T\simeq V/S$.
    \end{proof}
\end{cor}

\begin{thm}[teorema de la correspondencia de subespacios]
   Sea $S\subseteq V$ un subespacio. La función
   \begin{align*}
        \{\text{subespacios de $V$ que contienen a $S$}\} \to \{\text{subespacios de $V/S$}\}, 
    \end{align*}
    dada por $T \mapsto p(T)$, donde $p\colon V\to V/S$ es el epimorfismo canónico, es biyectiva.

    \begin{proof}
		Observemos que si $T$ es un subespacio de $V$ y $S\subseteq
		T$ entonces $p(T)$ es un subespacio de $V/S$ y así la
		función $T\mapsto p(T)$ del enunciado está bien definida.  Veamos que la función
   		\begin{align*}
    	    \{\text{subespacios de $V$ que contienen a $S$}\} &\to \{\text{subespacios de $V/S$}\}\\
			p^{-1}(L) & \mapsfrom L
	    \end{align*}
		está bien definida: 
		si
		$L\subseteq V/S$ es un subespacio, $p^{-1}(L)$ es un subespacio
		de $V$ que contiene a $S$ pues si $s\in S$ entonces $p(s)=0\in L$.

		Veamos que $p^{-1}(L)\mapsfrom L$ es la inversa de $T\mapsto p(T)$.
		Primero observemos que $p(p^{-1}(L))=L$ pues $p$ es sobreyectiva.  Por
		otro lado, debemos demostrar que $p^{-1}(p(T))=T$.  Como es evidente
		que $p^{-1}(p(T))\supseteq T$, basta demostrar que
		$p^{-1}(p(T))\subseteq T$.  Si $v\in p^{-1}(p(T))$, como entonces
		$p(v)\in p(T)$, se tiene que $p(v)=p(t)$ para algún $t\in T$. Luego
		$v-t\in\ker p=S\subseteq T$ y por lo tanto $v\in T$. 
    \end{proof}
\end{thm}

\begin{thm}[propiedad universal del cociente]
    \label{thm:propiedad_universal}
    Sean $V$ y $W$ espacios vectoriales, $f\in\hom(V,W)$ y $S\subseteq\ker f $
    un subespacio. Entonces existe una única $g\in\hom(V/S,W)$ tal que
    $gp_S=f$, es decir: existe una única $g\in\hom(V/S,W)$ que hace que el
    siguiente diagrama conmute:
    \[
    \xymatrix{ V\ar[r]^f\ar[d]_{p_S} & W\\ V/S\ar@{-->}[ru]_g & }
    \]
    Más aún, $\ker g=\ker f/S$ y
    $\im g=\im f$.

    \begin{proof}
        Definimos a $g$ como $g(v+S)=f(v)$. Veamos que está bien definida: si
        $v_1+S=v_2+S$ entonces $v_1-v_2\in S$. Como $S\subseteq\ker f$,
        $f(v_1-v_2)=0$ y luego $g(v_1+S)=f(v_1)=f(v_2)=g(v_2+S)$

        Como $f$ es una transformación lineal, es fácil ver que $g$ es también
        una transformación lineal. 
        Calculemos la imagen de $g$: 
        \begin{align*}
            \im g=\{g(v+S):v\in V\}=\{f(v):v\in V\}=\im f.
        \end{align*}
        Calculemos ahora el núcleo de $g$:
        \begin{align*}
            \ker g&=\{v+S:g(v+S)=0\}\\
            &=\{v+S:f(v)=0\}=\{v+S:v\in\ker f\}=\ker f+S.
        \end{align*}

        Queda demostrar la unicidad: si $h\in\hom(V/S,W)$ cumple que $hp_S=f$
        entonces $h(v+S)=hp_S(v)=f(v)=gp_S(v)=g(v+S)$ y luego $h=g$.
    \end{proof}
\end{thm}

\begin{cor}[primer teorema de isomorfismo]
    Sea $f\in\hom(V,W)$. Entonces la transformación lineal $V/\ker f\to W$,
    $v+\ker f\mapsto f(v)$, es inyectiva. En particular $V/\ker f\simeq \im f$.

    \begin{proof}
        Es consecuencia de la propiedad universal del cociente,
        teorema~\ref{thm:propiedad_universal}, aplicada al subespacio $S=\ker
        f$.
    \end{proof}
\end{cor}

\begin{cor}[segundo teorema de isomorfismo]
    Sean $S,T\subseteq V$ subespacios. Entonces 
    \[
    \frac{S+T}{T}\simeq \frac{S}{S\cap T}.
    \]

    \begin{proof}
        Sea $f\colon S+T\to S/S\cap T$ dada por $f(s+t)=s+(S\cap T)$. Primero
        debemos demostrar que $f$ está bien definida: si $s+t=s'+t'$ con
        $s,s'\in S$ y $t,t'\in T$ entonces $s'-s\in S\cap T$ y $s+(s'-s)=s'$.
        Luego $f(s+t)=f(s'+t')$. Como
        \[
            \ker f=\{s+t: s+S\cap T=S\cap T\}=\{s+t:s\in S\cap T\}=\{t+S\cap T\}=T,
        \]
        y $f$ es un epimorfismo, el primer teorema de isomorfismo demuestra el
        corolario.
    \end{proof}
\end{cor}

\begin{cor}[tercer teorema de isomorfismo]
    Sean $S\subseteq T\subseteq V$ subespacios. Entonces
    \[
        \frac{V/S}{T/S}\simeq \frac{V}{T}.
    \]

    \begin{proof}
        Sea $f\colon V/S\to T/S$ dada por $v+S\mapsto v+T$. Veamos la buena
        definición: si $v_1+S=v_2+S$ entonces $v_1-v_2\in S$ y luego, como
        $S\subseteq T$, $v_1+T=v_2+T$. Como
        \[
            \ker f=\{v+S:v+T=T\}=\{v+S:v\in T\}=T,
        \]
        y $f$ es epimorfismo, el primer teorema de isomorfismo nos da el
        isomorfismo que queríamos demostrar.
    \end{proof}
\end{cor}

%\include{variedades}
\chapter{Soluciones a los ejercicios}

\section{Capítulo 1}

\begin{solution}[ejercicio~\ref{xca:Ax=b_y_Ax=0}]
	Sea $y$ tal que $Ay=b$. Entonces $y-p$ es solución de $Ax=0$. Luego
	$y=(y-p)+p\in S+p$. Recíprocamente, si $y$ satisface $Ay=0$ entonces
	$A(y+p)=Ay+Ap=0+b=b$.
\end{solution}

\begin{solution}[ejercicio~\ref{xca:I^T=I}]
\end{solution}
 
\section{Capítulo 2}

\begin{solution}[ejercicio~\ref{xca:subespacio}]
    Si $S$ es un subespacio entonces es evidente que $v+\lambda w\in S$ para
    todo $v,w\in S$ y $\lambda\in\K$. Recíprocamente, como $S$ es no vacío,
    existe un elemento $v_0\in S$ y entonces $0=v_0+(-1)v_0\in S$. Si $v,w\in
    S$ entonces $v+w=v+1w\in S$. Por último, si $v\in S$ y $\lambda\in\K$
    entonces $\lambda v=v+(\lambda-1)v\in S$.
\end{solution}

\begin{solution}[ejercicio~\ref{xca:f(1)=0}]
	Si $f(1)=0$ entonces
	$X-1$ divide a $f$ y luego $f=(X-1)g$ para algún $g\in\R[X]$. 
\end{solution}
 
\begin{solution}[ejercicio~\ref{xca:X^2-3X+2|f}]
	Como $f(1)=f(2)=0$, el polinomio 
    \[
        X^2-3X+2=(X-1)(X-2)
    \]
    divide a $f$. Luego 
	$f=(X^2-3x+2)g$ para algún $g\in\R[X]$.
\end{solution}

\begin{solution}[ejercicio~\ref{xca:K[X]_no_fg}]
		
\end{solution}

\section{Capítulo 3}

\begin{solution}[ejercicio~\ref{xca:Silvester}]
    Para demostrar la primera afirmación, sea $g_1=g|_{\im f}$. Es evidente que
    $\ker g_1\subseteq\ker g$. Entonces, por el teorema de la dimensión, 
    \begin{align*}
        \dim\ker(gf)&=\dim U-\dim\im(gf)\\
        &=\dim U-\dim\im f+\dim\im f-\dim\im(gf)\\
        &=\dim\ker f+\dim\im f-\dim\im(gf)\\
        &=\dim\ker f+\dim\ker g_1\\
        &\leq \dim\ker f+\dim\ker g.
    \end{align*}

    Para demostrar la segunda desigualdad, primero observamos que 
    \[
        \dim\im(gf)\leq\dim\im g
    \]
    pues 
    $\im(gf)\subseteq\im g$. Por otro lado, como
    $\im(gf)\subseteq\im g_1$, tenemos 
    \begin{align*}
        \dim\im(gf)\leq\dim\im(g_1)=\dim\im f-\dim\ker g_1\leq\dim\im f.
    \end{align*}
    Luego,
    \[
    \dim\im(gf)\leq\min\{\dim\im g,\dim\im f\}.
    \]

    Para demostrar la tercera igualdad, usamos el teorema de la dimensión en
    $gf$ y el primer ítem y obtenemos
    \begin{align*}
        \dim\im(gf)&=\dim U-\dim\ker(gf)\\
        &\geq\dim U-\dim\ker f-\dim\ker g\\
        &=\dim U-\dim U+\dim\im f-\dim V+\dim\im g\\
        &=\dim\im f+\dim\im g-\dim V.
    \end{align*}
\end{solution}

\begin{solution}[ejercicio~\ref{xca:proyector(2f-1)^2}]
    Observemos que $(2f-\id_V)^2=\id_V$ si y sólo si
    $(2f-\id_V)(2f-\id_V)(v)=v$ para todo $v\in V$. Esto es equivalente a 
    \[
        4f^2(v)-4f(v)-v=v
    \]
    para todo $v\in V$, que a su vez equivale a $4f^2(v)=4f(v)$ para todo $v\in
    V$. Como el cuerpo de base es $\R$, esta última igualdad equivale a
    $f^2(v)=v$ para todo $v\in V$.
\end{solution}

\begin{solution}[ejercicio~\ref{xca:proyector}]
	Si $f$ es un proyector y $w\in \im f$ entonces $w=f(v)$ para algún $v\in V$
	y luego $f(w)=f(f(v))=f(v)=w$.  Recíprocamente, si $f(w)=w$ para todo
	$w\in\im f$ y $v\in V$ entonces $f(f(v))=f(v)$. 
\end{solution}

\begin{solution}[ejercicio~\ref{xca:pfp=fp}]
	Supongamos que $f(S)\subseteq S$ y sea $p\colon V\to V$ con $\im p=S$. Sea
	$v\in V$. Entonces $(fp)(v)\in S=\im p$ y luego existe $w\in V$ tal que
	$(fp)(v)=p(w)$. Al aplicar $p$ se obtiene
	\[
	(pfp)(v)=p^2(w)=p(w)=(fp)(v),
	\]
	que es lo que se quería demostrar.

	Supongamos ahora que $pfp=fp$ para todo proyector $p\colon V\to V$ con $\im
	p=S$. Si $s\in S$ entonces, como $S=\im p$ y $p(s)=s$,
	$f(s)=fp(s)=pfp(s)\in S$. 
\end{solution}

\begin{solution}[ejercicio~\ref{xca:oplus_proyectores}]
	Para demostrar que (1) implica (2) basta tomar las transformaciones
	lineales $p_i\colon V\to V$ dado por $v=s_1+\cdots+s_n\mapsto s_i$.
	Recíprocamente, si $s_1+\cdots+s_n=0$ entonces, al aplicar $p_i$ y utilizar
	que los $p_i$ son proyectores que satisfacen $p_ip_j=0$ si $i\ne j$, se
	obtiene que $s_i=0$ para todo $i$.
\end{solution}

\begin{solution}[ejercicio~\ref{xca:proyector_matriz}]
	Si $f$ es un proyector entonces $V=\ker f\oplus\im f$ y además $f(w)=w$
	para todo $w\in\im f$.  Sea $\{v_1,\dots,v_r\}$ una base de $\im f$ y
	$\{v_{r+1},\dots,v_n\}$ una base de $\ker f$.  Luego
	$\cB=\{v_1,\dots,v_n\}$ es base de $V$ y la matriz de $f$ en la base $\cB$
	tiene la forma deseada.

	Recíprocamente, si $\{v_1,\dots,v_r\}$ es base de $\im f$ y
	$\{v_{r+1},\dots,v_n\}$ es base de $\ker f$ entonces $f(w)=w$ para todo
	$w\in\im f$. Luego $f$ es un proyector por el
	ejercicio~\ref{xca:proyector}.
\end{solution}

\section{Capítulo 4}

\begin{solution}[ejercicio~\ref{xca:traza}]
    Si $i\ne j$ entonces 
    \[
        \delta(E^{ij})=\delta(E^{ii}E^{ij})=\delta(E^{ij}E^{ii})=\delta(0)=0.
    \]
    Por otro lado, para cada $i,k\in\{1,\dots,n\}$ tenemos 
    \[
        \delta(E^{ii})=\delta(E^{ik}E^{ki})=\delta(E^{ki}E^{ik})=\delta(E^{kk}).
    \]
    Sea entonces $\lambda=\delta(E^{11})$. Si
    escribimos a $A$ en la base de los $E^{ij}$ tenemos
    \begin{align*}
        \delta(A) & %\delta\left(\sum_{1\leq i,j\leq n}a_{ij}E^{ij}\right)
        =\sum_{1\leq i,j\leq n}a_{ij}\delta(E^{ij})
        =\sum_{i=1}^n a_{ii}\delta(E^{ii})
        =\lambda\sum_{i=1}^na_{ii}
        =\lambda\tr(A),
    \end{align*}
    tal como queríamos demostrar.
\end{solution}
\begin{solution}[ejercicio~\ref{xca:v=0<=>f(v)=0}]\
    Si $v=0$ entonces $f(v)=0$ para todo $f\in V^*$.
    Recíprocamente, si $v\ne0$, extendemos el conjunto linealmente
    independiente $\{v\}$ a una base $\cB$ de $V$. La existencia de la
    base dual a la base $\cB$ garantiza la existencia de un $f\in V^*$ tal que
    $f(v)=1$. 
\end{solution}

\begin{solution}[ejercicio~\ref{xca:V=<v>+kerf}]
    Como $f(v)\ne0$, todo $w\in V$ puede
    escribirse como 
    \begin{align*}
        w=w-\frac{f(w)}{f(v)}v+\frac{f(w)}{f(v)}v,
    \end{align*}
    donde $f\left(w-\frac{f(w)}{f(v)}v\right)=0$, y entonces $V=\ker
    f+\langle v\rangle$.  Si $w\in\ker f\cap\langle v\rangle$ entonces
    $f(w)=0$ y $w=\lambda v$ para algún $\lambda\in\K$. Luego $\lambda
    f(v)=0$ y entonces, como $f(v)\ne0$, $\lambda=0$ y $w=0$.
\end{solution}

\begin{solution}[ejercicio~\ref{xca:kerf=kerg}]
    Si existe $\lambda\in\K$ tal que $f=\lambda g$ entonces es
    evidente que $\ker f=\ker g$. Recíprocamente, supongamos que
    $f\ne0$ (de lo contrario, el resultado es trivial) y sea
    $\{v_1,\dots,v_{n-1}\}$ una base de $\ker f$. La extendemos a una
    base $\{v_1,\dots,v_n\}$ de $V$ y entonces, por construcción,
    $f(v_n)\ne0$ y $g(v_n)\ne0$.  Si definimos $\lambda=f(v_n)/g(v_n)$
    entonces $\lambda\ne0$ y luego $f=\lambda g$ pues coinciden en la
    base $\{v_1,\dots,v_n\}$.
\end{solution}

\begin{solution}[ejercicio~\ref{xca:annX=<X>}]
    Si $v\in\langle X\rangle$ entonces existen $x_1,\dots,x_n\in X$ y
    $\alpha_1,\dots,\alpha_n\in\K$ tales que $v=\sum_{i=1}^n\alpha_ix_i$.
    Luego $f(v)=0$ para toda $f\in\ann X$ y entonces $\ann
    X\subseteq\ann\langle X\rangle$. La recíproca es cierta pues si
    $f\in\ann\langle X\rangle$ entonces $X\subseteq\langle
    X\rangle\subseteq\ker f$.
\end{solution}

\begin{solution}[ejercicio~\ref{xca:dual:LI}]
	Supongamos que $\{f_1,\dots,f_n\}$ son linealmente independientes y sea
	$\{v_1,\dots,v_n\}$ una base de $V$ que tiene a la base de las $f_i$ como
	su base dual. Si escribimos $v=\sum_{i=1}^n\alpha_iv_i$ entonces para cada
	$j$ se tiene 
	\[
	0=f_j(v)=f_j\left(\sum_{i=1}^n\alpha_iv_i\right)=\sum_{i=1}^n\alpha_if_j(v_i)=\sum_{i=1}^n\alpha_i\delta_{ji}=\alpha_j.
	\]
	Luego $v=0$, una contradicción.
\end{solution}

\section{Capítulo 5}

\begin{solution}[ejercicio~\ref{xca:disjuntas_conmutan}]
	Sea $j\in\{1,\dots,n\}$. Si $\alpha(j)\ne j$ entonces $\beta(j)=j$ y además
	$\beta(\alpha(j))=\alpha(j)$ (pues de lo contrario tendríamos
	$\alpha(\alpha(j))=\alpha(j)$ que implica $\alpha(j)=j$). Luego
	\[
		(\alpha\beta)(j)=\alpha(\beta(j))=\alpha(j)=\beta(\alpha(j)).
	\]
	Similarmente se hace el caso en que $\beta(j)\ne j$. Por último, si
	$\alpha(j)=\beta(j)=j$ entonces trivialmente se obtiene
	$(\alpha\beta)(j)=(\beta\alpha)(j)$. 
\end{solution}

\begin{solution}[ejercicio~\ref{xca:permutaciones}]
	Si $k\geq0$ entonces
	\[
		\sigma^{k}(i)=(\alpha\beta)^{k}(i)=\alpha^{k}\left(\beta^{k}(i)\right)=\alpha^{k}(i). 
	\]
\end{solution}

\begin{solution}[ejercicio~\ref{xca:dA=(detA)dI}]
    Supongamos que $d$ es una función $n$-lineal y alternada sea $A\in\K^{n\times
    n}$. Escribamos a cada fila de $A$ como $A_i=\sum_{j=1}^n a_{ij}e_j$
    donde $\{e_1,\dots,e_n\}$ es la base canónica de $\K^{1\times n}$.
    Entonces
    \begin{align*}
        d(A)&=d(A_1,\dots,A_n)
        =\sum_{j_1=1}^n\cdots\sum_{j_n=1}^na_{1j_1}\cdots a_{nj_n}d(e_{j_1},\dots,e_{j_n}).
    \end{align*}
    Como $d$ es alternada, la suma es no nula únicamente cuando todos los
    $j_k$ son distintos, es decir cuando $|\{j_1,\dots,j_n\}|=n$. Entonces,
    la suma anterior se hace sobre todas las $n$-uplas $(j_1,\dots,j_n)$ de
    elementos distintos del conjunto $\{1,\dots,n\}$. Luego, por el
    lema~\ref{lem:sigma},  
    \begin{align*} 
        d(A)&=\sum_{\sigma\in\Sym_n}a_{1\sigma(1)}\cdots a_{n\sigma(n)}d(e_{\sigma(i)},\dots,e_{\sigma(n)})\\
        &=\sum_{\sigma\in\Sym_n}\sgn(\sigma)a_{1\sigma(1)}\cdots a_{n\sigma(n)}d(e_1,\dots,e_n)\\
        &=(\det A)d(I).
    \end{align*}
\end{solution}

\begin{solution}[ejercicio~\ref{xca:bloques_2x2}]
	Escribimos
	\[
	\det\begin{pmatrix}
		A & B\\
		0 & C
	\end{pmatrix}
	=\begin{pmatrix}
		I & 0\\
		0 & C
	\end{pmatrix}
	\begin{pmatrix}
		A & B\\
		0 & I
	\end{pmatrix}
	\]
	y observamos que 
	\begin{align*}
		\det\begin{pmatrix}
			I & 0\\
			0 & C
		\end{pmatrix}
		=\det C,
		&&
		\det\begin{pmatrix}
			A & B\\
			0 & I
		\end{pmatrix}
		=\det A.
	\end{align*}
\end{solution}

%\begin{solution}[ejercicio~\ref{xca:rango_submatriz}]
%	Si $\rg A\geq r$ entonces existen al menos $r$ columnas de $A$ que son
%	linealmente independientes. 
%\end{solution}


\begin{solution}[ejercicio~\ref{xca:lagrange}]
	Si $f=a_0+a_1X+\cdots+a_nX^n$ y $f(x_i)=y_i$ para todo
	$i\in\{1,\dots,n+1\}$ entonces 
	\[
	\begin{cases}
		\begin{aligned}
			& a_0+a_1x_1+\cdots+a_nx_1^n = y_1,\\
			& a_0+a_1x_2+\cdots+a_nx_2^n = y_2,\\
			&\quad\vdots\\
			& a_0+a_1x_{n+1}+\cdots+a_nx_{n+1}^{n} = y_{n+1}.
		\end{aligned}
	\end{cases}
	\]
	Este sistema tiene solución única pues el determinante de la matriz
	asociada es el determinante de Vandermonde \[
		V(x_1,\dots,x_{n+1})=\prod_{i<j}(x_i-x_j)\ne0.
	\]
\end{solution}

\begin{solution}[ejercicio~\ref{xca:adjadjA}]
	Como $A^{-1}=(\det A)^{-1}(\adj A)$ entonces $(\det A)A=(\adj A)^{-1}$. Por
	otro lado, al aplicar determinante en la igualdad $(\adj A)A=(\det A)I$, se
	obtiene $\det\adj A=(\det A)^{n-1}$. Luego
	\[
	\adj(\adj A)=(\det A)^{n-1}(\adj A)^{-1}=(\det A)^{n-2}A.
	\]
\end{solution}

\begin{solution}[ejercicio~\ref{xca:adj(BAB^(-1))}]
		Sabemos que \[
			\adj(BAB^{-1})(BAB^{-1})=\det(BAB^{-1})I=(\det A)I.
		\]
		Entonces
		\[
			\adj(BAB^{-1})=(\det A)BA^{-1}B=B(\adj A)B^{-1}
		\]
		pues $A^{-1}=(\det A)^{-1}\adj A$. 
\end{solution}

\begin{solution}[ejercicio~\ref{xca:determinante:A1000}]
    Sean $\alpha,\beta\in\R$ tales que $\alpha A^{1000}+\beta A^{1001}=0$. Si
    aplicamos la función determinante a la igualdad $\alpha A^{1000}=-\beta
    A^{1001}$, obtenemos $\alpha^4(\det A)^{1000}=(-\beta)^4(\det A)^{1001}$. Luego $\alpha^4+\beta^4=0$ y entonces
    $\alpha=\beta=0$.

\end{solution}
\section{Capítulo 6}

\begin{solution}[ejercicio~\ref{xca:spec(fg)=spec(gf)}]
    Sea $\lambda\in\spec(fg)$. Si $\lambda=0$ entonces $fg\not\in\Aut V$.
    Luego, como $f\not\in\Aut V$ o $g\not\in Aut V$, se tiene que
    $gf\not\in\Aut V$. Si $\lambda\ne0$ sea $v\in V\setminus\{0\}$ tal que
    $(fg)(v)=\lambda v\ne0$. Entonces, como $w=g(v)\ne0$, 
    \[
        (gf)(w)=(fgf)(v)=g(\lambda v)=\lambda g(v)=\lambda w.
    \]
    Luego $\lambda\in\spec(gf)$ y entonces $\spec(fg)\subseteq\spec(gf)$. 
\end{solution}
\section{Capítulo 7}

\begin{solution}[ejercicio~\ref{xca:f_invariante}]
    La primera afirmación es consecuencia de lo visto en~\ref{block:XY}.
    Demostremos la segunda afirmación. Sea 
	\[
        p=\lcm(m_{f|_S},m_{f|_T}).
    \]
    Como $m_{f|_S}$ y $m_{f|_T}$ dividen a $f$, entonces $p$ divide a $m_f$.
    Por otro lado, como $V=S\oplus T$, para cada $v\in V$ existe únicos $s\in
    T$ y $T\in T$ tales que $v=s+t$. Luego
    \[
        p(f)(v)=p(f)(s+t)=p(f)(s)+p(f)(t)=p(f|_S)(s)+p(f|_T)(t).
    \]
    Como $m_{f|_S}$ y $m_{f|_T}$ ambos dividen a $p$, entonces
    $p(f|_S)(s)=p(f|_T)(t)=0$. Luego $m_f$ divide a $p$ y entonces, como $m_f$
    y $p$ son mónicos, $m_f=p$.
\end{solution}

\begin{solution}[ejercicio~\ref{xca:auxiliar}]\
    \begin{enumerate}
        \item Primero observemos que, como $p(f)\circ f=f\circ p(f)$, entonces
            \[
            p(f)(f^i(v))=f^i(p(f)(v))\in C(p(f)(v))
            \]
            para todo $i\geq0$ y luego $p(C(v))\subseteq C(p(v))$. Recíprocamente, 
            \[
            f^i(p(f)(v))=p(f)(f^i(v))\in p(f)(C(v))
            \]
            para todo $i\geq0$ y luego $p(C(v))\supseteq C(p(v))$.
        \item Como $p(f)$ es un endomorfismo de $V$, entonces \[
            p(V)=p(V_1)+\cdots+p(V_k). 
            \]
            Para $i\in\{1,\dots,k\}$ sean $v_i\in V_i$ tales que
            $v_1+\cdots+v_k=0$. Entonces
            $p(f)(v_1)+\cdots+p(f)(v_k)=p(f)(v_1+\cdots+v_k)=0$. Como
            los $V_i$ son $p$-invariantes, $p(f)(v_i)\in V_i$ y luego, como
            los $V_i$ están en suma directa, $p(f)(v_i)=0$ para cada 
            $i\in\{1,\dots,k\}$. 
        \item Por definición, 
            \[
            0=m_{p(v)}(f)(p(f)(v))=(m_{p(v)}p)(f)(v)
            \]
            y entonces $m_v=m_w$ divide al polinomio $m_{p(v)}p$. En
            particular, como $m_{p(v)}p$ da cero al ser evaluado en $w$, obtememos
            \[
            0=(m_{p(v)}p)(f)(w)=m_{p(v)}(f)(p(f)(w)).
            \]
            Luego $m_{p(w)}$ divide a $m_{p(v)}$. Análogamente $p_{p(v)}$
            divide a $m_{p(w)}$ y  entonces, como $m_{p(v)}$ y $m_{p(w)}$ son
            mónicos, $m_{p(v)}=m_{p(w)}$. 

            Como $m_{p(v)}=m_{p(w)}$ y que $\deg
            m_{p(v)}=\dim C(p(v))$ por el lema~\ref{lem:dimC(v)=degm_v}, entonces
            $\dim C(p(v))=\dim C(p(w))$.
    \end{enumerate}
\end{solution}

\begin{solution}[ejercicio~\ref{xca:nilpotente_y_diagonalizable=0}]\
    Sean $\lambda_1,\dots,\lambda_n$ los autovalores de $f$. 
    Sea $\cB$ una base de $f$ tal que $[f]_{\cB,\cB}$ es una matriz diagonal
    cuya diagonal principal es $\lambda_1,\dots,\lambda_n$. Como $f$ es
    nilpotente, existe $m\in\N$ tal que $f^m=0$. Entonces 
    \[
    0=[f^m]_{\cB,\cB}=[f]_{\cB,\cB}^m.
    \]
    Esto implica que $\lambda_i^m=0$ para todo $i$ y por lo
    tanto $\lambda_i=0$ para todo $i$. 
\end{solution}

\begin{solution}[ejercicio~\ref{xca:nilpotente:autovalores}]\
    \begin{enumerate}
        \item Sea $r$ el índice de nilpotencia de $f$. Si $r=1$ entonces $f=0$
            y el ejercicio queda resuelto. Si $r>1$ entonces sea $v\in V$ tal
            que $f^r(v)=0$ y $f^{r-1}(v)\ne0$. Si $w=f^{r-1}(v)$ entonces
            $f(w)=0w$. 
        \item Sea $r$ el índice de nilpotencia de $f$  y sea $v\ne0$ tal que
            $f(v)=\lambda v$. Entonces $0=f^r(v)=\lambda^rv$. Como $v\ne0$
            entonces $\lambda^r=0$ y luego $\lambda=0$.
    \end{enumerate}
\end{solution}

\section{Capítulo 8}

\begin{solution}[ejercicio~\ref{xca:<v-w,x>=0}]
	Si $\langle v,x\rangle=\langle w,x\rangle$ entonces $\langle
	v-w,x\rangle=0$ y luego, al tomar $x=v-w$, se tiene $0=\|v-w\|^2=\langle
	v-w,v-w\rangle$, que implica que $v=w$. 
\end{solution}

\begin{solution}[ejercicio~\ref{xca:Pitagoras}]
	Si $v\perp w$ entonces $\langle v,w\rangle=\langle w,v\rangle=0$. Luego
	\begin{align*}
		\|v+w\|^2&=\langle v+w,v+w\rangle\\
		&=\langle v,v\rangle+\langle v,w\rangle+\langle w,v\rangle+\langle w,w\rangle=\|v\|^2+\|w\|^2.
	\end{align*}
\end{solution}

\begin{solution}[ejercicio~\ref{xca:Sperp}]\
	\begin{enumerate}
		\item Se deduce trivialmente de $\langle 0,v\rangle=\langle v,0\rangle=0$.
		\item Si $v\in T^\perp$ entonces $\langle v,w\rangle=0$ para todo $w\in
			T$. En particular, $v\in S^\perp$ pues, como $S\subseteq T$,
			$\langle v,w\rangle=0$ para todo $w\in S$.
		\item Como $S\subseteq S+T$ y $T\subseteq S+T$, entonces, por lo visto
			en ítem anterior, $(S+T)^\perp\subseteq S^\perp\cap T^\perp$.
			Recíprocamente, si $v\in S^\perp\cap T^\perp$ entonces, dados $x\in
			S$, $y\in T$, se tiene 
			\[
			\langle v,x+y\rangle=\langle v,x\rangle+\langle v,y\rangle=0+0=0. 
			\]
	\end{enumerate}
\end{solution}

\begin{solution}[ejercicio~\ref{xca:perpperp}]
	Por definición se tiene que $S\subseteq S^{\perp\perp}$. Por otro lado,
	como $V$ es de dimensión finita y $S$ y $S^\perp$ son subespacios de $V$,
	$\dim V=\dim S+\dim S^\perp=\dim S^\perp+\dim S^{\perp\perp}$. Como
	$\dim S=\dim S^{\perp\perp}$, entonces $S=S^{\perp\perp}$.
\end{solution}

\begin{solution}[ejercicio~\ref{xca:proyector_ortogonal}]\
	\begin{enumerate}
		\item Sea $v\in(im p)^\perp$. Como $v-p(v)\in\ker p$ y $\ker p\perp\im
			p$, entonces 
			\[
			\langle v,p(v)\rangle-\langle p(v),p(v)\rangle=\langle v-p(v),p(v)\rangle=0.
			\]
			Como $v\in(\im p)^\perp$ entonces $\langle v,p(v)\rangle=0$ y luego
			$\|p(v)\|^2$. Esto implica que $p(v)=0$ y que $v\in\ker p$. Por
			otro lado, como $p$ es un proyector ortogonal, $\ker p\perp\im p$,
			y entonces es evidente que $\ker p\subseteq(\im p)^\perp$. 
		\item Como $\ker p\perp\im p$ entonces $\im p\subseteq(\ker p)^\perp$.
			Por otro lado, si $v\in(\ker p)^\perp$ entonces, como
			$v-p(v)\in\ker p$ y $\ker p\perp\im p$, 
			\begin{align*}
				\|v-p(v)\|^2&=\langle v-p(v),v-p(v)\rangle\\
				&=\langle v,v-p(v)\rangle-\langle p(v),v-p(v)\rangle=0.
			\end{align*}
			Luego $v=p(v)\in\im p$. 
		\item Se deduce trivialmente de los ítems anteriores.
	\end{enumerate}
\end{solution}

\begin{solution}[ejercicio~\ref{xca:pSpT}]
	Como $\im p_S\perp\im p_T$, entonces 
	por el ejercicio~\ref{xca:proyector_ortogonal}, 
	$\im p_T\subseteq (\im p_S)^\perp=\ker
	p_S$. Luego $p_Sp_T=0$. 
\end{solution}


\printindex

\end{document}
