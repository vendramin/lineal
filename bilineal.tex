\chapter{Formas bilineales}

%\begin{block}[matriz de un producto interno]
%    \framebox{definicion}
%
%    \framebox{ejemplo}
%
%    \framebox{es hermitiana pero no vale la vuelta}
%
%    \framebox{calcular <v,w> con la matriz}
%\end{block}

\section{Formas bilineales simétricas}

\begin{block}
    Sea $V$ un espacio vectorial. Una \textbf{forma bilineal} sobre $V$ es una
    función $B\colon V\times V\to\K$ tal que
    \begin{enumerate}
        \item $B(u+v,w)=B(u,w)+B(v,w)$ para todo $u,v,w\in V$.
        \item $B(\lambda v,w)=\lambda B(v,w)$ para todo $v,w\in V$ y $\lambda\in\K$. 
        \item $B(u,v+w)=B(u,v)+B(u,w)$ para todo $u,v,w\in V$.
        \item $B(v,\lambda w)=\lambda B(v,w)$ para todo $v,w\in V$ y $\lambda\in\K$. 
    \end{enumerate}
    Una forma bilineal $B\colon V\times V\to\K$ es \textbf{simétrica} si
    $B(v,w)=B(w,v)$ para todo $v,w\in V$.
\end{block}

\begin{example}
    Si $A\in\R^{n\times n}$ es simétrica entonces $B\colon\R^n\times\R^n\to\R$
    dada por $B(x,y)=xAy^T$ es una forma bilineal simétrica. 
\end{example}

\begin{xca}
    \label{xca:simetrica:diagonal}
    Demuestre que si $B$ es una forma bilineal simétrica de un espacio
    vectorial real o complejo $V$ entonces
    \[
        B(v,w)=\frac12\left(B(v+w,v+w)-B(v,v)-B(w,w)\right)
    \]
    para todo $v,w\in V$. 
\end{xca}

\begin{block}
    Sea $\cB=\{v_1,\dots,v_n\}$ una base de $V$ y 
\end{block}





